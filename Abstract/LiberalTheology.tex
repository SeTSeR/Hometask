% Created 2020-04-09 Thu 17:27
% Intended LaTeX compiler: pdflatex
\documentclass[11pt]{article}
\usepackage[utf8]{inputenc}
\usepackage[T1]{fontenc}
\usepackage{graphicx}
\usepackage{grffile}
\usepackage{longtable}
\usepackage{wrapfig}
\usepackage{rotating}
\usepackage[normalem]{ulem}
\usepackage{amsmath}
\usepackage{textcomp}
\usepackage{amssymb}
\usepackage{capt-of}
\usepackage{hyperref}
\usepackage{minted}
\usepackage[english, russian]{babel}
\author{Sergey Makarov}
\date{}
\title{Фридрих Шлейермахер. Программа либеральной теологии и её критика}
\hypersetup{
 pdfauthor={Sergey Makarov},
 pdftitle={Фридрих Шлейермахер. Программа либеральной теологии и её критика},
 pdfkeywords={},
 pdfsubject={},
 pdfcreator={Emacs 26.3 (Org mode 9.3)}, 
 pdflang={Russian}}
\begin{document}

\maketitle
\section{Введение}
\label{sec:org3453b82}
Фридрих Шлейермахер оказал значительное влияние на развитие нынешней христианской мысли, его
называли {}<<отцом церкви XIX века>>{}\textsuperscript{1}. Он стоял в начале либеральной теологии -- крайне популярного
направления теологии в конце XIX-начале XX в. Либеральные теологи пытались {}<<осовременить>>{}
христианство и привести его в соответствие с достижениями науки, рационализировать Священное
Писание. Впоследствии это привело к сведению роли христианства до нравственного учения, а не
религиозного, и демифологизации Священного Писания. В этом реферате будет дан обзор идей
либеральной теологии и их критики более поздними теологами.
\section{Идеи либеральной теологии}
\label{sec:org2b52dbf}
В {}<<речах о религии>>{}\textsuperscript{2} Ф. Шлейермахер высказывает тезисы о сути религии, образовавшие фундамент для либеральной теологии.

Ключевой тезис Шлейермахера -- религия порождается чувством абсолютной взаимозависимости, связи всех вещей в мире и целого.
По мнению Шлейермахера, это чувство необходимо возникает в любой лучшей душе одним из двух способов: либо путём наблюдения за
внешним миром и отыскания внутренней связи между всеми явлениями и событиями, либо путём рефлексии и отыскания всевозможных
проявлений целого внутри себя: {}<<созерцайте самих себя с пристальным напряжением, отделите всё, что не есть ваше я, продолжайте
фиксировать ваше внутреннее содержание, -- и чем более, по мере удаления всего чуждого, ваша личность и ваше обособленное бытие
сужается и даже почти исчезает, тем яснее предстанет перед вами вселенная, тем прекраснее вы будете вознаграждены за ужас
самоуничтожения преходящего чувством вечного в вас>>{}\textsuperscript{2}. Поскольку религия порождается чувством, область религии есть область
чувства, поэтому религию невозможно свести ни к науке, ни к нравственности, ведь наука относится к сфере разума, а нравственность
относится к сфере действия и человеческих взаимоотношений. Все же претензии к религии от того, что религии пытаются приписать то,
чем она не является, например, механизмы познания и описания мира или функции регулятора взаимоотношений между людьми. Также
Шлейермахер показывает губительность вмешательства государства в религию, в частности, попыток возлагания на религию
ответственности за поддержание авторитета государства.

Другим важным следствием чувственной природы религии является вторичность религиозной догматики, понятий и принципов религии.
Система понятий и принципов есть лишь отражение религиозного чувства и потому по определению вторична и не может быть основой
религии. Религиозная догматика и обряды есть выражение религии на языке современной культуры, из чего вытекает два важных
следствия. Во-первых, все религии равноправны, поскольку являются одним и тем же, но только в конкретном культурном контексте
(вообще говоря, стоит указать, что это относится к так называемым {}<<позитивным>>{} религиям, вроде христианства или ислама;
языческие {}<<естественные>>{} религии Шлейермахер считает слишком неявными, неопределёнными, чтобы они могли служить выражением
религиозного чувства). Во-вторых, очень важно уметь трактовать священные тексты и адаптировать их к новым культурным условиям.
Эта идея легла в основу герменевтики.

Учение о герменевтике стало основой для демифологизации Священного Писания, вплоть до отказа от божественности Иисуса Христа
и признания его лишь идеалом нравственного человека, не закладывавшего никакого вероучения.
Стоит заметить, что такая редукция идёт вразрез с идеей Шлейермахера о несводимости религии к нравственности.
\section{Критика либеральной теологии}
\label{sec:org20c217b}
Кьеркегор в {}<<страхе и трепете>>{}\textsuperscript{3} показывает, что единственным способом постижения Бога является
осознание противоречия и асбурда, заложенного в возможности прямого общения с Богом. Тем самым
Кьеркегор критикует идею о возможности прихода к Богу через наблюдение за миром и рефлексию.
Карл Барт развивает концепцию трансцендентного Бога, утверждая что поскольку напрямую осознать
Бога невозможно, претензии либеральной теологии на адекватность толкования Священного Писания
не могут быть состоятельными(что, в принципе, согласуется с идеями Шлейерхмахера, поскольку
и Священное Писание, и попытки его толкования есть лишь {}<<проекции>>{} целого на современную
обстановку). Историческая критика Библии с точки зрения Барта же есть попытка поставить её
под свой контроль и фактически выстроить теологию на основе антропологии. По этой же причине
и догматика будет всегда оставаться относительной и неполной, в чём-то заблуждающейся.

Несмотря на то, что казалось бы в отношении толкования Священного Писания идеи Барта и Кьеркегора
казалось бы не противоречат идеям Шлейермахера, между воззрениями либеральных и диалектических
теологов есть одно существенное различие. Как было указано в предыдущей части, Шлейермахер
указывал религиозное чувство как нечто врождённое, присущее каждому человеку. У Кьеркегора и
Барта же Бог обязательно трансцендентален, он может быть воспринят только через противоречие.

Кьеркегор и Барт также чётко различают истинное Христианство и его институциональные формы,
Церковь, веру и религию. Поскольку вера это парадоксальное откровение Бога-инкогнито, её
выражением не может быть ни религия, ни Церковь, ни культура. У Шлейермахера также было
похожее разделение между Церковью и естественной верой, но у него это различие заключалось
в том, что Церковь должна быть посредником между религией и массами, и за счёт этого Церковь
будет неизбежно искажать религию.
\section{Заключение}
\label{sec:orgd35537a}
В заключение хотелось бы рассмотреть комментарий А. Уминского о {}<<Евангелии страданий>>{} Кьеркегора\textsuperscript{4},
чтобы показать, что идеи Кьеркегора актуальны и сейчас. В этом комментарии автор показывает,
что в результате долгого {}<<благополучия>>{} христианства произошло искажение самой
сути понятия {}<<христианин>>{}, что привело к дискредитации христианства в XX веке с одной стороны,
и появлению идеи {}<<богословия процветания>>{} -- с другой. Именно поэтому, несмотря на то, что
казалось бы Церковь сейчас много делает, её слова могут быть не услышанными. Благополучная Церковь
и благополучная церковная жизнь уже не воспринимаются как следование за Христом. В самом деле, ведь
следование за Христом подразумевает не только единоразовый отказ от всего и от себя({}<<взваливание на
себя креста>>{}), но каждодневную готовность оставить всё и броситься в неизвестность({}<<нести свой крест>>{}).
Более того, настоящее следование за Христом предполагает следование в одиночку, без учителя. Таким
образом, по мнению автора, Кьеркегор утверждает, что благочестивый благополучный человек, исправно
ходящий в церковь и придерживающийся традиций, тем не менее, не может считаться христианином, если он
не готов в любой момент отказаться от себя и своего благополучия. Вопрос {}<<благополучности>>{} христианства
сейчас актуален как никогда.
\section{Источники}
\label{sec:orgfd4a931}
\([1]\): Barth К. Die protecstantische Theologie im 19. Jahrhundert. ihre Vorgeschichle und ihre Geschichtc. Zürich, 1952. S. 379. \\
\([2]\): Friedrich Schleiermacher: Über die Religion. Hamburg 1958, S. 1.\\
\([3]\): Fear and Trembling; Copyright 1843 Søren Kierkegaard – Kierkegaard's Writings; 6 – copyright 1983 – Howard V. Hong\\
\([4]\): \url{https://www.pravmir.ru/o-knige-s-kerkegora/}
\end{document}
