% Created 2019-09-26 Thu 16:25
% Intended LaTeX compiler: pdflatex
\documentclass[11pt]{article}
\usepackage[utf8]{inputenc}
\usepackage[T1]{fontenc}
\usepackage{graphicx}
\usepackage{grffile}
\usepackage{longtable}
\usepackage{wrapfig}
\usepackage{rotating}
\usepackage[normalem]{ulem}
\usepackage{amsmath}
\usepackage{textcomp}
\usepackage{amssymb}
\usepackage{capt-of}
\usepackage{hyperref}
\usepackage{amsmath}
\usepackage{mathtools}
\usepackage[english, russian]{babel}
\author{Sergey Makarov}
\date{\today}
\title{}
\hypersetup{
 pdfauthor={Sergey Makarov},
 pdftitle={},
 pdfkeywords={},
 pdfsubject={},
 pdfcreator={Emacs 26.3 (Org mode 9.1.9)}, 
 pdflang={English}}
\begin{document}

\section{Задача 2.18}
\label{sec:orgdc2a4a0}
Найти интегралы:
\begin{enumerate}
\item \(\int_{-\infty}^\infty\frac{x\cos(x)}{x^2 - 2x + 10}\)
\end{enumerate}
\begin{multline*}
\int_{-\infty}^\infty\frac{x\cos(x)dx}{x^2 - 2x + 10} = \mathrm{Re}\int_{-\infty}^\infty\frac{xe^{ix}dx}{x^2-2x+10} = \\
= \mathrm{Re}\left[2\pi i \mathrm{Res}_{z=-1+3i}\left(\frac{z}{z^2-2z+10}\right)\right] = \\
= \mathrm{Re}\left[2\pi i(\frac{1}{2} - \frac{i}{6})\right] = \mathrm{Re}\left[\frac{\pi}{3} + \pi i\right] = \frac{\pi}{3}
\end{multline*}
$$\mathrm{Res}_{z=-1+3i}f(z) = \lim_{z \to -1 + 3i}f(z)(z - (-1 + 3i)) = \frac{-1 + 3i}{-1 + 3i - (-1 - 3i)} = \frac{-1 + 3i}{6i} = \frac{1}{2} - \frac{i}{6}$$
\begin{enumerate}
\setcounter{enumi}{1}
\item \(\int_{-\infty}^\infty\frac{x\sin(x)}{x^4 + 1}\)
\end{enumerate}
\begin{multline*}
\int_{-\infty}^\infty\frac{x\sin(x)}{x^4 + 1} = \mathrm{Im}\int_{-\infty}^\infty\frac{xe^{ix}dx}{x^4+1} = \\
= \mathrm{Im}\left[2\pi i\left(\mathrm{Res}_{z=e^{i\frac{\pi}{4}}}\frac{z}{z^4+1} + \mathrm{Res}_{z=e^{i\frac{3\pi}{4}}}\frac{z}{z^4+1}\right)\right] = \\
 = \mathrm{Im}\left[2\pi i\frac{1}{4}\left(-\frac{1}{\sqrt 2} - \frac{i}{\sqrt 2} + \frac{1}{\sqrt 2} - \frac{i}{\sqrt 2}\right)\right] = \mathrm{Im}\left[\frac{1}{\sqrt 2}\right] = 0
\end{multline*}
$$\mathrm{Res}_{z=e^{i\frac{\pi}{4}}} = \lim_{z \to e^{i\frac{\pi}{4}}}\frac{z}{z^4+1} = \lim_{z \to e^{i\frac{\pi}{4}}}\frac{1}{4z^3} = \frac{1}{4}e^{-i\frac{3\pi}{4}}$$
$$\mathrm{Res}_{z=e^{i\frac{3\pi}{4}}} = \lim_{z \to e^{i\frac{3\pi}{4}}}\frac{z}{z^4+1} = \lim_{z \to e^{i\frac{3\pi}{4}}}\frac{1}{4z^3} = \frac{1}{4}e^{-i\frac{9\pi}{4}}$$
\begin{enumerate}
\setcounter{enumi}{2}
\item \(\int_0^\infty\frac{\cos(x)dx}{(x^2+a^2)^2}, a \ne 0\)
\end{enumerate}
\begin{multline*}
\int_0^\infty\frac{\cos(x)dx}{(x^2+a^2)^2} = \frac{1}{2}\int_{-\infty}^\infty\frac{\cos(x)dx}{(x^2+a^2)^2} = \\
 = \frac{1}{2}\mathrm{Re}\int_{-\infty}^\infty\frac{e^{ix}dx}{(x^2+a^2)^2} = \frac{1}{2}\mathrm{Re}\left[2\pi i\mathrm{Res}_{z=ai}\frac{1}{(x^2+a^2)^2}\right] = \\
 = \frac{1}{2}\frac{\pi}{2a^2} = \frac{\pi}{4a^2}
\end{multline*}
$$\mathrm{Res}_{z=ai} = \lim_{z \to ai}\frac{d}{dx}\left[\frac{1}{(x + ai)^2}\right] = \lim_{z \to ai}-\frac{2x}{(x + ai)^3} = -\frac{2ai}{(2ai)^3} = \frac{1}{4a^2}$$
\begin{enumerate}
\setcounter{enumi}{3}
\item \(\int_0^\infty\frac{\sin^2(x)dx}{x^2}\)
\end{enumerate}
\begin{multline*}
\int_0^\infty\frac{\sin^2(x)dx}{x^2} = -\int_0^\infty\sin^2(x)d\left(\frac{1}{x}\right) = \\
= -\frac{sin^2(x)}{x}\bigg|_0^\infty + \int_0^\infty\frac{2\sin(x)\cos(x)dx}{x} = \int_0^\infty\frac{\sin(2x)d(2x)}{2x} = \frac{\pi}{2}
\end{multline*}
\section{Задача 2.16}
\label{sec:org0c48112}
Вычислить интегралы:
\begin{enumerate}
\setcounter{enumi}{2}
\item \(\int_0^\infty\frac{dx}{(x^2 + 1)^n}, n \in \mathbb{N}\)
\end{enumerate}
\begin{multline*}
\int_0^\infty\frac{dx}{(x^2+1)^n} = \frac{1}{2}\int_{-\infty}^{\infty}\frac{dx}{(x^2+1)^n}
 = 2\pi i\frac{1}{2}\mathrm{Res}_{z=i}\left[\frac{1}{(z^2+1)^n}\right] = 2\pi i(-1)^{n-1}\frac{(2n)!}{2^{2n+1}n!}
\end{multline*}
$$\mathrm{Res}_{z=i}\left[\frac{1}{(z^2+1)^n}\right] = \lim_{z \to i}\frac{d^n}{dz^n}\left[\frac{1}{(z + i)^n}\right]
 = \frac{(-1)^n}{-2^{2n}}\frac{(2n)!}{n!} = (-1)^{n-1}\frac{(2n)!}{2^{2n}{n!}}$$
\begin{enumerate}
\setcounter{enumi}{3}
\item \(\int_0^\infty\frac{x^2dx}{(x^2+a^2)^2}, a > 0\)
\end{enumerate}
$$\int_0^\infty\frac{x^2dx}{(x^2+a^2)^2} = \frac{1}{2}\int_{-\infty}^\infty\frac{x^2dx}{(x^2+a^2)^2}
 = \frac{1}{2}2\pi i\mathrm{Res}_{z=ai}\left[\frac{z^2}{(z^2+a^2)^2}\right] = \frac{\pi}{4a}$$
\begin{multline*}
\mathrm{Res}_{z=ai}\left[\frac{z^2}{(z^2+a^2)^2}\right] = \lim_{z \to ai}\frac{d}{dz}\left[\frac{z^2}{(z+ai)^2}\right] = \\
 = \lim_{z\to ai}\left[\frac{2z(z+ai)^2 - 2(z+ai)z^2}{(z+ai)^4}\right] = \frac{-4a^2 + 2a^2}{-8a^3i} = -\frac{i}{4a}
\end{multline*}
\section{Задача 2937}
\label{sec:orgcd3a7b2}
Каков будет ряд Фурье для тригонометрического многочлена:
$$P_n(x) = \sum_{i = 0}^n(\alpha_i\cos ix + \beta_i\sin ix)?$$
В общем случае ряд Фурье для \$(2l)\$-периодической функции на \([-l; l]\) имеет вид:
$$f(x) = \frac{a_0}{2} + \sum_{k = 1}^{\infty}(a_n\cos\frac{\pi k x}{l} + b_n\sin\frac{\pi k x}{l})$$
Или, при \(l = \pi\), как в нашем случае:
$$f(x) = \frac{a_0}{2} + \sum_{k = 1}^{\infty}(a_n\cos kx + b_n\sin kx)$$
Откуда видим, что \(a_k = \alpha_k, k = \overline{0, n}\) и \(a_k = 0, k > n\). Аналогично, \(b_k = \beta_k,
k = \overline{1, n}\) и \(b_k = 0, k > n\).

Иными словами, ряд Фурье для тригонометрического многочлена имеет вид:
$$P_n(x) = \frac{2\alpha_0}{2} + \sum_{i=1}^n(\alpha_i\cos ix + \beta_i\sin ix)$$
\section{Задача 2942}
\label{sec:org79d56aa}
Разложить в ряд Фурье функцию \(f(x) = |x|\) в интервале \((-\pi; \pi)\).

Ряд Фурье на \((-\pi, \pi)\) имеет вид:
$$f(x) = \frac{a_0}{2} + \sum_{n = 1}^{\infty}(a_n\cos nx + b_n\sin nx)$$
Где \(a_n = \frac{1}\pi\int_{-\pi}^\pi f(x)\cos nxdx, n \in \mathbb{N}_0\),
\(b_n = \frac{1}\pi\int_{-\pi}^\pi f(x)\sin nxdx, n \in \mathbb{N}\).
Функция чётная, поэтому \(b_n = 0\, \forall n\). Найдём \(a_n\):

При \(n = 0\):
$$a_0 = \frac{1}\pi\int_{-\pi}^\pi |x|dx = \frac{2}\pi\int_0^\pi xdx = \pi$$
При остальных \(n\):
\begin{multline*}
a_n = \frac{1}\pi\int_{-\pi}^\pi |x|\cos nx dx = \frac{2}\pi\int_0^\pi x\cos nx dx = \\
 = \frac{2}{\pi n}\int_0^\pi xd(\sin(nx)) = \frac{2}{\pi n}\left((x\sin nx)\bigg|_0^\pi - \int_0^\pi\sin nxdx\right) = \\
 = \frac{2}{\pi n^2}\cos nx\bigg|_0^\pi = \frac{4}{\pi n^2}
\end{multline*}
Таким образом, ряд Фурье для \(f(x)\) имеет вид:
$$f(x) = \frac{\pi}{2} + \sum_{n = 1}^\infty\frac{4}{\pi n^2}\cos nx$$
\section{Задача 2943}
\label{sec:orgc583f24}
Разложить в ряд Фурье на \((-\pi, \pi)\) функцию:
\begin{equation*}
f(x) = \begin{cases}
ax, -\pi < x < 0;\\
bx, 0 < x < \pi.
\end{cases}
\end{equation*}
Где \(a\) и \(b\) - постоянные.

Ряд Фурье имеет вид:
$$f(x) = \frac{\alpha_0}2 + \sum_{n = 0}^\infty(\alpha_n\cos nx + \beta_n\sin nx)$$
Где \(\alpha_n = \frac{2}\pi\int_{-\pi}^\pi f(x)\cos nxdx, n \in \mathbb{N}_0,
\beta_n = \frac{2}\pi\int_{-\pi}^\pi f(x)\sin nxdx, n \in \mathbb{N}\).

Найдём коэффициенты:
$$a_0 = \frac{1}\pi\int_{-\pi}^\pi f(x)dx = \frac{1}\pi\int_{-\pi}^0 axdx + \frac{1}\pi\int_0^\pi bxdx = (b - a)\frac{\pi}2$$
\begin{multline*}
a_n = \frac{1}\pi\int_{-\pi}^\pi f(x)\cos nx dx = \frac{1}\pi\int_{-\pi}^0 ax\cos nxdx + \frac{1}\pi\int_0^\pi bx\cos nxdx = \\
 = \frac{b - a}\pi\int_0^\pi x\cos nx dx = \frac{2(b - a)}{\pi n^2}
\end{multline*}
\begin{multline*}
b_n = \frac{1}\pi\int_{-\pi}^\pi f(x)\sin nx dx = \frac{1}\pi\int_{-\pi}^0 ax\sin nx + \frac{1}\pi\int_0^pi bx\sin nx dx = \\
 = \frac{b + a}\pi\int_0^\pi x\sin nx dx = -\frac{b + a}{\pi n}\int_0^\pi xd(\cos nx) = \\
 = -\frac{b + a}{\pi n}\left((x\cos nx)\bigg|_0^\pi - \int_0^\pi\cos nxdx\right) = \\
 = -\frac{b+a}{\pi n}\left((-1)^n\pi - \frac{1}n\sin nx\bigg|_0^\pi\right) = (-1)^{n+1}\frac{b+a}{\pi n}
\end{multline*}
Собирая всё вместе, получаем:
$$f(x) = \frac{\pi}2(b - a) + \sum_{n = 1}^\infty\left(\frac{2(b - a)}{\pi n^2}\cos nx + (-1)^{n + 1}\frac{b+a}{\pi n}\sin nx\right)$$
\section{Задача 2958}
\label{sec:orgf9ed520}
Разложить в ряд Фурье периодическую функцию \(f(x) = |\cos x|\).

Функция \(f(x)\) является \(\pi\)-периодической, поэтому достаточно её разложить на \(\left(-\frac{\pi}2; \frac{\pi}2\right)\).
Для \(2l\)-периодической функции ряд Фурье на \((-l; l)\) имеет вид:
$$f(x) = \frac{a_0}2 + \sum_{n = 0}^\infty\left(a_n\cos\frac{\pi nx}l + b_n\sin\frac{\pi nx}l\right)$$
Где
$$a_n = \frac{1}l\int_{-l}^lf(x)\cos\frac{\pi nx}ldx, n \in \mathbb{N}_0$$, $$b_n = \frac{1}l\int_{-l}^lf(x)\sin\frac{\pi nx}ldx, n \in \mathbb{N}$$
В нашем случае \(l = \frac{\pi}2\), поэтому формулы приводятся к виду:
$$f(x) = \frac{a_0}2 + \sum_{n = 0}^\infty\left(a_n\cos 2nx + b_n\sin 2nx\right)$$
$$a_n = \frac{2}\pi\int_{-\frac{\pi}2}^{\frac{\pi}2}f(x)\cos 2nxdx, n \in \mathbb{N}_0$$
$$b_n = \frac{2}\pi\int_{-\frac{\pi}2}^{\frac{\pi}2}f(x)\sin 2nxdx, n \in \mathbb{N}$$
Найдём коэффициенты:
$$a_0 = \frac{2}\pi\int_{-\frac{\pi}2}^\frac{\pi}2f(x)dx = \frac{4}\pi\int_0^\frac{\pi}2\cos xdx = \frac{4}\pi$$
\begin{multline*}
a_n = \frac{2}\pi\int_{-\frac{\pi}2}^\frac{\pi}2f(x)\cos 2nxdx = \frac{4}\pi\int_0^\frac{\pi}2\cos x\cos 2nxdx = \\
 = \frac{2}\pi\left(\int_0^\frac{\pi}2\cos(2n-1)xdx - \int_0^\frac{\pi}2\cos(2n+1)xdx\right) = \\
 = \frac{2}{\pi(2n - 1)}\sin(2n-1)dx\bigg|_0^\frac{\pi}2 - \frac{2}{\pi(2n+1)}\sin(2n+1)dx\bigg|_0^\frac{\pi}2 = \\
 = (-1)^n\frac{2}{\pi(2n-1)} + (-1)^{n-1}\frac{2}{\pi(2n+1)} = (-1)^n\frac{4}{\pi(4n^2 - 1)}
\end{multline*}
\begin{multline*}
b_n = \frac{2}\pi\int_{-\frac{\pi}2}^\frac{\pi}2f(x)\sin 2nxdx = \frac{4}\pi\int_0^\frac{\pi}2\cos x\sin 2nxdx = \\
 = \frac{2}\pi\left(\int_0^\frac{\pi}2\sin(2n+1)xdx - \int_0^\frac{\pi}2\sin(2n-1)x\right) = \\
 = -\frac{2}{\pi(2n+1)}\cos(2n+1)x\bigg|_0^\frac{\pi}2 + \frac{2}{\pi(2n-1)}\cos(2n-1)x\bigg|_0^\frac{\pi}2 = \\
 = \frac{2}{\pi(2n+1)} - \frac{2}{\pi(2n - 1)} = -\frac{4}{\pi(4n^2-1)}
\end{multline*}
Итого, ряд Фурье имеет вид:
$$f(x) = \frac{2}\pi + \sum_{n = 0}^\infty\left((-1)^n\frac{4}{\pi(4n^2-1)}\cos 2nx - \frac{4}{\pi(4n^2-1)}\sin 2nx\right)$$
\section{Задача 1}
\label{sec:orgcbab40e}
Найти и классифицировать особые точки функции:
$$g(z) = \frac{z^3e^{1/z}}{(1 - e^z)^2}$$
Предельной точкой может быть только точка \(z = 0\). Проверим эту точку:
$$g(z) \sim_{z \to 0} \frac{z^3e^{1/z}}{z^2} = ze^{1/z}$$
Последняя функция имеет существенную особую точку при \(z = 0\).(\(z_{n1} = \frac{1}n, z_{n2} = \frac{i}n\)).
\section{Задача 13.41(2)}
\label{sec:orga8770dd}
Найти ДЛО, отображающее точки \(z = -1, i, 1 + i\) соответственно в точки \(w = i, \infty, 1\).

Поскольку \(f(i) = \infty\), то \(f(z)\) имеет вид:
\(f(z) = \frac{az + b}{z - i}\). Запишем систему:
\begin{multline*}
\begin{dcases}
\frac{b - ai}{-1 - i} = i, \\
\frac{(a + b) + ai}i = 1
\end{dcases}
\Rightarrow
\begin{cases}
-ai + b = 1 - i, \\
a + (1 + i)b = 1,
\end{cases}
\Rightarrow
\begin{cases}
a = i, \\
b = -i,
\end{cases}
\end{multline*}
Откуда \(f(z)\) имеет вид:
$$f(z) = \frac{iz - i}{z - i}$$
\section{Задача 13.46(2)}
\label{sec:org39fbbc6}
Отобразить конформно верхнюю полуплоскость \(\{z : \mathrm{Im} z > 0\}\) на единичный круг \(\{w : |w| \leq 1\}\)
так, чтобы:
$$w(2i) = 0, \,\mathrm{arg}w'(2i) = 0$$
Поскольку точки, симметричные относительно границы, перейдут в точки, симметричные относительно границы, то \(f(-2i) = \infty\),
что даёт общий вид отображения:
\begin{multline*}
f(z) = \alpha\frac{z - 2i}{z + 2i} \Rightarrow f'(z) = \alpha\frac{z + 2i - z + 2i}{(z + 2i)^2} = \\
= \frac{4\alpha i}{(z + 2i)^2} \Rightarrow f(2i) = \frac{\alpha}{4i}
\end{multline*}
$$\mathrm{arg}\frac{\alpha}{4i} = 0 \Rightarrow \alpha \in i\mathbb{R}_+$$
Чтобы найти значение \(\alpha\), проверим, что значение на границе \(z = 0\) переходит на границу:
$$f(z) = \alpha\frac{-i}{i} = -\alpha \Rightarrow \alpha = i$$
Таким образом, \(f(z)\) имеет вид:
$$f(z) = \frac{iz + 2}{z + 2i}$$
\section{Задача 13.50(2)}
\label{sec:orgf1425a5}
Отобразить конформно внутренность единичного круга \(\{z: |z| < 1\}\) на внутренность единичного круга \(\{w: |w| < 1\}\)
так, чтобы:
$$w\left(\frac{1}2i\right) = 0, \,\mathrm{arg} w'\left(\frac{1}2i\right) = \frac{\pi}2$$

Симметричные относительно границы точки при ДЛО переходят в симметричные относительно границы, поэтому \(w(2i) = \infty\),
что даёт общий вид отображения:
\begin{multline*}
w(z) = \alpha\frac{z - i/2}{z - 2i} \Rightarrow w'(z) = \\
= \alpha\frac{z - 2i - z + i/2}{(z - 2i)^2} = -\alpha\frac{5i}{2(z - 2i)^2} \Rightarrow w'\left(\frac{i}2\right) = \frac{2\alpha}{5i}
\end{multline*}
$$\mathrm{arg}\frac{2\alpha}{5i} = \frac{\pi}2 \Rightarrow \alpha \in \mathrm{R}_+$$
Найдём \(\alpha\) из условия, что точка с границы \(z = i\) переходит на границу:
$$w(i) = \alpha\frac{-i/2}{-i} = \frac{\alpha}2 \Rightarrow \alpha = 2$$
Итого, \(w(z)\) имеет вид:
$$w(z) = \frac{2z - 1}{z - 2i}$$
\section{Задача 13.39(2)}
\label{sec:org33010fb}
Найти общий вид ДЛО, переводящего верхнюю полуплоскость \(\{z: \mathrm{Im}\,z > 0\}\) на правую полуплоскость
\(\{w: \mathrm{Re}\,w > 0\}\).
Пусть точка \(z = a\) переходит в точку \(w = 0\). Заметим, что \(a \in \mathbb{R}\), так как только точки границы
переходят в точки границы. Тогда \(f(z)\) имеет вид:
$$f(z) = \frac{z - a}{cz + d}$$
Точки \(z = a + i\) и \(z = a - i\) симметричны относительно границы, следовательно, их образы будут также симметричны
относительно границы:
\begin{multline*}
\begin{dcases}
\frac{i}{(ca + d) + ci} = \alpha + i\beta, \\
\frac{-i}{(ca + d) - ci} = -\alpha + i\beta
\end{dcases}
\Rightarrow
\begin{dcases}
\alpha = \frac{i(ca +d)}{(ca + d)^2 + c^2}, \\
\beta = \frac{-ci}{(ca + d)^2 + c^2}
\end{dcases}
\end{multline*}
Поскольку \(\alpha, \beta \in \mathbb{R}\), \(c, d \in i\mathbb{R}\), т. е. общий вид \(f(z)\) таков:
$$f(z) = i\frac{z - a}{cz + d}, a, c, d \in \mathbb{R}$$
\section{Задача 13.69}
\label{sec:orgd3561b5}
Доказать, что регулярные ветви функции \(w = \sqrt z\) конформно отображают плоскость \(\mathbb{C}\) с разрезом по
неотрицательной части действительной оси соответственно на нижнюю и верхнюю полуплоскости.

Пусть \(z = |z|(\cos\mathrm{arg}z + i\sin\mathrm{arg}z)\). Тогда
$$w(z) = \sqrt z = \sqrt{|z|}\left(\cos\left(\mathrm{arg} z + \frac{\pi k}2\right) + i\sin\left(\mathrm{arg} z + \frac{\pi k}2\right)\right), k = 0, 1$$
Для первой ветви при \(z \in \mathbb{C}\quad |w(z)| \in [0; \infty), \mathrm{arg} w(z) \in [0; \pi]\). Для второй при
\(z \in \mathbb{C}\quad |w(z)| \in [0; \infty), \mathrm{arg} w(z) \in [\pi; 2\pi]\), что и требовалось.
\section{Задача 13.75(2)}
\label{sec:org4fa0c84}
Отобразить конформно на верхнюю полуплоскость \(\{w: \mathrm{Im}\,w > 0\}\) область \(\mathbb{C} \setminus [z_1, z_2]\).

\begin{enumerate}
\item Построим ДЛО, отображающее \(\mathbb{C} \setminus [z_1, z_2]\) на \(\mathbb{C} \setminus \mathbb{R}_+\):
\begin{enumerate}
\item Обозначим \(z_0 = \frac{z_1 + z_2}2\), тогда \(f(z) = \frac{1}2\frac{z - z_0}{z_0}\) конформно отобразит исходную область на область \(\mathbb{C} \setminus [0; 1]\).
\item Отображением \(g(z) = \frac{1}z - 1\) можно конформно отобразить область \(\mathbb{C} \setminus [0; 1]\) на область \(\mathbb{C} \setminus \mathbb{R}_+\).
\end{enumerate}
\item По предыдущей задаче, первая ветвь корня конформно отображает полученное множество на верхнюю полуплоскость.
\end{enumerate}

Получили отображение:
$$f(z) = \sqrt{\frac{1}{\frac{z - z_0}{2z_0}} - 1} = \sqrt{\frac{2z_0}{z - z_0} - 1} = \sqrt{\frac{3z_0 - z}{z - z_0}}$$
Где под корнем подразумевается аналитическое продолжение арифметического квадратного корня.
\section{Задача 13.74(1)}
\label{sec:orge1da32f}
Отобразить конформно на верхнюю полуплоскость круговую "луночку":
$$\{z: |z| < 1, |z - i| < 1\}$$

Для начала с помощью ДЛО "выпрямим" область(\(A \to 0, B \to \infty\)).
\(A = -\frac{\sqrt 3}2  + \frac{i}2, B = \frac{sqrt 3}2 + \frac{i}2\).
$$z_1 = \frac{z - z_A}{z - z_B}$$
$$C \to \frac{\frac{\sqrt 3}2 - \frac{i}2}{-\frac{\sqrt 3}2 - \frac{i}2} = \frac{(\frac{\sqrt 3}2 - \frac{i}2)(-\frac{\sqrt 3}2 + \frac{i}2)}
= -\left(\frac{\sqrt 3}2 - \frac{i}2\right)^2 = -\frac{1}2 + \frac{\sqrt 3}2i$$
$$D \to \frac{\frac{\sqrt 3}2 + \frac{i}2}{-\frac{\sqrt 3}2 + \frac{i}2} = -\frac{1}2 - \frac{\sqrt 3}2i$$
Таким образом, область перешла в угол. Угол можно перевести в верхнюю полуплоскость путём композиции поворота и возведения в степень.
$$z^2 = z_1e^{-\frac{2\pi}3i}$$
$$z^3 = z_2^{3/2} = z\sqrt z$$
Где в качестве \(\sqrt z\) берётся аналитическое продолжение арифметического корня.
\section{Задача 12.31}
\label{sec:orgb5b40d5}
Вычислить интеграл: \(\int_0^\infty\frac{\ln xdx}{x^2 + a^2} = I, a > 0\)

Посчитаем интеграл по контуру \(\gamma\), составленному из двух полуокружностей радиусами \(R\) и \(\varepsilon\):
$$\oint_\Gamma\frac{\ln z}{z^2 + a^2}dz = I_M$$
Где \(\ln z = \ln |z| + i\mathrm{arg} z\).
Рассмотрим \(lim_{R \to +\infty, \varepsilon \to 0}I_M\)
$$L  = \lim_{R \to \infty, \varepsilon \to 0}2\pi i\operatorname{res}_{z = ai}\frac{\ln z}{z^2 + a^2} = \frac{\pi}{a}\left(\ln a + i\frac{\pi}2\right)$$
С другой стороны:
$$I = I_1 + I_2 + I_3 + I_4$$
$$I_1 = \int_{\varepsilon}^R\frac{\ln x}{x^2 + a^2} \to I$$
$$I_2 = \int_{-R}^{-\varepsilon}\frac{\ln xdx}{x^2 + a^2} = \int_{-\varepsilon}^{-R}\frac{\ln(-x) + i\pi}{(-x)^2 + a^2}dx
= \int_\varepsilon^R\frac{\ln xdx}{x^2 + a^2} + i\pi\int_\varepsilon^R\frac{dx}{x^2 + a^2} \to I + i\pi\frac{1}a\frac{\pi}2$$
$$I_3 = \int_\pi^0\frac{ln(\varepsilon e^{i\varphi})}{\varepsilon^2e^{2i\varphi} + a^2}\varepsilon ie^{i\varphi}d\varphi
= \varepsilon i\int_\pi^0\frac{\ln \varepsilon + i\varphi}{\varepsilon^2e^2i\varphi + a^2}d\varphi \to 0$$
$$I_4 = \int_0^\pi\frac{Re^{i\varphi}}{R^2e^{2i\varphi} + a^2}Rie^{i\varphi}d\varphi \to 0$$
Таким образом, \(\frac{\pi}a\left(\ln a + i\frac{\pi}2\right) = I + I + i\frac{\pi^2}{2a} \Rightarrow I = \frac{\pi}{2a}\ln a\)
\section{Задача 13.79}
\label{sec:org315f839}
Найти образы при отображении \(w = e^z, z = x + iy\):
\begin{enumerate}
\item Прямоугольной сетки \(x = c, y = c, c \in \mathbb{R}\).
\item Прямых \(y = kx + b, k, b \in \mathbb{R}\).
\item Полосы \(\alpha < y < \beta, 0 \leq \alpha < \beta < 2\pi\).
\item Полосы между прямыми \(y = x, y = x + 2\pi\).

\item Рассмотрим образ прямой \(x = c, c \in \mathbb{R}\). Точки на этой прямой имеют вид \(z = c + yi, y \in \mathbb{R}\).
\end{enumerate}
Тогда \(w = e^z = e^c(\cos y + i\sin y)\), т. е. вертикальные прямые перейдут в окружности с центром в нуле и радиусом \(e^c\).

Точки прямой \(y = c, c \in \mathbb{R}\) имеют вид \(z = x + ci, x \in \mathbb{R}\). Тогда \(w = e^{ci}\cdot e^x\), т. е.
образами горизонтальных прямых будут лучи, исходящие от начала координат.
\begin{enumerate}
\item \(y = kx + b: e^z = e^{x + i(kx + b)} = e^{ib}e^xe^{ikx} = e^x(\cos(kx + b) + i\sin(kx + b))\)
\end{enumerate}

Таким образом, образом прямой будет спираль.
\begin{enumerate}
\item Образом будет угол между \(\alpha\) и \(\beta\).
\item Разобьём полосу вертикальными отрезками. Отрезки перейдут в полные окружности с центром в нуле без одной точки,
\end{enumerate}
лежащей на спирали, в которую переходят границы, т. е. образ - вся плоскость без спирали.
\section{Задача 13.82}
\label{sec:org596f847}
Функция Жуковского: \(w = \frac{1}2\left(z + \frac{1}{z}\right)\)
Найти образы:
\begin{enumerate}
\item Единичного круга и его внутренности.
\end{enumerate}
Пусть \(z = Re^{i\varphi}\). Тогда \(w = \frac{1}2\left(Re^{i\varphi} + \frac{1}Re^{-i\varphi}\right)
= \cos\varphi\frac{R + R^{-1}}2 + i\sin\varphi\frac{R - R^{-1}}2 \equiv x + iy\)
$$1 = \cos^2\varphi + \sin^2\varphi = \frac{x^2}{(\frac{1}2(R+R^{-1}))^2} + \frac{y^2}{(\frac{1}2(R - R^{-1}))^2}$$
$$a^2 - b^2 = 1$$
Получили семейство софокусных эллипсов, переходящее при \(R = 1\) в отрезок \([-1, 1]\).
Внутренняя(и внешняя) часть единичного круга переходит в \(\mathbb{C} \setminus [-1, 1]\)
\begin{enumerate}
\item Верхней полуплоскости.
\end{enumerate}
Верхние полуокружности перейдут в нижние полуэллипсы. Верхние полуокружности с радиусами \(\frac{1}{R}\) дадут верхние
полуэллипсы.
\section{Задача 13.84}
\label{sec:org3acb421}
Отобразить на верхнюю полуплоскость \({w: \mathrm{Im} z > 0}\) области:
\begin{enumerate}
\item \(\{z: |z| < 1 \ \left[\frac{1}2; 1\right]\}\)
\end{enumerate}
Применим к области функцию Жуковского, получим \(\mathbb{C}_{i+} \setminus \left[1; \frac{5}4\right]\).
Применим ДЛО \(w_1 = \frac{w + 1}{w - \frac{5}4}\). Получим \(\mathbb{C}_{i+} \setminus \mathbb{R}_+\).
Применение второй ветви корня даст \(\mathbb{C}_+\).

13.79, 13.80, 13.82, 13.84(2, 4), 13.87
\section{Задача 2961}
\label{sec:org8aeac98}
Разложить функцию \(f(x) = x^2\):
\begin{enumerate}
\item На \((-\pi; \pi)\)
\item На \((0; \pi)\)
\item На \((0; 2\pi)\)

\item Ряд Фурье на \((-\pi; \pi)\) имеет вид:
\end{enumerate}
$$\frac{a_0}2 + \sum_{n = 1}^\infty(a_n\cos nx + b_n\sin nx)$$
Где
$$a_n = \frac{1}\pi \int_{-\pi}^\pi f(x)\cos nx dx, b_n = \frac{1}\pi \int_{-\pi}^\pi f(x)\sin nx dx, n \in \mathbb{N}$$
Поскольку \(y = x^2\) - чётная функция, \(b_n = 0, \forall n \in \mathbb{N}\).
$$a_0 = 0$$
\begin{multline*}
a_0 = \frac{2}\pi \int_0^\pi x^2dx = \frac{2}3\pi^2\\
a_n = \frac{2}\pi \int_0^\pi x^2\cos nx dx = \frac{2}{\pi n}\int_0^\pi x^2d(\sin nx) = \frac{2}{\pi n}\left(x^2\sin nx\bigg|_0^\pi -
\int_0^\pi2x\sin nx dx\right) = \\
 = \frac{4}{\pi n^2}\int_0^\pi xd(\cos nx) = \frac{4}{\pi n^2}\left(x\cos nx\bigg|_0^\pi - \int_0^\pi\cos nxdx\right) = (-1)^{n}\frac{4}{n^2}
\end{multline*}
Соотвественно, ряд имеет вид:
$$f(x) = \frac{\pi^2}{3} + \sum_{n = 1}^\infty(-1)^n\frac{4}{n^2}\cos nx, x \in [-\pi; \pi]$$
\begin{enumerate}
\item Ряд имеет вид \(x^2 = \sum_{n = 1}^\infty b_n\sin nx\). Для получения этого разложения разложим функцию \(f(x) = x^2\operatorname{sgn} x\)
\end{enumerate}
$$a_n = 0 \forall n \in \mathbb{N}_0$$
\begin{multline*}
b_n = \frac{2}\pi\int_0^\pi x^2\sin nxdx = \frac{2}{\pi n} x^2d(\cos nx) = \frac{2}{\pi n}\left(x^2\cos nx\bigg|_0^\pi - \int_0^\pi 2x\cos nxdx\right) = \\
= (-1)^n\frac{2\pi}n - \frac{4}{\pi n^2}\int_0^\pi xd(\sin nx) = (-1)^n\frac{2\pi}n - \frac{4}{\pi n^2}\left(x\sin nx\bigg|_0^\pi - \int_0^\pi\sin nx dx\right) = \\
= (-1)^n\frac{2\pi}n + \frac{4}{\pi n^3}((-1)^n - 1)
\end{multline*}
\begin{enumerate}
\item Продолжим периодически функцию на \(\mathbb{R}\). Тогда
\end{enumerate}
$$a_n = \frac{1}{\pi}\int_{-\pi}^\pi f(x)\cos nxdx$$
$$b_n = \frac{1}{\pi}\int_{-\pi}^\pi f(x)\sin nxdx$$
$$a_0 = \frac{1}\pi\int_0^{2\pi}x^2dx = \frac{8}3\pi^2$$
$$a_n = \frac{1}\pi\int_0^{2\pi}x^2\cos nxdx = \frac{4}{n^2}$$
$$b_n = \frac{1}\pi\int_0^{2\pi}x^2\sin nxdx = \frac{4\pi}n$$
Соответственно, ряд будет иметь вид:
$$x^2 = \frac{4\pi^2}3 + \sum_{n = 1}^\infty\left(\frac{4}{n^2}\cos nx + \frac{4\pi}n\sin nx\right), x \in (0; 2\pi)$$
\section{Задача 2955}
\label{sec:org9d2e4f8}
Разложить в ряд Фурье периодическую функцию \(f(x) = x - [x]\).

Раскладываем по промежутку \(\left[-\frac{1}2; \frac{1}2\right]\):
$$\{x\} = \frac{a_0}2 + \sum_{n = 1}^\infty(a_n\cos 2\pi nx + b_n\sin 2\pi nx)$$
$$a_0 = 2\int_0^1xdx = 1$$
\begin{multline*}
a_n = 2\int_{-1/2}^{1/2}\{x\}\cos 2\pi nxdx = 2\int_0^1x\cos 2\pi nxdx = \frac{x\sin 2\pi nx}{\pi n}\bigg|_0^1 - \frac{1}{\pi n}\int_0^1\sin 2\pi nxdx = \\
= \frac{1}{\pi n}\frac{\cos 2\pi nx}{2\pi n}\bigg|_0^1 = 0
\end{multline*}
\begin{multline*}
b_n = 2\int_0^1x\sin 2\pi nx dx = -\frac{x\cos 2\pi nx}{\pi n}\bigg|_0^1 + \frac{1}{\pi n}\int_0^1\cos 2\pi nxdx = -\frac{1}{\pi n}
\end{multline*}
Откуда ряд имеет вид:
$${x} = \frac{1}2 - \frac{1}\pi\sum_{n = 1}^\infty\frac{\sin 2\pi nx}n$$
\section{Задача 2479}
\label{sec:orgad28a17}
Пусть \(f(x + \pi) \equiv -f(x)\). Какой вид имеет ряд Фурье для \(f(x)\)?
\begin{multline*}
a_n = \frac{1}\pi\int_{-\pi}^\pi f(x)\cos nx = \frac{1}\pi\int_{-\pi}^0f(x)\cos nx + \frac{1}\pi\int_0^\pi f(x)\cos nx = \\
= \frac{1}\pi\int_0^\pi f(y - \pi)\cos n(y - \pi)dy + \frac{1}\pi\int_0^\pi f(x)\cos nxdx = 0, при n \in 2\mathbb{Z}
\end{multline*}
\section{Задача 2}
\label{sec:org74889f6}
Вычислить интеграл:
$$\int_0^\infty\frac{e^{-\alpha x^2} - e^{-2\alpha x^2}}xdx = I(\alpha), \alpha > 0$$

Продифференцируем интеграл по параметру \(\alpha\):
\begin{multline*}
I'(\alpha) = \int_0^\infty\frac{-x^2e^{-\alpha x^2} + 2x^2e^{-2\alpha x^2}}xdx =
\int_0^\infty(-xe^{-\alpha x^2} + 2xe^{-2\alpha x^2})dx = \\
= \frac1{2\alpha}e^{-\alpha x^2}\bigg|_0^\infty - \frac1{2\alpha}{e^{-2\alpha x^2}\bigg|_0^\infty} = 0
\end{multline*}
В силу признака Вейерштрасса полученный интеграл сходится равномерно на любой полупрямой
\([\alpha_0; +\infty)\), поэтому дифференцирование правомерно.

Отсюда находим \(I(\alpha)\):
$$I(\alpha) = C$$
Найдём \(C\):
\begin{multline*}
C = I(1) = \int_0^\infty\frac{e^{-x^2} - e^{-2x^2}}xdx =\bigg|_{x = \sqrt t}\int_0^\infty\frac{e^{-t} - e^{-2t}}{\sqrt t}d(\sqrt t) = \\
= -\frac12\int_0^\infty\frac{e^{-t} - e^{-2t}}tdt = -\frac12\ln2 = -\frac{\ln2}2
\end{multline*}
Поскольку \(f(t) = e^{-t} \in C[0; \infty)\) и \(\forall A > 0 \exists \int_A^\infty\frac{f(t)}tdt\), то справедлива формула Фруллани:
$$\int_0^\infty\frac{f(at) - f(bt)}tdt = f(0)\ln\left(\frac ba\right)$$
Окончательно получаем:
$$\int_0^\infty\frac{e^{-\alpha x^2} - e^{-2\alpha x^2}}xdx = -\frac{\ln2}2, \forall \alpha > 0$$
\end{document}
