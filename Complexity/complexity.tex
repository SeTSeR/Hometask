% Created 2020-03-05 Thu 15:23
% Intended LaTeX compiler: pdflatex
\documentclass[11pt]{article}
\usepackage[utf8]{inputenc}
\usepackage[T1]{fontenc}
\usepackage{graphicx}
\usepackage{grffile}
\usepackage{longtable}
\usepackage{wrapfig}
\usepackage{rotating}
\usepackage[normalem]{ulem}
\usepackage{amsmath}
\usepackage{textcomp}
\usepackage{amssymb}
\usepackage{capt-of}
\usepackage{hyperref}
\usepackage{minted}
\usepackage{amsmath}
\usepackage{esint}
\usepackage[english, russian]{babel}
\usepackage{mathtools}
\usepackage{amsthm}
\usepackage{listings}
\usepackage[top=0.8in, bottom=0.75in, left=0.625in, right=0.625in]{geometry}
\def\zall{\setcounter{lem}{0}\setcounter{cnsqnc}{0}\setcounter{th}{0}\setcounter{Cmt}{0}\setcounter{equation}{0}}
\newcounter{lem}\setcounter{lem}{0}
\def\lm{\par\smallskip\refstepcounter{lem}\textbf{\arabic{lem}}}
\newtheorem*{Lemma}{Лемма \lm}
\newcounter{th}\setcounter{th}{0}
\def\th{\par\smallskip\refstepcounter{th}\textbf{\arabic{th}}}
\newtheorem*{Theorem}{Теорема \th}
\newcounter{cnsqnc}\setcounter{cnsqnc}{0}
\def\cnsqnc{\par\smallskip\refstepcounter{cnsqnc}\textbf{\arabic{cnsqnc}}}
\newtheorem*{Consequence}{Следствие \cnsqnc}
\newcounter{Cmt}\setcounter{Cmt}{0}
\def\cmt{\par\smallskip\refstepcounter{Cmt}\textbf{\arabic{Cmt}}}
\newtheorem*{Note}{Замечание \cmt}
\author{Sergey Makarov}
\date{\today}
\title{}
\hypersetup{
 pdfauthor={Sergey Makarov},
 pdftitle={},
 pdfkeywords={},
 pdfsubject={},
 pdfcreator={Emacs 26.3 (Org mode 9.3)}, 
 pdflang={English}}
\begin{document}


\section{Задача 1}
\label{sec:org777cb9f}
Доказать, что \(n \in \mathbb{Z} \Leftrightarrow n = \left\lfloor\frac{n}2\right\rfloor + \left\lceil\frac{n}2\right\rceil\).
\subsection{Решение}
\label{sec:org49c322f}
Если \(n\) целое, то \(\left\lfloor\frac{n}2\right\rfloor = \frac{n - 1}2\) и \(\left\lceil\frac{n}2\right\rceil = \frac{n + 1}2\), что и даёт достаточность.
Пусть теперь \(n = \left\lfloor\frac{n}2\right\rfloor + \left\lceil\frac{n}2\right\rceil\).
В этом случае представления для пола и потолка \(\frac{n}2\) принимают вид:
$$\left\lfloor\frac{n}2\right\rfloor = \left\lfloor\frac{n - 1}2\right\rfloor$$
$$\left\lceil\frac{n}2\right\rceil = \left\lfloor\frac{n + 1}2\right\rfloor$$
Из этого представления получаем, что если \(n\) нецелое, то \(\left\lfloor\frac{n}2\right\rfloor + \left\lceil\frac{n}2\right\rceil < \frac{n - 1}2 + \frac{n + 1}2 = n\),
что противоречит условию.
\section{Задача 3}
\label{sec:orgda391b2}
Пусть \(P(n)\) - число простых множителей, входящих в разложение \(n\). Показать, что \(P(n) = O(\log_2n)\)
\subsection{Решение}
\label{sec:orgf0af0e3}
$$P(ab) = P(a) + P(b) \leq C\log_2(a) + C\log_2(b) = C\log_2(ab)$$
$$\log_2pn = \log_2p + \log_2n$$
В качестве базы индукции возьмём простое число:
$$P(p) = 1 \leq \log_2(p)$$


Рассмотрим множество входов размера \(s\):
\begin{equation}
X_s = \{x | ||x|| = s\} \subset X
\end{equation}
На этом множестве определена вероятность \(0 \leq P_s(x) \leq 1\). На пространствах \(X_s\)
функции временных и пространственных затрат являются случайными величинами.
\textbf{Сложностью в среднем} будем называть математическое ожидание такой величины.

Задача: построения с помощью циркуля и линейки, затраты -- количество проведённых прямых и кривых.
Задача: построить \(\frac1n\) часть отрезка и найти сложность алгоритма. Это можно сделать со сложностью
\(O(\ln n)\).
\begin{equation}
sum = \frac{q^n - 1}{q - 1}b_0 \equiv q^{n - 1}
\end{equation}
\section{Задача 4}
\label{sec:orgbe6f2bd}
Изменим алгоритм Грэхема: по сразу построенному многоугольнику ходим по кругу, "спрямляя"
"вдавленные" вершины, если они есть. Показать, что алгоритм становится квадратичным.
\section{Задача 5}
\label{sec:org892a0c1}
Есть операция сложения. Нужно проверить, является ли число полным квадратом. Проверяя с
помощью суммы нечётных чисел, получим сложность \(O(\sqrt{N}))\).

Пусть добавилась операция \(\left\lfloor\frac{N}2\right\rfloor\). Какой сложности мы добьёмся?
\section{Задача 6}
\label{sec:org456375b}
В среднем сложность QuickSort \(O(\log N)\), в худшем -- \(O(n^2)\). Доказать.
Худший вариант - обратно отсортированный массив. \((k - 1)^2 + (n - k)^2, k \in \overline{1, n}\).
\section{Задача 7}
\label{sec:org78ddb10}
Есть алгоритм \(A\) над натуральным числом \(n\). Найти связь между сложностями \(T_A(n)\) и \(T_A^*(m)\),
где \(m\) - битовая длина \(n\). Пусть \(T_A(n) \geq f(n)\). Верно ли, что \(T_A^*(m) \geq f(2^{m - 1})\)?
\section{Задача 8}
\label{sec:org41087fa}
Есть алгоритм \(A\) над числом \(n \in \mathbb{N}\). Рассматриваются сложности \(T_A(n)\) и \(T_A^*(m)\).
Пусть \(T_A^*(m) \geq f(m)\). Верно ли, что \(T_A(n) \geq f(\lfloor\log_2n\rfloor)\)?
\section{Задача 9}
\label{sec:org61acb39}
Пусть \(l(n)\) - длина кратчайшей аддитивной цепочки для числа \(n\) - набора
\(1 = n_1 \leq n_2 \leq \ldots \leq n_k = n\)
такого, что \(n_s = n_k + n_l\) для любого \(s\) и некоторых \(k, l\).
Проверить, что \(l(ab) \leq l(a) + l(b) - 1\). Верно ли, что \(l(ab) = l(a) + l(b) - 1\)?
\section{Сложность в среднем}
\label{sec:org6092c97}
В задаче сортировок каждую конфигурацию массива длины \(n\) соответствует \(\Pi_n\) - перестановка массива \(1\ldots n\).
Введём \(\Pi_n\) - вероятностное пространство, в котором все перестановки равновероятны.
Перестановки могут рассматриваться как отображения множества \(\{1, 2, \ldots, n\}\) на себя.

\textbf{Неподвижной точкой} отображения \(f(x)\) будет значение аргумента \(x\), такая, что \(f(x) = x\).
Количество неподвижных точек может быть любым числом от 1 до \(n\). Сколько их в среднем?

Мастемер "50 занимательных вероятностных задач с решениями".

Рассмотрим событие \(B_n^v\) - событие, при котором \(v\) - fixpoint. Его вероятность равна
\(\frac{(n - 1)!}{n!} = \frac1n\). Рассмотрим случайную величину \(\xi_n\) - количество
неподвижных точек перестановки и
\begin{equation}
\zeta_n^v = \begin{cases}
1, \text{ если v - fixpoint}, \\
0, \text{ иначе.}
\end{cases}
\end{equation}
Тогда
\begin{equation}
P_n(\zeta_n^v = 1) = \frac1n \Rightarrow \mathbb{E}\zeta_n^v = \frac1n
\end{equation}
и, соответственно,
\begin{equation}
\mathbb{E}\xi_n = \mathbb{E}\left(\sum_{v = 1}^n\zeta_n^v\right) =
\sum_{v = 1}^n\mathbb{E}\zeta_n^v = \frac{n}n = 1
\end{equation}
Введём \(H_n^{u, v}\) - событие, при котором перед \(a_v\) стоит ровно \(u\) элементов меньше его.
Тогда \(P_n(H_n^{u, v}) = \frac1v\). Будем обозначать \(p(1, 2, \ldots, v)\) все перестановки, у
которых первые \(v\) элементов имеют тот же относительный порядок. Количество таких перестановок
будет
\begin{equation}
\begin{pmatrix}
n \\
v
\end{pmatrix}(n - v)!
\end{equation},
соответственно, количество перестановок, индуцирующих $H_n^{u, v}$:
\begin{equation}
\frac{n!}{v!}(v - 1)! = \frac{n!}v
\end{equation}
Пусть \(W = W_1 \cup \ldots \cup W_l\) -- полная группа событий. Полное математическое ожидание:
\begin{equation}
\mathbb{E}\xi = \sum_{k = 1}^l\mathbb{E}(\xi | W_k)\mathbb{P}(W_k)
\end{equation}
Для исследования сложности введём случайные величины \(\xi_n^i\) - затраты на \(i-м\) шаге сортировки,
применяемой к \(a = (a_1, \ldots, a_n) \in \Pi_n\). Тогда
\begin{equation}
\overline{T}_{I_1}(n) = \sum_{i = 1}^{n - 1}\mathbb{E}\xi_n^i
\end{equation}
По формуле полного матожидания имеем:
\begin{equation}
\mathbb{E}\xi_n^i = \frac1{i + 1}\sum_{k = 0}^i\mathbb{E}(\xi_n^i | H_n^{k, i + 1})
\end{equation}
\begin{equation}
\xi_n^i = \begin{cases}
i - k + 1, k > 0, \\
i, k = 0.
\end{cases}
\end{equation}
\begin{equation}
\mathbb{E}(\xi_n^i | H_n^{k, i + 1}) = \begin{cases}
i - k + 1, k > 0, \\
i, k = 0.
\end{cases}
\end{equation}
Подставляя в (10), находим:
\begin{equation}
\frac1{i + 1}\left(i + \sum_{k = 1}^i(i - k + 1)\right) = \frac{i}2 + 1 - \frac1{i + 1}
\end{equation}
Подставляя найденное в (9), получим:
\begin{equation}
\overline{T}_{I_1}(n) = \sum_{i = 1}^{n - 1}\left(\frac{i}2 + 1 - \frac1{i + 1}\right)
= \frac{(n + 4)(n - 1)}4 - \ln n + O(1) = \frac{n^2}4 + O(n)
\end{equation}
Заметим, что в отличие от сложности в худшем случае, сложность в среднем аддитивна.
\subsection{Задача 1}
\label{sec:org879f0e2}
Показать, что сложность по числу обменов сортировки выбором \(n - 2 + \frac1n\).
\subsection{Задача 2}
\label{sec:org3bb8508}
\begin{lstlisting}
 m := x_1;
 for i := 2 to n do
   if x_i < m then m := x_i;
 end;
\end{lstlisting}
Какой рост у количества присваиваний?
\subsection{Задача 3}
\label{sec:org0eb7b5a}
Посчитать сложность в среднем для задачи сортировки вагонов при условии, что все расположения
вагонов равновероятны.
\subsection{Сложность в среднем для быстрой сортировки}
\label{sec:org7334f49}
   \begin{equation}
\overline{T}_{QS}(n) = n - 1 + \frac1n\sum_{i = 1}^n\left(\overline{T}_{QS}(i - 1) +
\overline{T}_{QS}(n - i)\right)
   \end{equation}
Вообще говоря, нужно проверить, что все порядки в левой и правой частях равновероятны.
\begin{equation}
n\overline{T}_{QS}(n) = n(n - 1) + 2\sum_{k = 0}^{n - 1}\overline{T}_{QS}(k)
\end{equation}
\begin{equation}
n\overline{T}_{QS}(n) - (n - 1)\overline{T}_{QS}(n - 1) = 2(n - 1)
\end{equation}
Положим $t(n) = \frac{\overline{T_{QS}(n)}}{n + 1}, t(1) = t(0) = 0$. Тогда
\begin{equation}
t(n) - t(n - 1) = 2\frac{n - 1}{n(n + 1)}
\end{equation}
и
\begin{equation}
t(n) = t(n_0) + 2\sum_{k = n_0 + 1}^n\frac{k - 1}{k(k + 1)}
\end{equation}
Подставляя $n_0 = 1, t(n_0) = 0$, получаем:
\begin{multline}
t(n) = 2\sum_{k = 1}^n\frac{k - 1}{k(k + 1)} = 2\sum_{k = 1}^n\frac1{k + 1} - 2\sum_{k = 1}^n\frac1{k(k + 1)}
= 2\ln n + O(1) \Rightarrow \\
\Rightarrow \overline{T(n)} = 2(n + 1)\ln n + O(n) = 2n\ln n + O(n) = \Theta(n\log n)
\end{multline}

Число перемещений оценивается примерно таким же образом и так же равна \(\Theta(n\log n)\).
Соответственно, в силу аддитивности сложности общая сложность так же имеет вид \(\Theta(n\log n)\).

Пространственная сложность алгоритма с точки зрения алгебраической сложности \(O(1)\), но с
учётом рекурсии и стека получается логарифмическая сложность(вообще говоря, это верно только
при выборе меньшего массива после разбиения, другая стратегия может дать худшую сложность).
\subsection{Задача 4}
\label{sec:orgc83fe9c}
Доказать, что \(\overline{T}_{QS}(n) \leq 2n\ln n\).
\subsection{Задача 5}
\label{sec:org5ed10f3}
Доказать, что \(\overline{T}_{QS}(n) = 2(n + 1)H_n - 4n\).
\subsection{Задача 6}
\label{sec:orgc3fbf54}
Понять максимальную длину "навесов" из костей домино и сколько для этого понадобится костей?
\subsection{Рандомизированные алгоритмы}
\label{sec:org2de9343}
Рандомизированные алгоритмы содержат в себе элемент случайности - вызовы генератора случайных
чисел. Каждому входу сопоставляется вероятностное пространство сценариев выполнения алгоритма,
а в нём -- случайная величина затрат. Соответственно, размеру входа сопоставляются максимальные
усреднённые затраты на всех входа.

Будем называть массив \(x_1, \ldots, x_n\) массивом, \textbf{содержащим большинство}, если больше
половины его элементов имеют одно и то же значение. Пусть теперь массив содержит большинство.
Найти \(x_i\), входящий в большинство. Рассмотрим алгоритм, выбирающий случайный элемент, и
затем проверяющий его вхождение в большинство. Сценарием будет набор индексов
\((i_1, \ldots, i_m)\)(вообще говоря, бесконечный). Пусть \(p\) - вероятность получения элемента
из большинства, тогда количество сравнений на равенство будет:
\begin{equation}
a = p(n - 1) + (1 - p)(n - 1 + a) = n - 1 + (1 - p)a \Rightarrow a = \frac{n - 1}{p}
\end{equation}
Если массив заведомо имеет большинство, то $p > \frac12 \Rightarrow a < 2(n - 1)$.

Теперь рассмотрим быструю сортировку с выбором pivot-элемента случайным образом. Сценарии в
этом случае имеют вид двоичных деревьев с длинами частей массивов.

Определим три случайные величины:
\begin{equation}
P_n(s) = P_{n - 1}(s') + P_{n - 1}(s'')
\end{equation}
\begin{equation}
\chi_n(s), \chi_n'(s) = \chi_{i - 1}(s'), \chi_n''(s) = \chi_{n - 1}(s'')
\end{equation}
\begin{equation}
\chi_n(s) = n - 1 + \chi'_n(s) + \chi''_n(s)
\end{equation}
\begin{equation}
S_n = S^1_n \cup S^2_n \cup \ldots \cup S^n_n
\end{equation}
Заметим, что
\begin{equation}
P_n(S_n^1) = P_n(S_n^2) = \ldots = P_n(S_n^n) = \frac1n
\end{equation}
Тогда
\begin{multline}
\mathbb{E}\chi_n = \sum_{i = 1}^n\mathbb{E}(\chi_n | S_n^i)\frac1n =
n - 1 + \frac1n\sum_{i = 1}^n(\mathbb{E}(\chi_n' | S_n^i) + \mathbb{E}(\chi_n'' | S_n^i)) =
n - 1 + \frac1n\sum_{i = 1}^n(\mathbb{E}\chi_{i - 1} + \mathbb{E}\chi_{n - i}) = \\
= n - 1 + \frac2n\sum_{i = 1}^n\mathbb{E}\chi_{i - 1} =
n - 1 + \frac2n\sum_{i = 0}^{n - 1}\mathbb{E}\chi_i
\end{multline}
Откуда
\begin{equation}
\mathbb{E}\chi_n = 2n\ln n + O(n)
\end{equation}
Поскольку для всех входов длины $n$ затраты одинаковы, они и составляют сложность.
\subsection{Задача 7}
\label{sec:org7451674}
Найти сложность по числу обращений к генератору случайных чисел.
Асимптотика \(\frac23n - \frac13\)
\subsection{Задача 8}
\label{sec:orgebb2609}
m-я порядковая статистика
\subsection{Задача 9}
\label{sec:org00ff936}
Восстановление перестановки по инверсионному вектору.
\section{Оценивание числа шагов (итераций) алгоритма}
\label{sec:orgac4d056}
Рассмотрим алгоритм Евклида. Временные затраты этого алгоритма имеют оценку:
\begin{equation}
C_E(a_0, a_1) \leq a_1
\end{equation}
Пусть
\begin{equation}
k, l \in \mathbb{N}, k > l, r \text{ - остаток.}
\end{equation}
Тогда
\begin{equation}
\lambda(k) > \lambda(r)
\end{equation}
\begin{equation}
L(k, l) = \lambda(k) + \lambda(l) - 2: L(k, l) > L(l, r)
\end{equation}
Отсюда следует, что
\begin{equation}
C_E(a_0, a_1) \leq \lambda(a_0) + \lambda(a_1) - 2
\end{equation}
С другой стороны,
\begin{equation}
C_E(a_0, a_1) = C_E(a_1, a_2) + 1 \leq \lambda(a_1) + \lambda(a_2) - 1 \leq 2\lambda(a_1) - 1
\leq 2\log_2a_1 + 1
\end{equation}
Отсюда
\begin{equation}
T_E(a_1) = \max_{a_0 \geq a_1}C_E(a_0, a_1) \leq 2\log_2a_1 + 1
\end{equation}
Расширенный алгоритм Евклида имеет такую же сложность, как и обычный. Он позволяет найти
$s$ и $t$ такие, что $sa_0 + ta_1 = d = (a_0, a_01)$. Для них верны неравенства:
\begin{equation}
|s_1| \leq |s_2| \leq |s_3|, |t_1| \leq |t_2| < |t_3|
\end{equation}
и при $n > 2$:
\begin{equation}
|s_3| < |s_4| < \ldots < |s_{n + 1}|, |t_3| < |t_4| < \ldots < |t_{n = 1}|
\end{equation}
Отсюда следуют неравенства:
\begin{equation}
|s_n| \leq a_1, |t_n| < a_0
\end{equation}
\subsection{Задача 10}
\label{sec:orgddb6d54}
Алгоритм построния опорных лучей к данному многоугольнику.
\subsection{Сортировка фон Неймана}
\label{sec:orgb673729}
Используется вспомогательный массив дляны \(n\). Уже упорядоченные части сливаются во
вспомогательном массиве.
\end{document}
