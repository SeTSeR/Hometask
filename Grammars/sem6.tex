% Created 2020-04-26 Sun 13:36
% Intended LaTeX compiler: pdflatex
\documentclass[11pt]{article}
\usepackage[utf8]{inputenc}
\usepackage[T1]{fontenc}
\usepackage{graphicx}
\usepackage{grffile}
\usepackage{longtable}
\usepackage{wrapfig}
\usepackage{rotating}
\usepackage[normalem]{ulem}
\usepackage{amsmath}
\usepackage{textcomp}
\usepackage{amssymb}
\usepackage{capt-of}
\usepackage{hyperref}
\usepackage{minted}
\usepackage{amsmath}
\usepackage{esint}
\usepackage[english, russian]{babel}
\usepackage{mathtools}
\usepackage{amsthm}
\usepackage[top=0.8in, bottom=0.75in, left=0.625in, right=0.625in]{geometry}
\author{Sergey Makarov}
\date{\today}
\title{}
\hypersetup{
 pdfauthor={Sergey Makarov},
 pdftitle={},
 pdfkeywords={},
 pdfsubject={},
 pdfcreator={Emacs 26.3 (Org mode 9.3)}, 
 pdflang={Russian}}
\begin{document}


\section{Задача 1}
\label{sec:org9b70f6d}
Вычислить функции FIRST и FOLLOW для всех нетерминалов грамматики `Even`:
\begin{equation}
\begin{cases}
S \rightarrow BaBaS | B, \\
B \rightarrow bB | \varepsilon.
\end{cases}
\end{equation}
\subsection{Решение}
\label{sec:org19baf43}
Найдём FIRST(B). Первое правило даёт нам, что $b \in$ FIRST(B), второе -- что $\varepsilon \in$ FIRST(B),
таким образом, $FIRST(B) = \{\varepsilon, b\}$. Второе правило для S даёт нам, что $\varepsilon, b \in$ FIRST(S),
первое -- то же самое и вдобавок, $a \in$ FIRST(S), поскольку из первого нетерминала этого
правила $B$ выводима пустая цепочка.

Вычислим FOLLOW(S). Поскольку в грамматике нет правил, в которых за S следует какой-либо символ грамматики, но
есть правило $S \rightarrow BaBaS$, в котором S -- самый правый символ, то FOLLOW(S) = $\{\$\}$. Теперь вычислим
FOLLOW(B). В грамматике есть правило $S \rightarrow B$, из которого следует, что $\$ \in$ FOLLOW(B). Из правила
$S \rightarrow BaBaS$ следует также, что $a \in$ FOLLOW(B). Таким образом:
\begin{equation}
\begin{cases}
FIRST(S) = \{\varepsilon, a, b\}, \\
FIRST(B) = \{\varepsilon, b\}, \\
FOLLOW(S) = \{\$\}, \\
FOLLOW(B) = \{a, \$\}.
\end{cases}
\end{equation}
\section{Задача 2}
\label{sec:orga83ce67}
Привести грамматику `Even` к LL(1)-грамматике и построить для этой грамматики LL(1)-анализатор.
\subsection{Решение}
\label{sec:org9a9ff93}
Грамматика `Even` не является LL(1)-грамматикой, поскольку находясь в начальном состоянии и видя,
что входной символ \(b\), автомат не сможет выбрать из двух правил для \(S\) нужное. Приведём эту
грамматику к LL(1)-грамматике, добавив новое правило:
\begin{equation}
\begin{cases}
S \rightarrow BC, \\
B \rightarrow bB | \varepsilon, \\
C \rightarrow aBaS | \varepsilon.
\end{cases}
\end{equation}
Пересчитаем множества FIRST и FOLLOW:
\begin{equation*}
\begin{cases}
FIRST(S) = \{\varepsilon, a, b\}, \\
FIRST(B) = \{\varepsilon, b\}, \\
FIRST(C) = \{\varepsilon, a\}, \\
FOLLOW(S) = \{\$\}, \\
FOLLOW(B) = \{a, \$\}, \\
FOLLOW(C) = \{\$\}.
\end{cases}
\end{equation*}
Проверим, что получившаяся грамматика действительно будет LL(1)-грамматикой. Рассмотрим пару
продукций \(B \rightarrow bB | \varepsilon\). Для этой пары продукций:
\begin{enumerate}
\item Не существует терминала, с которого бы начинались строки, порождаемые правыми частями обеих продукций.
\item Только одна из этих продукций порождает пустую строку.
\item Строка \(bB\) не порождает ни одной строки, начинающейся с символа из FOLLOW(B).
\end{enumerate}
Аналогичные три условия выполняются и для продукций \(C \rightarrow aBaS | \varepsilon\).

Построим таблицу разбора для полученной грамматики:
\begin{center}
\begin{tabular}{llll}
\hline
Нетерминал & Входной символ &  & \\
 & \(a\) & \(b\) & \$\\
\hline
\(S\) & \(S \rightarrow BC\) & \(S \rightarrow BC\) & \(S \rightarrow BC\)\\
\(B\) & \(B \rightarrow \varepsilon\) & \(B \rightarrow bB\) & \(B \rightarrow \varepsilon\)\\
\(C\) & \(C \rightarrow aBaS\) &  & \(C \rightarrow \varepsilon\)\\
\hline
\end{tabular}
\end{center}
Здесь и далее пустые ячейки таблицы означают состояние ошибки.
\section{Задача 3}
\label{sec:orge2821fa}
Привести грамматику
\begin{equation}
\begin{cases}
S \rightarrow aAB | ac, \\
A \rightarrow Aa | b, \\
B \rightarrow Bb | c | \varepsilon
\end{cases}
\end{equation}
к LL(1)-грамматике, построить LL(1)-анализатор и продемонстрировать его работу на слове
\(\omega = abaa\).
\subsection{Решение}
\label{sec:orgfc8ec73}
Входная грамматика не является LL(1)-грамматикой, поскольку:
\begin{enumerate}
\item Содержит леворекурсивные правила для нетерминалов \(A\) и \(B\).
\item Содержит пару продукций \(S \rightarrow aAB | ac\), правые части которых начинаются с одного и того же символа.
\end{enumerate}
Преобразуем эту грамматику к LL(1)-грамматике:
\begin{equation}
\begin{cases}
S \rightarrow aC, \\
C \rightarrow AB | c, \\
A \rightarrow bD, \\
B \rightarrow cE | E, \\
D \rightarrow aD | \varepsilon, \\
E \rightarrow bE | \varepsilon.
\end{cases}
\end{equation}
Рассчитаем множества FIRST и FOLLOW для символов грамматики:
\begin{equation*}
\begin{cases}
FIRST(S) = \{a\}, \\
FIRST(C) = \{b, c\}, \\
FIRST(A) = \{b\}, \\
FIRST(B) = \{b, c, \varepsilon\}, \\
FIRST(D) = \{a, \varepsilon\}, \\
FIRST(E) = \{b, \varepsilon\}, \\
FOLLOW(S) = \{\$\}, \\
FOLLOW(C) = \{\$\}, \\
FOLLOW(A) = \{b, c, \$\}, \\
FOLLOW(B) = \{\$\}, \\
FOLLOW(D) = \{b, c, \$\}, \\
FOLLOW(E) = \{\$\}.
\end{cases}
\end{equation*}
Используя эти множества, можно построить таблицу разбора:
\begin{center}
\begin{tabular}{lllll}
\hline
Нетерминал & Входной символ &  &  & \\
 & \(a\) & \(b\) & \(c\) & \$\\
\hline
\(S\) & \(S \rightarrow aC\) &  &  & \\
\(C\) &  & \(C \rightarrow AB\) & \(C \rightarrow c\) & \\
\(A\) &  & \(A \rightarrow bD\) &  & \\
\(B\) &  & \(B \rightarrow E\) & \(B \rightarrow cE\) & \(B \rightarrow E\)\\
\(D\) & \(D \rightarrow aD\) & \(D \rightarrow \varepsilon\) & \(D \rightarrow \varepsilon\) & \(D \rightarrow \varepsilon\)\\
\(E\) &  & \(E \rightarrow bE\) &  & \(E \rightarrow \varepsilon\)\\
\hline
\end{tabular}
\end{center}

Рассмотрим работу этого автомата на примере строки \(\omega = abaa\):
\begin{center}
\begin{tabular}{llll}
\hline
Прочитанная часть входа & Стек & Вход & Действие\\
\hline
 & S\$ & abaa\$ & \\
 & aC\$ & abaa\$ & вывести \(S \rightarrow aC\)\\
a & C\$ & baa\$ & прочитать \(a\)\\
a & AB\$ & baa\$ & вывести \(C \rightarrow AB\)\\
a & bDB\$ & baa\$ & вывести \(A \rightarrow bD\)\\
ab & DB\$ & aa\$ & прочитать \(b\)\\
ab & aDB\$ & aa\$ & вывести \(D \rightarrow aD\)\\
aba & DB\$ & a\$ & прочитать \(a\)\\
aba & aDB\$ & a\$ & вывести \(D \rightarrow aD\)\\
abaa & DB\$ & \$ & прочитать \(a\)\\
abaa & B\$ & \$ & вывести \(D \rightarrow \varepsilon\)\\
abaa & E\$ & \$ & вывести \(B \rightarrow E\)\\
abaa & \$ & \$ & вывести \(E \rightarrow \varepsilon\)\\
\hline
\end{tabular}
\end{center}
\section{Задача 4}
\label{sec:org5ca31a9}
Построить по грамматике LR(1)-анализатор и продемонстрировать его работу на слове \(\omega = abab\):
\begin{equation}
S \rightarrow SaSb | \varepsilon,
\end{equation}
\subsection{Решение}
\label{sec:org4dc1cd1}
Введём в грамматику новый стартовый символ \(S'\) и дополним её правилом \(S' \rightarrow S\), получим грамматику:
\begin{equation}
\begin{cases}
S' \rightarrow S \\
S \rightarrow SaSb \\
S \rightarrow \varepsilon.
\end{cases}
\end{equation}
Вычислим множества FIRST:
\begin{equation*}
FIRST(S') = FIRST(S) = \{\varepsilon, a\}.
\end{equation*}
Вычислим множество ситуаций для дополненной грамматики:
\begin{equation*}
CLOSURE(\{[S' \rightarrow \cdot S, \$]\}) = \{[S' \rightarrow \cdot S, \$],
[S \rightarrow \cdot SaSb, \$], [S \rightarrow \cdot, \$],
[S \rightarrow \cdot SaSb, a], [S \rightarrow \cdot, a]\} = I_1
\end{equation*}
\begin{equation*}
GOTO(I_1, a) = GOTO(I_1, b) = \emptyset
\end{equation*}
\begin{multline*}
GOTO(I_1, S) = CLOSURE(\{[S' \rightarrow S\cdot, \$], [S \rightarrow S\cdot aSb, \$],
[S \rightarrow S\cdot aSb, a]\}) = \\
= \{[S' \rightarrow S\cdot, \$], [S \rightarrow S\cdot aSb, \$],
[S \rightarrow S\cdot aSb, a]\} = I_2
\end{multline*}
\begin{equation*}
GOTO(I_2, S) = GOTO(I_2, b) = \emptyset
\end{equation*}
\begin{multline*}
GOTO(I_2, a) = CLOSURE(\{[S \rightarrow Sa\cdot Sb, \$], [S \rightarrow Sa\cdot Sb, a]\}) = \\
= \{[S \rightarrow Sa\cdot Sb, \$], [S \rightarrow Sa\cdot Sb, a],
[S \rightarrow \cdot SaSb, b], [S \rightarrow \cdot, a], [S \rightarrow \cdot, b],
[S \rightarrow \cdot SaSb, a]\} = I_3
\end{multline*}
\begin{equation*}
GOTO(I_3, a) = GOTO(I_3, b) = \emptyset
\end{equation*}
\begin{multline*}
GOTO(I_3, S) = CLOSURE(\{[S \rightarrow SaS\cdot b, \$], [S \rightarrow SaS\cdot b, a],
[S \rightarrow S\cdot aSb, a], [S \rightarrow S\cdot aSb, b]\}) = \\
= \{[S \rightarrow SaS\cdot b, \$], [S \rightarrow SaS\cdot b, a],
[S \rightarrow S\cdot aSb, a], [S \rightarrow S\cdot aSb, b]\} = I_4
\end{multline*}
\begin{equation*}
GOTO(I_4, S) = \emptyset
\end{equation*}
\begin{multline*}
GOTO(I_4, a) = CLOSURE(\{[S \rightarrow Sa\cdot Sb, a], [S \rightarrow Sa\cdot Sb, b]\}) = \\
= \{[S \rightarrow Sa\cdot Sb, a], [S \rightarrow Sa\cdot Sb, b],
[S \rightarrow \cdot SaSb, b], [S \rightarrow \cdot a], [S \rightarrow \cdot, b],
[S \rightarrow \cdot SaSb, a]\} = I_5
\end{multline*}
\begin{equation*}
GOTO(I_4, b) = CLOSURE(\{[S \rightarrow SaSb\cdot, a], [S \rightarrow SaSb\cdot, \$]\})
= \{[S \rightarrow SaSb\cdot, b], [S \rightarrow SaSb\cdot, \$]\} = I_6
\end{equation*}
\begin{equation*}
GOTO(I_5, a) = GOTO(I_5, b) = \emptyset
\end{equation*}
\begin{multline*}
GOTO(I_5, S) = CLOSURE(\{[S \rightarrow SaS\cdot b, a], [S \rightarrow SaS\cdot b, b],
[S \rightarrow S\cdot aSb, a], [S \rightarrow S\cdot aSb, b]\}) = \\
= \{[S \rightarrow SaS\cdot b, a], [S \rightarrow SaS\cdot b, b],
[S \rightarrow S\cdot aSb, a], [S \rightarrow S\cdot aSb, b]\} = I_7
\end{multline*}
\begin{equation*}
GOTO(I_6, S) = GOTO(I_6, a) = GOTO(I_6, b) = \emptyset
\end{equation*}
\begin{equation*}
GOTO(I_7, S) = \emptyset
\end{equation*}
\begin{equation*}
GOTO(I_7, a) = CLOSURE(\{[S \rightarrow Sa\cdot Sb, a], [S \rightarrow Sa\cdot Sb, b]\}) = I_5
\end{equation*}
\begin{equation*}
GOTO(I_7, b) = CLOSURE(\{[S \rightarrow SaSb\cdot, a], [S \rightarrow SaSb\cdot, b]\}) = I_8
\end{equation*}
\begin{equation*}
GOTO(I_8, S) = GOTO(I_8, a) = GOTO(I_8, b) = \emptyset
\end{equation*}

Теперь можно построить таблицы ACTION и GOTO LR(1)-автомата:
\begin{center}
\begin{tabular}{rlllr}
\hline
Состояние & ACTION &  &  & GOTO\\
\hline
 & a & b & \$ & S\\
\hline
1 & r2 &  & r2 & 2\\
2 & s3 &  & acc & \\
3 & r2 & r2 &  & 4\\
4 & s5 & s6 &  & \\
5 & r2 & r2 &  & 7\\
6 & r1 &  & r1 & \\
7 & s5 & s8 &  & \\
8 & r1 & r1 &  & \\
\hline
\end{tabular}
\end{center}
Здесь \(ri\) обозначает свёртку по правилу c номером \(i\), \(sj\) обозначает сдвиг символа на стек
с переходом в состояние с номером \(j\)(и правила, и состояния нумеруются с единицы,
добавленное правило не учитывается), \(acc\) обозначает принятие слова автоматом. Пустые клетки
означают состояние ошибки.

Рассмотрим работу автомата на слове \(\omega = abab\):
\begin{center}
\begin{tabular}{llll}
\hline
Стек & Прочитанная часть входа & Вход & Действие\\
\hline
1 &  & abab\$ & свёртка по правилу \(S \rightarrow \varepsilon\)\\
1 2 &  & abab\$ & считывание {}<<a>>{}\\
1 2 3 & a & bab\$ & свёртка по правилу \(S \rightarrow \varepsilon\)\\
1 2 3 4 & a & bab\$ & считывание {}<<b>>{}\\
1 2 3 4 6 & ab & ab\$ & свёртка по правилу \(S \rightarrow SaSb\)\\
1 2 & ab & ab\$ & считывание {}<<a>>{}\\
1 2 3 & aba & b\$ & свёртка по правилу \(S \rightarrow \varepsilon\)\\
1 2 3 4 & aba & b\$ & считывание {}<<b>>{}\\
1 2 3 4 6 & abab & \$ & свёртка по правилу \(S \rightarrow SaSb\)\\
1 & abab & \$ & свёртка по правилу \(S \rightarrow \varepsilon\)\\
1 2 & abab & \$ & принятие строки\\
\hline
\end{tabular}
\end{center}
\section{Задача 5}
\label{sec:orgc70028f}
Построить по грамматике LR(1)-анализатор и продемонстрировать его работу на слове \(\omega = abab\):
\begin{equation}
\begin{cases}
S \rightarrow Ab | Ac, \\
A \rightarrow aA | b.
\end{cases}
\end{equation}
\subsection{Решение}
\label{sec:org18d8e6a}
Дополним грамматику новым стартовым символом \(S'\) и правилом \(S' \rightarrow S\):
\begin{equation}
\begin{cases}
S' \rightarrow S \\
S \rightarrow Ab \\
S \rightarrow Ac \\
A \rightarrow aA \\
A \rightarrow b.
\end{cases}
\end{equation}
Вычислим множества FIRST для нетерминалов:
\begin{equation}
\begin{cases}
FIRST(S') = FIRST(S) = \{a, b\}, \\
FIRST(A) = \{a, b\}, \\
\end{cases}
\end{equation}
Вычислим множество ситуаций:
\begin{multline*}
I_1 = CLOSURE(\{[S' \rightarrow \cdot S, \$]\}) = \\
= \{[S' \rightarrow \cdot S, \$],
[S \rightarrow \cdot Ab, \$], [S \rightarrow \cdot Ac, \$], [A \rightarrow \cdot aA, b],
[A \rightarrow \cdot b, b], [A \rightarrow \cdot aA, c], [A \rightarrow \cdot b, c]\}
\end{multline*}
\begin{equation*}
GOTO(I_1, c) = \emptyset
\end{equation*}
\begin{equation*}
GOTO(I_1, S) = CLOSURE(\{[S' \rightarrow S\cdot, \$]\}) = \{[S'\rightarrow S\cdot, \$]\} = I_2
\end{equation*}
\begin{equation*}
GOTO(I_1, A) = CLOSURE(\{[S \rightarrow A\cdot b, \$], [S \rightarrow A\cdot c, \$]\}) =
\{[S \rightarrow A\cdot b, \$], [S \rightarrow A\cdot c, \$]\} = I_3
\end{equation*}
\begin{multline*}
GOTO(I_1, a) = CLOSURE(\{[A \rightarrow a\cdot A, b], [A \rightarrow a\cdot A, c]\}) = \\
= \{[A \rightarrow a\cdot A, b], [A \rightarrow a\cdot A, c],
[A \rightarrow \cdot aA, b], [A \rightarrow \cdot b, b],
[A \rightarrow \cdot aA, c], [A \rightarrow \cdot b, c]\} = I_4
\end{multline*}
\begin{equation*}
GOTO(I_1, b) = CLOSURE(\{[A \rightarrow b\cdot, b], [A \rightarrow b\cdot, c]\}) =
\{[A \rightarrow b\cdot, b], [A \rightarrow b\cdot, c]\} = I_5
\end{equation*}
\begin{equation*}
GOTO(I_2, S) = GOTO(I_2, A) = GOTO(I_2, a) = GOTO(I_2, b) = GOTO(I_2, c) = \emptyset
\end{equation*}
\begin{equation*}
GOTO(I_3, S) = GOTO(I_3, A) = GOTO(I_3, a) = \emptyset
\end{equation*}
\begin{equation*}
GOTO(I_3, b) = CLOSURE(\{[S \rightarrow Ab\cdot, \$]\}) = \{[S \rightarrow Ab\cdot, \$]\} = I_6
\end{equation*}
\begin{equation*}
GOTO(I_3, c) = CLOSURE(\{[S \rightarrow Ac\cdot, \$]\}) = \{[S \rightarrow Ac\cdot, \$]\} = I_7
\end{equation*}
\begin{equation*}
GOTO(I_4, S) = GOTO(I_4, c) = \emptyset
\end{equation*}
\begin{equation*}
GOTO(I_4, A) = CLOSURE(\{[A \rightarrow aA\cdot, b], [A \rightarrow aA\cdot, c]\}) =
\{[A \rightarrow aA\cdot, b], [A \rightarrow aA\cdot, c]\} = I_8
\end{equation*}
\begin{equation*}
GOTO(I_4, a) = CLOSURE(\{[A \rightarrow a\cdot A, b], [A \rightarrow a\cdot A, c]\}) = I_4
\end{equation*}
\begin{equation*}
GOTO(I_4, b) = CLOSURE(\{[A \rightarrow b\cdot, b], [A \rightarrow b\cdot, c]\}) = I_5
\end{equation*}
\begin{equation*}
GOTO(I_5, S) = GOTO(I_5, A) = GOTO(I_5, a) = GOTO(I_5, b) = GOTO(I_5, c) = \emptyset
\end{equation*}
\begin{equation*}
GOTO(I_6, S) = GOTO(I_6, A) = GOTO(I_6, a) = GOTO(I_6, b) = GOTO(I_6, c) = \emptyset
\end{equation*}
\begin{equation*}
GOTO(I_7, S) = GOTO(I_7, A) = GOTO(I_7, a) = GOTO(I_7, b) = GOTO(I_7, c) = \emptyset
\end{equation*}
\begin{equation*}
GOTO(I_8, S) = GOTO(I_8, A) = GOTO(I_8, a) = GOTO(I_8, b) = GOTO(I_8, c) = \emptyset
\end{equation*}
Построим таблицы ACTION и GOTO:
\begin{center}
\begin{tabular}{rllllrr}
\hline
Состояние & ACTION &  &  &  & GOTO & \\
 & a & b & c & \$ & S & A\\
\hline
1 & s4 & s5 &  &  & 2 & 3\\
2 &  &  &  & acc &  & \\
3 &  & s6 & s7 &  &  & \\
4 & s4 & s5 &  &  &  & 8\\
5 &  & r4 & r4 &  &  & \\
6 &  &  &  & r1 &  & \\
7 &  &  &  & r2 &  & \\
8 &  & r3 & r3 &  &  & \\
\hline
\end{tabular}
\end{center}
Рассмотрим работу автомата на слове \(\omega = abab\):
\begin{center}
\begin{tabular}{llll}
\hline
Стек & Прочитанная часть входа & Вход & Действие\\
\hline
1 &  & abab\$ & Считать {}<<a>>{}\\
1 4 & a & bab\$ & Считать {}<<b>>{}\\
1 4 5 & ab & ab\$ & Выдать ошибку\\
\hline
\end{tabular}
\end{center}
\section{Задача 6}
\label{sec:org5fb1620}
Заполнить таблицу Кока-Янгера-Касами для языка, заданного грамматикой \(G\) и выяснить, может
ли \(G\) порождать цепочку \(a + b * c\).
\begin{equation}
G = \{\{A, C, D\}, \{a, b, c, +, *\}, P, D\},
\end{equation}
где
\begin{equation}
\begin{cases}
D \rightarrow D + A | A \\
A \rightarrow A * C | C \\
C \rightarrow a | b | c
\end{cases}
\end{equation}
\subsection{Решение}
\label{sec:org14d41b9}
Для начала приведём грамматику к нормальной форме Хомского. Исходная грамматика уже не
содержит epsilon-продукций, но содержит цепные продукции \(D \rightarrow A\) и \(A \rightarrow C\).
Удалим эти продукции:
\begin{equation*}
\begin{cases}
D \rightarrow D + A | A \\
A \rightarrow A * C | C \\
C \rightarrow a | b | c
\end{cases}
\rightarrow
\begin{cases}
D \rightarrow D + A | A \\
A \rightarrow A * C | a | b | c \\
C \rightarrow a | b | c
\end{cases}
\rightarrow
\begin{cases}
D \rightarrow D + A | A * C | a | b | c \\
A \rightarrow A * C | a | b | c \\
C \rightarrow a | b | c
\end{cases}
\end{equation*}
Полученная грамматика не содержит бесполезных символов, поэтому путём введения новых переменных
получем нормальную форму Хомского для неё:
\begin{equation*}
\begin{cases}
D \rightarrow DR_D \\
D \rightarrow AR_A \\
D \rightarrow a \\
D \rightarrow b \\
D \rightarrow c \\
R_D \rightarrow PA \\
P \rightarrow + \\
R_A \rightarrow MC \\
M \rightarrow * \\
A \rightarrow AR_A \\
A \rightarrow a \\
A \rightarrow b \\
A \rightarrow c \\
C \rightarrow a \\
C \rightarrow b \\
C \rightarrow c
\end{cases}
\end{equation*}
Построим таблицу Кока-Янгера-Касами размера 5x5 для цепочки \(a+b*c\):
\begin{center}
\begin{tabular}{rlllll}
\hline
\(i\backslash j\) & 1 & 2 & 3 & 4 & 5\\
\hline
1 & \(\{D, A, C\}\) & \(\emptyset\) & \(\{D\}\) & \(\emptyset\) & \(\{D\}\)\\
2 &  & \(\{P\}\) & \(\{R_D\}\) & \(\emptyset\) & \(\{R_D\}\)\\
3 &  &  & \(\{D, A, C\}\) & \(\emptyset\) & \(\{A, D\}\)\\
4 &  &  &  & \(\{M\}\) & \(\{R_A\}\)\\
5 &  &  &  &  & \(\{D, A, C\}\)\\
\hline
\end{tabular}
\end{center}
В этой таблице \(T_{ij}\) -- множество нетерминалов, порождающих подцепочку цепочки \(a+b*c\),
начинающуюся с индекса \(i\) и заканчивающуюся индексом \(j\). Порядок заполнения таблицы:
\begin{enumerate}
\item Сначала заполняются диагональные ячейки: \(T_{ii}\) -- множество левых частей продукций вида \(P \rightarrow a_i\).
\item Далее заполняются ячейки \(T_{ij}, i < j\). Пусть заполнены ячейки \(T_{kj}, i < k \leq j\) и \(T_{im}, i \leq m < j\). Тогда построим множество \(L = \{PQ | P \in T_{ik}, Q \in T_{(k + 1)j}, k \in \overline{i, (j - 1)}\}\). Тогда \(T_{ij}\) -- множество левых частей продукций, правые части которых совпадают с элементами \(L\).
\end{enumerate}

Поскольку \(D \in T_{15}\), цепочка \(a+b*c\) принадлежит языку, порождаемому \(G\).
\end{document}
