% Created 2020-02-25 Tue 17:27
% Intended LaTeX compiler: pdflatex
\documentclass[11pt]{article}
\usepackage[utf8]{inputenc}
\usepackage[T1]{fontenc}
\usepackage{graphicx}
\usepackage{grffile}
\usepackage{longtable}
\usepackage{wrapfig}
\usepackage{rotating}
\usepackage[normalem]{ulem}
\usepackage{amsmath}
\usepackage{textcomp}
\usepackage{amssymb}
\usepackage{capt-of}
\usepackage{hyperref}
\usepackage{minted}
\usepackage{amsmath}
\usepackage{esint}
\usepackage[english, russian]{babel}
\usepackage{mathtools}
\usepackage{amsthm}
\usepackage[top=0.8in, bottom=0.75in, left=0.625in, right=0.625in]{geometry}
\def\zall{\setcounter{lem}{0}\setcounter{cnsqnc}{0}\setcounter{th}{0}\setcounter{Cmt}{0}\setcounter{equation}{0}\setcounter{stnmt}{0}}
\newcounter{lem}\setcounter{lem}{0}
\def\lm{\par\smallskip\refstepcounter{lem}\textbf{\arabic{lem}}}
\newtheorem*{Lemma}{Лемма \lm}
\newcounter{stnmt}\setcounter{stnmt}{0}
\def\st{\par\smallskip\refstepcounter{stnmt}\textbf{\arabic{stnmt}}}
\newtheorem*{Statement}{Утверждение \st}
\newcounter{th}\setcounter{th}{0}
\def\th{\par\smallskip\refstepcounter{th}\textbf{\arabic{th}}}
\newtheorem*{Theorem}{Теорема \th}
\newcounter{cnsqnc}\setcounter{cnsqnc}{0}
\def\cnsqnc{\par\smallskip\refstepcounter{cnsqnc}\textbf{\arabic{cnsqnc}}}
\newtheorem*{Consequence}{Следствие \cnsqnc}
\newcounter{Cmt}\setcounter{Cmt}{0}
\def\cmt{\par\smallskip\refstepcounter{Cmt}\textbf{\arabic{Cmt}}}
\newtheorem*{Note}{Замечание \cmt}
\author{Sergey Makarov}
\date{\today}
\title{}
\hypersetup{
 pdfauthor={Sergey Makarov},
 pdftitle={},
 pdfkeywords={},
 pdfsubject={},
 pdfcreator={Emacs 26.3 (Org mode 9.3)}, 
 pdflang={English}}
\begin{document}

\tableofcontents

\begin{itemize}
\item Интеграл Лебега без меры
\item Пространство с обобщёнными производными и обобщённые функции
\item Задача управления подъёмным краном в штормящем море.
\item Задача управления колебаниями понтографа. Найти материалы для проводов и перемычек.
\end{itemize}

\section{Уравнения смешанного типа}
\label{sec:orgb950b37}
\begin{itemize}
\item Трансзвуковая гидродинамка
\end{itemize}
Уравнение кого-то-кого-то:
u = u(x, y)
\begin{equation}
yu_{xx} + u_{yy} = 0
\end{equation}
В верхней полуплоскости уравнение эллиптическое, в нижней -- гиперболическое. Нужно найти
решение $u \in C(\overline{D}) \cap C^1(D) \cap C^2(D^+) \cap C^2(D^-)$.

Положим $\nu(x) = \frac{\partial u(x, 0)}{\partial y}$.
Уравнение Лаврентьева-Бицадзе:
\begin{equation}
\operatorname{sgn}(y)u_{xx} + u_{yy} = 0
\end{equation}
Метод разделения переменных приводит к уравнениям:
\begin{equation}
\begin{cases}
X''(x) + \lambda X(x) = 0, \\
X(0) = X(1) = 0
\end{cases}
\end{equation}
\begin{equation}
X_n(x) = \sin \pi nx, n = 1, 2, 3, \ldots.
\end{equation}
Задача нагруженной струны приводит к уравнению вида:
\begin{equation}
\begin{cases}
X'' + \lambda X = 0, \\
X(0) = 0, X'(1) = d\lambda X(1).
\end{cases}
\end{equation}
\begin{equation}
X_n = \sin \sqrt{\lambda_n}x, \ctg\sqrt{\lambda} = d\sqrt{\lambda}
\end{equation}
\end{document}
