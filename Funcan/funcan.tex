% Created 2020-02-22 Sat 16:39
% Intended LaTeX compiler: pdflatex
\documentclass[11pt]{article}
\usepackage[utf8]{inputenc}
\usepackage[T1]{fontenc}
\usepackage{graphicx}
\usepackage{grffile}
\usepackage{longtable}
\usepackage{wrapfig}
\usepackage{rotating}
\usepackage[normalem]{ulem}
\usepackage{amsmath}
\usepackage{textcomp}
\usepackage{amssymb}
\usepackage{capt-of}
\usepackage{hyperref}
\usepackage{minted}
\usepackage{amsmath}
\usepackage{esint}
\usepackage[english, russian]{babel}
\usepackage{mathtools}
\usepackage{amsthm}
\usepackage[top=0.8in, bottom=0.75in, left=0.625in, right=0.625in]{geometry}
\def\zall{\setcounter{lem}{0}\setcounter{cnsqnc}{0}\setcounter{th}{0}\setcounter{Cmt}{0}\setcounter{equation}{0}\setcounter{stnmt}{0}}
\newcounter{lem}\setcounter{lem}{0}
\def\lm{\par\smallskip\refstepcounter{lem}\textbf{\arabic{lem}}}
\newtheorem*{Lemma}{Лемма \lm}
\newcounter{stnmt}\setcounter{stnmt}{0}
\def\st{\par\smallskip\refstepcounter{stnmt}\textbf{\arabic{stnmt}}}
\newtheorem*{Statement}{Утверждение \st}
\newcounter{th}\setcounter{th}{0}
\def\th{\par\smallskip\refstepcounter{th}\textbf{\arabic{th}}}
\newtheorem*{Theorem}{Теорема \th}
\newcounter{cnsqnc}\setcounter{cnsqnc}{0}
\def\cnsqnc{\par\smallskip\refstepcounter{cnsqnc}\textbf{\arabic{cnsqnc}}}
\newtheorem*{Consequence}{Следствие \cnsqnc}
\newcounter{Cmt}\setcounter{Cmt}{0}
\def\cmt{\par\smallskip\refstepcounter{Cmt}\textbf{\arabic{Cmt}}}
\newtheorem*{Note}{Замечание \cmt}
\author{Sergey Makarov}
\date{\today}
\title{}
\hypersetup{
 pdfauthor={Sergey Makarov},
 pdftitle={},
 pdfkeywords={},
 pdfsubject={},
 pdfcreator={Emacs 26.3 (Org mode 9.3)}, 
 pdflang={English}}
\begin{document}

\tableofcontents


\section{Структура курса}
\label{sec:org7fe6210}
\begin{enumerate}
\item Структура множеств на прямой
\item Мера и измеримые множества
\item Измеримые функции
\item Интеграл Лебега
\item Пространства NP(?)
\item Метрические пространства
\item Линейные пространства
\item Обратный оператор
\item Линейный функционал в бесконечномерном пространстве
\item Гильбертово пространство
\item Сопряжённый оператор(без скалярного произведения)
\item Вполне непрерывные компактные операторы
\item Теория Фредгольма
\item Спектральная теория в бесконечномерном пространстве
\end{enumerate}

\section{Список литературы}
\label{sec:orge3ac9c8}
Капустин "Функциональный анализ".

Зачёт из двух частей: теоретический вопрос и задача из списка.
Список литературы:
\begin{enumerate}
\item В. А. Ильин, Э. Г. Позняк "Основы математического анализа. Часть II".
\item Л. Н. Королёв, А. Н. Колмогоров, С. В. Фомин "Элементы теории функций и функционального анализа".
\item Л. А. Люстерник, В. И. Соболев "Элементы функционального анализа".
\item Ю. С. Очан "Сборник задач по математическому анализу".
\item Т. А. Леонтьева, А. В. Домрина "Задачи по теории функций и функциональному анализу".
\end{enumerate}
\section{Открытые и замкнутые множества на прямой}
\label{sec:org7f871f4}
$$E = E_1 \cup E_2 = \{e | e \in E_1 || e \in E_2\}$$
$$E = E_1 \cap E_2 = \{e | e \in E_1 \&\& e \in E_2\}$$
$$E = E_1 \backslash E_2 = \{e | e \in E_1 \&\& e \notin E_2\}$$
$$E_1\triangle E_2 = (E_1 \backslash E_2) \cup (E_2 \backslash E_1)$$
$$CE = R \backslash E$$
Предельная точка $x_0$ множества $E$ - точка $x_0$, в любой окрестности которой есть точки $E$.
Пусть $E'$ - множество предельных точек $E$. Возможны случаи:
\begin{enumerate}
\item \(E' \subset E\). Тогда \(E\) замкнуто.
\item \(E \subset E'\). Тогда \(E\) - плотное в себе.
\item \(E = E'\). Тогда \(E\) - совершенное.
\item \(\overline{E} = E \cup E'\)
\end{enumerate}
Свойства:
\begin{enumerate}
\item $E'$ - замкнутое.
\item $\overline{E}$ - замкнутое.
\item конечное объединение замкнутых множеств замкнуто.
\end{enumerate}
Бесконечное(и даже счётное) объединение замкнутых множеств может быть незамкнуто.

Точка множества называется \textbf{внутренней}, если она содержится в нём вместе с некоторой
окрестностью. $\operatorname{int} E$ - множество внутренних точек $E$. Множество, совпадающее
с множеством своих внутренних точек, называется \textbf{открытым}. Дополнение открытого
множества является замкнутым и наоборот, дополнение замкнутого множества является открытым.
Объединение любого числа открытых множеств является открытым.
Пересечение конечного числа открытых множеств есть множество открытое, для бесконечного числа
это уже неверно.
Пересечение любого числа замкнутых множеств замкнуто(доказывается переходом к дополнению).
Если $A$ замкнутое, а $B$ открытое, то $A\backslash B$ замкнутое.
Если $A$ открытое, а $B$ замкнутое, то $A\backslash B$ открытое.
\begin{Theorem}
Любое открытое множество на прямой представляет собой объединение конечного или счётного числа
попарно непересекающихся интервалов(в том числе неограниченных).
\end{Theorem}
\begin{proof}
Возьмём $x \in E$ и рассмотрим $V(x)$ - объединение окрестностей $x$, целиком лежащих в $E$.
Обозначим $a = \inf V(x), b = \sup V(x)$. Обе эти точки лежат вне $V(x)$. Возьмём точку
$a < y < x$. Тогда $\exists (\alpha, \beta) \in E, x \in (\alpha, \beta), y' \in (\alpha, beta) a < y' < y < x$.

$\forall x_1 \neq x_2 V(x_1) = V(x_2)$, либо $V(x_1) \cap V(x_2) = 0$.
\end{proof}
Рассмотрим $I = [0, 1]$. Пусть $G_1 = \left(\frac13, \frac23\right)$. Вырезаем середину, из
оставшихся сегментов вырезаем середину, и т. д. Остаток называется \textbf{канторовым множеством} K.
Канторово множество замкнуто как дополнение открытого. Суммарная канторова множества равна нулю.

Исследуем счётность канторова множества. Представим все числа $[0, 1]$ в троичном виде. Тогда
точки канторова множества - все точки, не содержащие в своём представлении единиц. Множество
таких точек континуально.
\section{Измеримые множества. Мера Лебега.}
\label{sec:orge4089a8}
Покрытием множества $E$ будем называть счётную систему интервалов, объеднинение которых содержит $E$
Составим число $\sigma(s)$ - сумму длин отрезков, входящих в покрытие.

\textbf{Внешняя мера} $|E|^* = \inf_{s(E)}\sigma(S)$.
\textbf{Расстоянием} между множествами назовём минимальное расстояние между их точками.
Свойства внешней меры:
\begin{enumerate}
\item $E_1 \subset E_2 \Rightarrow |E_1|^* \leq |E_2|^*$(\textbf{монотонность}).
\item $E = U_{n = 1}^{\infty}E_n \Rightarrow |E|^* \leq \sum_{n = 1}^{\infty}|E_n|^*$(\textbf{полу-аддитивность}).
\item $\rho(E_1, E_2) = d > 0 \Rightarrow |E_1 \cup E_2|^* = |E_1|^* + |E_2|^*$.
\item $\forall E \forall\varepsilon > 0 \exists\text{ открытое }G E \subset G |G|^* < |E|^* + \varepsilon$
\end{enumerate}
\textbf{Доказательство}:\\
1. \ldots\\
2. $\forall \varepsilon > 0 \exists \{\Delta^n_k\}_{k = 1}^{\infty} E_n \subset U_{k = 1}^{\infty}\Delta_n^k \sum_{k = 1}^{\infty}|\Delta_k^n| < |E_n|^* + \frac{\varepsilon}{2^n}$.
Тогда $|E|^* \leq \sum_{n = 1}^{\infty}\sum_{k = 1}^{\infty}|\Delta_k^n| \leq \sum_{n = 1}^{\infty}\left(|E_n|^* + \frac{\varepsilon}{2^n}\right) \leq \sum_{n = 1}^{\infty}|E_n|^* + \varepsilon$.\\
3. Для любого $\varepsilon > 0$ можно построить покрытие длины меньше $|E|^* + \frac{\varepsilon}2$, такое, что длина каждого интервала
меньше любого наперёд заданного числа. Для этого нужно "раздробить" отрезок, покрыв точки "склейки" интервалами длины $\frac{\varepsilon}4, \frac{\varepsilon}8, \ldots$, учитывая
заданную максимальную длину. Такое покрытие можно разделить на два покрытия, одно только для $E_1$,
другое для $E_2$. Переходя к пределу при $\varepsilon \to 0$, получаем искомое утверждение.
Спецкурс "Избранные главы функционального анализа" по вторникам в 16:20.
"Актуальные задачи функционального анализа и математической физики".

Множество $E$ на прямой называется \textbf{измеримым по Лебегу}(измеримым), если
$\forall \varepsilon > 0 \exists \text{ открытое множество } G: E \subset G, |G \backslash E|^* < \varepsilon: |E| = |E|^*$.
\begin{Theorem}
$|E| = 0 \Leftrightarrow |E|^* = 0$
\end{Theorem}
\begin{proof}
Если $|E| = 0$, то по определению $|E|^* = 0$.

Пусть $|E|^* = 0$. Тогда $\forall \varepsilon > 0 \exists G: E \subset G: |G|^* < |E|^* + \varepsilon$
$G \backslash E \subset G \Rightarrow |G \ E|^* \leq |G|^* < |E|^* + \varepsilon = \varepsilon$.
\end{proof}
\begin{Theorem}
Всякое открытое множество измеримо по Лебегу, и его мера равна сумме мер попарно непересекающихся
покрывающих его интервалов.
\end{Theorem}
\begin{Theorem}
Объединение конечного или счётного числа измеримых множеств есть множество измеримое.
\end{Theorem}
\begin{proof}
Из измеримости множеств следует, что $\forall \varepsilon \exists G_n \supset E_n$
$|G_n \backslash E_n|^* < \frac{\varepsilon}{2^n}$. Рассмотрим $G = \cup_{n = 1}^{\infty}G_n$. Тогда
$E \subset G, G \backslash E \subset \cup_{n = 1}^{\infty}(G_n \backslash E_n)$, откуда
\begin{equation*}
|G \backslash E|^* \leq \sum_{n = 1}^{\infty}|G_n \backslash E_n|* \leq \sum_{n = 1}^{\infty}\frac{\varepsilon}{2^n} = \varepsilon
\end{equation*}
\end{proof}
\begin{Theorem}
Любое замкнутое множество измеримо по Лебегу.
\end{Theorem}
\begin{proof}
Рассмотрим сначала случай ограниченного множества. Пусть $\Delta = (a, b)$. Введём обозначение:
\begin{equation*}
\Delta^{\alpha} = \begin{cases}
(a + \alpha, b - \alpha), \alpha < \frac{b - a}2, \alpha > 0, \\
\emptyset, \alpha \geq \frac{b - a}2.
\end{cases}
\end{equation*}
$\overline{\Delta^{\alpha}}$ - замыкание $\Delta^{\alpha}$.
\begin{equation*}
E_n = \cup_{n = 1}^{\infty}\Delta_k
\end{equation*}
\begin{equation*}
E_n^{\alpha} = \cup_{n = 1}^{\infty}\Delta_k^{\alpha}
\end{equation*}
$\overline{E_n}$ и $\overline{E_n^{\alpha}}$ - замыкания $E_n$ и $E_n^{\alpha}$ соответственно.

$G \supset F |G|^* < |F|^* + \varepsilon |G \backslash F| = \cup_{n = 1}^{\infty}\Delta_n$,
$\Delta_i \cap \Delta_j = \emptyset$ при $i \neq j$. $\overline{E_n^{\alpha}} \subset G \backslash F$,
поэтому $|\overline{E_n^{\alpha}} \cup F| = |\overline{E_n^{\alpha}}|^* + |F|^* < |G|^* < |F|^* + \varepsilon$,
откуда $|\overline{E_n^{\alpha}}|^* < \varepsilon$.

Перейдём теперь к неограниченному случаю. Положим в нём $F_n = F \cap [-n, n], F = \cup_{n = 1}^{\infty}F_n$
и перейдём в предыдущем равенстве к пределу при $n \to \infty$.
\end{proof}
\begin{Theorem}
Если $E$ измеримо, то и $CE$ измеримо.
\end{Theorem}
\begin{proof}
\begin{equation*}
\forall n \in \mathbb{N} \exists G_n |G_n \backslash E|^* < \frac1n \Rightarrow
CE \backslash CG_n = G_n \backslash E, F = \cup_{n = 1}^{\infty}F_n.
\end{equation*}
\begin{equation*}
CE \backslash F \subset CE \backslash F_n
\end{equation*}
\begin{equation*}
|CE \backslash F|^* \leq |CE \backslash F_n|^* < \frac1n \Rightarrow
|CE \backslash F|^* = 0 \Rightarrow |CE \backslash F| = 0.
\end{equation*}
\end{proof}
\begin{Theorem}[Критерий измеримости по Лебегу]
$E$ измеримо $\Leftrightarrow \forall \varepsilon > 0 \exists F\text{ - замкнутое} \subset E: |E \ F|^* < \varepsilon$.
\end{Theorem}
\begin{proof}
Следует из теоремы 6.
\end{proof}
\begin{Theorem}
Пересечение конечного и счётного числа измеримых множеств измеримо по Лебегу.
\end{Theorem}
\begin{proof}
\begin{equation*}
CE = \cup_{n = 1}^{\infty}CE_n \Rightarrow \text{ измеримо} \Rightarrow E\text{ также измеримо.}
\end{equation*}
\end{proof}
\begin{Theorem}
Для измеримых $A$ и $B$ $A \backslash B = A \cap CB$ измеримо.
\end{Theorem}
\begin{Theorem}[$\sigma$-аддитивность меры]
Если множество $E$ представимо в виде объединения не более чем счётного числа попарно
непересекающихся множеств, его мера равна сумме множеств мер объединения.
\end{Theorem}
\begin{proof}
Рассмотрим случай ограниченных множеств $E_n$. Тогда по критерию
\begin{equation*}
\forall \varepsilon \exists F_n \subset E_n |E_n \backslash F_n| < \frac{\varepsilon}{2^n}
\end{equation*}
\begin{equation*}
E_n = (E_n \backslash F_n) \cup F_n \Rightarrow |E_n| < |E_n \backslash F_n| + |F_n|
\end{equation*}
Тогда
\begin{equation*}
\sum_{k = 1}^n|E_k| < \sum_{k = 1}^n|E_k \backslash F_k| + \sum_{k = 1}^n|F_k| <
\sum_{k = 1}^n\frac{\varepsilon}{2^k} + |\cup_{k = 1}^nF_k| < |E| + \varepsilon
\end{equation*}
Переходя к пределу при $n \to \infty$, получаем, что
\begin{equation*}
\sum_{k = 1}^n|E_k| \leq |E| + \varepsilon
\end{equation*}
откуда при $\varepsilon \to 0 \sum_{k = 1}^n|E_k| \leq |E|$. Из свойства 3 внешней меры вытекает,
что $\sum_{k = 1}^n|E_k| \geq |E| \Rightarrow \sum_{k = 1}^n |E_k| = |E|$.

Перейдём теперь к неограниченному случаю. Рассмотрим множества $E_n^k = E_n \cap (-k, k + 1]$.
Тогда
\begin{equation}
E = \cup_{k = -\infty}^{+\infty}\cup_{n = 1}^{\infty}E^k_n,
\end{equation}
а для $E^k_n$ уже доказано свойство счётной аддитивности.
\end{proof}
Множество $G$ является \textbf{множеством типа $G_\delta$}, если $G = \cap_{n = 1}^{\infty}G_n$
($G_n$ - открытые). Множество $F$ называется \textbf{множеством типа $F_\sigma$}, если
$F = \cup_{n = 1}^{\infty}F_n$($F_n$ - замкнутые).
\begin{Theorem}
\begin{equation*}
\forall \text{ измеримого } E \exists G\text{ типа }G_{\delta}, F\text{ типа }F_{\sigma} \text{ такие,
что } G \supset E \supset F \text{ и } |G| = |E| = |F|.
\end{equation*}
\end{Theorem}
\begin{proof}
\begin{equation*}
\forall n \in \mathbb{N} \exists G_n \supset E \supset F_n: |G_n \backslash E| < \frac1n,
G = \cap_{n = 1}^{\infty}G_n, |E \backslash F_n| < \frac1n F = \cup_{n = 1}^{\infty}F_n
\end{equation*}
Тогда
\begin{equation*}
G \backslash E \subset G_n \backslash E \Rightarrow |G \backslash E| \leq |G_n \backslash E| < \frac1n
\Rightarrow |G \backslash E| = 0
\end{equation*}
Аналогично для $F$.
\end{proof}
Существуют неизмеримые множества. Пусть $\alpha$ - иррациональные, будем выбирать на
окружности классы точек, совместимые поворотом на $\pi n\alpha, n \in \mathbb{Z}$($\Phi_n$). Тогда
$C = \cup_{n = -\infty}^{+\infty}\Phi_n$. Но $|C| = 1$, а $\Phi_n$ конгруэтнтны, поэтому не могут
быть измеримы.

Непустая система множеств называется \textbf{кольцом}, если она замкнута относительно операций
пересечения и симметрической разности множеств. Множество кольца называется \textbf{единицей},
если $\forall A E\cap A = A$. Полукольцо.

Общий принцип построения меры Лебега на множестве: строим меру на полукольце, по аддитивности
продолжаем её на кольцо, затем с помощью аппроксимации продолжаем её на оставшуюся часть.

\textbf{Пример} -- Построение меры Лебега-Стилтьеса:
Рассмотрим функцию $F(x)$. Построим меру: $\mu([a, b]) = F(b + 0) - F(a)$, $\mu((a, b)) = F(b) - F(a + 0)$.
На остальные множества мера распространяется по аддитивности и покрытию.

Построим меру Лебега на плоскости:
Рассмотрим открытые и закрытые прямоугольники на $\mathbb{R}_2$. Мера прямоугольника -- его площадь.
\textbf{Элементарное} множество - множество, элементы которого получаются объединением
прямоугольников. Мера вводится как сумма мер множеств объединения. На элементарных множествах
с помощью леммы Гейне-Бореля можно показать счётную аддитивность меры. Вводим внешнюю меру
множества как точную нижнюю грань мер элементарных множеств, покрывающих данное.

\textbf{Абстрактная мера} - положительная действительнозначная функция на полукольце,
обладающая свойством конечной аддитивности.
\end{document}
