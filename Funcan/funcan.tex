% Created 2020-03-05 Thu 15:10
% Intended LaTeX compiler: pdflatex
\documentclass[11pt]{article}
\usepackage[utf8]{inputenc}
\usepackage[T1]{fontenc}
\usepackage{graphicx}
\usepackage{grffile}
\usepackage{longtable}
\usepackage{wrapfig}
\usepackage{rotating}
\usepackage[normalem]{ulem}
\usepackage{amsmath}
\usepackage{textcomp}
\usepackage{amssymb}
\usepackage{capt-of}
\usepackage{hyperref}
\usepackage{minted}
\usepackage{amsmath}
\usepackage{esint}
\usepackage[english, russian]{babel}
\usepackage{mathtools}
\usepackage{amsthm}
\usepackage[top=0.8in, bottom=0.75in, left=0.625in, right=0.625in]{geometry}
\def\zall{\setcounter{lem}{0}\setcounter{cnsqnc}{0}\setcounter{th}{0}\setcounter{Cmt}{0}\setcounter{equation}{0}\setcounter{stnmt}{0}}
\newcounter{lem}\setcounter{lem}{0}
\def\lm{\par\smallskip\refstepcounter{lem}\textbf{\arabic{lem}}}
\newtheorem*{Lemma}{Лемма \lm}
\newcounter{stnmt}\setcounter{stnmt}{0}
\def\st{\par\smallskip\refstepcounter{stnmt}\textbf{\arabic{stnmt}}}
\newtheorem*{Statement}{Утверждение \st}
\newcounter{th}\setcounter{th}{0}
\def\th{\par\smallskip\refstepcounter{th}\textbf{\arabic{th}}}
\newtheorem*{Theorem}{Теорема \th}
\newcounter{cnsqnc}\setcounter{cnsqnc}{0}
\def\cnsqnc{\par\smallskip\refstepcounter{cnsqnc}\textbf{\arabic{cnsqnc}}}
\newtheorem*{Consequence}{Следствие \cnsqnc}
\newcounter{Cmt}\setcounter{Cmt}{0}
\def\cmt{\par\smallskip\refstepcounter{Cmt}\textbf{\arabic{Cmt}}}
\newtheorem*{Note}{Замечание \cmt}
\author{Sergey Makarov}
\date{\today}
\title{}
\hypersetup{
 pdfauthor={Sergey Makarov},
 pdftitle={},
 pdfkeywords={},
 pdfsubject={},
 pdfcreator={Emacs 26.3 (Org mode 9.3)}, 
 pdflang={English}}
\begin{document}

\tableofcontents


\section{Структура курса}
\label{sec:org051f5d8}
\begin{enumerate}
\item Структура множеств на прямой
\item Мера и измеримые множества
\item Измеримые функции
\item Интеграл Лебега
\item Пространства NP(?)
\item Метрические пространства
\item Линейные пространства
\item Обратный оператор
\item Линейный функционал в бесконечномерном пространстве
\item Гильбертово пространство
\item Сопряжённый оператор(без скалярного произведения)
\item Вполне непрерывные компактные операторы
\item Теория Фредгольма
\item Спектральная теория в бесконечномерном пространстве
\end{enumerate}

\section{Список литературы}
\label{sec:orge8e827b}
Капустин "Функциональный анализ".

Зачёт из двух частей: теоретический вопрос и задача из списка.
Список литературы:
\begin{enumerate}
\item В. А. Ильин, Э. Г. Позняк "Основы математического анализа. Часть II".
\item Л. Н. Королёв, А. Н. Колмогоров, С. В. Фомин "Элементы теории функций и функционального анализа".
\item Л. А. Люстерник, В. И. Соболев "Элементы функционального анализа".
\item Ю. С. Очан "Сборник задач по математическому анализу".
\item Т. А. Леонтьева, А. В. Домрина "Задачи по теории функций и функциональному анализу".
\end{enumerate}
\section{Открытые и замкнутые множества на прямой}
\label{sec:org730ddbe}
\zall
$$E = E_1 \cup E_2 = \{e | e \in E_1 || e \in E_2\}$$
$$E = E_1 \cap E_2 = \{e | e \in E_1 \&\& e \in E_2\}$$
$$E = E_1 \backslash E_2 = \{e | e \in E_1 \&\& e \notin E_2\}$$
$$E_1\triangle E_2 = (E_1 \backslash E_2) \cup (E_2 \backslash E_1)$$
$$CE = R \backslash E$$
Предельная точка $x_0$ множества $E$ - точка $x_0$, в любой окрестности которой есть точки $E$.
Пусть $E'$ - множество предельных точек $E$. Возможны случаи:
\begin{enumerate}
\item \(E' \subset E\). Тогда \(E\) замкнуто.
\item \(E \subset E'\). Тогда \(E\) - плотное в себе.
\item \(E = E'\). Тогда \(E\) - совершенное.
\item \(\overline{E} = E \cup E'\)
\end{enumerate}
Свойства:
\begin{enumerate}
\item $E'$ - замкнутое.
\item $\overline{E}$ - замкнутое.
\item конечное объединение замкнутых множеств замкнуто.
\end{enumerate}
Бесконечное(и даже счётное) объединение замкнутых множеств может быть незамкнуто.

Точка множества называется \textbf{внутренней}, если она содержится в нём вместе с некоторой
окрестностью. $\operatorname{int} E$ - множество внутренних точек $E$. Множество, совпадающее
с множеством своих внутренних точек, называется \textbf{открытым}. Дополнение открытого
множества является замкнутым и наоборот, дополнение замкнутого множества является открытым.
Объединение любого числа открытых множеств является открытым.
Пересечение конечного числа открытых множеств есть множество открытое, для бесконечного числа
это уже неверно.
Пересечение любого числа замкнутых множеств замкнуто(доказывается переходом к дополнению).
Если $A$ замкнутое, а $B$ открытое, то $A\backslash B$ замкнутое.
Если $A$ открытое, а $B$ замкнутое, то $A\backslash B$ открытое.
\begin{Theorem}
Любое открытое множество на прямой представляет собой объединение конечного или счётного числа
попарно непересекающихся интервалов(в том числе неограниченных).
\end{Theorem}
\begin{proof}
Возьмём $x \in E$ и рассмотрим $V(x)$ - объединение окрестностей $x$, целиком лежащих в $E$.
Обозначим $a = \inf V(x), b = \sup V(x)$. Обе эти точки лежат вне $V(x)$. Возьмём точку
$a < y < x$. Тогда $\exists (\alpha, \beta) \in E, x \in (\alpha, \beta), y' \in (\alpha, beta) a < y' < y < x$.

$\forall x_1 \neq x_2 V(x_1) = V(x_2)$, либо $V(x_1) \cap V(x_2) = 0$.
\end{proof}
Рассмотрим $I = [0, 1]$. Пусть $G_1 = \left(\frac13, \frac23\right)$. Вырезаем середину, из
оставшихся сегментов вырезаем середину, и т. д. Остаток называется \textbf{канторовым множеством} K.
Канторово множество замкнуто как дополнение открытого. Суммарная канторова множества равна нулю.

Исследуем счётность канторова множества. Представим все числа $[0, 1]$ в троичном виде. Тогда
точки канторова множества - все точки, не содержащие в своём представлении единиц. Множество
таких точек континуально.
\section{Измеримые множества. Мера Лебега.}
\label{sec:orgd69e7a4}
\zall
Покрытием множества $E$ будем называть счётную систему интервалов, объеднинение которых содержит $E$
Составим число $\sigma(s)$ - сумму длин отрезков, входящих в покрытие.

\textbf{Внешняя мера} $|E|^* = \inf_{s(E)}\sigma(S)$.
\textbf{Расстоянием} между множествами назовём минимальное расстояние между их точками.
Свойства внешней меры:
\begin{enumerate}
\item $E_1 \subset E_2 \Rightarrow |E_1|^* \leq |E_2|^*$(\textbf{монотонность}).
\item $E = U_{n = 1}^{\infty}E_n \Rightarrow |E|^* \leq \sum_{n = 1}^{\infty}|E_n|^*$(\textbf{полу-аддитивность}).
\item $\rho(E_1, E_2) = d > 0 \Rightarrow |E_1 \cup E_2|^* = |E_1|^* + |E_2|^*$.
\item $\forall E \forall\varepsilon > 0 \exists\text{ открытое }G E \subset G |G|^* < |E|^* + \varepsilon$
\end{enumerate}
\textbf{Доказательство}:\\
1. \ldots\\
2. $\forall \varepsilon > 0 \exists \{\Delta^n_k\}_{k = 1}^{\infty} E_n \subset U_{k = 1}^{\infty}\Delta_n^k \sum_{k = 1}^{\infty}|\Delta_k^n| < |E_n|^* + \frac{\varepsilon}{2^n}$.
Тогда $|E|^* \leq \sum_{n = 1}^{\infty}\sum_{k = 1}^{\infty}|\Delta_k^n| \leq \sum_{n = 1}^{\infty}\left(|E_n|^* + \frac{\varepsilon}{2^n}\right) \leq \sum_{n = 1}^{\infty}|E_n|^* + \varepsilon$.\\
3. Для любого $\varepsilon > 0$ можно построить покрытие длины меньше $|E|^* + \frac{\varepsilon}2$, такое, что длина каждого интервала
меньше любого наперёд заданного числа. Для этого нужно "раздробить" отрезок, покрыв точки "склейки" интервалами длины $\frac{\varepsilon}4, \frac{\varepsilon}8, \ldots$, учитывая
заданную максимальную длину. Такое покрытие можно разделить на два покрытия, одно только для $E_1$,
другое для $E_2$. Переходя к пределу при $\varepsilon \to 0$, получаем искомое утверждение.
Спецкурс "Избранные главы функционального анализа" по вторникам в 16:20.
"Актуальные задачи функционального анализа и математической физики".

Множество $E$ на прямой называется \textbf{измеримым по Лебегу}(измеримым), если
$\forall \varepsilon > 0 \exists \text{ открытое множество } G: E \subset G, |G \backslash E|^* < \varepsilon: |E| = |E|^*$.
\begin{Theorem}
$|E| = 0 \Leftrightarrow |E|^* = 0$
\end{Theorem}
\begin{proof}
Если $|E| = 0$, то по определению $|E|^* = 0$.

Пусть $|E|^* = 0$. Тогда $\forall \varepsilon > 0 \exists G: E \subset G: |G|^* < |E|^* + \varepsilon$
$G \backslash E \subset G \Rightarrow |G \ E|^* \leq |G|^* < |E|^* + \varepsilon = \varepsilon$.
\end{proof}
\begin{Theorem}
Всякое открытое множество измеримо по Лебегу, и его мера равна сумме мер попарно непересекающихся
покрывающих его интервалов.
\end{Theorem}
\begin{Theorem}
Объединение конечного или счётного числа измеримых множеств есть множество измеримое.
\end{Theorem}
\begin{proof}
Из измеримости множеств следует, что $\forall \varepsilon \exists G_n \supset E_n$
$|G_n \backslash E_n|^* < \frac{\varepsilon}{2^n}$. Рассмотрим $G = \cup_{n = 1}^{\infty}G_n$. Тогда
$E \subset G, G \backslash E \subset \cup_{n = 1}^{\infty}(G_n \backslash E_n)$, откуда
\begin{equation*}
|G \backslash E|^* \leq \sum_{n = 1}^{\infty}|G_n \backslash E_n|* \leq \sum_{n = 1}^{\infty}\frac{\varepsilon}{2^n} = \varepsilon
\end{equation*}
\end{proof}
\begin{Theorem}
Любое замкнутое множество измеримо по Лебегу.
\end{Theorem}
\begin{proof}
Рассмотрим сначала случай ограниченного множества. Пусть $\Delta = (a, b)$. Введём обозначение:
\begin{equation*}
\Delta^{\alpha} = \begin{cases}
(a + \alpha, b - \alpha), \alpha < \frac{b - a}2, \alpha > 0, \\
\emptyset, \alpha \geq \frac{b - a}2.
\end{cases}
\end{equation*}
$\overline{\Delta^{\alpha}}$ - замыкание $\Delta^{\alpha}$.
\begin{equation*}
E_n = \cup_{n = 1}^{\infty}\Delta_k
\end{equation*}
\begin{equation*}
E_n^{\alpha} = \cup_{n = 1}^{\infty}\Delta_k^{\alpha}
\end{equation*}
$\overline{E_n}$ и $\overline{E_n^{\alpha}}$ - замыкания $E_n$ и $E_n^{\alpha}$ соответственно.

$G \supset F |G|^* < |F|^* + \varepsilon |G \backslash F| = \cup_{n = 1}^{\infty}\Delta_n$,
$\Delta_i \cap \Delta_j = \emptyset$ при $i \neq j$. $\overline{E_n^{\alpha}} \subset G \backslash F$,
поэтому $|\overline{E_n^{\alpha}} \cup F| = |\overline{E_n^{\alpha}}|^* + |F|^* < |G|^* < |F|^* + \varepsilon$,
откуда $|\overline{E_n^{\alpha}}|^* < \varepsilon$.

Перейдём теперь к неограниченному случаю. Положим в нём $F_n = F \cap [-n, n], F = \cup_{n = 1}^{\infty}F_n$
и перейдём в предыдущем равенстве к пределу при $n \to \infty$.
\end{proof}
\begin{Theorem}
Если $E$ измеримо, то и $CE$ измеримо.
\end{Theorem}
\begin{proof}
\begin{equation*}
\forall n \in \mathbb{N} \exists G_n |G_n \backslash E|^* < \frac1n \Rightarrow
CE \backslash CG_n = G_n \backslash E, F = \cup_{n = 1}^{\infty}F_n.
\end{equation*}
\begin{equation*}
CE \backslash F \subset CE \backslash F_n
\end{equation*}
\begin{equation*}
|CE \backslash F|^* \leq |CE \backslash F_n|^* < \frac1n \Rightarrow
|CE \backslash F|^* = 0 \Rightarrow |CE \backslash F| = 0.
\end{equation*}
\end{proof}
\begin{Theorem}[Критерий измеримости по Лебегу]
$E$ измеримо $\Leftrightarrow \forall \varepsilon > 0 \exists F\text{ - замкнутое} \subset E: |E \ F|^* < \varepsilon$.
\end{Theorem}
\begin{proof}
Следует из теоремы 6.
\end{proof}
\begin{Theorem}
Пересечение конечного и счётного числа измеримых множеств измеримо по Лебегу.
\end{Theorem}
\begin{proof}
\begin{equation*}
CE = \cup_{n = 1}^{\infty}CE_n \Rightarrow \text{ измеримо} \Rightarrow E\text{ также измеримо.}
\end{equation*}
\end{proof}
\begin{Theorem}
Для измеримых $A$ и $B$ $A \backslash B = A \cap CB$ измеримо.
\end{Theorem}
\begin{Theorem}[$\sigma$-аддитивность меры]
Если множество $E$ представимо в виде объединения не более чем счётного числа попарно
непересекающихся множеств, его мера равна сумме множеств мер объединения.
\end{Theorem}
\begin{proof}
Рассмотрим случай ограниченных множеств $E_n$. Тогда по критерию
\begin{equation*}
\forall \varepsilon \exists F_n \subset E_n |E_n \backslash F_n| < \frac{\varepsilon}{2^n}
\end{equation*}
\begin{equation*}
E_n = (E_n \backslash F_n) \cup F_n \Rightarrow |E_n| < |E_n \backslash F_n| + |F_n|
\end{equation*}
Тогда
\begin{equation*}
\sum_{k = 1}^n|E_k| < \sum_{k = 1}^n|E_k \backslash F_k| + \sum_{k = 1}^n|F_k| <
\sum_{k = 1}^n\frac{\varepsilon}{2^k} + |\cup_{k = 1}^nF_k| < |E| + \varepsilon
\end{equation*}
Переходя к пределу при $n \to \infty$, получаем, что
\begin{equation*}
\sum_{k = 1}^n|E_k| \leq |E| + \varepsilon
\end{equation*}
откуда при $\varepsilon \to 0 \sum_{k = 1}^n|E_k| \leq |E|$. Из свойства 3 внешней меры вытекает,
что $\sum_{k = 1}^n|E_k| \geq |E| \Rightarrow \sum_{k = 1}^n |E_k| = |E|$.

Перейдём теперь к неограниченному случаю. Рассмотрим множества $E_n^k = E_n \cap (-k, k + 1]$.
Тогда
\begin{equation}
E = \cup_{k = -\infty}^{+\infty}\cup_{n = 1}^{\infty}E^k_n,
\end{equation}
а для $E^k_n$ уже доказано свойство счётной аддитивности.
\end{proof}
Множество $G$ является \textbf{множеством типа $G_\delta$}, если $G = \cap_{n = 1}^{\infty}G_n$
($G_n$ - открытые). Множество $F$ называется \textbf{множеством типа $F_\sigma$}, если
$F = \cup_{n = 1}^{\infty}F_n$($F_n$ - замкнутые).
\begin{Theorem}
\begin{equation*}
\forall \text{ измеримого } E \exists G\text{ типа }G_{\delta}, F\text{ типа }F_{\sigma} \text{ такие,
что } G \supset E \supset F \text{ и } |G| = |E| = |F|.
\end{equation*}
\end{Theorem}
\begin{proof}
\begin{equation*}
\forall n \in \mathbb{N} \exists G_n \supset E \supset F_n: |G_n \backslash E| < \frac1n,
G = \cap_{n = 1}^{\infty}G_n, |E \backslash F_n| < \frac1n F = \cup_{n = 1}^{\infty}F_n
\end{equation*}
Тогда
\begin{equation*}
G \backslash E \subset G_n \backslash E \Rightarrow |G \backslash E| \leq |G_n \backslash E| < \frac1n
\Rightarrow |G \backslash E| = 0
\end{equation*}
Аналогично для $F$.
\end{proof}
Существуют неизмеримые множества. Пусть $\alpha$ - иррациональные, будем выбирать на
окружности классы точек, совместимые поворотом на $\pi n\alpha, n \in \mathbb{Z}$($\Phi_n$). Тогда
$C = \cup_{n = -\infty}^{+\infty}\Phi_n$. Но $|C| = 1$, а $\Phi_n$ конгруэтнтны, поэтому не могут
быть измеримы.

Непустая система множеств называется \textbf{кольцом}, если она замкнута относительно операций
пересечения и симметрической разности множеств. Множество кольца называется \textbf{единицей},
если $\forall A E\cap A = A$. Полукольцо.

Общий принцип построения меры Лебега на множестве: строим меру на полукольце, по аддитивности
продолжаем её на кольцо, затем с помощью аппроксимации продолжаем её на оставшуюся часть.

\textbf{Пример} -- Построение меры Лебега-Стилтьеса:
Рассмотрим функцию $F(x)$. Построим меру: $\mu([a, b]) = F(b + 0) - F(a)$, $\mu((a, b)) = F(b) - F(a + 0)$.
На остальные множества мера распространяется по аддитивности и покрытию.

Построим меру Лебега на плоскости:
Рассмотрим открытые и закрытые прямоугольники на $\mathbb{R}_2$. Мера прямоугольника -- его площадь.
\textbf{Элементарное} множество - множество, элементы которого получаются объединением
прямоугольников. Мера вводится как сумма мер множеств объединения. На элементарных множествах
с помощью леммы Гейне-Бореля можно показать счётную аддитивность меры. Вводим внешнюю меру
множества как точную нижнюю грань мер элементарных множеств, покрывающих данное.

\textbf{Абстрактная мера} - положительная действительнозначная функция на полукольце,
обладающая свойством конечной аддитивности.
\section{Измеримые функции}
\label{sec:orgcc45e06}
\zall
Будем использовать обозначение \(E[f > a] = \{x \in E | f(x) > a\}\).

Функция \(f\), определённая на измеримом множестве \(E\), называется \textbf{измеримой} по Лебегу на нём, если
\(\forall a \in \mathbb{R} E[f(x) \geq a]\) измеримо.

Свойства измеримых функций:
\begin{Statement}
\begin{enumerate}
\item Функция $f$ измерима тогда и только тогда, когда измеримо одно из множеств $E[f > a], E[f \leq a], E[f < a]$.
\item Если $E_1 \subset E$ и $f$ измерима на $E$, то $f$ измерима на $E_1$.
\item Если $f$ измерима на $E_1, E_2, \ldots$, то $f$ измерима на $E = \cup_{n = 1}^{\infty}E_n$.
\item Любая функция измерима на множестве меры ноль.
\item Если функция $f$ измерима на $E$ и $f \sim g$, то $g$ измерима на $E$.
\item Если $f(x)$ почти всюду непрерывна на $E$, то она измерима на $E$.
\end{enumerate}
\end{Statement}
\begin{proof}
\begin{equation}
E[f > a] = \cup_{n = 1}^{\infty}E[f \geq a + \frac1n],
\end{equation}
\begin{equation}
E[f \geq a] = \cap_{n = 1}^{\infty}E[f > a - \frac1n],
\end{equation}
\begin{equation}
E[f \leq a] = E \backslash E[f > a]
\end{equation}
\begin{equation}
E[f < a] = E \backslash E[f \geq a]
\end{equation}
\begin{equation}
E_1[f \geq a] = E_1 \cap E[f \geq a]
\end{equation}
\begin{equation}
E[f \geq a] = \cup_{n = 1}^{\infty}E_n[f \geq a]
\end{equation}
Заметим, что если множество $F$ замкнуто, то $F[f \geq a]$ также замкнуто. Представим $E$ в
виде:
\begin{equation}
E = E_1 \cup E_2 \cup E_3,
\end{equation}
где $E_1$ - множество точек разрыва функции $f$, $E_2$ - множество типа $F_{\sigma}$, $|E_1| = |E_3| = 0$.
\end{proof}
\textbf{Эквивалентные функции}:
$f \sim g$, если $|E[f \neq g]| = 0$.
Будем говорить, что свойство выполнено \textbf{почти всюду}, если оно выполнено на всём множестве,
кроме может быть подмножества точек меры ноль.
\begin{Theorem}
Пусть $f$ - измеримая функция. Тогда функции $|f|$, $cf$ и $f + c$ также измеримы. Если $g$
измерима, то множество $E[f > g]$ также измеримо.
\end{Theorem}
\begin{proof}
\begin{equation}
E[|f| \geq a] = \begin{cases}
E[f \geq a] \cup E[f \leq -a], a > 0, \\
E, a \leq 0.
\end{cases}
\end{equation}
\begin{equation}
E[cf \geq a] = \begin{cases}
E[f \geq \frac{a}{c}], c > 0, \\
E[f \leq \frac{a}{c}], c < 0, \\
E, c = 0, a \geq 0, \\
\emptyset, c = 0, a < 0.
\end{cases}
\end{equation}
Для доказательства последнего пункта пронумеруем все точки. Тогда
\begin{equation}
E[f > g] = \cup_{k = 1}^{\infty}E[f > r_k] \cap E[g < r_k]
\end{equation}
\end{proof}
\begin{Theorem}[Арифметические операции над измеримыми функциями]
Пусть $E$ -- измеримое множество, $f$ и $g$ -- измеримые функции. Тогда $f \pm g$, $f\cdot g$,
$\frac{f}g(g \neq 0)$ -- измеримые функции.
\end{Theorem}
\begin{proof}
\begin{equation}
E[f \pm g \geq a] = E[f \geq \mp g + a]
\end{equation}
\begin{equation}
fg = \frac{(f + g)^2}4 - \frac{(f - g)^2}4
\end{equation}
\begin{equation}
E[\frac1g > a] = \begin{cases}
E[g > 0] \cap E[g < \frac1a], a > 0, \\
E[g > 0], a = 0, \\
E[g > 0] \cup E[g < \frac1a], a < 0.
\end{cases}
\end{equation}
\end{proof}
\begin{Theorem}
Пусть $f_1, f_2, \ldots, f_n, \ldots$ -- измеримые функции. Тогда
$\underline{f}(x) = \underline{\lim}_{n \to \infty}f_n(x)$ и
$\overline{f}(x) = \overline{\lim}_{n \to \infty}f_n(x)$ -- измеримые функции.
\end{Theorem}
\begin{proof}
Функции $\phi(x) = \inf_nf_n(x)$ и $\psi(x) = \sup_nf_n(x)$ являются измеримыми, что видно из
следующих соотношений:
\begin{equation}
E[\phi > a] = \cup_{n = 1}^{\infty}E[f_n < a],
\end{equation}
\begin{equation}
E[\psi > a] = \cup_{n = 1}^{\infty}E[f_n > a],
\end{equation}
Остаётся заметить, что $\underline{f}(x) = \sup_{n \geq 1}\inf_{k \geq n}f_k(x)$ и
$\overline{f}(x) = \inf_{n \geq 1}\sup_{k \geq n}f_k(x)$.
\end{proof}
\begin{Theorem}
Пусть $f_1(x), \ldots$ -- последовательность измеримых функций, почти всюду сходящаяся к $f(x)$.
Тогда $f(x)$ -- измеримая функция.
\end{Theorem}
\begin{proof}
Множество $E$ разбивается на две части -- множество сходимости, на котором есть верхний и нижний
пределы, равные $f(x)$ и остаток меры нуль.
\end{proof}
Будем говорить, что последовательность \(f_1(x), \ldots, f_n(x), \ldots\) почти всюду
ограниченных измеримых функций \textbf{сходится по мере} к почти всюду ограниченной функции \(f(x)\),
если \(\lim_{n \to \infty}|E[|f_n - f| \geq \varepsilon]| = 0\), т. е.
\(\forall \varepsilon > 0 \delta > 0 \exists N = N(\varepsilon, \delta) \forall n \geq N: |E[|f_n - f| \geq \varepsilon]| < \delta\).
\begin{Theorem}
Пусть $|E| < +\infty$ и $f_n(x)$ сходится почти всюду к $f(x)$. Тогда $f_n(x)$ сходится к
$f(x)$ по мере.
\end{Theorem}
\begin{proof}
Фиксируем $\varepsilon > 0$. Положим $E_n = E[|f_n - f| \geq \varepsilon]$, $R_n = \cup_{k = 1}^{\infty}E_k$.
Поскольку $|E_n| \geq |R_n|$, достаточно показать, что $|R_n| \to 0$ при $n \to \infty$.
Введём множества $A_0 = E[|f| = +\infty]$, $A_n = E[|f_n| = +\infty]$, $A = \cup_{n = 0}^{\infty}A_n$,
$B = E \backslash E[\lim_{n \to \infty}f_n(x) = f(x)]$, $C = A \cup B$, $|C| = 0$, $R = \cap_{n = 1}^{\infty}$.
\begin{equation}
\cup_{k = n}^{\infty}(R_k \backslash R_{k + 1}) = R_n \backslash R
\end{equation}
\begin{equation}
|R_n \backslash R| = \sum_{k = n}^{\infty}|R_k \backslash R_{k + 1}| \Rightarrow |R_n \backslash R|
\to 0 \Rightarrow |R_n| \to |R|.
\end{equation}
Покажем, что $R \subset C$. Возьмём $x_0 \notin C$. Тогда $\lim_{n \to \infty}f_n(x_0) = f(x_0)$,
т. е. $\forall \varepsilon > 0 \exists N = N(\varepsilon, x_0) \forall n \geq N |f_n(x_0) - f(x_0)| < \varepsilon$,
соответственно, $x_0 \notin R_n$ и $x_0 \notin E_n \Rightarrow x_0 \notin R$.
\end{proof}
Заметим, что в общем случае из сходимости почти всюду сходимость по мере не следует. Рассмотрим
функцию:
\begin{equation}
f_n(x) = \begin{cases}
1, x \in [n, n + 1],\\
0, x \notin [n, n + 1],
\end{cases}
\end{equation}
на $E = R$. Эта последовательность сходится к $f(x) \equiv 1$ всюду, но не сходится по мере.

Из сходимости по мере не следует сходимость даже в какой-то одной точке.
\begin{Theorem}
Пусть $|E| < +\infty$ и $f_n(x)$ по мере сходится к $f(x)$. Тогда $\exists\{f_{n_k}(x)\}$,
$f_{n_k}(x) \to f(x)$ почти всюду.
\end{Theorem}
\begin{proof}
Введём множества($k \in \mathbb{N}$):
\begin{equation}
E_k = E[|f_{n_k} - f| \geq \frac1k], |E| < \frac1{2^k}
\end{equation}
\begin{equation}
R_n = \cup_{k = n}^{\infty}E_k, |R_n| \leq \sum_{k = n}|E_k| < \frac1{2^{k - 1}}
\end{equation}
Тогда $|R_n| \to 0$ и $|R| = 0$, так как $|R_n| \to |R|$.

$\forall x_0 \notin R \exists N = N(x_0) x_0 \notin R_N \Rightarrow x_0 \notin E_k, k \geq N$,
т. е. $|f_{n_k}(x_0) - f(x_0)| < \frac{1}k$.
\end{proof}
\begin{Theorem}[Первая теорема 7]
Пусть $f_n \to f$ и $f_n \to g$ по мере. Тогда $f \sim g$.
\end{Theorem}
\begin{proof}
\begin{equation}
\forall \varepsilon > 0 E[|f - g| \geq \varepsilon] \subset E\left[|f_n - f| \geq \frac{\varepsilon}2\right]
\cup E\left[|f_n - g| \geq \frac{\varepsilon}2\right]
\end{equation}
\begin{equation}
E[f \neq g] \subset \cup_{n = 1}^{\infty}E\left[|f - g| \geq \frac1n\right]
\end{equation}
\end{proof}
\begin{Theorem}[Теорема Егорова]
Пусть $|E| < +\infty$ и $f_n \to f$ почти всюду, все функции почти всюду конечны и измеримы.
Тогда $\forall\delta > 0 \exists E_{\delta} \subset E$ и $|E_{\delta}| > |E| - \delta$
$f_n(x) \rightrightarrows f(x)$ на $E_{\delta}$.
\end{Theorem}
\begin{Theorem}[Теорема Лузина]
Пусть
\begin{equation}
\forall\varepsilon > 0 \exists \varphi(x) \in C(E) |E[f \neq \varphi]| < \varepsilon
\end{equation}
Если $|f| \leq k$, то $|\varphi| \leq k$.
\end{Theorem}
\section{Интеграл Лебега}
\label{sec:orgeb1302c}
\subsection{Интеграл Лебега от ограниченной функции на множестве конечной меры}
\label{sec:org1937225}
Пусть \(|E| < +\infty\). \textbf{Лебеговым разбиением} множества \(E\) назовём совокупность конечного
числа непересекающихся измеримых множеств, объединение которых даёт \(E\):
\begin{equation}
T = \{E_k\}|_{k = 1}^n, \cup_{k = 1}E_k = E, E_i \cap E_j = \emptyset, \forall i, j: i \neq j
\end{equation}
\textbf{Верхней} и \textbf{нижней} интегральной суммой назовём суммы:
\begin{equation}
S = \sum_{k = 1}^nM_k|E_k|,
\end{equation}
\begin{equation}
s = \sum_{k = 1}^nm_k|E_k|,
\end{equation}
где
\begin{equation}
M_k = \sup_{E_k}f(x), m_k = \inf_{E_k}f(x)
\end{equation}
\textbf{Верхним} и \textbf{нижним} интегралами Лебега называются пределы:
\begin{equation}
\overline{I} = \inf_TS_t, \underline{I} = \sup_{T}s_T
\end{equation}
\textbf{Интегралом} Лебега называется значение \(I = \overline{I} = \underline{I} = \int_Ef(x)dx\),
если эти множества совпадают. В этом случае функция \(f(x)\) называется \textbf{интегрируемой по Лебегу}
на \(E\).

Разбиение \(T^*\) называется \textbf{измельчением} разбиения \(T\), если:
1. $1 \leq i \leq m \exists \nu(i): 1 \leq \nu(i) \leq n: E_i^* \subset E_{\nu(i)}$
2. $\cup_{\nu(i) = k}E^*_i = E_k$

\begin{Statement}
Если $T^*$ - измельчение $T$, то
\begin{equation}
S_{T^*} \leq S_T, s_{T^*} \geq S_T.
\end{equation}
\end{Statement}
\begin{proof}
Докажем только первое неравенство, второе аналогично.
\begin{equation}
S_{T^*} = \sum_{i = 1}^mM_i^*|E_i^*| = \sum_{k = 1}^n\sum_{\nu(i) = k}M_i^*|E_i^*| \leq
\sum_{k = 1}^n\sum_{\nu(i) = k}M_k|E_i^*| = \sum_{k = 1}M_k|E_k| = S_T
\end{equation}
\end{proof}
Разбиение $\hat{T} = T_1\times T_2$ будем называть \textbf{произведением} разбиений $T_1$ и
$T_2$, если оно состоит из всевозможных пересечений множеств из $T_1$ с множествами из $T_2$.
\begin{Consequence}
\begin{equation}
\forall T_1, T_2 s_{T_1} \leq s_{\hat{T}} \leq S_{\hat{T}} \leq S_{T_2}
\end{equation}
\end{Consequence}
В частности, из этого следует, что $\underline{I} \leq \overline{I}$.
\begin{Theorem}
Если функция интегрируема по Риману на $[a, b]$, то она интегрируема по Лебегу на $[a, b]$.
\end{Theorem}
\begin{proof}
Разбиение по Риману есть частный случай разбиения по Лебегу, поэтому:
\begin{equation}
\underline{I}_R \leq \underline{I}_L \leq \overline{I}_L \leq \overline{I}_R.
\end{equation}
Поскольку функция интегрируема по Риману, $\underline{I}_R = \overline{I}_R = I \Rightarrow$
$\underline{I}_L = \overline{I}_L = I$.
\end{proof}
Пример функции, интегрируемой по Лебегу, но не интегрируемой по Риману - функция Дирихле:
$$E = [0, 1]$$
\begin{equation}
f(x) = \begin{cases}
1, x \in \mathbb{R} \backslash \mathbb{Q}, \\
0, x \in \mathbb{Q}
\end{cases}
\end{equation}
Пусть $E_1 = E \cap Q, E_2 = E \backslash E_1$. Рассмотрим $T: {E_1, E_2}$. Тогда
\begin{equation}
s_t = 0|E_1| + 1|E_2| = 1 = S_T \Rightarrow I_L = 1
\end{equation}
\begin{Theorem}[Теорема Лебега]
Пусть ограниченная функция $f(x)$ измерима на множестве конечной меры. Тогда эта функция
интегрируема на нём.
\end{Theorem}
\begin{proof}
\begin{equation}
m \leq f(x) \leq M
\end{equation}
Составим лебегово разбиение $m = y_0 < y_1 < \ldots < y_n = M$. Введём множества
$E_0 = E[y_0 \leq f(x) \leq y_1], E_k = E[y_{k - 1} < f(x) \leq y_k], k = \overline{2, n}$.
На них $s_T = \sum_{k = 1}^nM_k|E_k|, S_T = \sum_{k = 1}^nm_k|E_k|, y_{k = 1} \leq m_k \leq M_k \leq y_k$.
Тогда
\begin{equation}
\sum_{k = 1}^ny_{k - 1}|E_k| \leq s_T \leq S_T \leq \sum_{k = 1}^ny_k|E_k|,
\end{equation}
соответственно
\begin{equation}
0 \leq S_T - s_t \leq \sum_{k = 1}^n(y_k - y_{k - 1})|E_k| \leq \delta|E|,
\end{equation}
где $\delta = \max_{1 \leq k \leq n}(y_k - y_{k - 1}) \Rightarrow 0 \leq$
$0 \leq \overline{I} - \underline{I} \leq S_T - s_T$.
\end{proof}
Спецкурс в четверг в 16:20.
\subsection{Свойства интеграла Лебега}
\label{sec:orgeed9315}
\begin{enumerate}
\item \(\int_Edx = |E|\).
\item \(\int_E\alpha fdx = \alpha\int_Efdx, \alpha = const\).
\item \(\int_E(f_1 + f_2)dx = \int_Ef_1dx + \int_Ef_2dx\).
\item Если \(E_1 \cap E_2 = \emptyset, E = E_1 \cup E_2\), то \(\int_Efdx = \int_{E_1}fdx + \int_{E_2}fdx\).
\item Если \(f_1 \leq f_2\) почти всюду, то \(\int_Ef_1dx \leq \int_Ef_2dx\).
\end{enumerate}
\subsection{Интеграл Лебега от измеримой(вообще говоря, неограниченной), неотрицательной функции на множестве конечной меры}
\label{sec:org2beb85e}
Введём \textbf{срез} функции \(f(x)\):
\zall
\begin{equation}
f_N(x) = \begin{cases}
f(x), f(x) \leq N, \\
N, f(x) > N.
\end{cases}
\end{equation}
Тогда
\begin{equation}
E[f_N(x) > a] = \begin{cases}
E[f(x) > a] = E[f(x) > a], a < N, \\
\emptyset, a \geq N
\end{cases}
\end{equation}
По теореме Лебега $\exists I_N = \int_Ef_N(x)dx$. Если
$\exists I = \lim_{N \to +\infty}I_N = \int_Ef(x)dx$, то функция $f(x)$ называется \textbf{интегрируемой
по Лебегу}. $E_{\infty} = E[f = +\infty]$. Для интегрируемых функций $|E_{\infty}| = 0$.
\begin{Theorem}
Пусть $E = \cup_{n = 1}^{\infty}E_k, E_i \cap E_j = \emptyset, i \neq j$. Тогда:
\begin{equation}
\int_Efdx = \sum_{n = 1}^{\infty}\int_{E_k}fdx
\end{equation}
Причём если $f$ интегрируема на $E$, то $f$ интегрируема на $E_k$ и выполнено (3). Кроме того,
если $f$ интегрируема на каждом $E_k$ и ряд в правой части (3) сходится, то $f$ интегрируема
на $E$ и выполнено (3)
\end{Theorem}
\begin{proof}
Пусть $0 \leq f \leq M$. Тогда
\begin{equation}
\int_Efdx - \sum_{k = 1}^N\int_{E_k}fdx = \int_{\cup_{n = N + 1}^{\infty}}fdx \leq M|\cup_{n = N + 1}^{\infty}E_n|
= M\sum_{n = N + 1}^{\infty}|E_n| \to 0
\end{equation}

Пусть теперь $f$ не ограничена. Тогда
\begin{equation}
\int_{E_k}f_Ndx \leq \int_Ef_Ndx = I_N \leq \int_Efdx \Rightarrow \int_{E_k}fdx \leq \int_Efdx
\end{equation}
В силу первой части для среза выполнено равенство
\begin{equation}
\int_Ef_Ndx = \sum_{n = 1}^{\infty}\int_{E_n}fdx \leq \sum_{n = 1}^{\infty}\int_{E_n}fdx
\end{equation}
С другой стороны,
\begin{equation}
\int_Ef_Ndx \geq \sum_{k = 1}^m\int_{E_k}f_Ndx
\end{equation}
при переходе к пределу при $N \to +\infty$:
\begin{equation}
\int_Efdx \geq \sum_{k = 1}^{\infty}\int_Efdx
\end{equation}
Из (6) и (8) получаем (3).
\end{proof}
\end{document}
