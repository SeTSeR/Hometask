% Created 2019-05-26 Sun 20:02
% Intended LaTeX compiler: pdflatex
\documentclass[11pt]{article}
\usepackage[utf8]{inputenc}
\usepackage[T1]{fontenc}
\usepackage{graphicx}
\usepackage{grffile}
\usepackage{longtable}
\usepackage{wrapfig}
\usepackage{rotating}
\usepackage[normalem]{ulem}
\usepackage{amsmath}
\usepackage{textcomp}
\usepackage{amssymb}
\usepackage{capt-of}
\usepackage{hyperref}
\usepackage{amsmath}
\usepackage{mathtools}
\usepackage[english, russian]{babel}
\author{Sergey Makarov}
\date{\today}
\title{}
\hypersetup{
 pdfauthor={Sergey Makarov},
 pdftitle={},
 pdfkeywords={},
 pdfsubject={},
 pdfcreator={Emacs 26.2 (Org mode 9.1.9)}, 
 pdflang={English}}
\begin{document}

\section{Задача 2}
\label{sec:orgc4526dc}
Вычислить интеграл:
$$\int_0^\infty\frac{e^{-\alpha x^2} - e^{-2\alpha x^2}}xdx = I(\alpha), \alpha > 0$$

Продифференцируем интеграл по параметру \(\alpha\):
\begin{multline*}
I'(\alpha) = \int_0^\infty\frac{-x^2e^{-\alpha x^2} + 2x^2e^{-2\alpha x^2}}xdx =
\int_0^\infty(-xe^{-\alpha x^2} + 2xe^{-2\alpha x^2})dx = \\
= \frac1{2\alpha}e^{-\alpha x^2}\bigg|_0^\infty - \frac1{2\alpha}{e^{-2\alpha x^2}\bigg|_0^\infty} = 0
\end{multline*}
В силу признака Вейерштрасса полученный интеграл сходится равномерно на любой полупрямой
\([\alpha_0; +\infty)\), поэтому дифференцирование правомерно.

Отсюда находим \(I(\alpha)\):
$$I(\alpha) = C$$
Найдём \(C\):
\begin{multline*}
C = I(1) = \int_0^\infty\frac{e^{-x^2} - e^{-2x^2}}xdx =\bigg|_{x = \sqrt t}\int_0^\infty\frac{e^{-t} - e^{-2t}}{\sqrt t}d(\sqrt t) = \\
= -\frac12\int_0^\infty\frac{e^{-t} - e^{-2t}}tdt = -\frac12\ln2 = -\frac{\ln2}2
\end{multline*}
Поскольку \(f(t) = e^{-t} \in C[0; \infty)\) и \(\forall A > 0 \exists \int_A^\infty\frac{f(t)}tdt\), то справедлива формула Фруллани:
$$\int_0^\infty\frac{f(at) - f(bt)}tdt = f(0)\ln\left(\frac ba\right)$$
Окончательно получаем:
$$\int_0^\infty\frac{e^{-\alpha x^2} - e^{-2\alpha x^2}}xdx = -\frac{\ln2}2, \forall \alpha > 0$$
\end{document}
