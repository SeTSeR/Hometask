% Created 2020-04-30 Thu 10:17
% Intended LaTeX compiler: pdflatex
\documentclass[11pt]{article}
\usepackage[utf8]{inputenc}
\usepackage[T1]{fontenc}
\usepackage{graphicx}
\usepackage{grffile}
\usepackage{longtable}
\usepackage{wrapfig}
\usepackage{rotating}
\usepackage[normalem]{ulem}
\usepackage{amsmath}
\usepackage{textcomp}
\usepackage{amssymb}
\usepackage{capt-of}
\usepackage{hyperref}
\usepackage{minted}
\usepackage{amsmath}
\usepackage{esint}
\usepackage[english, russian]{babel}
\usepackage{mathtools}
\usepackage{amsthm}
\usepackage[top=0.8in, bottom=0.75in, left=0.625in, right=0.625in]{geometry}
\newcommand{\grad}{\operatorname{grad}}
\def\zall{\setcounter{lem}{0}\setcounter{cnsqnc}{0}\setcounter{th}{0}\setcounter{Cmt}{0}\setcounter{equation}{0}\setcounter{stnmt}{0}}
\newcounter{lem}\setcounter{lem}{0}
\def\lm{\par\smallskip\refstepcounter{lem}\textbf{\arabic{lem}}}
\newtheorem*{Lemma}{Лемма \lm}
\newcounter{th}\setcounter{th}{0}
\def\th{\par\smallskip\refstepcounter{th}\textbf{\arabic{th}}}
\newtheorem*{Theorem}{Теорема \th}
\newcounter{cnsqnc}\setcounter{cnsqnc}{0}
\def\cnsqnc{\par\smallskip\refstepcounter{cnsqnc}\textbf{\arabic{cnsqnc}}}
\newtheorem*{Consequence}{Следствие \cnsqnc}
\newcounter{Cmt}\setcounter{Cmt}{0}
\def\cmt{\par\smallskip\refstepcounter{Cmt}\textbf{\arabic{Cmt}}}
\newtheorem*{Note}{Замечание \cmt}
\newcounter{stnmt}\setcounter{stnmt}{0}
\def\st{\par\smallskip\refstepcounter{stnmt}\textbf{\arabic{stnmt}}}
\newtheorem*{Statement}{Утверждение \st}
\author{Sergey Makarov}
\date{\today}
\title{}
\hypersetup{
 pdfauthor={Sergey Makarov},
 pdftitle={},
 pdfkeywords={},
 pdfsubject={},
 pdfcreator={Emacs 26.3 (Org mode 9.3)}, 
 pdflang={Russian}}
\begin{document}


\section{Задача 1}
\label{sec:orgd3ef870}
Построить двойственную к задаче линейного программирования:
\begin{equation}
\max_{-1 \leq x \leq 1} (x + 1)
\end{equation}
Найти значения оптимума для прямой и двойственной задачи, проверить, что они совпадают.
\subsection{Решение}
\label{sec:org20095c9}
Запишем задачу в форме озЛП:
\begin{equation}
(\max_{Ax \leq b}(c, x)) + 1,
\end{equation}
где
\begin{equation}
\begin{cases}
A = \begin{pmatrix}
1 \\
-1
\end{pmatrix}\\
b = \begin{pmatrix}
1 \\
1
\end{pmatrix}\\
c = 1
\end{cases}
\end{equation}
Соответственно, двойственная задача имеет вид:
\begin{equation}
(\min_{\lambda A = c, \lambda \geq 0}(\lambda, b)) + 1,
\end{equation}
или
\begin{equation}
\min_{\lambda_1 - \lambda_2 = 1, \lambda_1 \geq 0, \lambda_2 \geq 0}(\lambda_1 + \lambda_2) + 1 = 
\min_{\lambda_1 \geq 1}(2\lambda_1 - 1) + 1 = 2\min_{\lambda_1 \geq 1}\lambda_1
\end{equation}

Максимум для прямой задачи достигается при $x = 1$, значение этого максимума равно 2.
Для двойственной задачи оптимум достигается при $\lambda_1 = 2, \lambda_2 = 0$ и равен так же 2.
\end{document}
