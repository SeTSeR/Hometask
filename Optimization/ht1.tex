% Created 2020-04-24 Fri 17:32
% Intended LaTeX compiler: pdflatex
\documentclass[11pt]{article}
\usepackage[utf8]{inputenc}
\usepackage[T1]{fontenc}
\usepackage{graphicx}
\usepackage{grffile}
\usepackage{longtable}
\usepackage{wrapfig}
\usepackage{rotating}
\usepackage[normalem]{ulem}
\usepackage{amsmath}
\usepackage{textcomp}
\usepackage{amssymb}
\usepackage{capt-of}
\usepackage{hyperref}
\usepackage{minted}
\usepackage{amsmath}
\usepackage{esint}
\usepackage[english, russian]{babel}
\usepackage{mathtools}
\usepackage{amsthm}
\usepackage[top=0.8in, bottom=0.75in, left=0.625in, right=0.625in]{geometry}
\newcommand{\grad}{\operatorname{grad}}
\def\zall{\setcounter{lem}{0}\setcounter{cnsqnc}{0}\setcounter{th}{0}\setcounter{Cmt}{0}\setcounter{equation}{0}\setcounter{stnmt}{0}}
\newcounter{lem}\setcounter{lem}{0}
\def\lm{\par\smallskip\refstepcounter{lem}\textbf{\arabic{lem}}}
\newtheorem*{Lemma}{Лемма \lm}
\newcounter{th}\setcounter{th}{0}
\def\th{\par\smallskip\refstepcounter{th}\textbf{\arabic{th}}}
\newtheorem*{Theorem}{Теорема \th}
\newcounter{cnsqnc}\setcounter{cnsqnc}{0}
\def\cnsqnc{\par\smallskip\refstepcounter{cnsqnc}\textbf{\arabic{cnsqnc}}}
\newtheorem*{Consequence}{Следствие \cnsqnc}
\newcounter{Cmt}\setcounter{Cmt}{0}
\def\cmt{\par\smallskip\refstepcounter{Cmt}\textbf{\arabic{Cmt}}}
\newtheorem*{Note}{Замечание \cmt}
\newcounter{stnmt}\setcounter{stnmt}{0}
\def\st{\par\smallskip\refstepcounter{stnmt}\textbf{\arabic{stnmt}}}
\newtheorem*{Statement}{Утверждение \st}
\author{Sergey Makarov}
\date{\today}
\title{}
\hypersetup{
 pdfauthor={Sergey Makarov},
 pdftitle={},
 pdfkeywords={},
 pdfsubject={},
 pdfcreator={Emacs 26.3 (Org mode 9.3)}, 
 pdflang={Russian}}
\begin{document}


\section{Домашняя работа}
\label{sec:orgca0c065}
Выписать три шага метода Ньютона с постоянным шагом для задачи поиска минимума функции
\begin{equation}
f(x_1, x_2) = x_1^2 + 2x_2^2
\end{equation}
и начальном приближении $(x_1, x_2) = (-1, 1)$ и шаге $a = 1$. Найти максимальный шаг,
при котором метод градиентного спуска сойдётся.
\subsection{Решение}
\label{sec:org2359f54}
По формуле Тейлора:
\begin{equation*}
f(M) - f(M_0) = (\grad f(M_0), M - M_0) + \frac12(H(f)(M - M_0), M - M_0) + o(||M - M_0||^2),
\end{equation*}
где
\begin{equation*}
H(f) = \begin{bmatrix}
\dfrac{\partial^2 f}{\partial x_1^2} & \dfrac{\partial^2 f}{\partial x_1\partial x_2} \\
\dfrac{\partial^2 f}{\partial x_1\partial x_2} & \dfrac{\partial^2 f}{\partial x_2^2}.
\end{bmatrix}
\end{equation*}
Выберем точку $M_1$ так, чтобы минимизировать квадратичную часть разницы $f(M_1) - f(M_0)$:
\begin{equation}
(\grad f(M_0) - H(f)(M_1 - M_0), M_1 - M_0) \rightarrow \min \Rightarrow
M_1 = M_0 - (H(f))^{-1}\grad f(M_0)
\end{equation}

Выпишем градиент и гессиан функции (1):
\begin{equation*}
\grad f = (2x_1, 4x_2)^T,
H(f) = \begin{bmatrix}
2 & 0 \\
0 & 4
\end{bmatrix}
\Rightarrow H^{-1}(f) =
\begin{bmatrix}
\frac12 & 0 \\
0 & \frac14
\end{bmatrix}
\end{equation*}

Первые приближения метода Ньютона:
\begin{equation*}
M_0 = (-1, 1)^T, M_1 = M_0 - (H(f))^{-1}\grad f(M_0) = (-1, 1)^T - \begin{bmatrix}
\frac12 & 0 \\
0 & \frac14
\end{bmatrix}
\cdot\begin{bmatrix}
-2 \\
4
\end{bmatrix} = \begin{bmatrix}
-1 \\
1
\end{bmatrix} - \begin{bmatrix}
-1 \\
1
\end{bmatrix} = \begin{bmatrix}
0 \\
0
\end{bmatrix}
\end{equation*}
Поскольку градиент функции в точке $M_1$ равен нулю, считать дальнейшие приближения нет смысла --
они все будут равны нулю, т. е. метод сошёлся за один шаг.

Пусть $M_n = (x_1, x_2)$, тогда $M_{n + 1} = (x_3, x_4)$, где:
\begin{equation}
x_3 = x_1 - 2ax_1 = (1 - 2a)x_1, x_4 = x_2 - 4ax_2 = (1 - 4a)x_2
\end{equation}
Таким образом, последовательности координат образуют геометрическую прогрессию со знаменателем
$(1 - 2a)$ для первой координаты и $(1 - 4a)$ для второй. Соответственно, последовательность
первых координат сходится при $0 < a < 1$, а последовательность вторых -- при $0 < a < \frac12$.
Таким образом, последовательность точек, получаемая методом градиентного спуска, сходится при
$0 < a < \frac12$. При граничном значении $a = \frac12$ последовательность расходится.
\end{document}
