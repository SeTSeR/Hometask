% Created 2019-05-14 Tue 11:22
% Intended LaTeX compiler: pdflatex
\documentclass[11pt]{article}
\usepackage[utf8]{inputenc}
\usepackage[T1]{fontenc}
\usepackage{graphicx}
\usepackage{grffile}
\usepackage{longtable}
\usepackage{wrapfig}
\usepackage{rotating}
\usepackage[normalem]{ulem}
\usepackage{amsmath}
\usepackage{textcomp}
\usepackage{amssymb}
\usepackage{capt-of}
\usepackage{hyperref}
\usepackage{amsmath}
\usepackage[english, russian]{babel}
\author{Sergey Makarov}
\date{\today}
\title{}
\hypersetup{
 pdfauthor={Sergey Makarov},
 pdftitle={},
 pdfkeywords={},
 pdfsubject={},
 pdfcreator={Emacs 26.2 (Org mode 9.1.9)}, 
 pdflang={English}}
\begin{document}

\section{Задача 4.11}
\label{sec:org297c2a1}
$$X_1 = -0.114$$
$$X_2 = 0.196$$
$$H_0: X_i \sim R[-0.5, 0.5]$$
$$H_1: X_i \sim N(0, 0,009)$$
$$\alpha = 0.1$$
Критерий - ?

$$L_0 = \frac{1}1\mathbb{1}(\min X_i \geq -0.5)\mathbb{1}(\max X_i \leq 0.5)$$
$$L_1 = \frac{1}{2\pi 0.009}\exp\left(-\frac{1}{2(0.009)^2(X_1^2+X_2^2)}\right)$$
\begin{equation*}
\phi^*(\mathbb{X}) = \begin{cases}
1, \frac{L_1}{L_0} \geq 0.5\\
0, \frac{L_1}{L_0} \leq 0.5
\end{cases} \Leftrightarrow \phi^*(\mathbb{X}) = \begin{cases}
1, \max |X_i| > 0.5\\
0, \max |X_i| \leq 0.5
\end{cases}
\end{equation*}
$$\mathbb{E}_{H_0} = 1\mathbb{P}{(X_1^2 + X_2^2 \leq C_\alpha)} = \alpha
\Rightarrow C_\alpha = \frac{\alpha}{\pi} \approx 0.0031$$
$$X_1^2 + X_2^2 = 0.0514 > 0.031$$
Таким образом, гипотезу нужно принять.
\section{Задача 2}
\label{sec:org19a60f0}
Найти характеристическую функцию распределения Коши:
$$\xi \sim \rho(x) = \frac{1}{\pi(1 + x^2)}$$

\begin{multline*}
\varphi(t, \xi) = \mathbb{E}e^{it\xi} = \int_{-\infty}^\infty\frac{1}{\pi(1+x^2)}e^{itx}dx = \\
 = \lim_{R \to \infty}\oint_{\gamma_R}\frac{1}{\pi(1+z^2)}e^{itz}dz = 2\pi i\, res_{z=i} f(z) = \\
 = 2\pi i \lim_{z=i}\frac{1}{\pi(1+i)}e^{itz} = e^{-t}
\end{multline*}
С учётом чётности характеристической функции получим $$\varphi(t, \xi) = e^{-|t|}$$

Пусть теперь \(\xi_1, \ldots, \xi_n \sim \rho(x)\). Найти распределение \(\frac{\sum_{i = 1}^n\xi}{n}\).
\begin{multline*}
\varphi(t, \frac{\xi_1 + \ldots + \xi_n}{n}) = \mathbb{E}\exp{it\frac{\sum_{i = 1}^n\xi_n}{n}} = \ldots = e^{-|t|}
\end{multline*}
\section{Задача 3}
\label{sec:orga061e39}
Есть выборка из распределения Рэлея:
$$F(x, \theta) = 1 - e^{-\left(\frac{x}{\sqrt\theta}\right)^2}, x \geq -, \theta > 0$$
Найти оценку максимального правдоподобия и её свойства.
Найдём распределение \(X_i\):
\begin{latex}
\begin{equation*}
\begin{cases}
0, x < 0, \\
2e^{-\frac{x^2}{\theta}}\frac{x}\theta, x \geq 0
\end{cases}
\end{equation*}
\end{latex}
Функция правдоподобия:
$$L(x, \theta) = \prod_{i = 1}^n2e^{-\frac{X_i^2}{\theta}}\frac{X_i}{\theta}
= 2^n\prod_{i = 1}^n{X_i}\frac{e^{-\frac{1}\theta\sum_{i = 1}^n{X_i^2}}}{\theta^n}$$
Откуда получаем, что \(T(\mathbb{X}) = (\sum_{i = 1}^nX_i)^2\) - достаточная оценка. По критерию факторизации, она
является полной. Оценка максимального правдоподобия - \(\theta = \frac{\sum_{i = 1}^nX_i^2}n\)
Несмещённость оценки:
$$\mathbb{E}(\theta) = \frac{1}n\sum_{i = 1}^n\mathbb{E}X_i^2 = \mathbb{E}X_i^2 = \ldots = \theta$$
Из доказанного и закона больших чисел следует, что оценка также является состоятельной. Оценка является
эффективной как функция достаточной статистики.
\section{Задача 4}
\label{sec:orgeea67d8}
Дана выборка из логистического распределения:
$$\rho(x, \theta) = \frac{e^{-x + \theta}}{(1 + e^{-x + \theta})^2}$$
Плотность распределения симметрична относительно \(\theta\), поэтому \(\mathbb{E}X_i = \theta\), что и означает
несмещённость и состоятельность.
\end{document}
