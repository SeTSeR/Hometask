% Created 2020-03-31 Tue 18:42
% Intended LaTeX compiler: pdflatex
\documentclass[11pt]{article}
\usepackage[utf8]{inputenc}
\usepackage[T1]{fontenc}
\usepackage{graphicx}
\usepackage{grffile}
\usepackage{longtable}
\usepackage{wrapfig}
\usepackage{rotating}
\usepackage[normalem]{ulem}
\usepackage{amsmath}
\usepackage{textcomp}
\usepackage{amssymb}
\usepackage{capt-of}
\usepackage{hyperref}
\usepackage{minted}
\usepackage{amsmath}
\usepackage{esint}
\usepackage[english, russian]{babel}
\usepackage{mathtools}
\usepackage{amsthm}
\usepackage[top=0.8in, bottom=0.75in, left=0.625in, right=0.625in]{geometry}
\def\zall{\setcounter{lem}{0}\setcounter{cnsqnc}{0}\setcounter{th}{0}\setcounter{Cmt}{0}\setcounter{equation}{0}}
\newcounter{lem}\setcounter{lem}{0}
\def\lm{\par\smallskip\refstepcounter{lem}\textbf{\arabic{lem}}}
\newtheorem*{Lemma}{Лемма \lm}
\newcounter{th}\setcounter{th}{0}
\def\th{\par\smallskip\refstepcounter{th}\textbf{\arabic{th}}}
\newtheorem*{Theorem}{Теорема \th}
\newcounter{cnsqnc}\setcounter{cnsqnc}{0}
\def\cnsqnc{\par\smallskip\refstepcounter{cnsqnc}\textbf{\arabic{cnsqnc}}}
\newtheorem*{Consequence}{Следствие \cnsqnc}
\newcounter{Cmt}\setcounter{Cmt}{0}
\def\cmt{\par\smallskip\refstepcounter{Cmt}\textbf{\arabic{Cmt}}}
\newtheorem*{Note}{Замечание \cmt}
\author{Sergey Makarov}
\date{\today}
\title{}
\hypersetup{
 pdfauthor={Sergey Makarov},
 pdftitle={},
 pdfkeywords={},
 pdfsubject={},
 pdfcreator={Emacs 26.3 (Org mode 9.3)}, 
 pdflang={English}}
\begin{document}

\tableofcontents


\section{Семинар 1}
\label{sec:orgb0c3822}
\subsection{Задача 1}
\label{sec:orgf582e74}
Построить СДНФ функции:
\begin{center}
\begin{tabular}{rrrr}
\hline
\(x_1\) & \(x_2\) & \(x_3\) & f\\
\hline
0 & 0 & 0 & 1\\
0 & 0 & 1 & 0\\
0 & 1 & 0 & 1\\
0 & 1 & 1 & 1\\
1 & 0 & 0 & 0\\
1 & 0 & 1 & 1\\
1 & 1 & 0 & 0\\
1 & 1 & 1 & 1\\
\hline
\end{tabular}
\end{center}
\subsubsection{Решение}
\label{sec:org5eea59d}
   \begin{equation}
D_f^s(x_1, x_2, x_3) = \overline{x_1}\overline{x_2}\overline{x_3}\vee\overline{x_1}x_2\overline{x_3}
\vee\overline{x_1}x_2x_3\vee x_1\overline{x_2}x_3
   \end{equation}
\begin{equation}
K_f^s(x_1, x_2, x_3) = (x_1\vee x_2\vee\overline{x_3})(\overline{x_1}\vee x_2\vee x_3)
(\overline{x_1}\vee\overline{x_2}\vee x_3)(\overline{x_1}\vee\overline{x_2}\vee\overline{x_3})
\end{equation}
\subsection{Задача 2}
\label{sec:orgdda6d34}
  \begin{equation}
f = (00101111). \text{ Найти простые импликанты.}
  \end{equation}
\subsubsection{Решение}
\label{sec:org9d68c7e}
   \begin{equation}
A = \{x_1, \overline{x_3}, x_1x_2, x_2\overline{x_3}\}\text{ - импликанты $f$.}
   \end{equation}
$x_1x_2$ - не простая импликанта.
\subsection{Задача 3}
\label{sec:org9bd8b81}
Найти простые импликанты функции
  \begin{equation}
f = (01111110).
  \end{equation}
\subsubsection{Решение}
\label{sec:org3c4465f}
   \begin{equation}
A = \{x_1\overline{x_2}, x_2x_3, x_1x_2x_3\}\text{ - импликанты}
   \end{equation}
$x_1\overline{x_2}$ - простая импликанта, $x_2x_3$ и $x_1x_2x_3$ - не импликанты.
\subsection{Задача 4}
\label{sec:org5e55209}
Построить сокращённую ДНФ функции
\begin{equation}
\tilde{\alpha_f} = (1111 1000 0100 1100)
\end{equation}
\subsubsection{Решение}
\label{sec:orgd05cacf}
Код максимальной грани: \((0022)\), соответствующая простая импликанта \(\overline{x_1}\overline{x^2}\).
Далее идёт ребро \((0200)\), соответствующее импликанте \(\overline{x_1}\overline{x_2}\overline{x_3}\).
Далее идёт ребро \((2100)\), соответствующее импликанте \(x_2\overline{x_3}\overline{x_4}\).
Следующее ребро \((1102) \rightarrow x_1x_2\overline{x_3}\),
\((1201) \rightarrow x_1\overline{x_3}x_4\), \((2001) \rightarrow \overline{x_2}\overline{x_3}x_4\).
\subsection{Задача 2.6}
\label{sec:org353fadb}
Найти сокращённую ДНФ методом карты:
\begin{equation}
\tilde{\alpha_f} = (0101 0111)
\end{equation}
\subsubsection{Решение}
\label{sec:orgdcc2203}
\begin{center}
\begin{tabular}{rrr}
\hline
x\textsubscript{1x}\textsubscript{2x}\textsubscript{3} & 0 & 1\\
\hline
00 & 0 & 1\\
01 & 0 & 1\\
11 & 1 & 1\\
10 & 0 & 1\\
\hline
\end{tabular}
\end{center}
Откуда \(D_f = x_3\vee x_1x_2\).
\subsection{Задача 2.6(5)}
\label{sec:orga9279c8}
  \begin{equation}
\tilde{\alpha_f} = (0001 1011 1101 1111)
  \end{equation}
\subsubsection{Решение}
\label{sec:org466427c}
\begin{center}
\begin{tabular}{rrrrr}
\hline
\(x_1\) & \(x_2\) & \(x_3\) & \(x_4\) & \(f\)\\
\hline
0 & 0 & 0 & 0 & 0\\
0 & 0 & 0 & 1 & 0\\
0 & 0 & 1 & 0 & 0\\
0 & 0 & 1 & 1 & 1\\
0 & 1 & 0 & 0 & 1\\
0 & 1 & 0 & 1 & 0\\
0 & 1 & 1 & 0 & 1\\
0 & 1 & 1 & 1 & 1\\
1 & 0 & 0 & 0 & 1\\
1 & 0 & 0 & 1 & 1\\
1 & 0 & 1 & 0 & 0\\
1 & 0 & 1 & 1 & 1\\
1 & 1 & 0 & 0 & 1\\
1 & 1 & 0 & 1 & 1\\
1 & 1 & 1 & 0 & 1\\
1 & 1 & 1 & 1 & 1\\
\hline
\end{tabular}
\end{center}
Тогда карта Карно будет иметь вид:
\begin{center}
\begin{tabular}{rrrrrl}
\hline
\(x_1x_2x_3x_4\) & 00 & 01 & 11 & 10 & \\
\hline
00 & 0 & 0 & 1 & 0 & \(x_2x_3, x_2\overline{x_4}\)\\
01 & 1 & 0 & 1 & 1 & \(x_1x_2, x_1\overline{x_3}\)\\
11 & 1 & 1 & 1 & 1 & \(x_1x_4\)\\
10 & 1 & 1 & 1 & 0 & \(x_3x_4\)\\
\hline
\end{tabular}
\end{center}
Тогда сокращённая ДНФ имеет вид:
\begin{equation}
D_f = x_2\overline{x_4}\vee x_2x_3\vee x_1x_2\vee x_1x_4\vee x_3x_4\vee x_1\overline{x_3}
\end{equation}
\subsection{Задача 2.3(1)}
\label{sec:orgb96a267}
Получить сокращённую ДНФ по КНФ:
\begin{equation}
(x_1\vee x_2\vee\overline{x_3})(\overline{x_1}\vee x_2\vee x_3)(\overline{x_2}\vee\overline{x_3})
\end{equation}
\subsubsection{Решение}
\label{sec:orgc695d8d}
\begin{multline}
   (x_1\vee x_2\vee\overline{x_3})(\overline{x_1}\vee x_2\vee x_3)(\overline{x_2}\vee\overline{x_3})
= (x_1x_2\vee x_1x_3 \vee x_2\overline{x_1}\vee x_2\vee x_2x_3\vee\overline{x_1}\overline{x_3}
\vee x_2\overline{x_3})(\overline{x_2}\vee\overline{x_3}) = \\
= (x_2\vee x_1x_3\vee\overline{x_1}\overline{x_3})(\overline{x_2}\vee\overline{x_3})
= x_2\overline{x_3}\vee x_1\overline{x_2}x_3\vee\overline{x_1}\overline{x_2}\overline{x_3}
\vee\overline{x_1}\overline{x_3} = x_2\overline{x_3}\vee\overline{x_1}\overline{x_3}\vee x_1\overline{x_2}x_3
\end{multline}
\subsection{Задача 2.2(1)}
\label{sec:orgd8e2276}
Построить сокращённую ДНФ по данной ДНФ методом Блейка:
\begin{equation}
\overline{x_1}\overline{x_2}\vee x_1\overline{x_2}x_4\vee x_2\overline{x_3}x_4\vee\overline{x_2}x_4|
\vee\overline{x_1}\overline{x_3}x_4\vee x_1\overline{x_3}x_4|\vee\overline{x_2}\overline{x_3}x_4
\vee\overline{x_3}x_4 = \overline{x_1}\overline{x_2}\vee\overline{x_2}x_4\vee\overline{x_3}x_4
\end{equation}
\subsection{Задача 2.2(2)}
\label{sec:org2b1574d}
  \begin{equation}
x_1\overline{x_2}x_3\vee\overline{x_1}x_2\overline{x_4}\vee\overline{x_2}\overline{x_3}x_4|
\vee x_1\overline{x_2}x_4
  \end{equation}
\subsection{Задача 2.9(1)}
\label{sec:orgfb90b5b}
  \begin{equation}
f(\tilde{x_n}) = x_1\oplus\ldots\oplus x_n
  \end{equation}
Длина сокращённой ДНФ - ?
\subsubsection{Решение}
\label{sec:org5691b3d}
   Максимальные грани - точки $\Rightarrow$ длина сокращённой ДНФ, она же длина СДНФ равна
$2^n - 1$.
\subsection{Задача 2.9(2)}
\label{sec:org151c9df}
Найти длину сокращённой ДНФ функции:
  \begin{equation}
f(\tilde{x_n}) = (x_1\vee x_2\vee x_3)(\overline{x_1}\vee\overline{x_2}\overline{x_3})\oplus
x_4\oplus\ldots\oplus x_n
  \end{equation}
\subsubsection{Решение}
\label{sec:org3969ae7}
\begin{equation}
f(\tilde{\alpha}) = 1 \Leftrightarrow \begin{cases}
g(\tilde{\alpha}) = 1, \\
h(\tilde{\alpha}) = 0, \\
g(\tilde{\alpha}) = 0, \\
h(\tilde{\alpha}) = 1.
   \end{cases}
\end{equation}
Первому случаю соответствует $2^{n - 4}\cdot6$ максимальных грани, второму - $2\cdot2^{n - 4}$,
итого длина ДНФ составляет $2^{n - 1}$.

\url{http://mk.cs.msu.ru}, лекционные курсы, ОКи, домашние задания там.
\section{Домашняя работа 1}
\label{sec:org78cad67}
\subsection{Задача 2.3(4)}
\label{sec:orgc4529ca}
Представить в виде СДНФ функцию:
\zall
\begin{equation}
f(\tilde{x}^3) = (01010001)
\end{equation}
\subsubsection{Решение}
\label{sec:org9d61d4f}
Функция принимает единичные значения на наборах $001, 011$ и $111$, откуда СДНФ имеет вид:
\begin{equation}
f(\tilde{x}^3) = \overline{x_1}\overline{x_2}x_3\vee\overline{x_1}x_2x_3\vee x_1x_2x_3
\end{equation}
\subsection{Задача 2.1(3)}
\label{sec:org97d16e3}
Из множества \(A\) ЭК выделить простые импликанты функции \(f\):
\begin{equation}
A = \{x_1, \overline{x_4}, x_2\overline{x_3}, \overline{x_1}\overline{x_2}\overline{x_4}\},
f(\tilde{x}^4) = (1010 1110 1101 1110)
\end{equation}
\subsubsection{Решение}
\label{sec:orgf3f09f3}
ЭК $x_1$ не является импликантой функции $f$, так как $f(1111) = 0$.

ЭК $\overline{x_4}$ является простой импликантой функции $f$.

ЭК $x_2\overline{x_3}$ является импликантой функции $f$. ЭК $x_2$ не является импликантой
функции $f$, так как $f(1111) = 0$. ЭК $\overline{x_3}$ также не является импликантой $f$,
так как $f(0001) = 0$, значит $x_2\overline{x_3}$ -- простая импликанта функции $f$.

ЭК $\overline{x_1}\overline{x_2}\overline{x_4}$ является импликантой функции $f$, но она не
является простой, так как содержит множитель $\overline{x_4}$, который, как было указано ранее,
является импликантой $f$.
\subsection{Задача 2.5(2, 6)}
\label{sec:org8b87839}
Изобразив множество \(N_f\) функции \(f(\tilde{x}^n)\) в \(B^n\), найти коды максимальных интервалов
и построить сокращённую ДНФ:
\begin{equation}
\tilde{\alpha}_f = (0101 0011)
\end{equation}
\begin{equation}
\tilde{\alpha}_f = (0001 0111 1110 1111)
\end{equation}
\subsubsection{Решение}
\label{sec:org7056999}
Для первой функции находим две максимальные грани с кодами \((112)\) и \((021)\), попытки расширить
которые приводят к выходу из \(N_f\). Отсюда получаем сокращённую ДНФ:
\begin{equation}
f(\tilde{x}^3) = \overline{x_1}x_3\vee x_1x_2
\end{equation}
Для второй функции максимальные грани имеют коды \((2112)\), \((1202)\), \((0121)\) и \((0211)\).
Попытки расширить эти грани приводят к выходу из \(N_f\). Отсюда получаем сокращённую ДНФ:
\begin{equation}
f(\tilde{x}^4) = x_2x_3\vee x_1\overline{x_3}\vee\overline{x_1}x_2x_4\vee\overline{x_1}x_3x_4
\end{equation}
\subsection{Задача 2.6(2, 6)}
\label{sec:orgb906bac}
Найти сокращённую ДНФ функции с помощью минимизирующей карты:
\begin{equation}
\tilde{\alpha_f} = (1101 1011)
\end{equation}
\begin{equation}
\tilde{\alpha_f} = (0011 1101 1111 1101)
\end{equation}
Карта Карно для первой функции:
\begin{center}
\begin{tabular}{rrrrr}
\hline
\(x_1\) $\backslash$ \(x_2x_3\) & 00 & 01 & 11 & 10\\
\hline
0 & 1 & 1 & 1 & 0\\
1 & 1 & 0 & 1 & 1\\
\hline
\end{tabular}
\end{center}
В данном случае максимальными по включению будут прямоугольники \((00-)\), \((-00)\), \((0-1)\),
\((-11)\), \((11-)\), \((1-0)\), откуда сокращённая ДНФ имеет вид:
\begin{equation}
f(\tilde{x}^3) = \overline{x_1}\overline{x_2}\vee\overline{x_2}\overline{x_3}\vee\overline{x_1}x_3
\vee x_2x_3\vee x_1x_2\vee x_1\overline{x_3}
\end{equation}
Карта Карно для второй функции:
\begin{center}
\begin{tabular}{rrrrr}
\hline
\(x_1x_2\) $\backslash$ \(x_3x_4\) & 00 & 01 & 11 & 10\\
\hline
00 & 0 & 0 & 1 & 1\\
01 & 1 & 1 & 1 & 0\\
11 & 1 & 1 & 1 & 0\\
10 & 1 & 1 & 1 & 1\\
\hline
\end{tabular}
\end{center}
Максимальными по включению здесь будут прямоугольники \((10--), (-10-), (--11), (-010)\),
которые дают сокращённую ДНФ:
\begin{equation}
f(\tilde{x}^4) = x_1\overline{x_2}\vee x_2\overline{x_3}\vee x_3x_4\vee\overline{x_2}x_3\overline{x_4}
\end{equation}
\subsection{Задача 2.2(3, 4)}
\label{sec:org54b28c3}
По заданной ДНФ методом Блейка получить сокращённую ДНФ:
\begin{equation}
D = x_1\vee\overline{x_1}x_2\vee\overline{x_1}\overline{x_2}x_3\vee\overline{x_1}x_2x_3x_4
\end{equation}
\begin{equation}
D = x_1\overline{x_2}x_4\vee\overline{x_1}\overline{x_2}x_3\vee\overline{x_3}\overline{x_4}
\end{equation}
\subsubsection{Решение}
\label{sec:org44a853b}
\begin{multline}
x_1\vee\overline{x_1}x_2\vee\overline{x_1}\overline{x_2}x_3\vee\overline{x_1}x_2x_3x_4 = \\
= x_1\vee\overline{x_1}x_2\vee\overline{x_1}\overline{x_2}x_3\vee\overline{x_1}x_2x_3x_4
\vee x_2\vee\overline{x_2}x_3\vee x_2x_3x_4\vee\overline{x_1}x_3\vee\overline{x_1}x_3x_4 = \\
= x_1\vee x_2\vee\overline{x_1}x_3\vee\overline{x_2}x_3\vee
= x_1\vee x_2\vee\overline{x_1}x_3\vee\overline{x_2}x_3\vee x_3
= x_1\vee x_2\vee x_3
\end{multline}
\begin{multline}
x_1\overline{x_2}x_4\vee\overline{x_1}\overline{x_2}x_3\vee\overline{x_3}\overline{x_4}
= x_1\overline{x_2}x_4\vee\overline{x_1}\overline{x_2}x_3\vee\overline{x_3}\overline{x_4}\vee
\overline{x_2}x_3x_4\vee\overline{x_1}\overline{x_2}\overline{x_4}\vee x_1\overline{x_2}\overline{x_3} = \\
= x_1\overline{x_2}x_4\vee\overline{x_1}\overline{x_2}x_3\vee\overline{x_3}\overline{x_4}\vee
\overline{x_2}x_3x_4\vee\overline{x_1}\overline{x_2}\overline{x_4}\vee x_1\overline{x_2}\overline{x_3}\vee
\overline{x_2}\overline{x_3}\overline{x_4}\vee x_1\overline{x_2}x_4\vee\overline{x_1}\overline{x_2}x_3 = \\
= x_1\overline{x_2}x_4\vee\overline{x_1}\overline{x_2}x_3\vee\overline{x_3}\overline{x_4}
\vee\overline{x_2}x_3x_4\vee\overline{x_1}\overline{x_2}\overline{x_4}\vee x_1\overline{x_2}\overline{x_3}
\end{multline}
\subsection{Задача 2.3(3, 4)}
\label{sec:org1f5a3e6}
Построить сокращённую ДНФ по заданной КНФ:
\begin{equation}
(x_1\vee\overline{x_2}\vee\overline{x_3})(\overline{x_1}\vee x_2)(x_1\vee x_2\vee x_3)
\end{equation}
\begin{equation}
(x_1\vee x_2\vee x_3)(\overline{x_1}\vee\overline{x_2}\vee\overline{x_3})
\end{equation}
\subsubsection{Решение}
\label{sec:org2871ddb}
\begin{multline}
(x_1\vee\overline{x_2}\vee\overline{x_3})(\overline{x_1}\vee x_2)(x_1\vee x_2\vee x_3) =
(x_1x_2\vee\overline{x_1}\overline{x_2}\vee\overline{x_1}\overline{x_3}\vee x_2\overline{x_3})
(x_1\vee x_2\vee x_3) = \\
= x_1x_2\vee x_1x_2x_3\vee\overline{x_1}\overline{x_2}x_3\vee\overline{x_1}x_2\overline{x_3}\vee
x_1x_2\overline{x_3}\vee x_2\overline{x_3} = x_1x_2\vee x_2\overline{x_3}\vee\overline{x_1}\overline{x_2}x_3
\end{multline}
\begin{equation}
(x_1\vee x_2\vee x_3)(\overline{x_1}\vee\overline{x_2}\vee\overline{x_3})
= x_1\overline{x_2}\vee x_1\overline{x_3}\vee\overline{x_1}x_2\vee x_2\overline{x_3}\vee
\overline{x_1}x_3\vee\overline{x_2}x_3
\end{equation}
\subsection{Задача 2.9(6)}
\label{sec:org123a610}
Найти длину сокращённой ДНФ функции \(f\):
\begin{equation}
f(\tilde{x}^n) = (x_1\vee\ldots\vee x_n)(x_1\vee\ldots\vee x_k\vee\overline{x}_{k + 1}\vee\ldots\vee\overline{x}_n)
\end{equation}
\subsubsection{Решение}
\label{sec:org5e3b4f4}
\begin{multline}
(x_1\vee\ldots\vee x_n)(x_1\vee\ldots\vee x_k\vee\overline{x}_{k + 1}\vee\ldots\overline{x}_n) =
(\vee_{i = 1}^kx_k)\vee(\vee_{i, j = 1}^{k, n}x_ix_j)\vee(\vee_{i = k + 1, j = 1}^{n}\overline{x}_ix_j) = \\
= (\vee_{i = 1}^kx_k)\vee(\vee_{i, j = k + 1}^n\overline{x_i}x_j)\vee(\vee_{i, j = k + 1}^nx_ix_j)
\end{multline}
Всего получаем
\begin{equation}
k + 2\frac{(n - k)(n - k + 1)}2 = k + (n - k)(n - k + 1) = n + (n - k)^2 = n^2 - 2nk + n + k^2
\end{equation}
слагаемых.
\section{Семинар 2}
\label{sec:org51337dd}
\subsection{Задача 3.1(1, 5)}
\label{sec:org98da806}
\zall
\begin{equation}
   D = x_1x_2 \vee\overline{x_1}\text{ - не тупиковая, не минимальная, кратчайшая.}
\end{equation}
\begin{equation}
   D = x_1x_2x_3\vee\overline{x_2}x_3\vee x_2\overline{x_3}\text{ - не тупиковая, не минимальная, кратчайшая.}
\end{equation}
\subsection{Задача 3.3(1)}
\label{sec:org0912121}
Построить ядро, ДНФ Квайна, ДНФ \(\Sigma T\) по сокращённой ДНФ:
   \begin{equation}
D_f = xy\vee\overline{x}\overline{z}\vee y\overline{z}
   \end{equation}
\subsubsection{Решение}
\label{sec:org93ea01d}
Таблица Квайна:
\begin{center}
\begin{tabular}{lrlrlllrrl}
\hline
 & 000 & 001 & 010 & 011 & 100 & 101 & 110 & 111 & \\
\hline
\(xy\) &  & x &  & x & x & x & 1 & 1 & я\\
\(\overline{x}\overline{z}\) & 1 & x & 1 & x & x & x &  &  & я\\
\(y\overline{z}\) &  & x & 1 & x & x & x & 1 &  & пя, р\\
\hline
\end{tabular}
\end{center}
000 и 111 - ядровые грани, 010 и 110 покрыты ядром => \(y\overline{z}\) покрыта ядром.
Ядро: \(\{N_{xy}, N_{\overline{x}\overline{z}}\}\)
ДНФ Квайна: \(D_f = xy\vee\overline{x}\overline{z}\)
Точки 010 и 110 регулярны, поэтому ДНФ \(\Sigma T\) совпадает с ДНФ Квайна.
\subsection{Задача 3.3(2)}
\label{sec:orgf18d489}
Построить ядро, ДНФ Квайна, ДНФ \(\Sigma t\):
   \begin{equation}
D = \overline{z}\overline{w}\vee\overline{y}zw\vee x\overline{y}w\vee \overline{x}\overline{y}z
\vee x\overline{y}\overline{z}\vee\overline{x}\overline{y}\overline{w}
   \end{equation}
\subsubsection{Решение}
\label{sec:org486e1e6}
Таблица Квайна:
\begin{center}
\begin{tabular}{lrlrrrlllrrlrrlll}
\hline
 & 0000 & 0001 & 0010 & 0011 & 0100 & 0101 & 0110 & 0111 & 1000 & 1001 & 1010 & 1011 & 1100 & 1101 & 1110 & 1111\\
\hline
\(\overline{z}\overline{w}\) & 1 & x &  &  & 1 & x & x & x & 1 &  & x &  & 1 & x & x & x\\
\(\overline{y}zw\) &  & x &  & 1 &  & x & x & x &  &  & x & 1 &  & x & x & x\\
\(x\overline{y}w\) &  & x &  &  &  & x & x & x &  & 1 & x & 1 &  & x & x & x\\
\(\overline{x}\overline{y}z\) &  & x & 1 & 1 &  & x & x & x &  &  & x &  &  & x & x & x\\
\(x\overline{y}\overline{z}\) &  & x &  &  &  & x & x & x & 1 & 1 & x &  &  & x & x & x\\
\(\overline{x}\overline{y}\overline{w}\) & 1 & x & 1 &  &  & x & x & x &  &  & x &  &  & x & x & x\\
\hline
\end{tabular}
\end{center}
Точки 0100 и 1100 ядровые, точки 0000 и 1000 покрыты ядром и регулярны, откуда ядро: \(\{N_{\overline{z}\overline{w}}\}\),
и ДНФ Квайна совпадает с ДНФ \(\Sigma T\) и равна исходной.
\subsection{Задача 3.4(3)}
\label{sec:orgacade00}
   \begin{equation}
D_f = \overline{x}w\vee\overline{y}w\vee zw\vee xz\vee yz
   \end{equation}
\subsubsection{Решение}
\label{sec:org8e83afa}
Таблица Квайна:
\begin{center}
\begin{tabular}{llrlrlrrrlrrrlrrr}
\hline
 & 0000 & 0001 & 0010 & 0011 & 0100 & 0101 & 0110 & 0111 & 1000 & 1001 & 1010 & 1011 & 1100 & 1101 & 1110 & 1111\\
\hline
\(\overline{x}w\) & x & 1 & x & 1 & x & 1 &  & 1 & x &  &  &  & x &  &  & \\
\(\overline{y}w\) & x & 1 & x & 1 & x &  &  &  & x & 1 &  & 1 & x &  &  & \\
\(zw\) & x &  & x & 1 & x &  &  & 1 & x &  &  & 1 & x &  &  & 1\\
\(xz\) & x &  & x &  & x &  &  &  & x &  & 1 & 1 & x & 1 & 1 & \\
\(yz\) & x &  & x &  & x &  & 1 & 1 & x &  &  &  & x &  & 1 & 1\\
\hline
\end{tabular}
\end{center}
Точки 0101, 0110, 1001, 1010 ядровые, 0001, 0011, 0111, 1011, 1110, 1111 покрыты ядром и регулярны,
соответственно, ДНФ \(\Sigma M\), ДНФ \(\Sigma T\) и ДНФ Квайна совпадают и равны:
\begin{equation}
D_f = \overline{x}w\vee\overline{y}w\vee xz\vee yz
\end{equation}
\subsection{Задача 3.6(1)}
\label{sec:orgd1046b5}
Построить все тупиковые ДНФ функции
\begin{equation}
\tilde{\alpha}_f = (01111100)
\end{equation}
\subsubsection{Решение}
\label{sec:orga9048b5}
Карта Карно:
\begin{center}
\begin{tabular}{rrr}
\hline
\(xy\backslash w\) & 0 & 1\\
\hline
00 & 0 & 1\\
01 & 1 & 1\\
11 & 0 & 0\\
10 & 1 & 1\\
\hline
\end{tabular}
\end{center}
Отсюда получаем \(\overline{x}w\vee\overline{x}y\vee x\overline{y}\vee\overline{y}w\). Таблица
Квайна:
\begin{center}
\begin{tabular}{lrrrrrl}
\hline
 & 001 & 010 & 011 & 100 & 101 & \\
\hline
\(K_1 = \overline{x}w\) & 1 &  & 1 &  &  & \\
\(K_2 = \overline{x}y\) &  & 1 & 1 &  &  & я\\
\(K_3 = x\overline{y}\) &  &  &  & 1 & 1 & я\\
\(K_4 = \overline{y}w\) & 1 &  &  &  & 1 & \\
\hline
\end{tabular}
\end{center}
Точки 010 и 100 ядровые, точки 011 и 101 покрыты ядром.
\begin{equation}
(K_1\vee K_4)\cdot K_2\cdot K_3 = K_1K_2K_3\vee K_2K_3K_4
\end{equation}
Все тупиковые ДНФ:
\begin{equation}
\overline{x}w\vee\overline{x}y\vee x\overline{y}
\end{equation}
\begin{equation}
\overline{x}y\vee x\overline{y}\vee\overline{y}w
\end{equation}
\subsection{Задача 3.6(4)}
\label{sec:org38d2537}
   \begin{equation}
\tilde{\alpha}_f = (1111 1000 0100 1100)
   \end{equation}
\subsubsection{Решение}
\label{sec:org6b01b7f}
Карта Карно:
\begin{center}
\begin{tabular}{rrrrr}
\hline
\(xy\backslash zw\) & 00 & 01 & 11 & 10\\
\hline
00 & 1 & 1 & 1 & 1\\
01 & 1 & 0 & 0 & 0\\
11 & 1 & 1 & 0 & 0\\
10 & 0 & 1 & 0 & 0\\
\hline
\end{tabular}
\end{center}
Получаем сокращённую ДНФ:
\begin{equation}
\overline{x}\overline{y}\vee\overline{x}\overline{z}\overline{w}
\vee xy\overline{z}\vee x\overline{z}w\vee \overline{y}\overline{z}w
\end{equation}
Таблица Квайна:
\begin{center}
\begin{tabular}{lrrrrrrrrl}
\hline
 & 0000 & 0001 & 0010 & 0011 & 0100 & 1001 & 1100 & 1101 & \\
\hline
\(K_1 = \overline{x}\overline{y}\) & 1 & 1 & 1 & 1 &  &  &  &  & я\\
\(K_2 = \overline{x}\overline{z}\overline{w}\) & 1 &  &  &  & 1 &  &  &  & \\
\(K_3 = y\overline{z}\overline{w}\) &  &  &  &  & 1 &  & 1 &  & \\
\(K_4 = xy\overline{z}\) &  &  &  &  &  &  & 1 & 1 & \\
\(K_5 = x\overline{z}w\) &  &  &  &  &  & 1 &  & 1 & \\
\(K_6 = \overline{y}\overline{z}w\) &  & 1 &  &  &  & 1 &  &  & \\
\hline
\end{tabular}
\end{center}
Ядровые точки 0010 и 0011, точки 0000 и 0001 покрыты ядром.
Функция покрытия для оставшейся таблицы:
\begin{equation}
(K_2\vee K_3)(K_5\vee K_6)(K_3\vee K_4)(K_4\vee K_5) =
(K_3\vee K_2K_4)(K_5\vee K_4K_6) =
K_3K_5\vee K_3K_4K_6\vee K_2K_4K_5\vee K_2K_4K_6
\end{equation}
Получаем 4 тупиковых ДНФ:
\begin{equation}
\overline{x}\overline{y}\vee y\overline{z}\overline{w}\vee x\overline{z}w,
\end{equation}
\begin{equation}
\overline{x}\overline{y}\vee y\overline{z}\overline{w}\vee xy\overline{z}\vee\overline{y}\overline{z}w,
\end{equation}
\begin{equation}
\overline{x}\overline{y}\vee \overline{x}\overline{z}\overline{w}\vee xy\overline{z}\vee x\overline{z}w,
\end{equation}
\begin{equation}
\overline{x}\overline{y}\vee\overline{x}\overline{z}\overline{w}\vee xy\overline{z}\vee \overline{y}\overline{z}w
\end{equation}
\section{Домашняя работа 2}
\label{sec:org3d797e2}
\subsection{Задача 3.1(4, 6)}
\label{sec:org7ded8dd}
Выяснить, является ли ДНФ D тупиковой, кратчайшей, минимальной:
\zall
\begin{equation}
D = x_1\overline{x}_2 \vee \overline{x}_1x_2
\end{equation}
\begin{equation}
D = x_1x_2 \vee \overline{x}_1x_3\overline{x}_4 \vee \overline{x}_2x_3x_4
\end{equation}
\subsubsection{Решение}
\label{sec:org15d64e6}
4) $D$ является тупиковой, кратчайшей и минимальной.\\
6) $D$ является тупиковой, кратчайшей и минимальной.
\subsection{Задача 3.3(3, 4 -- ядро, ДНФ Квайна и \(\Sigma\) T)}
\label{sec:org16be2cf}
Построить ядро, ДНФ Квайна и \(\Sigma T\) по сокращённой ДНФ:
   \begin{equation}
D = x\overline{y}z \vee x\overline{y}\overline{w} \vee \overline{x}y\overline{w} \vee
\overline{x}\overline{z}\overline{w} \vee \overline{y}\overline{z}\overline{w}
   \end{equation}
\begin{equation}
D = \overline{z}\overline{w} \vee \overline{x}\overline{y}\overline{w} \vee
\overline{x}\overline{y}\overline{z} \vee xyz \vee xy\overline{w}
\end{equation}
\subsubsection{Решение}
\label{sec:orgd438d0c}
Строим таблицу Квайна:
\begin{center}
\begin{tabular}{lrlllrlrrrlrrlllrl}
\hline
 & 0000 & 0001 & 0010 & 0011 & 0100 & 0101 & 0110 & 0111 & 1000 & 1001 & 1010 & 1011 & 1100 & 1101 & 1110 & 1111 & \\
\hline
\(x\overline{y}z\) &  & x & x & x &  & x &  &  &  & x & 1 & 1 & x & x & x &  & я\\
\(x\overline{y}\overline{w}\) &  & x & x & x &  & x &  &  & 1 & x & 1 &  & x & x & x &  & я\\
\(\overline{x}y\overline{w}\) &  & x & x & x & 1 & x & 1 &  &  & x &  &  & x & x & x &  & я\\
\(\overline{x}\overline{z}\overline{w}\) & 1 & x & x & x & 1 & x &  &  &  & x &  &  & x & x & x &  & я\\
\(\overline{y}\overline{z}\overline{w}\) &  & x & x & x &  & x &  & 1 &  & x &  &  & x & x & x & 1 & я\\
\hline
 & я &  &  &  & р &  & я & я & я &  & р & я &  &  &  & я & \\
\hline
\end{tabular}
\end{center}
Таким образом, ядро: \(\{N_{x\overline{y}z}, N_{x\overline{y}\overline{w}}, N_{\overline{x}y\overline{w}}, N_{\overline{x}\overline{z}\overline{w}}, N_{\overline{y}\overline{z}\overline{w}}\}\),
соответственно, ДНФ Квайна и ДНФ \(\Sigma T\) совпадают с ДНФ \((3)\)

Таблица Квайна для второй ДНФ(наборы, где все слагаемые обращаются в ноль, опущены для краткости):
\begin{center}
\begin{tabular}{lrrrrrrrrrl}
\hline
 & 0000 & 0001 & 0010 & 0100 & 1000 & 1100 & 1101 & 1110 & 1111 & \\
\hline
\(\overline{z}\overline{w}\) & 1 &  &  & 1 & 1 & 1 &  &  &  & я\\
\(\overline{x}\overline{y}\overline{w}\) & 1 &  & 1 &  &  &  &  &  &  & я\\
\(\overline{x}\overline{y}\overline{z}\) & 1 & 1 &  &  &  &  &  &  &  & я\\
\(xyz\) &  &  &  &  &  &  &  & 1 & 1 & я\\
\(xy\overline{w}\) &  &  &  &  &  &  & 1 &  & 1 & я\\
\hline
 & р & я & я & я & я & я & я & я & р & \\
\hline
\end{tabular}
\end{center}
Таким образом, ядро: \(\{N_{\overline{z}\overline{w}}, N_{\overline{x}\overline{y}\overline{w}}, N_{\overline{x}\overline{y}\overline{z}}, N_{xyz}, N_{xy\overline{w}}\}\).
Соответственно, ДНФ Квайна и ДНФ \(\Sigma T\) совпадают с (4)
\subsection{Задача 3.4(4)}
\label{sec:orgc08eacf}
По сокращённой ДНФ построить ДНФ \(\Sigma M\):
\begin{equation}
D = \overline{x}z \vee yz \vee \overline{x}\overline{y}w \vee xy \vee \overline{y}zw \vee xzw
\end{equation}
\subsubsection{Решение}
\label{sec:org3ee0695}
Таблица Квайна:
\begin{center}
\begin{tabular}{lrrrrrrrrrrl}
\hline
 & 0001 & 0010 & 0011 & 0110 & 0111 & 1011 & 1100 & 1101 & 1110 & 1111 & \\
\hline
\(\overline{x}z\) &  & 1 & 1 & 1 & 1 &  &  &  &  &  & я\\
\(yz\) &  &  &  & 1 & 1 &  &  &  & 1 & 1 & пя\\
\(\overline{x}\overline{y}w\) & 1 &  & 1 &  &  &  &  &  &  &  & я\\
\(xy\) &  &  &  &  &  &  & 1 & 1 & 1 & 1 & я\\
\(\overline{y}zw\) &  &  & 1 &  &  & 1 &  &  &  &  & \\
\(xzw\) &  &  &  &  &  & 1 &  &  &  & 1 & \\
\hline
 & я & я & пя, р & пя, р & пя, р &  & я & я & пя, р & пя, р & \\
\hline
\end{tabular}
\end{center}
Получаем две тупиковые ДНФ:
\begin{equation}
D_1 = \overline{x}z \vee \overline{x}\overline{y}w \vee xy \vee \overline{y}zw
\end{equation}
и
\begin{equation}
D_2 = \overline{x}z \vee \overline{x}\overline{y}w \vee xy \vee xzw
\end{equation}
Обе ДНФ являются минимальными, поэтому ДНФ $\Sigma M$ имеет вид:
\begin{equation}
D = \overline{x}z \vee yz \vee \overline{x}\overline{y}w \vee xy \vee \overline{y}zw \vee xzw
\end{equation}
\subsection{Задача 3.6(3, 6, 8)}
\label{sec:org2554122}
С помощью таблицы Квайна построить все тупиковые ДНФ функции:
\begin{equation}
\tilde{\alpha}_f = (0001 1111)
\end{equation}
\begin{equation}
\tilde{\alpha}_f = (1110 0110 0001 0101)
\end{equation}
\begin{equation}
\tilde{\alpha}_f = (0001 1011 1110 0111)
\end{equation}
\subsubsection{Решение}
\label{sec:org9a94987}
Карта Карно:
\begin{center}
\begin{tabular}{rrr}
\hline
\(xy\backslash z\) & 0 & 1\\
\hline
00 &  & \\
01 &  & 1\\
11 & 1 & 1\\
10 & 1 & 1\\
\hline
\end{tabular}
\end{center}
Получаем сокращённую ДНФ:
\begin{equation}
D = x \vee yz
\end{equation}
Строим таблицу Квайна:
\begin{center}
\begin{tabular}{lrrrrrl}
\hline
 & 011 & 100 & 101 & 110 & 111 & \\
\hline
\(K_1 = x\) &  & 1 & 1 & 1 & 1 & я\\
\(K_2 = yz\) & 1 & 1 &  &  &  & я\\
\hline
 & я & пя, р & я & я & я & \\
\hline
\end{tabular}
\end{center}
Получили, что все грани ядровые, т. е. единственной тупиковой днф является (10).

Карта Карно для (10)
\begin{center}
\begin{tabular}{rrrrr}
\hline
\(xy\backslash zw\) & 00 & 01 & 11 & 10\\
\hline
00 & 1 & 1 &  & 1\\
01 &  & 1 &  & 1\\
11 &  & 1 & 1 & \\
10 &  &  & 1 & \\
\hline
\end{tabular}
\end{center}
Получаем сокращённую ДНФ:
\begin{equation}
D = \overline{x}\overline{y}\overline{z} \vee \overline{x}\overline{z}w \vee y\overline{z}w \vee
xyw \vee xzw \vee \overline{x}z\overline{w} \vee \overline{x}\overline{y}\overline{w}
\end{equation}
Таблица Квайна:
\begin{center}
\begin{tabular}{lrrrrrrrrl}
\hline
 & 0000 & 0001 & 0010 & 0101 & 0110 & 1011 & 1101 & 1111 & \\
\hline
\(K_1 = \overline{x}\overline{y}\overline{z}\) & 1 & 1 &  &  &  &  &  &  & \\
\(K_2 = \overline{x}\overline{z}w\) &  & 1 &  & 1 &  &  &  &  & \\
\(K_3 = y\overline{z}w\) &  &  &  & 1 &  &  & 1 &  & \\
\(K_4 = xyw\) &  &  &  &  &  &  & 1 & 1 & \\
\(K_5 = xzw\) &  &  &  &  &  & 1 &  & 1 & я\\
\(K_6 = \overline{x}z\overline{w}\) &  &  & 1 &  & 1 &  &  &  & я\\
\(K_7 = \overline{x}\overline{y}\overline{w}\) & 1 &  & 1 &  &  &  &  &  & \\
\hline
 &  &  & пя &  & я & я &  & пя & \\
\hline
\end{tabular}
\end{center}
Построим покрытие таблицы:
\begin{multline}
(K_1 \vee K_7)(K_1 \vee K_2)(K_2 \vee K_3)(K_3 \vee K_4)K_5K_6 = (K_1 \vee K_2K_7)(K_3 \vee K_2K_4)K_5K_6 = \\
= K_1K_3K_5K_6 \vee K_1K_2K_4K_5K_6 \vee K_2K_3K_5K_6K_7 \vee K_2K_4K_5K_6K_7
\end{multline}
Откуда получаем 4 тупиковых ДНФ:
\begin{equation}
D_1 = \overline{x}\overline{y}\overline{z} \vee y\overline{z}w \vee xzw \vee \overline{x}z\overline{w}
\end{equation}
\begin{equation}
D_2 = \overline{x}\overline{y}\overline{z} \vee \overline{x}\overline{z}w \vee xzw \vee
\overline{x}z\overline{w} \vee \overline{x}\overline{y}\overline{w}
\end{equation}
\begin{equation}
D_3 = \overline{x}\overline{z}w \vee y\overline{z}w \vee xzw \vee \overline{x}z\overline{w} \vee
\overline{x}\overline{y}\overline{w}
\end{equation}
\begin{equation}
D_4 = \overline{x}\overline{z}w \vee xyw \vee xzw \vee \overline{x}z\overline{w} \vee
\overline{x}\overline{y}\overline{w}
\end{equation}

Карта Карно для (11):
\begin{center}
\begin{tabular}{rrrrr}
\hline
\(xy\backslash zw\) & 00 & 01 & 11 & 10\\
\hline
00 &  &  & 1 & \\
01 & 1 &  & 1 & 1\\
11 & 1 & 1 &  & 1\\
10 &  & 1 & 1 & 1\\
\hline
\end{tabular}
\end{center}
Получаем сокращённую ДНФ:
\begin{equation}
D = y\overline{w} \vee xy\overline{z} \vee x\overline{z}w \vee x\overline{y}w \vee
x\overline{y}z \vee xz\overline{w} \vee \overline{x}yz \vee \overline{x}zw
\end{equation}
Таблица Квайна:
\begin{center}
\begin{tabular}{lrrrrrrrrrrl}
\hline
 & 0011 & 0100 & 0110 & 0111 & 1001 & 1010 & 1011 & 1100 & 1101 & 1110 & \\
\hline
\(K_1 = y\overline{w}\) &  & 1 & 1 &  &  &  &  & 1 &  & 1 & я\\
\(K_2 = xy\overline{z}\) &  &  &  &  &  &  &  & 1 & 1 &  & \\
\(K_3 = x\overline{z}w\) &  &  &  &  & 1 &  &  &  & 1 &  & \\
\(K_4 = x\overline{y}w\) &  &  &  &  & 1 &  & 1 &  &  &  & \\
\(K_5 = x\overline{y}z\) &  &  &  &  &  & 1 & 1 &  &  &  & \\
\(K_6 = xz\overline{w}\) &  &  &  &  &  & 1 &  &  &  & 1 & \\
\(K_7 = \overline{x}yz\) &  &  & 1 & 1 &  &  &  &  &  &  & пя\\
\(K_8 = \overline{x}zw\) & 1 &  &  & 1 &  &  &  &  &  &  & я\\
\hline
 & я & я & пя & пя &  &  &  & пя &  & пя & \\
\hline
\end{tabular}
\end{center}
Покрытие таблицы:
\begin{multline}
(K_3 \vee K_4)(K_5 \vee K_6)(K_4 \vee K_5)(K_2 \vee K_3)K_1K_8 = (K_3 \vee K_2K_4)(K_5 \vee K_6K_4)K_1K_8 = \\
= K_1K_3K_5K_8 \vee K_1K_2K_4K_5K_8 \vee K_1K_3K_4K_6K_8 \vee K_1K_2K_4K_6K_8
\end{multline}
Получаем 4 тупиковых ДНФ:
\begin{equation}
D_1 = y\overline{w} \vee x\overline{z}w \vee x\overline{y}z \vee \overline{x}zw
\end{equation}
\begin{equation}
D_2 = y\overline{w} \vee xy\overline{z} \vee x\overline{y}w \vee x\overline{y}z \vee \overline{x}zw
\end{equation}
\begin{equation}
D_3 = y\overline{w} \vee x\overline{z}w \vee xz\overline{w} \vee \overline{x}zw
\end{equation}
\begin{equation}
D_4 = y\overline{w} \vee xy\overline{z} \vee x\overline{y}w \vee xz\overline{w} \vee \overline{x}zw
\end{equation}
\end{document}
