% Created 2020-02-22 Sat 16:27
% Intended LaTeX compiler: pdflatex
\documentclass[11pt]{article}
\usepackage[utf8]{inputenc}
\usepackage[T1]{fontenc}
\usepackage{graphicx}
\usepackage{grffile}
\usepackage{longtable}
\usepackage{wrapfig}
\usepackage{rotating}
\usepackage[normalem]{ulem}
\usepackage{amsmath}
\usepackage{textcomp}
\usepackage{amssymb}
\usepackage{capt-of}
\usepackage{hyperref}
\usepackage{minted}
\usepackage{amsmath}
\usepackage{esint}
\usepackage[english, russian]{babel}
\usepackage{mathtools}
\usepackage{amsthm}
\usepackage[top=0.8in, bottom=0.75in, left=0.625in, right=0.625in]{geometry}
\def\zall{\setcounter{lem}{0}\setcounter{cnsqnc}{0}\setcounter{th}{0}\setcounter{Cmt}{0}\setcounter{equation}{0}}
\newcounter{lem}\setcounter{lem}{0}
\def\lm{\par\smallskip\refstepcounter{lem}\textbf{\arabic{lem}}}
\newtheorem*{Lemma}{Лемма \lm}
\newcounter{th}\setcounter{th}{0}
\def\th{\par\smallskip\refstepcounter{th}\textbf{\arabic{th}}}
\newtheorem*{Theorem}{Теорема \th}
\newcounter{cnsqnc}\setcounter{cnsqnc}{0}
\def\cnsqnc{\par\smallskip\refstepcounter{cnsqnc}\textbf{\arabic{cnsqnc}}}
\newtheorem*{Consequence}{Следствие \cnsqnc}
\newcounter{Cmt}\setcounter{Cmt}{0}
\def\cmt{\par\smallskip\refstepcounter{Cmt}\textbf{\arabic{Cmt}}}
\newtheorem*{Note}{Замечание \cmt}
\author{Sergey Makarov}
\date{\today}
\title{}
\hypersetup{
 pdfauthor={Sergey Makarov},
 pdftitle={},
 pdfkeywords={},
 pdfsubject={},
 pdfcreator={Emacs 26.3 (Org mode 9.3)}, 
 pdflang={English}}
\begin{document}

\tableofcontents


\section{Задача 1}
\label{sec:org4afbed6}
Построить СДНФ функции:
\begin{center}
\begin{tabular}{rrrr}
\hline
\(x_1\) & \(x_2\) & \(x_3\) & f\\
\hline
0 & 0 & 0 & 1\\
0 & 0 & 1 & 0\\
0 & 1 & 0 & 1\\
0 & 1 & 1 & 1\\
1 & 0 & 0 & 0\\
1 & 0 & 1 & 1\\
1 & 1 & 0 & 0\\
1 & 1 & 1 & 1\\
\hline
\end{tabular}
\end{center}
\subsection{Решение}
\label{sec:orgd04c2a0}
   \begin{equation}
D_f^s(x_1, x_2, x_3) = \overline{x_1}\overline{x_2}\overline{x_3}\vee\overline{x_1}x_2\overline{x_3}
\vee\overline{x_1}x_2x_3\vee x_1\overline{x_2}x_3
   \end{equation}
\begin{equation}
K_f^s(x_1, x_2, x_3) = (x_1\vee x_2\vee\overline{x_3})(\overline{x_1}\vee x_2\vee x_3)
(\overline{x_1}\vee\overline{x_2}\vee x_3)(\overline{x_1}\vee\overline{x_2}\vee\overline{x_3})
\end{equation}
\section{Задача 2}
\label{sec:org0b7fc0d}
  \begin{equation}
f = (00101111). \text{ Найти простые импликанты.}
  \end{equation}
\subsection{Решение}
\label{sec:org1fdb4d2}
   \begin{equation}
A = \{x_1, \overline{x_3}, x_1x_2, x_2\overline{x_3}\}\text{ - импликанты $f$.}
   \end{equation}
$x_1x_2$ - не простая импликанта.
\section{Задача 3}
\label{sec:orgab61f1f}
Найти простые импликанты функции
  \begin{equation}
f = (01111110).
  \end{equation}
\subsection{Решение}
\label{sec:orge4bc60c}
   \begin{equation}
A = \{x_1\overline{x_2}, x_2x_3, x_1x_2x_3\}\text{ - импликанты}
   \end{equation}
$x_1\overline{x_2}$ - простая импликанта, $x_2x_3$ и $x_1x_2x_3$ - не импликанты.
\section{Задача 4}
\label{sec:org0b480dc}
Построить сокращённую ДНФ функции
\begin{equation}
\tilde{\alpha_f} = (1111 1000 0100 1100)
\end{equation}
\subsection{Решение}
\label{sec:org5f8b21e}
Код максимальной грани: \((0022)\), соответствующая простая импликанта \(\overline{x_1}\overline{x^2}\).
Далее идёт ребро \((0200)\), соответствующее импликанте \(\overline{x_1}\overline{x_2}\overline{x_3}\).
Далее идёт ребро \((2100)\), соответствующее импликанте \(x_2\overline{x_3}\overline{x_4}\).
Следующее ребро \((1102) \rightarrow x_1x_2\overline{x_3}\),
\((1201) \rightarrow x_1\overline{x_3}x_4\), \((2001) \rightarrow \overline{x_2}\overline{x_3}x_4\).
\section{Задача 2.6}
\label{sec:org7304dba}
Найти сокращённую ДНФ методом карты:
\begin{equation}
\tilde{\alpha_f} = (0101 0111)
\end{equation}
\subsection{Решение}
\label{sec:orgadd7e81}
\begin{center}
\begin{tabular}{rrr}
\hline
x\textsubscript{1x}\textsubscript{2x}\textsubscript{3} & 0 & 1\\
\hline
00 & 0 & 1\\
01 & 0 & 1\\
11 & 1 & 1\\
10 & 0 & 1\\
\hline
\end{tabular}
\end{center}
Откуда \(D_f = x_3\vee x_1x_2\).
\section{Задача 2.6(5)}
\label{sec:orgfc1c3b3}
  \begin{equation}
\tilde{\alpha_f} = (0001 1011 1101 1111)
  \end{equation}
\subsection{Решение}
\label{sec:org8659f20}
\begin{center}
\begin{tabular}{rrrrr}
\hline
\(x_1\) & \(x_2\) & \(x_3\) & \(x_4\) & \(f\)\\
\hline
0 & 0 & 0 & 0 & 0\\
0 & 0 & 0 & 1 & 0\\
0 & 0 & 1 & 0 & 0\\
0 & 0 & 1 & 1 & 1\\
0 & 1 & 0 & 0 & 1\\
0 & 1 & 0 & 1 & 0\\
0 & 1 & 1 & 0 & 1\\
0 & 1 & 1 & 1 & 1\\
1 & 0 & 0 & 0 & 1\\
1 & 0 & 0 & 1 & 1\\
1 & 0 & 1 & 0 & 0\\
1 & 0 & 1 & 1 & 1\\
1 & 1 & 0 & 0 & 1\\
1 & 1 & 0 & 1 & 1\\
1 & 1 & 1 & 0 & 1\\
1 & 1 & 1 & 1 & 1\\
\hline
\end{tabular}
\end{center}
Тогда карта Карно будет иметь вид:
\begin{center}
\begin{tabular}{rrrrrl}
\hline
\(x_1x_2x_3x_4\) & 00 & 01 & 11 & 10 & \\
\hline
00 & 0 & 0 & 1 & 0 & \(x_2x_3, x_2\overline{x_4}\)\\
01 & 1 & 0 & 1 & 1 & \(x_1x_2, x_1\overline{x_3}\)\\
11 & 1 & 1 & 1 & 1 & \(x_1x_4\)\\
10 & 1 & 1 & 1 & 0 & \(x_3x_4\)\\
\hline
\end{tabular}
\end{center}
Тогда сокращённая ДНФ имеет вид:
\begin{equation}
D_f = x_2\overline{x_4}\vee x_2x_3\vee x_1x_2\vee x_1x_4\vee x_3x_4\vee x_1\overline{x_3}
\end{equation}
\section{Задача 2.3(1)}
\label{sec:org92f7df4}
Получить сокращённую ДНФ по КНФ:
\begin{equation}
(x_1\vee x_2\vee\overline{x_3})(\overline{x_1}\vee x_2\vee x_3)(\overline{x_2}\vee\overline{x_3})
\end{equation}
\subsection{Решение}
\label{sec:org73bf76f}
\begin{multline}
   (x_1\vee x_2\vee\overline{x_3})(\overline{x_1}\vee x_2\vee x_3)(\overline{x_2}\vee\overline{x_3})
= (x_1x_2\vee x_1x_3 \vee x_2\overline{x_1}\vee x_2\vee x_2x_3\vee\overline{x_1}\overline{x_3}
\vee x_2\overline{x_3})(\overline{x_2}\vee\overline{x_3}) = \\
= (x_2\vee x_1x_3\vee\overline{x_1}\overline{x_3})(\overline{x_2}\vee\overline{x_3})
= x_2\overline{x_3}\vee x_1\overline{x_2}x_3\vee\overline{x_1}\overline{x_2}\overline{x_3}
\vee\overline{x_1}\overline{x_3} = x_2\overline{x_3}\vee\overline{x_1}\overline{x_3}\vee x_1\overline{x_2}x_3
\end{multline}
\section{Задача 2.2(1)}
\label{sec:org0ab1b71}
Построить сокращённую ДНФ по данной ДНФ методом Блейка:
\begin{equation}
\overline{x_1}\overline{x_2}\vee x_1\overline{x_2}x_4\vee x_2\overline{x_3}x_4\vee\overline{x_2}x_4|
\vee\overline{x_1}\overline{x_3}x_4\vee x_1\overline{x_3}x_4|\vee\overline{x_2}\overline{x_3}x_4
\vee\overline{x_3}x_4 = \overline{x_1}\overline{x_2}\vee\overline{x_2}x_4\vee\overline{x_3}x_4
\end{equation}
\section{Задача 2.2(2)}
\label{sec:orgb59fd51}
  \begin{equation}
x_1\overline{x_2}x_3\vee\overline{x_1}x_2\overline{x_4}\vee\overline{x_2}\overline{x_3}x_4|
\vee x_1\overline{x_2}x_4
  \end{equation}
\section{Задача 2.9(1)}
\label{sec:org8d2a2ee}
  \begin{equation}
f(\tilde{x_n}) = x_1\oplus\ldots\oplus x_n
  \end{equation}
Длина сокращённой ДНФ - ?
\subsection{Решение}
\label{sec:org039b8be}
   Максимальные грани - точки $\Rightarrow$ длина сокращённой ДНФ, она же длина СДНФ равна
$2^n - 1$.
\section{Задача 2.9(2)}
\label{sec:orgb3cf195}
Найти длину сокращённой ДНФ функции:
  \begin{equation}
f(\tilde{x_n}) = (x_1\vee x_2\vee x_3)(\overline{x_1}\vee\overline{x_2}\overline{x_3})\oplus
x_4\oplus\ldots\oplus x_n
  \end{equation}
\subsection{Решение}
\label{sec:org3d6364a}
\begin{equation}
f(\tilde{\alpha}) = 1 \Leftrightarrow \begin{cases}
g(\tilde{\alpha}) = 1, \\
h(\tilde{\alpha}) = 0, \\
g(\tilde{\alpha}) = 0, \\
h(\tilde{\alpha}) = 1.
   \end{cases}
\end{equation}
Первому случаю соответствует $2^{n - 4}\cdot6$ максимальных грани, второму - $2\cdot2^{n - 4}$,
итого длина ДНФ составляет $2^{n - 1}$.

\url{http://mk.cs.msu.ru}, лекционные курсы, ОКи, домашние задания там.
\end{document}
