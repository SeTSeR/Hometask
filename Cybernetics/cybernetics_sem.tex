% Created 2020-02-29 Sat 23:31
% Intended LaTeX compiler: pdflatex
\documentclass[11pt]{article}
\usepackage[utf8]{inputenc}
\usepackage[T1]{fontenc}
\usepackage{graphicx}
\usepackage{grffile}
\usepackage{longtable}
\usepackage{wrapfig}
\usepackage{rotating}
\usepackage[normalem]{ulem}
\usepackage{amsmath}
\usepackage{textcomp}
\usepackage{amssymb}
\usepackage{capt-of}
\usepackage{hyperref}
\usepackage{minted}
\usepackage{amsmath}
\usepackage{esint}
\usepackage[english, russian]{babel}
\usepackage{mathtools}
\usepackage{amsthm}
\usepackage[top=0.8in, bottom=0.75in, left=0.625in, right=0.625in]{geometry}
\def\zall{\setcounter{lem}{0}\setcounter{cnsqnc}{0}\setcounter{th}{0}\setcounter{Cmt}{0}\setcounter{equation}{0}}
\newcounter{lem}\setcounter{lem}{0}
\def\lm{\par\smallskip\refstepcounter{lem}\textbf{\arabic{lem}}}
\newtheorem*{Lemma}{Лемма \lm}
\newcounter{th}\setcounter{th}{0}
\def\th{\par\smallskip\refstepcounter{th}\textbf{\arabic{th}}}
\newtheorem*{Theorem}{Теорема \th}
\newcounter{cnsqnc}\setcounter{cnsqnc}{0}
\def\cnsqnc{\par\smallskip\refstepcounter{cnsqnc}\textbf{\arabic{cnsqnc}}}
\newtheorem*{Consequence}{Следствие \cnsqnc}
\newcounter{Cmt}\setcounter{Cmt}{0}
\def\cmt{\par\smallskip\refstepcounter{Cmt}\textbf{\arabic{Cmt}}}
\newtheorem*{Note}{Замечание \cmt}
\author{Sergey Makarov}
\date{\today}
\title{}
\hypersetup{
 pdfauthor={Sergey Makarov},
 pdftitle={},
 pdfkeywords={},
 pdfsubject={},
 pdfcreator={Emacs 26.3 (Org mode 9.3)}, 
 pdflang={English}}
\begin{document}

\tableofcontents


\section{Семинар 1}
\label{sec:orga6cbc83}
\subsection{Задача 1}
\label{sec:org21e084a}
Построить СДНФ функции:
\begin{center}
\begin{tabular}{rrrr}
\hline
\(x_1\) & \(x_2\) & \(x_3\) & f\\
\hline
0 & 0 & 0 & 1\\
0 & 0 & 1 & 0\\
0 & 1 & 0 & 1\\
0 & 1 & 1 & 1\\
1 & 0 & 0 & 0\\
1 & 0 & 1 & 1\\
1 & 1 & 0 & 0\\
1 & 1 & 1 & 1\\
\hline
\end{tabular}
\end{center}
\subsubsection{Решение}
\label{sec:org973e9d1}
   \begin{equation}
D_f^s(x_1, x_2, x_3) = \overline{x_1}\overline{x_2}\overline{x_3}\vee\overline{x_1}x_2\overline{x_3}
\vee\overline{x_1}x_2x_3\vee x_1\overline{x_2}x_3
   \end{equation}
\begin{equation}
K_f^s(x_1, x_2, x_3) = (x_1\vee x_2\vee\overline{x_3})(\overline{x_1}\vee x_2\vee x_3)
(\overline{x_1}\vee\overline{x_2}\vee x_3)(\overline{x_1}\vee\overline{x_2}\vee\overline{x_3})
\end{equation}
\subsection{Задача 2}
\label{sec:org1f5b043}
  \begin{equation}
f = (00101111). \text{ Найти простые импликанты.}
  \end{equation}
\subsubsection{Решение}
\label{sec:org7eabc6a}
   \begin{equation}
A = \{x_1, \overline{x_3}, x_1x_2, x_2\overline{x_3}\}\text{ - импликанты $f$.}
   \end{equation}
$x_1x_2$ - не простая импликанта.
\subsection{Задача 3}
\label{sec:org871d2d4}
Найти простые импликанты функции
  \begin{equation}
f = (01111110).
  \end{equation}
\subsubsection{Решение}
\label{sec:org4319fd5}
   \begin{equation}
A = \{x_1\overline{x_2}, x_2x_3, x_1x_2x_3\}\text{ - импликанты}
   \end{equation}
$x_1\overline{x_2}$ - простая импликанта, $x_2x_3$ и $x_1x_2x_3$ - не импликанты.
\subsection{Задача 4}
\label{sec:orgf016445}
Построить сокращённую ДНФ функции
\begin{equation}
\tilde{\alpha_f} = (1111 1000 0100 1100)
\end{equation}
\subsubsection{Решение}
\label{sec:org4e33bf7}
Код максимальной грани: \((0022)\), соответствующая простая импликанта \(\overline{x_1}\overline{x^2}\).
Далее идёт ребро \((0200)\), соответствующее импликанте \(\overline{x_1}\overline{x_2}\overline{x_3}\).
Далее идёт ребро \((2100)\), соответствующее импликанте \(x_2\overline{x_3}\overline{x_4}\).
Следующее ребро \((1102) \rightarrow x_1x_2\overline{x_3}\),
\((1201) \rightarrow x_1\overline{x_3}x_4\), \((2001) \rightarrow \overline{x_2}\overline{x_3}x_4\).
\subsection{Задача 2.6}
\label{sec:orge68e1ac}
Найти сокращённую ДНФ методом карты:
\begin{equation}
\tilde{\alpha_f} = (0101 0111)
\end{equation}
\subsubsection{Решение}
\label{sec:orga31129b}
\begin{center}
\begin{tabular}{rrr}
\hline
x\textsubscript{1x}\textsubscript{2x}\textsubscript{3} & 0 & 1\\
\hline
00 & 0 & 1\\
01 & 0 & 1\\
11 & 1 & 1\\
10 & 0 & 1\\
\hline
\end{tabular}
\end{center}
Откуда \(D_f = x_3\vee x_1x_2\).
\subsection{Задача 2.6(5)}
\label{sec:org7588cbe}
  \begin{equation}
\tilde{\alpha_f} = (0001 1011 1101 1111)
  \end{equation}
\subsubsection{Решение}
\label{sec:org444d87e}
\begin{center}
\begin{tabular}{rrrrr}
\hline
\(x_1\) & \(x_2\) & \(x_3\) & \(x_4\) & \(f\)\\
\hline
0 & 0 & 0 & 0 & 0\\
0 & 0 & 0 & 1 & 0\\
0 & 0 & 1 & 0 & 0\\
0 & 0 & 1 & 1 & 1\\
0 & 1 & 0 & 0 & 1\\
0 & 1 & 0 & 1 & 0\\
0 & 1 & 1 & 0 & 1\\
0 & 1 & 1 & 1 & 1\\
1 & 0 & 0 & 0 & 1\\
1 & 0 & 0 & 1 & 1\\
1 & 0 & 1 & 0 & 0\\
1 & 0 & 1 & 1 & 1\\
1 & 1 & 0 & 0 & 1\\
1 & 1 & 0 & 1 & 1\\
1 & 1 & 1 & 0 & 1\\
1 & 1 & 1 & 1 & 1\\
\hline
\end{tabular}
\end{center}
Тогда карта Карно будет иметь вид:
\begin{center}
\begin{tabular}{rrrrrl}
\hline
\(x_1x_2x_3x_4\) & 00 & 01 & 11 & 10 & \\
\hline
00 & 0 & 0 & 1 & 0 & \(x_2x_3, x_2\overline{x_4}\)\\
01 & 1 & 0 & 1 & 1 & \(x_1x_2, x_1\overline{x_3}\)\\
11 & 1 & 1 & 1 & 1 & \(x_1x_4\)\\
10 & 1 & 1 & 1 & 0 & \(x_3x_4\)\\
\hline
\end{tabular}
\end{center}
Тогда сокращённая ДНФ имеет вид:
\begin{equation}
D_f = x_2\overline{x_4}\vee x_2x_3\vee x_1x_2\vee x_1x_4\vee x_3x_4\vee x_1\overline{x_3}
\end{equation}
\subsection{Задача 2.3(1)}
\label{sec:org5f83d2b}
Получить сокращённую ДНФ по КНФ:
\begin{equation}
(x_1\vee x_2\vee\overline{x_3})(\overline{x_1}\vee x_2\vee x_3)(\overline{x_2}\vee\overline{x_3})
\end{equation}
\subsubsection{Решение}
\label{sec:org3c9dad9}
\begin{multline}
   (x_1\vee x_2\vee\overline{x_3})(\overline{x_1}\vee x_2\vee x_3)(\overline{x_2}\vee\overline{x_3})
= (x_1x_2\vee x_1x_3 \vee x_2\overline{x_1}\vee x_2\vee x_2x_3\vee\overline{x_1}\overline{x_3}
\vee x_2\overline{x_3})(\overline{x_2}\vee\overline{x_3}) = \\
= (x_2\vee x_1x_3\vee\overline{x_1}\overline{x_3})(\overline{x_2}\vee\overline{x_3})
= x_2\overline{x_3}\vee x_1\overline{x_2}x_3\vee\overline{x_1}\overline{x_2}\overline{x_3}
\vee\overline{x_1}\overline{x_3} = x_2\overline{x_3}\vee\overline{x_1}\overline{x_3}\vee x_1\overline{x_2}x_3
\end{multline}
\subsection{Задача 2.2(1)}
\label{sec:orgf2770d3}
Построить сокращённую ДНФ по данной ДНФ методом Блейка:
\begin{equation}
\overline{x_1}\overline{x_2}\vee x_1\overline{x_2}x_4\vee x_2\overline{x_3}x_4\vee\overline{x_2}x_4|
\vee\overline{x_1}\overline{x_3}x_4\vee x_1\overline{x_3}x_4|\vee\overline{x_2}\overline{x_3}x_4
\vee\overline{x_3}x_4 = \overline{x_1}\overline{x_2}\vee\overline{x_2}x_4\vee\overline{x_3}x_4
\end{equation}
\subsection{Задача 2.2(2)}
\label{sec:org55982d9}
  \begin{equation}
x_1\overline{x_2}x_3\vee\overline{x_1}x_2\overline{x_4}\vee\overline{x_2}\overline{x_3}x_4|
\vee x_1\overline{x_2}x_4
  \end{equation}
\subsection{Задача 2.9(1)}
\label{sec:org6035a54}
  \begin{equation}
f(\tilde{x_n}) = x_1\oplus\ldots\oplus x_n
  \end{equation}
Длина сокращённой ДНФ - ?
\subsubsection{Решение}
\label{sec:orga3e345a}
   Максимальные грани - точки $\Rightarrow$ длина сокращённой ДНФ, она же длина СДНФ равна
$2^n - 1$.
\subsection{Задача 2.9(2)}
\label{sec:orgac0fc14}
Найти длину сокращённой ДНФ функции:
  \begin{equation}
f(\tilde{x_n}) = (x_1\vee x_2\vee x_3)(\overline{x_1}\vee\overline{x_2}\overline{x_3})\oplus
x_4\oplus\ldots\oplus x_n
  \end{equation}
\subsubsection{Решение}
\label{sec:org6d145d5}
\begin{equation}
f(\tilde{\alpha}) = 1 \Leftrightarrow \begin{cases}
g(\tilde{\alpha}) = 1, \\
h(\tilde{\alpha}) = 0, \\
g(\tilde{\alpha}) = 0, \\
h(\tilde{\alpha}) = 1.
   \end{cases}
\end{equation}
Первому случаю соответствует $2^{n - 4}\cdot6$ максимальных грани, второму - $2\cdot2^{n - 4}$,
итого длина ДНФ составляет $2^{n - 1}$.

\url{http://mk.cs.msu.ru}, лекционные курсы, ОКи, домашние задания там.
\section{Домашняя работа 1}
\label{sec:org2618424}
\subsection{Задача 2.3(4)}
\label{sec:org996ec8d}
Представить в виде СДНФ функцию:
\zall
\begin{equation}
f(\tilde{x}^3) = (01010001)
\end{equation}
\subsubsection{Решение}
\label{sec:org27b2fd2}
Функция принимает единичные значения на наборах $001, 011$ и $111$, откуда СДНФ имеет вид:
\begin{equation}
f(\tilde{x}^3) = \overline{x_1}\overline{x_2}x_3\vee\overline{x_1}x_2x_3\vee x_1x_2x_3
\end{equation}
\subsection{Задача 2.1(3)}
\label{sec:org96b7ca1}
Из множества \(A\) ЭК выделить простые импликанты функции \(f\):
\begin{equation}
A = \{x_1, \overline{x_4}, x_2\overline{x_3}, \overline{x_1}\overline{x_2}\overline{x_4}\},
f(\tilde{x}^4) = (1010 1110 1101 1110)
\end{equation}
\subsubsection{Решение}
\label{sec:org39662df}
ЭК $x_1$ не является импликантой функции $f$, так как $f(1111) = 0$.

ЭК $\overline{x_4}$ является простой импликантой функции $f$.

ЭК $x_2\overline{x_3}$ является импликантой функции $f$. ЭК $x_2$ не является импликантой
функции $f$, так как $f(1111) = 0$. ЭК $\overline{x_3}$ также не является импликантой $f$,
так как $f(0001) = 0$, значит $x_2\overline{x_3}$ -- простая импликанта функции $f$.

ЭК $\overline{x_1}\overline{x_2}\overline{x_4}$ является импликантой функции $f$, но она не
является простой, так как содержит множитель $\overline{x_4}$, который, как было указано ранее,
является импликантой $f$.
\subsection{Задача 2.5(2, 6)}
\label{sec:orgba1a846}
Изобразив множество \(N_f\) функции \(f(\tilde{x}^n)\) в \(B^n\), найти коды максимальных интервалов
и построить сокращённую ДНФ:
\begin{equation}
\tilde{\alpha}_f = (0101 0011)
\end{equation}
\begin{equation}
\tilde{\alpha}_f = (0001 0111 1110 1111)
\end{equation}
\subsubsection{Решение}
\label{sec:orgdba724c}
Для первой функции находим две максимальные грани с кодами \((112)\) и \((021)\), попытки расширить
которые приводят к выходу из \(N_f\). Отсюда получаем сокращённую ДНФ:
\begin{equation}
f(\tilde{x}^3) = \overline{x_1}x_3\vee x_1x_2
\end{equation}
Для второй функции максимальные грани имеют коды \((2112)\), \((1202)\), \((0121)\) и \((0211)\).
Попытки расширить эти грани приводят к выходу из \(N_f\). Отсюда получаем сокращённую ДНФ:
\begin{equation}
f(\tilde{x}^4) = x_2x_3\vee x_1\overline{x_3}\vee\overline{x_1}x_2x_4\vee\overline{x_1}x_3x_4
\end{equation}
\subsection{Задача 2.6(2, 6)}
\label{sec:org412b794}
Найти сокращённую ДНФ функции с помощью минимизирующей карты:
\begin{equation}
\tilde{\alpha_f} = (1101 1011)
\end{equation}
\begin{equation}
\tilde{\alpha_f} = (0011 1101 1111 1101)
\end{equation}
Карта Карно для первой функции:
\begin{center}
\begin{tabular}{rrrrr}
\hline
\(x_1\) $\backslash$ \(x_2x_3\) & 00 & 01 & 11 & 10\\
\hline
0 & 1 & 1 & 1 & 0\\
1 & 1 & 0 & 1 & 1\\
\hline
\end{tabular}
\end{center}
В данном случае максимальными по включению будут прямоугольники \((00-)\), \((-00)\), \((0-1)\),
\((-11)\), \((11-)\), \((1-0)\), откуда сокращённая ДНФ имеет вид:
\begin{equation}
f(\tilde{x}^3) = \overline{x_1}\overline{x_2}\vee\overline{x_2}\overline{x_3}\vee\overline{x_1}x_3
\vee x_2x_3\vee x_1x_2\vee x_1\overline{x_3}
\end{equation}
Карта Карно для второй функции:
\begin{center}
\begin{tabular}{rrrrr}
\hline
\(x_1x_2\) $\backslash$ \(x_3x_4\) & 00 & 01 & 11 & 10\\
\hline
00 & 0 & 0 & 1 & 1\\
01 & 1 & 1 & 1 & 0\\
11 & 1 & 1 & 1 & 0\\
10 & 1 & 1 & 1 & 1\\
\hline
\end{tabular}
\end{center}
Максимальными по включению здесь будут прямоугольники \((10--), (-10-), (--11), (-010)\),
которые дают сокращённую ДНФ:
\begin{equation}
f(\tilde{x}^4) = x_1\overline{x_2}\vee x_2\overline{x_3}\vee x_3x_4\vee\overline{x_2}x_3\overline{x_4}
\end{equation}
\subsection{Задача 2.2(3, 4)}
\label{sec:orgd290ba3}
По заданной ДНФ методом Блейка получить сокращённую ДНФ:
\begin{equation}
D = x_1\vee\overline{x_1}x_2\vee\overline{x_1}\overline{x_2}x_3\vee\overline{x_1}x_2x_3x_4
\end{equation}
\begin{equation}
D = x_1\overline{x_2}x_4\vee\overline{x_1}\overline{x_2}x_3\vee\overline{x_3}\overline{x_4}
\end{equation}
\subsubsection{Решение}
\label{sec:orgb2026b3}
\begin{multline}
x_1\vee\overline{x_1}x_2\vee\overline{x_1}\overline{x_2}x_3\vee\overline{x_1}x_2x_3x_4 = \\
= x_1\vee\overline{x_1}x_2\vee\overline{x_1}\overline{x_2}x_3\vee\overline{x_1}x_2x_3x_4
\vee x_2\vee\overline{x_2}x_3\vee x_2x_3x_4\vee\overline{x_1}x_3\vee\overline{x_1}x_3x_4 = \\
= x_1\vee x_2\vee\overline{x_1}x_3\vee\overline{x_2}x_3\vee
= x_1\vee x_2\vee\overline{x_1}x_3\vee\overline{x_2}x_3\vee x_3
= x_1\vee x_2\vee x_3
\end{multline}
\begin{multline}
x_1\overline{x_2}x_4\vee\overline{x_1}\overline{x_2}x_3\vee\overline{x_3}\overline{x_4}
= x_1\overline{x_2}x_4\vee\overline{x_1}\overline{x_2}x_3\vee\overline{x_3}\overline{x_4}\vee
\overline{x_2}x_3x_4\vee\overline{x_1}\overline{x_2}\overline{x_4}\vee x_1\overline{x_2}\overline{x_3} = \\
= x_1\overline{x_2}x_4\vee\overline{x_1}\overline{x_2}x_3\vee\overline{x_3}\overline{x_4}\vee
\overline{x_2}x_3x_4\vee\overline{x_1}\overline{x_2}\overline{x_4}\vee x_1\overline{x_2}\overline{x_3}\vee
\overline{x_2}\overline{x_3}\overline{x_4}\vee x_1\overline{x_2}x_4\vee\overline{x_1}\overline{x_2}x_3 = \\
= x_1\overline{x_2}x_4\vee\overline{x_1}\overline{x_2}x_3\vee\overline{x_3}\overline{x_4}
\vee\overline{x_2}x_3x_4\vee\overline{x_1}\overline{x_2}\overline{x_4}\vee x_1\overline{x_2}\overline{x_3}
\end{multline}
\subsection{Задача 2.3(3, 4)}
\label{sec:orga31e1b8}
Построить сокращённую ДНФ по заданной КНФ:
\begin{equation}
(x_1\vee\overline{x_2}\vee\overline{x_3})(\overline{x_1}\vee x_2)(x_1\vee x_2\vee x_3)
\end{equation}
\begin{equation}
(x_1\vee x_2\vee x_3)(\overline{x_1}\vee\overline{x_2}\vee\overline{x_3})
\end{equation}
\subsubsection{Решение}
\label{sec:org6b8ea13}
\begin{multline}
(x_1\vee\overline{x_2}\vee\overline{x_3})(\overline{x_1}\vee x_2)(x_1\vee x_2\vee x_3) =
(x_1x_2\vee\overline{x_1}\overline{x_2}\vee\overline{x_1}\overline{x_3}\vee x_2\overline{x_3})
(x_1\vee x_2\vee x_3) = \\
= x_1x_2\vee x_1x_2x_3\vee\overline{x_1}\overline{x_2}x_3\vee\overline{x_1}x_2\overline{x_3}\vee
x_1x_2\overline{x_3}\vee x_2\overline{x_3} = x_1x_2\vee x_2\overline{x_3}\vee\overline{x_1}\overline{x_2}x_3
\end{multline}
\begin{equation}
(x_1\vee x_2\vee x_3)(\overline{x_1}\vee\overline{x_2}\vee\overline{x_3})
= x_1\overline{x_2}\vee x_1\overline{x_3}\vee\overline{x_1}x_2\vee x_2\overline{x_3}\vee
\overline{x_1}x_3\vee\overline{x_2}x_3
\end{equation}
\subsection{Задача 2.9(6)}
\label{sec:orgaaa3a41}
Найти длину сокращённой ДНФ функции \(f\):
\begin{equation}
f(\tilde{x}^n) = (x_1\vee\ldots\vee x_n)(x_1\vee\ldots\vee x_k\vee\overline{x}_{k + 1}\vee\ldots\vee\overline{x}_n)
\end{equation}
\subsubsection{Решение}
\label{sec:org5900cfb}
\begin{multline}
(x_1\vee\ldots\vee x_n)(x_1\vee\ldots\vee x_k\vee\overline{x}_{k + 1}\vee\ldots\overline{x}_n) =
(\vee_{i = 1}^kx_k)\vee(\vee_{i, j = 1}^{k, n}x_ix_j)\vee(\vee_{i = k + 1, j = 1}^{n}\overline{x}_ix_j) = \\
= (\vee_{i = 1}^kx_k)\vee(\vee_{i, j = k + 1}^n\overline{x_i}x_j)\vee(\vee_{i, j = k + 1}^nx_ix_j)
\end{multline}
Всего получаем
\begin{equation}
k + 2\frac{(n - k)(n - k + 1)}2 = k + (n - k)(n - k + 1) = n + (n - k)^2 = n^2 - 2nk + n + k^2
\end{equation}
слагаемых.
\end{document}
