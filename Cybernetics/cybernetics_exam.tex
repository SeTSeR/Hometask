% Created 2020-06-06 Sat 20:05
% Intended LaTeX compiler: xelatex
\documentclass[11pt]{article}
\usepackage{graphicx}
\usepackage{grffile}
\usepackage{longtable}
\usepackage{wrapfig}
\usepackage{rotating}
\usepackage[normalem]{ulem}
\usepackage{amsmath}
\usepackage{textcomp}
\usepackage{amssymb}
\usepackage{capt-of}
\usepackage{hyperref}
\usepackage{minted}
\usepackage{polyglossia}
\setmainlanguage{russian}
\usepackage{esint}
\usepackage{mathtools}
\usepackage{amsthm}
\usepackage[top=0.8in, bottom=0.75in, left=0.625in, right=0.625in]{geometry}
\usepackage{fontspec}
\setmainfont{CMU Serif}
\setsansfont{CMU Sans Serif}
\setmonofont{CMU Typewriter Text}
\def\zall{\setcounter{lem}{0}\setcounter{cnsqnc}{0}\setcounter{th}{0}\setcounter{Cmt}{0}\setcounter{equation}{0}\setcounter{stnmt}{0}}
\newcounter{lem}\setcounter{lem}{0}
\def\lm{\par\smallskip\refstepcounter{lem}\textbf{\arabic{lem}}}
\newtheorem*{Lemma}{Лемма \lm}
\newcounter{th}\setcounter{th}{0}
\def\th{\par\smallskip\refstepcounter{th}\textbf{\arabic{th}}}
\newtheorem*{Theorem}{Теорема \th}
\newcounter{cnsqnc}\setcounter{cnsqnc}{0}
\def\cnsqnc{\par\smallskip\refstepcounter{cnsqnc}\textbf{\arabic{cnsqnc}}}
\newtheorem*{Consequence}{Следствие \cnsqnc}
\newcounter{Cmt}\setcounter{Cmt}{0}
\def\cmt{\par\smallskip\refstepcounter{Cmt}\textbf{\arabic{Cmt}}}
\newtheorem*{Note}{Замечание \cmt}
\newcounter{stnmt}\setcounter{stnmt}{0}
\def\st{\par\smallskip\refstepcounter{stnmt}\textbf{\arabic{stnmt}}}
\newtheorem*{Statement}{Утверждение \st}
\author{Sergey Makarov}
\date{\today}
\title{}
\hypersetup{
 pdfauthor={Sergey Makarov},
 pdftitle={},
 pdfkeywords={},
 pdfsubject={},
 pdfcreator={Emacs 28.0.50 (Org mode 9.3)}, 
 pdflang={Russian}}
\begin{document}

\tableofcontents

\pagebreak
\section{Представление функций алгебры логики(ФАЛ) дизъюнктивными нормальными формами (ДНФ) и его геометрическая интерпретация. Совершенная ДНФ и критерий единственности ДНФ.}
\label{sec:org525808a}
Множество \(B^n\), где \(B = \{0, 1\}, n \in \mathbb{N}\) называется \emph{единичным кубом} или \emph{гиперкубом} размерности \(n\). Отношение перестановочности разбивает куб \(B^n\) на классы эквивалентности -- \emph{слои} куба \(B_0^n, \ldots, B_n^n\), где \(B_i^n\) -- множество наборов с \(i\) единицами, \(|B_i^n| = \begin{pmatrix}n \\ i\end{pmatrix}\).

На гиперкубе введём отношение лексикографического линейного порядка, которое задаётся нумерацией \(\nu\):
\begin{equation*}
\nu(\alpha_1, \ldots, \alpha_n) = \sum_{i = 1}^n\alpha_i2^{n - i}
\end{equation*}
Иными словами, в качестве номера набора \(\alpha_1, \ldots, \alpha_n\) объявляется число, двоичная запись которого, дополненная слева нулями до длины \(n\), совпадает с набором \(\alpha_1, \ldots, \alpha_n\). Множество наборов, имеющих номера из отрезка \([a, b]\), называется \emph{отрезком куба} B\textsuperscript{n}.

\emph{Расстоянием Хэмминга} \(\rho\)(a, b) между наборами \(a\) и \(b\) называется количество разрядов, в которых наборы отличаются друг от друга. Наборы, находящиеся друг от друга на расстоянии \(n\), называются \emph{противоположными}, а наборы, отличающиеся только в одном разряде, называются \emph{соседними}. Множество \(\{b: \rho(a, b) = t\}\) называется \emph{сферой} радиуса \(t\) с центром в \(a\), соответственно, множество \(\{b: \rho(a, b) \leq t\}\) называется \emph{шаром} радиуса \(t\) с центром в \(a\). i-й слой куба \(B^n\) является сферой радиуса \(i\) с центром в наборе \(\widetilde{0} = (0, 0, \ldots, 0)\) и сферой радиуса \(n - i\) с центром в наборе \(\widetilde{1} = (1, 1, \ldots, 1)\).

На множестве \(B^n\) введём отношение частичного порядка:
\begin{equation*}
\alpha = (\alpha_1, \ldots, \alpha_n) \leq \beta = (\beta_1, \ldots, \beta_n) \Leftrightarrow \alpha_i \leq \beta_i \forall i = \overline{1, n}.
\end{equation*}
При этом \(a < b\), если \(a \leq b\)  и \(a \neq b\). Наборы \(a, b\) для которых \(a \leq b\) или \(b \leq a\) называются \emph{сравнимыми}, в противном случае \emph{несравнимыми}.

Для набора \(\gamma = (\gamma_1, \ldots, \gamma_n)\) длины \(n\) над множеством \(\{0, 1, 2\}\) через \(\Gamma_{\gamma}\) обозначим множество всех тех наборов \(\alpha = (\alpha_1, \ldots, \alpha_n)\) куба \(B^n\), для которых \(\alpha_i = \gamma_i \forall i \in \overline{1, n}: \gamma_i \neq 2\). Это множество называется \emph{гранью} куба \(B^n\), число \(n - r\), равное числу двоек в наборе \(\gamma\), называется \emph{размерностью} этой грани, а число \(r\) -- её \emph{рангом}. Грань \(\Gamma_{\gamma}\) представляет собой подкуб размерности \(n - r\) куба \(B^n\).

Будем рассматривать счётный упорядоченный алфавит булевых переменных(БП) \(\mathfrak{X} = \{x_1, x_2, \ldots, x_n, \ldots\}\) и будем рассматривать функции алгебры логики(ФАЛ) над переменными из \(\mathfrak{X}\). Множество таких функций будем обозначать \(P_2(\mathfrak{X})\) или \(P_2\), или \(P_2(n)\). Для задания ФАЛ \(f\) из \(P_2(n)\) можно использовать её таблицу значений -- матрицу \(M\) из множества \(B^{2^n\times(n + 1)}\), i-я строка, \(i \in \overline{1, 2^n}\) которой имеет вид
\begin{equation*}
M(i, [1, n + 1]) = (\alpha, f(\alpha))
\end{equation*}
Столбец \(M([1, 2^n], n = 1)\), однозначно задающий ФАЛ \(f\), считается её столбцом значений и записывается в виде строки, обозначаемой \(\widetilde{\alpha}_f\). Кроме того, ФАЛ \(f\) однозначно задаётся своим \emph{характеристическим множеством}:
\begin{equation*}
N_f = \{\alpha | \alpha \in B^n, f(\alpha) = 1\}
\end{equation*}
В этом случае ФАЛ \(f\) является характеристической функцией множества \(N_f\).

Функции \(\&, \vee\) и \(\oplus\) коммутативны и ассоциативны, а \(\&\) также дистрибутивна относительно \(\vee\) и \(\oplus\). Кроме того, имеют место следующие тождества:
\begin{equation}
x\cdot0 = x\cdot\overline{x} = x\oplus x = 0, x\vee1 = x\vee\overline{x} = x\oplus\overline{x} = 1,
\end{equation}
\begin{equation}
x\cdot x = x\vee x = x\vee 0 = x\oplus 0
\end{equation}
\begin{equation}
\label{eq:merge}
x_1\vee x_1x_2 = x_1
\end{equation}

ФАЛ вида
\begin{equation*}
f(x_1, \ldots, x_n) = a_1x_1 \oplus \ldots \oplus a_nx_n \oplus a_0
\end{equation*}
называются \emph{линейными}. Существенными БП этой ФАЛ являются те и только те \(x_i\), для которых \(a_i = 1\).

Рассмотрим формулы над множеством
\begin{equation*}
\text{Б}_0 = \{x_1\cdot x_2, x_1\vee x_2, \overline{x}_1\}
\end{equation*}
Функции \(x_i\) и \(\overline{x}_i\) будем называть \emph{буквами} БП \(x_i\) и будем считать, что
\(x_i^0 = \overline{x}_i, x_i^1 = x_i\). Конъюнкция(дизъюнкция) \(r, 1 \leq r \leq n\), букв различных БП из \(X(n)\) называется \emph{элементарной конъюнкцией(элементарной дизъюнкцией) ранга \(r\) от БП \(X(n)\)}. Дизъюнкция различных элементарных конъюнкций называется \emph{дизъюнктивной нормальной формой} (ДНФ), а конъюнкция различных элементарных дизъюнкций называется \emph{конъюнктивной нормальной формой} (КНФ). При этом ДНФ (КНФ) считается \emph{совершенной}, если все её ЭК(соответственно, ЭД) существенно зависят от одних и тех же БП, а их ранг равен числу этих БП. Число ЭК в ДНФ(ЭД в КНФ) \(\mathfrak{A}\) называется \emph{длиной} ДНФ(КНФ) и обозначается \(\lambda(\mathfrak{A})\). Любую ФАЛ \(f(x_1, \ldots, x_n)\), отличную от константы, можно представить в виде её совершенных ДНФ и КНФ следующим образом:
\begin{equation}
f(x_1, \ldots, x_n) = \bigvee\limits_{(\alpha_1, \ldots, \alpha_n) \in N_f}x_1^{\alpha_1}\ldots x_n^{\alpha_n} =
\bigwedge\limits_{(\beta_1, \ldots, \beta_n) \in \overline{N}_f}(x_1^{\overline{\beta}_1}\lor\ldots\lor x_n^{\overline{\beta}_n})
\end{equation}
Поскольку любую ФАЛ \(f\) из \(P_2(n)\), отличную от нуля, можно представить в виде совершенной ДНФ, а ФАЛ \(f \equiv 0\) представляется формулой \(x_1\cdot\overline{x}_1\), то множество Б\textsubscript{0} является базисом \(P_2\).

Совершенную ДНФ можно обобщить с помощью \emph{представления Шеннона}:
\begin{equation}
f(x', x'') = \bigwedge\limits_{\sigma'' = (\sigma_{q + 1}, \ldots, \sigma_n)}x_{q + 1}^{\sigma_{q + 1}}\ldots
x_n^{\sigma_n}f_{\sigma''}(x')
\end{equation}
Здесь \(q \in [0, n], x' = (x_1, \ldots, x_q), x'' = (x_{q + 1}, \ldots, x_n)\) и \(f_{\sigma''}(x') = f(x', \sigma'')\). При \(q = 0\) все ФАЛ \(f_{\sigma''}(x')\) являются константными.

Представление ФАЛ в виде ДНФ или КНФ имеет простую геометрическую интерпретацию. Пусть
\begin{equation}
\label{eq:dnf}
f(x_1, \ldots, x_n) = K_1 \lor \ldots K_s = \mathfrak{A},
\end{equation}
\begin{equation}
\label{eq:knf}
f(x_1, \ldots, x_n) = J_1\ldots J_t = \mathfrak{B},
\end{equation}
где \(K_1, \ldots, K_s(J_1, \ldots, J_t)\) -- различные ЭК(ЭД) от БП \(x_1, \ldots, x_n\). Эти представления эквивалентны следующим покрытиям множеств \(N_f\) и \(\overline{N}_f\) гранями куба \(B^n\):
\begin{equation}
\label{eq:covnf}
N_f = N_{K_1}\cup\ldots\cup N_{K_s},
\end{equation}
\begin{equation}
\label{eq:convnotnf}
\overline{N}_f = \overline{N}_{J_1}\cup\ldots\cup\overline{N}_{J_t}.
\end{equation}

\begin{Lemma}
Совершенная ДНФ ФАЛ $f, f \in P_2(n)$, является единственной ДНФ от БП $X(n)$, которая реализует эту ФАЛ, тогда и только тогда, когда в $N_f$ нет соседних наборов.
\end{Lemma}
\begin{proof}
Совершенная ДНФ ФАЛ $f$ является единственной ДНФ от БП $X(n)$, которая реализует эту ФАЛ, тогда и только тогда, когда в множестве $N_f$ нет граней размерности больше 0, что и означает, что в $N_f$ нет соседних наборов.
\end{proof}
\pagebreak
\zall
\section{Сокращённая ДНФ и способы её построения.}
\label{sec:orgc5f4e66}
Будем говорить, что ФАЛ \(f'\) \emph{имплицирует} ФАЛ \(f''\) или ФАЛ \(f''\) \emph{поглощает} ФАЛ \(f'\), если \(N_{f'} \subseteq N_{f''}\), т. е. \(f' \rightarrow f'' \equiv 1\). ЭК, имплицирующая ФАЛ \(f\), называется \emph{импликантой} этой ФАЛ. \(f' \rightarrow f'' \Leftrightarrow f'' = f'\lor f'' \text{ или } f' = f'\cdot f''\).

ДНФ \(\mathfrak{A}\) вида \eqref{eq:dnf} будем называть ДНФ \emph{без поглощений} ЭК, если ни одна из граней \(N_{K_1}, \ldots, N_{K_s}\) не содержится ни в одной из других граней покрытия \eqref{eq:covnf}. Это означает, что ни одна из ЭК \(K_i, i = \overline{1, s}\) не является импликантой другой ЭК \(K_j, j = \overline{1, s}, i \neq j\). Используя тождество поглощения \eqref{eq:merge}, можно из любой ДНФ \(\mathfrak{A}\) получить ДНФ \(\widehat{\mathfrak{A}}\) без поглощений ЭК.

Импликанта \(K\) ФАЛ \(f\) называется \emph{простой импликантой} этой ФАЛ, если она не поглощается никакой другой импликантой этой ФАЛ, т. е. не существует такой импликанты \(f K'\) такой, что \(K = K'\cdot K''\). Это означает, что в простую импликанту \(f\) не входят буквы несущественных БП \(f\) и из любой импликанты \(f\) можно получить простую импликанту удалением некоторых букв. Из последнего следует, что любая импликанта ФАЛ \(f\) имплицирует некоторую простую импликанту \(f\), что даёт геометрический смысл простой импликанты: простые импликанты ФАЛ \(f\) соответствуют максимальным по включению граням \(N_f\).

Дизъюнкция всех простых импликант ФАЛ \(f\) называется её \emph{сокращённой} ДНФ. Сокращённая ДНФ является ДНФ без поглощений и ей соответствует покрытие множества \(N_f\) всеми максимальными по включению гранями множества \(N_f\) этой ФАЛ, что даёт геометрический метод построения сокращённой ДНФ.

Для более наглядного построения сокращённой ДНФ для ФАЛ \(f\) в случае небольшого количества переменных, её часто представляют в виде \emph{карты Карно}. Переменные \(f\) делятся на две группы, количество переменных в которых отличается не более, чем на 1, в шапке таблицы записываются значения наборов из этих групп в порядке кода Грея(т. е. так, чтобы соседние наборы отличались ровно в одном разряде). В ячейках таблицы записываются соответствующие значения функции, а противоположные стороны таблицы отождествляются по принципу "тора". Квадрат или прямоугольник из соседних клеток таблицы с площадью, равной степенью двойки, соответствует некоторой грани булева куба.
  \begin{Theorem}
Пусть $\mathfrak{A'}$ и $\mathfrak{A''}$ -- сокращённые ДНФ ФАЛ $f'$ и $f''$ соответственно, а ДНФ $\mathfrak{A}$ без поглощений получается из формулы $\mathfrak{A'}\cdot\matfrak{A''}$ путём раскрытия скобок и приведения подобных слагаемых. Тогда $\mathfrak{A}$ -- сокращённая ДНФ ФАЛ $f = f'\cdot f''$.
  \end{Theorem}
  \begin{proof}
Достаточно показать, что в $\mathfrak{A}$ входит любая простая импликанта ФАЛ $f$. Пусть ЭК $K$ является простой импликантой ФАЛ $f$, а значит, является импликантой как $f'$, так и $f''$. Поскольку $\mathfrak{A'}$ и $\mathfrak{A''}$ являются сокращёнными ДНФ, в них найдутся ЭК $K'$ и $K''$, имплицириуемые ЭК $K$. Это значит, что в ДНФ $\mathfrak{A}$ войдёт ЭК $\widetilde{K}$, имплицируемая $K'\cdot K''$. Эта ЭК получится в результате раскрытия скобок и приведения подобных в формуле $\mathfrak{A'}\cdot\mathfrak{A''}$. Поскольку ЭК $K$ имплицирует ФАЛ $K'\cdot K''$, то $K$ имплицирует и $\widetilde{K}$. Поскольку $\widetilde{K}$ является импликантой $f$ и при этом имплицируется $K$, то $\widetile{K} = K$, так как $K$ -- простая импликанта $f$.
  \end{proof}
\begin{Consequence}
Если ДНФ $\mathfrak{A}$ без поглощений получается из КНФ $\mathfrak{B}$ ФАЛ $f$ в результате раскрытия скобок и приведения подобных, то $\mathfrak{A}$ -- сокращённая ДНФ ФАЛ $f$.
\end{Consequence}
\begin{proof}
Доказательство проводится индукцией по числу множителей в КНФ $\mathfrak{B}$.
\end{proof}

Метод Блейка позволяет получить сокращённую ДНФ ФАЛ \(f\) из произвольной ДНФ этой ФАЛ с помощью эквивалентных преобразований на основе тождества обобщённого склеивания:
  \begin{equation*}
x_1x_2\lor\overline{x}_1x_3 = x_1x_2\lor\overline{x}_1x_3\lor x_2x_3
  \end{equation*}

Любая ДНФ \(\mathfrak{A'}\), которую можно получить из ДНФ \(\mathfrak{A}\) путём формирования в ней с помощью тождеств ассоциативности и коммутативности подформул вида \(x_iK'\lor\overline{x}_iK''\), применения к этим подформулам тождества обобщённого склеивания:
  \begin{equation}
  \label{eq:gengluing}
x_iK'\lor\overline{x}_iK'' = x_iK'\lor\overline{x}_iK''\lor K'K''
  \end{equation}
и последующего приведения подобных слагаемых, называется \emph{расширением} ДНФ \(\mathfrak{A}\). Расширение \(\mathfrak{A'}\) ДНФ \(\mathfrak{A}\) называется \emph{строгим}, если \(\mathfrak{A'}\) содержит ЭК, не являющуюся импликантой ни одной ЭК из \(\mathfrak{A}\). Сокращённая ДНФ не имеет строгих расширений. Также, в результате построения последовательных строгих расширений из любой ДНФ можно получить ДНФ без поглощений ЭК, которая не имеет строгих расширений. Процесс построения строгих расширений сойдётся, поскольку число импликант функции \(f\) конечно, а все слагаемые, добавляемые при построении расширений, являются импликантами \(f\).
\begin{Theorem}
ДНФ без поглощений ЭК является сокращённой ДНФ тогда и только тогда, когда она не имеет строгих расширений.
\end{Theorem}
\begin{proof}
Достаточно убедиться в том, что ДНФ $\mathfrak{A}$ без поглощений ЭК, не имеющая строгих расширений, содержит все простые импликанты реализуемой ею ФАЛ $f$. Пусть $X(n) = \{x_1, \ldots, x_n\}$ -- множество БП ДНФ $\mathfrak{A}$, а $K$ -- простая импликанта $f$, которая не входит в $\mathfrak{A}$. Рассмотрим множество $\mathcal{K}$, состоящее из всех тех элементарных конъюнкций от БП $X(n)$, которые являются импликантами $f$, но не являются импликантами ни одной ЭК из $\mathfrak{A}$. Множество $\mathcal{K}$ непусто, поскольку содержит $K$, но при этом не содержит ЭК ранга $n$, поскольку любая ЭК вида $x_1^{\alpha_1}\ldots x_n^{\alpha_n}$, где $\alpha = (\alpha_1, \ldots, \alpha_n) \in N_f$, является импликантой той ЭК из $\mathfrak{A}$, которая обращается в 1 на наборе $\alpha$.

Пусть теперь $k$ -- ЭК максимального ранга в $\mathcal{K}$, причём $R(k) < n$, и пусть буквы некоторой БП $x_i, 1 \leq i \leq n$, не входят в $k$. Тогда, в силу выбора ЭК $k$ и свойств ДНФ $\mathfrak{A}$, ЭК вида $x_i\cdot k$ (вида $\overline{x}_i\cdot k$) должна быть импликантой некоторой ЭК вида $x_i\cdot K'$ (вида $\overline{x}_i\cdot K''$) из $\mathfrak{A}$, где ЭК $K'$ и $K''$ состоят из букв ЭК $k$. Следовательно, ЭК $k$ является импликантой ЭК $\widetilde{K}$, равной $K'\cdot K''$, а ЭК $\widetilde{K}$, в свою очередь, является импликантой некоторой ЭК из $\mathfrak{A}$. Действительно, ДНФ $\mathfrak{A}$ не имеет строгих расширений и поэтому содержит ЭК, которая имплицируется ЭК $\widetilde{K}$, получающейся из подформулы $x_iK'\lor\overline{x}_iK''$ в результате преобразования \eqref{eq:gengluing}. Таким образом, ЭК $k$ является импликантой некоторой ЭК из $\mathfrak{A}$ и не может входить в $\mathcal{K}$. Полученное противоречие доказывает, что ЭК $K$ входит в $\mathcal{A}$.
\end{proof}
\begin{Consequence}
Из любой ДНФ $\mathfrak{A}$ ФАЛ $f$ можно получить сокращённую ДНФ этой ФАЛ в результате построения последовательных строгих расширений до получения ДНФ без поглощений ЭК, не имеющих строгих расширений.
\end{Consequence}
\begin{proof}
Из любой ДНФ $\mathfrak{A}$ ФАЛ $f$ можно получить ДНФ без поглощений ЭК, которая не имеет строгих расширений. По доказанной теореме эта ДНФ является сокращённой ДНФ.
\end{proof}
\pagebreak
\zall
\section{Тупиковая ДНФ, ядро и ДНФ пересечение тупиковых. ДНФ Квайна, критерий вхождения простых импликант в тупиковые ДНФ и его локальность.}
\label{sec:org9be750a}
Будем говорить, что ДНФ \(\mathfrak{A}\), реализующая ФАЛ \(f\), является \emph{тупиковой} ДНФ, если \(f \neq \mathfrak{A'}\) для любой ДНФ \(\mathfrak{A'}\), полученной из \(\mathfrak{A}\) в результате удаления некоторых букв или целых ЭК. Таким образом, в тупиковую ДНФ могут входить только простые импликанты этой ФАЛ и \(\mathfrak{A}\) является ДНФ без поглощений ЭК. С геометрической точки зрения тупиковая ДНФ \(\mathfrak{A}\) ФАЛ \(f\) задаёт тупиковое покрытие множества \(N_f\) максимальными гранями ФАЛ \(f\) и обратно. Это означает, что никакая из граней, соответствующих некоторой ЭК ДНФ \(f\) не содержится в объединении граней, соответствующих остальным ЭК.

Построение тупиковых ДНФ является промежуточным этапом при построении \emph{минимальной} (\emph{кратчайшей}) ДНФ, т. е. ДНФ, имеющей минимальный ранг (длину) среди всех ДНФ, реализующих \(f\). Минимальная ДНФ всегда есть тупиковая, а среди кратчайших ДНФ всегда найдётся тупиковая.
При построении тупиковых ДНФ бывает полезно знать ДНФ \emph{пересечение тупиковых} (ДНФ \(\cap T\)) ФАЛ \(f\), т. е. дизъюнкцию всех тех различных простых импликант этой ФАЛ, которые входят в любую тупиковую ФАЛ \(f\).

Набор \(\alpha \in B^n\) называется \emph{ядровой точкой} ФАЛ \(f\), если \(\alpha \in N_f\) и \(\alpha\) покрывается только одной максимальной гранью \(f\). При этом максимальная грань \(N_k\), содержащая эту точку, называется \emph{ядровой гранью} \(f\), а совокупность различных ядровых граней называется \emph{ядром} \(f\).

  \begin{Lemma}
ДНФ $\cap T$ ФАЛ $f$ состоит из тех и только тех простых импликант ФАЛ $f$, которые соответствуют ядровым граням этой ФАЛ.
  \end{Lemma}
  \begin{proof}
Пусть тупиковая ДНФ $\mathfrak{A}$ ФАЛ $f$ не включает в себя простую импликанту $K$, которая соответствует ядровой грани $N_K$ ФАЛ $f$, содержащей ядровую точку $\alpha$ этой ФАЛ. Тогда ни одна другая максимальная грань не содержит точки $\alpha$ и поэтому все отличные от $K$ простые импликанты $f$ обращаются в 0 на наборе $\alpha$. Это значит, что ДНФ $\mathfrak{A}$ также равна нулю на этом наборе, т. е. $f(\alpha) = 0$. Полученное противоречие с тем, что $\alpha \in N_f$ означает, что ЭК $K$ должна входить в любую тупиковую ДНФ ФАЛ $f$.

Пусть теперь простая импликанта $K$ ФАЛ $f$ соответствует грани $N_K$, не входящей в ядро ФАЛ $f$. Это значит, что каждая точка грани $N_K$ покрывается некоторой другой максимальной гранью, отличной от $N_K$. Следовательно, все отличные от $N_K$ максимальные грани ФАЛ $f$ образуют покрытие множества $N_f$, из которого можно выделить тупиковое подпокрытие, соответствующее тупиковой ДНФ $f$, не содержащей ЭК $K$.
  \end{proof}

При построении тупиковых ДНФ ФАЛ \(f\) наряду с ДНФ пересечение тупиковых полезно знать ДНФ \emph{сумма тупиковых} (ДНФ \(\Sigma T\)) ФАЛ \(f\), т. е. дизъюнкцию всех тех различных простых импликант этой ФАЛ, которые входят хотя бы в одну тупиковую ДНФ ФАЛ \(f\). ДНФ \(\Sigma T\) \(f\) в общем случае не реализует функцию \(f\) и даже может быть пустой, в то время как ДНФ \(\Sigma T\) ФАЛ \(f\) всегда реализует эту ФАЛ, содержится в её сокращённой ДНФ и может с ней совпадать.

Будем называть ФАЛ \emph{ядровой}, если все её максимальные грани являются ядровыми. Из только что доказанной леммы следует, что сокращённая ДНФ ядровой ФАЛ является её единственной тупиковой ДНФ.

ДНФ, получающаяся из сокращённой ДНФ ФАЛ \(f\) удалением тех ЭК \(K\), для которых грань \(N_K\) покрывается ядром ФАЛ \(f\), но не входит в него, называется \emph{ДНФ Квайна} этой ФАЛ. ДНФ Квайна \(f\) включает в себя ДНФ \(\Sigma T\) и содержится в её сокращённой ДНФ.

Для ФАЛ \(f\) и набора \(\alpha \in N_f\) обозначим через \(\Pi_{\alpha}(f)\) множество всех происходящих через \(\alpha\) максимальных граней \(f\), которое мы будем называть \emph{пучком} \(f\) \emph{через точку \(\alpha\)}. Точку \(\alpha \in N_f\) будем называть \emph{регулярной точкой} \(f\), если \(\exists \beta \in N_f: \Pi_{\beta}(f) \subset \Pi_{\alpha}(f)\). Для любой регулярной точки \(\alpha\) найдётся нерегулярная точка \(\beta \in N_f: \Pi_{\beta}(f) \subset \Pi_{\alpha}(f)\).

Любая неядровая точка ядровой грани регулярна. Грань \(N_K\) ФАЛ \(f\) называется \emph{регулярной гранью} этой ФАЛ, если все точки \(N_K\) регулярны. Грань, которая не входит в ядро, но покрывается им, является регулярной.
  \begin{Theorem}
Простая импликанта $K$ ФАЛ $f$ входит в ДНФ $\Sigma T$ тогда и только тогда, когда грань $N_K$ не является регулярной гранью этой ФАЛ.
  \end{Theorem}
\begin{proof}
Пусть $\alpha_1, \ldots, \alpha_s$ -- все регулярные точки $f$. Тогда для любого $j, j = \overline{1, s}$ найдётся нерегулярная точка $\beta_j$ $f$ такая, что любая максимальная грань, проходящая через $\beta_j$, проходит и через $\alpha_j$. Тогда любой набор максимальных граней, покрывающих точки $\beta_1, \ldots, \beta_s$, покрывает и точки $\alpha_1, \ldots, \alpha_s$. Это значит, что грань $N_K$, состоящая только из регулярных точек, не может входить в тупиковое покрытие множества $N_f$ максимальными гранями, что означает, что ЭК $K$ не может входить в ДНФ $\Sigma T$ ФАЛ $f$.

Пусть теперь $N_K$ -- нерегулярная грань $f$, которая содержит нерегулярную точку $\alpha$ и пусть $N_f \backslash N_K = \{\beta_1, \ldots, \beta_q\}$. Из нерегулярности точки $\alpha$ следует, что для любого $j, j = \overline{1, q}$ пучок $\Pi_{\beta_j}(f)$ не может быть строго вложен в $\Pi_{\alpha}(f)$. Кроме того, $\Pi_{\beta_j}(f) \neq \Pi_{\alpha}(f)$, так как $N_K \in \Pi_{\alpha}(f) \backslash \Pi_{\beta_j}(f) \Rightarrow \exists N_{K_j} \in \Pi_{\beta_j}: \beta_j \in N_{K_j}, \alpha \notin N_{K_j}$. Это означает, что из покрытия $N_f$ максимальными гранями $N_K, N_{K_1}, \ldots, N_{K_q}$ нельзя удалить грань $N_K$, т. к. только она покрывает точку $\alpha$. Это значит, что любое тупиковое подпокрытие этого покрытия будет соответствовать тупиковой ДНФ, содержащей ЭК $K$.
\end{proof}

Для каждой максимальной грани \(\mathcal{N}\) ФАЛ \(f\) положим \(S_0(\mathcal{N}, f) = \{\mathcal{N}\}\), а затем индукцией по \(r, r = 1, 2, \ldots\) определим множество \(S_r(\mathcal{N}, f)\) как множество всех максимальных граней \(f\), которые имеют непустое пересечение c хотя бы одной гранью из множества \(S_{r - 1}(\mathcal{N}, f)\). Множество \(S_r(\mathcal{N}, f)\) будем называть \emph{окрестностью порядка \(r\) грани \(\mathcal{N}\) функции \(f\)}.

Вопрос о вхождении простой импликанты \(K\) ФАЛ \(f\) в ДНФ \(\cap T\) (ДНФ \(\Sigma T\)) этой ФАЛ можно решить, рассматривая окрестность \(S_1(N_K, f)(S_2(N_K, f))\). Для проверки грани \(N_K\) на её вхождение в ДНФ Квайна ФАЛ \(f\) также достаточно рассмотреть окрестность порядка 2.
\section{Особенности ДНФ линейных и монотонных ФАЛ. Функция покрытия, таблица Квайна и построение всех тупиковых ДНФ.}
\label{sec:orgfd2abeb}
\section{Градиентный алгоритм и оценка длины градиентного покрытия, лемма о протыкающих наборах. Использование градиентного алгоритма для построения ДНФ.}
\label{sec:orgbd6e93b}
\section{Задача минимизации ДНФ. Поведение функции Шеннона и оценки типичных значений для ранга и длины ДНФ.}
\label{sec:org5d50808}
\section{Алгоритмические трудности минимизации ДНФ и оценки максимальных значений некоторых связанных с ней параметров. Теорема Журавлёва о ДНФ сумма минимальных}
\label{sec:orgcd41a84}
\section{Формулы алгебры логики, их эквивалентные преобразования с помощью тождеств. Полнота системы основных тождеств для эквивалентных преобразований формул базиса \text{Б}\textsubscript{0} = \{\&, \(\lor\), \(\neg{}\)\}.}
\label{sec:orgaca4e00}
\section{Задание формул с помощью деревьев, функционалы их сложности и соотношения между ними. Оптимизация подобных формул по глубине.}
\label{sec:org46527ab}
\section{Схемы из функциональных элементов (СФЭ). Изоморфизм и эквивалентность схем, функционалы их сложности, операции приведения. Верхние оценки числа формул и СФЭ в базисе Б\textsubscript{0}}
\label{sec:orgf65ede7}
\section{Контактные схемы (КС) и \(\pi\)-схемы, их изоморфизм, эквивалентность, сложность, операции приведения. Структурное моделирование некоторых формул и \(\pi\)-схем. Оценки числа КС и числа \(\pi\)-схем. Особенности функционирования многополюсных КС.}
\label{sec:org262ae5f}
\section{Эквивалентные преобразования СФЭ и моделирование с их помощью формульных преобразований. Моделирование эквивалентных преобразований формул и схем в различных базисах, теорема перехода.}
\label{sec:org2d33080}
\section{Эквивалентные преобразования КС. Основные тождества, вывод вспомогательных (дополнительных) и обобщённых тождеств.}
\label{sec:org7cf23ca}
\section{Полнота системы основных тождеств. Отсутствие конечной полной системы тождеств в классе всех КС.}
\label{sec:orgf8290e8}
\section{Задачи синтеза. Методы синтеза схем на основе ДНФ и связанные с ними верхние оценки сложности функций.}
\label{sec:org898c45e}
\section{Простые нижние оценки сложности ФАЛ, реализация некоторых ФАЛ и минимальность некоторых схем.}
\label{sec:orgfaeb58a}
\section{Разложение ФАЛ и операция суперпозиции схем. Корректность суперпозиции для некоторых типов схем, разделительные КС и лемма Шеннона.}
\label{sec:org1ec9238}
\section{Каскадные КС и СФЭ. Метод каскадов и примеры его применения, метод Шеннона.}
\label{sec:orge732ef4}
\section{Нижние мощностные оценки функций Шеннона, их обобщение на случай синтеза схем для ФАЛ из специальных классов.}
\label{sec:orgbf76d56}
\section{Дизъюнктивно-универсальные множества ФАЛ. Асимптотически наилучший метод О. Б. Лупанова для синтеза СФЭ в базисе Б\textsubscript{0}.}
\label{sec:org6bdbe37}
\section{Регулярные разбиения единичного куба и моделирование ФАЛ переменными. Асимптотически наилучший метод синтеза формул в базисе Б\textsubscript{0}.}
\label{sec:orgfe25d3f}
\section{Асимптотически наилучший метод синтеза КС. Синтез схем для ФАЛ из некоторых специальных классов.}
\label{sec:org766e04a}
\section{Задача контроля схем и тесты для таблиц. Построение всех тупиковых тестов, оценки длины диагностического теста.}
\label{sec:org97222b1}
\section{Самокорректирующиеся КС и методы их построения. Асимптотически наилучший метод синтеза КС, корректирующих 1 обрыв (1 замыкание).}
\label{sec:orge35abf9}
\section{Доказательство теоремы Кука. Примеры NP-полных проблем, связанных с графами.}
\label{sec:org75d2ee6}
\section{Некоторые модификации основных классов схем (BDD, вычисляющие программы, схемы на КМОП-транзисторах и др.), ориентированные на программно-аппаратную реализацию ФАЛ.}
\label{sec:orge790954}
\section{Реализация автоматных функций схемами из функциональных элементов и элементов задержки, схемы с "мгновенными" обратными связями.}
\label{sec:org6f43e42}
\end{document}
