% Created 2020-06-15 Mon 16:13
% Intended LaTeX compiler: pdflatex
\documentclass[11pt]{article}
\usepackage[utf8]{inputenc}
\usepackage[T1]{fontenc}
\usepackage{graphicx}
\usepackage{grffile}
\usepackage{longtable}
\usepackage{wrapfig}
\usepackage{rotating}
\usepackage[normalem]{ulem}
\usepackage{amsmath}
\usepackage{textcomp}
\usepackage{amssymb}
\usepackage{capt-of}
\usepackage{hyperref}
\usepackage[russian]{babel}
\usepackage{amsmath}
\usepackage{esint}
\usepackage{mathtools}
\usepackage{amsthm}
\usepackage{listings}
\usepackage{float}
\usepackage[top=0.8in, bottom=0.75in, left=0.625in, right=0.625in]{geometry}
\usepackage{dot2texi}
\usepackage{tikz}
\usetikzlibrary{shapes, arrows, positioning}
\def\zall{\setcounter{lem}{0}\setcounter{cnsqnc}{0}\setcounter{th}{0}\setcounter{Cmt}{0}\setcounter{equation}{0}}
\newcounter{lem}\setcounter{lem}{0}
\def\lm{\par\smallskip\refstepcounter{lem}\textbf{\arabic{lem}}}
\newtheorem*{Lemma}{Лемма \lm}
\newcounter{th}\setcounter{th}{0}
\def\th{\par\smallskip\refstepcounter{th}\textbf{\arabic{th}}}
\newtheorem*{Theorem}{Теорема \th}
\newcounter{cnsqnc}\setcounter{cnsqnc}{0}
\def\cnsqnc{\par\smallskip\refstepcounter{cnsqnc}\textbf{\arabic{cnsqnc}}}
\newtheorem*{Consequence}{Следствие \cnsqnc}
\newcounter{Cmt}\setcounter{Cmt}{0}
\def\cmt{\par\smallskip\refstepcounter{Cmt}\textbf{\arabic{Cmt}}}
\newtheorem*{Note}{Замечание \cmt}
\author{Sergey Makarov}
\date{\today}
\title{}
\hypersetup{
 pdfauthor={Sergey Makarov},
 pdftitle={},
 pdfkeywords={},
 pdfsubject={},
 pdfcreator={Emacs 28.0.50 (Org mode 9.3)}, 
 pdflang={Russian}}
\begin{document}


\section{Задача 10}
\label{sec:org031702f}
Промоделировать \(\pi\)-схемой формулу
\begin{equation*}
x_1x_2\lor x_3\overline{x}_4x_5\lor \overline{x}_3x_4(\overline{x}_1\lor \overline{x}_5)x_6\lor x_7
\end{equation*}
и выписать с расстановкой всех скобок или задать деревом подобную ей формулу минимальной глубины. Указать и обосновать изменение данной глубины при инвертировании БП $x_2$.
\subsection{Решение}
\label{sec:orgadd7fe8}
Оценка глубины для подобной формулы имеет вид:
\begin{equation*}
\lceil\log(L(F) + 1)\rceil = 4 \leq D(F) \leq \lceil\log(L(F) + 1)\rceil + \operatorname{Alt}(F) = \lceil\log(L(F) + 1)\rceil + 2 = 6
\end{equation*}
При инвертировании БП \(x_2\) сложность формулы увеличится на 1, поэтому для глубины изменённой формулы будет справедлива та же оценка.

Исходной формуле отвечает \(\pi\)-схема:
\begin{figure}[H]
\centering
\begin{tikzpicture}[>=latex']
\tikzstyle{pole} = [draw, shape=circle, inner sep=0pt, minimum size=1.5mm]
\tikzstyle{inner} = [draw, shape=circle, inner sep=0pt, minimum size=1.5mm, fill=black]

\begin{dot2tex}[dot, tikz, codeonly, options=-tmath, scale = 0.8]
graph G {
    rankdir=LR;
    node[style=pole, label=""]a; b;
    node[style=inner];
    a -- v1[label=x_1];
    v1 -- b[label=x_2];
    a -- v2[label=x_3];
    v2 -- v3[label=" ", texlbl="$\overline{x}_4$"];
    v3 -- b[label=x_5];
    a -- v4[label=" ", texlbl="$\overline{x}_3$"];
    v4 -- v5[label=x_4];
    v5 -- v6[label=" ", texlbl="$\overline{x}_1$"];
    v5 -- v6[label=" ", texlbl="$\overline{x}_5$"];
    v6 -- b[label=x_6];
    a -- b[label=x_7];
}
\end{dot2tex}
\end{tikzpicture}
\end{figure}
\newpage

И дерево:
\begin{center}
\begin{dot2tex}[dot, tikz, options=-s -tmath, scale = 0.8]
graph G {
    node[shape=plaintext];
    xx_1[label=x_1];
    xx_3[label=x_3];
    xx_4[label=x_4];
    xx_5[label=x_5];

    v_1[label="\neg"];
    v_2[label="\neg"];
    v_3[label="\neg"];
    v_4[label="\land"];
    v_5[label="\neg"];
    v_6[label="\land"];
    v_7[label="\land"];

    v1[label="\lor"];
    v2[label="\land"];
    v3[label="\land"];
    v4[label="\lor"];

    v5[label="\land"];
    v6[label="\lor"];
    v7[label="\lor"];

    x_1 -- v_1; x_5 -- v_2; x_3 -- v_3;
    x_4 -- v_4; x_6 -- v_4;
    xx_4 -- v_5; xx_3 -- v_6; xx_5 -- v_6;
    xx_1 -- v_7; x_2 -- v_7;

    v_1 -- v1; v_2 -- v1; v_3 -- v2; v_4 -- v2;
    v_5 -- v3; v_6 -- v3; v_7 -- v4; x_7 -- v4;

    v1 -- v5; v2 -- v5; v3 -- v6; v4 -- v6;
    v5 -- v7; v6 -- v7;
}
\end{dot2tex}
\end{center}
\end{document}
