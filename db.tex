% Created 2020-11-18 Wed 12:43
% Intended LaTeX compiler: pdflatex
\documentclass[11pt]{article}
\usepackage[utf8]{inputenc}
\usepackage[T1]{fontenc}
\usepackage{graphicx}
\usepackage{grffile}
\usepackage{longtable}
\usepackage{wrapfig}
\usepackage{rotating}
\usepackage[normalem]{ulem}
\usepackage{amsmath}
\usepackage{textcomp}
\usepackage{amssymb}
\usepackage{capt-of}
\usepackage{hyperref}
\usepackage[russian]{babel}
\usepackage{listings}
\usepackage{amsmath}
\usepackage{esint}
\usepackage{mathtools}
\usepackage{amsthm}
\usepackage{minted}
\usepackage[top=0.8in, bottom=0.75in, left=0.625in, right=0.625in]{geometry}
\usepackage{float}
\usepackage{dot2texi}
\usepackage{tikz}
\usetikzlibrary{shapes, arrows, positioning}
\def\zall{\setcounter{lem}{0}\setcounter{cnsqnc}{0}\setcounter{th}{0}\setcounter{Cmt}{0}\setcounter{equation}{0}\setcounter{stnmt}{0}}
\newcounter{lem}\setcounter{lem}{0}
\def\lm{\par\smallskip\refstepcounter{lem}\textbf{\arabic{lem}}}
\newtheorem*{Lemma}{Лемма \lm}
\newcounter{th}\setcounter{th}{0}
\def\th{\par\smallskip\refstepcounter{th}\textbf{\arabic{th}}}
\newtheorem*{Theorem}{Теорема \th}
\newcounter{cnsqnc}\setcounter{cnsqnc}{0}
\def\cnsqnc{\par\smallskip\refstepcounter{cnsqnc}\textbf{\arabic{cnsqnc}}}
\newtheorem*{Consequence}{Следствие \cnsqnc}
\newcounter{Cmt}\setcounter{Cmt}{0}
\def\cmt{\par\smallskip\refstepcounter{Cmt}\textbf{\arabic{Cmt}}}
\newtheorem*{Note}{Замечание \cmt}
\newcounter{stnmt}\setcounter{stnmt}{0}
\def\st{\par\smallskip\refstepcounter{stnmt}\textbf{\arabic{stnmt}}}
\newtheorem*{Statement}{Утверждение \st}
\author{Макаров Сергей, группа 427}
\date{\today}
\title{Базы данных}
\hypersetup{
 pdfauthor={Макаров Сергей, группа 427},
 pdftitle={Базы данных},
 pdfkeywords={},
 pdfsubject={},
 pdfcreator={Emacs 27.1 (Org mode 9.4)}, 
 pdflang={Russian}}
\begin{document}

\maketitle

\section{Введение}
\label{sec:org1a56f17}
Требования к СУБД:
\begin{enumerate}
\item Эффективный доступ и хранение данных
\item Поддержка языка запросов
\item Поддержка целостности данных
\item Поддержка транзакций и многопользовательского режима
\item Поддержка журналирования и архивирования данных
Необходимость СУБД как отдельного компонента:
\item Журналирование и другие метаданные.
\item Нормализация и более тонкое управление внешней памятью
\item Управление оперативной(основной) памятью, доступ к физической основной памяти для кэшей и буферов
\end{enumerate}
\section{Реляционная модель}
\label{sec:orgb86b8b9}
\subsection{Алгебра Кодда}
\label{sec:orgbe72c8d}
Операции:
\begin{itemize}
\item UNION -- объединение таблиц с совпадающими заголовками
\item INTERSECT -- пересечение таблиц с совпадающими заголовками
\item MINUS -- разность таблиц с совпадающими заголовками
\item TIMES -- декартово произведение таблиц с непересекающимися заголовками, кортежи результата -- попарное объединение исходных кортежей
\item WHERE -- ограничение таблицы по условию
\item PROJECT -- проекция таблицы на множество атрибутов
\item JOIN -- соединение кортежей по условию
\item DIVIDE BY -- операция, обратная декартову произведению
\item RENAME -- переименование атрибутов
\item Присваивание
\end{itemize}
Приоритеты:
\begin{equation*}
RENAME \geq WHERE = PROJECT \geq TIMES = JOIN = INTERSECT = DIVIDE BY \geq UNION = MINUS
\end{equation*}
\subsection{Алгебра A*}
\label{sec:orga9146fe}
\begin{itemize}
\item Реляционное дополнение -- дополнение тела операнда до всей таблицы
\item Операция удаления атрибута
\item Операция переименования
\item Реляционная конъюкция -- множество кортежей, представляющих попарное объединение кортежей-операндов
\item Реляционная дизъюнкция -- множество кортежей, представляющее объединение пар кортежей, хотя бы одна из которых содержится в одном из операндов
\end{itemize}
Операция дизъюнкции выражается через операцию конъюкции и дополнения по закону де Моргана
\section{Нормализация}
\label{sec:org082e95a}
\begin{itemize}
\item Теорема Хита
Пусть задано отношение \(r = \{A, B, C\}\) и выполняется FD \(A \to B\). Тогда
\end{itemize}
\begin{minted}[]{text}
r = (r PROJECT {A, B}) NATURAL JOIN (r PROJECT {A, C})
\end{minted}
\begin{itemize}
\item Первая нормальная форма
Значения всех атрибутов атомарны, т. е. их внутренняя структура не видна. Несоблюдение этого требования усложняет структуру тублицы и работу со значениями в ней.
\item Вторая нормальная форма
Выполняются требования первой нормальной формы, и каждый неключевой атрибут минимально функционально зависит от первичного ключа. Несоблюдение этого требования усложняет работу с БД, так как первичный ключ содержит избыточную информацию.
\item Третья нормальная форма
Отношение находится во второй нормальной форме, и каждый неключевой атрибут нетранзитивно функционально зависит от первичного ключа. При несоблюдении этого требования возникают проблемы с обновлением сущности, входящей в транзитивную зависимость.
\item Теорема Риссанена
Проекции \(r_1\) и \(r_2\) отношения \(r\) являются независимыми тогда и только тогда, когда:
\begin{enumerate}
\item каждая FD в \(r\) логически следует из FD в \(r_1\) и \(r_2\).
\item общие атрибуты \(r_1\) и \(r_2\) образуют возможный ключ хотя бы одного из этих отношений
\end{enumerate}
\item Нормальная форма Бойса-Кодда
Отношение находится в BCNF тогда и только тогда, когда любая выполняемая для него нетривиальная и минимальная FD имеет в качестве детерминанта некоторый возможный ключ этого отношения.
\item Теорема Фейджина
\end{itemize}
Пусть имеется переменная отношения \(R\{A, B, C\}\). Отношение \(R\) декомпозируется без потерь на проекции \(\{A, B\}\) и \(\{A, C\}\) тогда и только тогда, когда для него выполняется MVD \(A \to\to B | C\).
\begin{itemize}
\item Четвёртая нормальная форма
Переменная отношения \(r\) находится в четвёртой нормальной форме тогда и только тогда, когда она находится в BCNF, и все MVD \(r\) являются FD, детерминантами которых являются возможные ключи \(r\).
\item Пятая нормальная форма
Переменная отношения \(r\) находится в пятой нормальной форме тогда и только тогда, когда любая её зависимость проекции-соединения подразумевается возможными ключами отношения \(R\).
\end{itemize}
\section{Синхронизация}
\label{sec:org605dd81}
\begin{itemize}
\item Atomicy(атомарность) -- результаты транзакции после фиксации либо отражены в состоянии БД полностью, либо не отображены вообще.
\item Consistency(согласованность, целостность) -- транзакция может быть зафиксирована тогда и только тогда, когда её действие не нарушает целостности БД.
\item Isolation(изоляция) -- две параллельно работающие транзакции не должны действовать одна на другую.
\item Durability(долговечность) -- после успешного завершения транзакции все внесённые ей изменения должны сохраняться даже в случае сбоев.
\end{itemize}

Степени изоляции транзакций:
\begin{enumerate}
\item Отсутствие потерянных изменений.
\item Отсутствие чтения грязных данных.
\item Отсутствие неповторяющихся чтений.
\item Сериализация транзакций -- выполнение их действий в таком порядке, результаты которого были бы такие же, как при последовательном их выполнении в некотором порядке.

Таблица совместимости блокировок
\end{enumerate}
\section{B+-деревья}
\label{sec:orge9261b8}
\end{document}
