% Created 2019-09-26 Thu 16:25
% Intended LaTeX compiler: pdflatex
\documentclass[11pt]{article}
\usepackage[utf8]{inputenc}
\usepackage[T1]{fontenc}
\usepackage{graphicx}
\usepackage{grffile}
\usepackage{longtable}
\usepackage{wrapfig}
\usepackage{rotating}
\usepackage[normalem]{ulem}
\usepackage{amsmath}
\usepackage{textcomp}
\usepackage{amssymb}
\usepackage{capt-of}
\usepackage{hyperref}
\usepackage{amsmath}
\usepackage{esint}
\usepackage[english, russian]{babel}
\usepackage{mathtools}
\usepackage{amsthm}
\usepackage[top=0.8in, bottom=0.75in, left=0.625in, right=0.625in]{geometry}
\def\zall{\setcounter{lem}{0}\setcounter{cnsqnc}{0}\setcounter{th}{0}\setcounter{Cmt}{0}\setcounter{equation}{0}}
\newcounter{lem}\setcounter{lem}{0}
\def\lm{\par\smallskip\refstepcounter{lem}\textbf{\arabic{lem}}}
\newtheorem*{Lemma}{Лемма \lm}
\newcounter{th}\setcounter{th}{0}
\def\th{\par\smallskip\refstepcounter{th}\textbf{\arabic{th}}}
\newtheorem*{Theorem}{Теорема \th}
\newcounter{cnsqnc}\setcounter{cnsqnc}{0}
\def\cnsqnc{\par\smallskip\refstepcounter{cnsqnc}\textbf{\arabic{cnsqnc}}}
\newtheorem*{Consequence}{Следствие \cnsqnc}
\newcounter{Cmt}\setcounter{Cmt}{0}
\def\cmt{\par\smallskip\refstepcounter{Cmt}\textbf{\arabic{Cmt}}}
\newtheorem*{Note}{Замечание \cmt}
\author{Sergey Makarov}
\date{\today}
\title{}
\hypersetup{
 pdfauthor={Sergey Makarov},
 pdftitle={},
 pdfkeywords={},
 pdfsubject={},
 pdfcreator={Emacs 26.3 (Org mode 9.1.9)}, 
 pdflang={English}}
\begin{document}

\zall

Лектор Гуров Сергей Исаевич

На одной из лекции будет КР. Допуск к экзамену <- зачёт по КР.

\textbf{Группа} - тройка \(<G, \circ, e>\), где \(G\) - непустое множество, e - единичный(нейтральный) элемент, причём выполнены следующие аксиомы:
\begin{enumerate}
\item Замкнутость(устойчивость) \(G\) относительно операции.
\item Ассоциативность операции.
\item \((x \circ e) = (e \circ x) = x\)
\item \(\forall x \in G \exists y \in G x \circ y = e\)
\end{enumerate}
Таблица Кэли - аналог таблицы умножения.

\(V_4 = \{e, a, b, c\}\)

\begin{tabular}{|c|c|c|c|c|}
\hline
$\circ$  & e & a & b & c \\
\hline
e        & e & a & b & c \\
\hline
a        & a & e & c & b \\
\hline
b        & b & c & e & a \\
\hline
c        & c & b & a & e \\
\hline
\end{tabular}

Примеры групп: \(\mathbb{Q}, \mathbb{Z}, \mathbb{R}, \mathbb{C}\) относительно сложения.
\(n\mathbb{Z}, B^n, S_n\).

В случае, если операция коммутативна, группа называется \textbf{коммутативной} или \textbf{абелевой}.

Пусть \(a \in G\). Наименьшее \(n\) такое, что \(a^n = e\) называется \textbf{порядком} элемента \(a\)(\(\operatorname{ord} a\)).

Подгруппой G называется \(H \subset G\) являющееся подгруппой: \(H \leq G\).
Каждый элемент порождает группу \(\{a, a^2, \ldots, a^{\operatorname{ord} a}\}\).

Пусть \(H \leq G, x \in G\). \textbf{Правым(левым) смежным классом} x называют множество:
$$xH = \{x \circ n | n \in H\}, Hx = \{n \circ x | n \in H\}$$
Подгруппа, для которой левые и правые классы с одинаковым представителем совпадают, называется
\textbf{нормальной}.

\begin{Theorem}
Правые(левые) смежные классы разных элементов либо совпадают, либо не пересекаются.
\end{Theorem}

\textbf{Изоморфизм} - биекция, сохраняющая операцию. Группы, между которыми существуют изоморфизмы,
называются \textbf{изоморфными}.

\begin{Theorem}[Теорема Кэли]
Любая конечная группа порядка $n$ изоморфна некоторой подгруппе $S_n$.
\end{Theorem}

\textbf{Гомоморфизм} - отображение меджу группами, сохраняющее операцию.

\begin{Theorem}
Пусть $H$ - нормальная подгруппа $G$. Тогда
$$|G| = |H| \cdot [G: H]$$
Число $[G: H]$ называется индексом группы $G$ относительно нормальной подгруппы $H$.
\end{Theorem}

\textbf{Циклическими} называют группы, порождённые одним элементом. Бесконечные циклические группы изоморфны
группе целых чисел по сложению. Циклические группы порядка n изоморфны группе вычетов порядка n.

Все порождающие элементы \(\mathbb{Z}_n\) это числа, взаимно простые с \(n\).

\textbf{Функция Эйлера} - количество чисел, взаимно простых с p, меньших p.
Свойства функции Эйлера:
\begin{enumerate}
\item \(\varphi(p^k) = p^{k - 1}\varphi(p)\).
\item Если \((a, b) = 1\), то \(\varphi(ab) = \varphi(a)\varphi(b)\).
\end{enumerate}

Абелева группа называется \textbf{кольцом}, если на ней определена операция умножения, связанная
с операцией сложения дистрибутивностью.

Если умножение ассоциативно(коммутативно), кольцо называется \textbf{ассоциативным(коммутативным)}.
Если у умножения существует нейтральный элемент, кольцо называется \textbf{кольцом с единицей}.
Если \(\forall a \neq 0, b \neq 0 ab \neq 0\), кольцо называется \textbf{кольцом без делителей нуля}.
Ассоциативное коммутативное кольцо с единицей без делителей нуля называется \textbf{целостным}.

Пусть R - кольцо.
Множество обратимых элементов R обозначаем R\(^{\text{*}}\).

Элемент кольца называется \textbf{неразложимым}, если он не может быть представлен в виде произведения двух других.

\textbf{Факториальным} называется кольцо, каждый ненулевой элемент которого либо обратим, либо однозначно
с точностью до порядка сомножителей и умножения на обратимые элементы раскладывается на неразложимые
множители. Такое разложение числа называется \textbf{примарным}.

Рассмотрим \(S \subset R\), являющееся кольцом. Такое множество называется \textbf{подкольцом}. Условия:
\begin{enumerate}
\item S - подгруппа по сложению
\item S замкнуто по умножению.
\end{enumerate}

(Двухсторонним)*Идеалом* называется подкольцо коммутативного кольца, замкнутое относительно
и умножения на элементы кольца.

Идеал \(I\) коммутативного кольца \(R\) называется \textbf{главным} с представителем \(a \in R\)(идеалом, порождённым \(a\)), если
$$I = \{r\cdot a | r \in R\} = (a)$$

Кольца, в которых все идеалы являются главными, называются \textbf{кольцами главных идеалов}. \textbf{Максимальным}
называется идеал, такой что \(I_{max} \subset I \Rightarrow I = R\).

\textbf{Утверждение}: В коммутативном кольце всегда существует максимальный идеал.

\textbf{Классом вычетов} по модулю идеала \(I\) коммутативного кольца \(R\) с представителем \(r \in R\)
называется множество \(r + I = \{r + i | i \in I\} = \overline{r_I}\).

\textbf{Фактор-кольцом} \(R/I\) называется кольцо классов вычетов \(\overline{r_I}\).

\textbf{Утверждение}: Фактор-кольцо по максимальному идеалу является полем.

Целостное кольцо \(R\) называется \textbf{евклидовым}, если \(\forall a \in R, a \neq 0 \exists N(a) \in \mathbb{N}\),
такая, что \(\forall b \neq 0 a = bq + r\), причём \(r = 0\) или \(N(r) < N(b)\).

Целостное кольцо, в котором каждый ненулевой элемент обратим, называется \textbf{полем}. У поля есть
мультипликативная группа(абелева группа по умножению).

Будем обозначать \(R^*\) множество обратимых элементов кольца \(R\).

У поля существуют только тривиальные идеалы.

Структура, аналогичная полю, в которой умножение некоммутативно, называется \textbf{телом}.
\begin{Theorem}
В теле нет нетривиальных идеалов.
\end{Theorem}
\textbf{Линейным векторным пространством} V над полем P называется аддитивная группа по сложению,
для элементов которой определено умножение на элементы поля, обладающая свойствами:
$$a(v_1 + v_2) = av_1 + av_2 \forall a \in P, v_1, v_2 \in V$$
$$(a + b)v = av + bv \forall a, b \in P, v \in V$$
$$a(bv) = (ab)v \forall a, b \in P, v \in V$$
$$1v = v \forall v \in V$$
и замкнутая относительно линейной комбинации с коэффициентами из P.

Поля вычетов по модулю \(p\), где \(p\) - простое, называются \textbf{простыми полями Галуа}. Минимальное
число \(p\) такое, что \(\underbrace{1 + 1 + \ldots + 1}_{p} = 0\), называется \textbf{характеристикой} поля.
Если \(p = \infty\), считается, что \(p = 0\). Поле дробей-многочленов имеет конечную характеристику,
но является бесконечным.

\textbf{Утверждение(тождество Фробениуса)}: \(\forall a, b \in GF(p) (a + b)^p = a^p + b^p\)

Пусть \(F^*_p = F_p \ \{0\}\).

\textbf{Утверждение}: \(|F^*_q| = q - 1\).

Рассмотрим \(K[x]\) -  кольцо многочленов над полем \(K\) от переменной \(x\). Будем считать, что \(a_n = 1\).
Рассмотрим \(\mathbb{F}_p[x]\).
В \(\mathbb{F}_2[x]\) неприводимыми являются многочлены \(x^2 + x + 1, x^3 + x^2 + 1, x^3 + x + 1\).
В \(\mathbb{F}_5[x]\) неприводимыми являются 6 многочленов.
\begin{Theorem}
В $\mathbb{F}_p \forall n < p$ существует неприводимый многочлен степени n.
\end{Theorem}
Пусть $a(x) \in \mathbb{F}_p[x]$ - неприводимый многочлени степени n. Рассмотрим
$(a(x)) = \{q(x)a(x) | q(x) \in \mathbb{F}_p[x]\}$. Тогда $\mathbb{F}_p[x] / (a(x))$ - множество
остатков от деления многочленов на a(x) - является полем. Если $a(x)$ - многочлен степени $n$, то
все остатки - многочлены степени до $n - 1$. Получили \textbf{расширение} поля Галуа $\mathbb{F}_p^n$,
$GF(p^n)$.

\textbf{Пример}:\\
$\mathbb{F}_3^2$ - ?

\begin{equation}
\mathbb{F}_3^2[x] = \mathbb{F}_3^2[x]/(x^2 + 1) = \{0, 1, 2, x, 2x, x + 1, x + 2, 2x + 1, 2x + 2\}
\end{equation}

\textbf{Пример 2}:\\
Рассмотрим $\mathbb{R}[x], a(x) = x^2 + 1$. Тогда
\begin{equation}
\mathbb{R}[x]/(x^2 + 1) = \{ax + b | a, b \in \mathbb{R}\} \text{ - поле комплексных чисел}
\end{equation}
\begin{Theorem}
Поля расширения по разным многочленам изоморфны.
\end{Theorem}
\begin{Theorem}[Соотношение Безу]
$\forall a, b \in \mathbb{N} \exists d \in \mathbb{N}, x, y \in \mathbb{Z}: ax + by = d, d = (a, b)$.
\end{Theorem}

\textbf{Расширенный алгоритм Евклида}:
\begin{equation*}
E = \begin{Vmatrix}
1 & 0 \\
0 & 1 \\
\end{Vmatrix},
r = 0.
\end{equation*}
Если $r = 0$, то второй столбец $E$ даёт $x$ и $y$.

Иначе \begin{equation}
E \to E \cdot \begin{Vmatrix}
0 & 1 \\
1 & -q
\end{Vmatrix}
\end{equation}
и $(a, b) \to (b, r)$.

Алгоритм Евклида позволяет искать обратный элемент в $\mathbb{Z}_m$:

1. Пусть $(c, m) = 1$. \\
2. Рассмотрим матрицу \begin{equation}
\begin{Vmatrix}
m & 0 \\
c & 1.
\end{Vmatrix}
\end{equation}\\
3. Поделим $m$ на $c$ с остатком: $m = qc + r$.\\
4. Вторую строку домножаем на $q$ и вычитаем из первой.\\
5. Когда первый элемент последней строки становится равным нулю, второй элемент даёт $c^{-1}$.

\textbf{Обобщённый алгоритм Евклида для нахождения в} \(\mathbb{F}_p/(a(x)) y(x)\), \textbf{обратного к} \(b(x)\):
Шаг 0: $r_{-2}(x) = a(x), r_{-1}(x) = b(x), y_{-2}(x) = 0, y_{-1}(x) = 1$.\\
Шаг 1: $$r_{-2} / r_{-1} \Rightarrow q_0, r_0$$
$$r_{-2}(x) = r_{-1}(x)q_0(x) + r_0(x)$$
$$y_0(x) = -q_0(x)$$
Если $\operatorname{deg} r_0(x) \geq 1$ - к следующему шагу, иначе к $(n + 1)$-му шагу.\\
Шаг 2: $$r_{i - 3}(x) = r_{i - 2}(x)q_{i - 1}(x) + r_{i - 1}(x)$$
$$y_{i - 1}(x) = y_{i - 3}(x) - y_{i - 2}(x)q_{i - 1}(x)$$
Если $\operatorname{deg}r_{i - 1} > 0$, продолжаем итерации.
\end{document}
