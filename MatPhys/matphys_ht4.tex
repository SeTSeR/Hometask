% Created 2019-09-30 Mon 17:59
% Intended LaTeX compiler: pdflatex
\documentclass[11pt]{article}
\usepackage[utf8]{inputenc}
\usepackage[T1]{fontenc}
\usepackage{graphicx}
\usepackage{grffile}
\usepackage{longtable}
\usepackage{wrapfig}
\usepackage{rotating}
\usepackage[normalem]{ulem}
\usepackage{amsmath}
\usepackage{textcomp}
\usepackage{amssymb}
\usepackage{capt-of}
\usepackage{hyperref}
\usepackage{amsmath}
\usepackage{esint}
\usepackage[english, russian]{babel}
\usepackage{mathtools}
\usepackage{amsthm}
\usepackage[top=0.8in, bottom=0.75in, left=0.625in, right=0.625in]{geometry}
\def\zall{\setcounter{lem}{0}\setcounter{cnsqnc}{0}\setcounter{th}{0}\setcounter{Cmt}{0}\setcounter{equation}{0}}
\author{Sergey Makarov}
\date{\today}
\title{}
\hypersetup{
 pdfauthor={Sergey Makarov},
 pdftitle={},
 pdfkeywords={},
 pdfsubject={},
 pdfcreator={Emacs 26.3 (Org mode 9.1.9)}, 
 pdflang={English}}
\begin{document}

\zall

ДЗ: 4.4, 4.6, 4.7, 4.8, 4.11

\section{Задача 4.4}
\label{sec:orgcd4397b}
\begin{equation}
\begin{cases}
u_t = a^2u_{xx} + f_0, 0 < x < l, t > 0, f_0 = const, \\
u_x(0, t) = u_x(l, t) = 0, t > 0, \\
u(x, 0) = 0, 0 \leq x \leq l.
\end{cases}
\end{equation}
\subsection{Решение}
\label{sec:org68ccd06}
Собственные значения и собственные функции соответствующей ЗШЛ:
\begin{equation*}
\begin{cases}
\lambda_n = \left(\frac{\pi n}l\right)^2, \\
X_n = \cos\frac{\pi n}l.
\end{cases}
\end{equation*}
Ищем решение (1) в виде
\begin{equation}
u(x, t) = \sum_{n = 0}^{\infty}u_n(t)\cos\frac{\pi n}lx.
\end{equation}
Для $f_0$ получим:
\begin{equation}
f_0 = \sum_{n = 0}^{\infty}f_n\cos\frac{\pi n}lx \Rightarrow f_n = 0, n \neq 0.
\end{equation}
Подставив в (1) (2) и (3), получим систему:
\begin{equation*}
\begin{cases}
u_n'(t) = -\left(\frac{\pi na}l\right)^2u_n(t) + f_n, \\
u_n(0) = 0.
\end{cases}
\end{equation*}
При $n = 0$:
\begin{equation*}
\begin{cases}
u_0'(t) = f_0, \\
u_0(0) = 0.
\end{cases}
\end{equation*}
Решением этой задачи будет функция $u_0(t) = f_0t$.
При $n \neq 0$ задача Коши будет иметь только нулевое решение, поэтому окончательно для $u(x, t)$ получаем:
\begin{equation}
u(x, t) = f_0t.
\end{equation}
\section{Задача 4.6}
\label{sec:orgedc39b2}
\begin{equation}
\begin{cases}
u_t = a^2u_{xx} + f_0, 0 < x < l, t > 0, f_0 = const, \\
u_x(0, t) = u(l, t) = 0, t > 0, \\
u(x, 0) = 0, 0 \leq x \leq l.
\end{cases}
\end{equation}
\subsection{Решение}
\label{sec:org61a8eab}
Собственные значения и собственные функции соответствующей ЗШЛ:
\begin{equation*}
\begin{cases}
\lambda_n = \left(\frac{\pi(2n + 1)}{2l}\right)^2, \\
X_n = \cos\frac{\pi(2n + 1)}{2l}x.
\end{cases}
\end{equation*}
Ищем решение в виде:
\begin{equation*}
u(x, t) = \sum_{n = 0}^{\infty}u_n(t)\cos\frac{\pi(2n + 1)}{2l}x
\end{equation*}
Найдём разложение $f_0$:
\begin{equation*}
f_0 = \sum_{n = 0}^{\infty}f_n\cos\frac{\pi(2n + 1)}{2l}x, \text{ где}
\end{equation*}
\begin{equation*}
f_n = \frac2l\int_0^lf_0\cos\frac{\pi(2n + 1)}{2l}xdx =
\frac{2f_0}{l}\frac{2l}{\pi(2n + 1)}\sin\frac{\pi(2n + 1)}{2l}x\bigg|_0^l =
\frac{4f_0}{\pi(2n + 1)}(-1)^n
\end{equation*}
Подставляя эти разложения в (5), получим систему задач Коши на $u_n(t)$:
\begin{equation}
\begin{cases}
u_n'(t) = -\left(\frac{\pi(2n + 1)}{2l}\right)^2u_n(t) + \frac{4f_0}{\pi(2n + 1)}(-1)^n, \\
u_n(0) = 0.
\end{cases}
\end{equation}
Решение будем искать в виде:
\begin{equation*}
u_n(t) = C(t)\exp\left\{-\left(\frac{\pi(2n + 1)}{2l}\right)^2t\right\}
\end{equation*}
Тогда
\begin{equation*}
C'(t)\exp\left\{-\left(\frac{\pi(2n + 1)}{2l}\right)^2t\right\} = \frac{4f_0}{\pi(2n + 1)}(-1)^n.
\end{equation*}
\begin{equation*}
C'(t) = \frac{4f_0}{\pi(2n + 1)}(-1)^n\exp\left\{\left(\frac{\pi(2n + 1)}{2l}\right)^2t\right\}
\end{equation*}
\begin{multline*}
C(t) = (-1)^n\frac{4f_0}{\pi(2n + 1)}\left(\frac{2l}{\pi(2n + 1)}\right)^2\exp\left\{\left(\frac{\pi(2n + 1)}{2l}\right)^2t\right\} + C_0 = \\
= (-1)^n\frac{16f_0l^2}{\pi^3(2n + 1)^3}\exp\left\{\left(\frac{\pi(2n + 1)}{2l}\right)^2t\right\} + C_0
\end{multline*}
Тогда для $u_n$ получаем:
\begin{equation*}
u_n(t) = (-1)^n\frac{16f_0l^2}{\pi^3(2n + 1)^3} + C_0\exp\left\{-\left(\frac{\pi(2n + 1)}{2l}\right)^2t\right\}
\end{equation*}
Подставив в начальное условие, окончательно находим:
\begin{equation*}
u_n(t) = (-1)^n\frac{16f_0l^2}{\pi^3(2n + 1)^3}\left(1 - \exp\left\{-\left(\frac{\pi(2n + 1)}{2l}\right)^2t\right\}\right)
\end{equation*}
И для $u(x, t)$ получаем выражение:
\begin{equation}
u(x, t) = \sum_{n = 0}^{\infty}(-1)^n\frac{16f_0l^2}{\pi^3(2n + 1)^3}\left(1 - \exp\left\{-\left(\frac{\pi(2n + 1)}{2l}\right)^2t\right\}\right)\cos\frac{\pi(2n + 1)}{2l}x
\end{equation}
\section{Задача 4.7}
\label{sec:org0dff4da}
\begin{equation}
\begin{cases}
u_t = a^2u_{xx}, 0 < x < l, t > 0, \\
u_x(0, t) = 0, u_x(l, t) = p, t > 0, p = const, \\
u(x, 0) = 0, 0 \leq x \leq \pi.
\end{cases}
\end{equation}
\subsection{Решение}
\label{sec:orgcaf0e1a}
Ищем решение в виде $u = U + v$, где $U = a(t)x^2 + b(t)x$ и удовлетворяет краевым условиям,
а $v$ - решение задачи с однородными краевыми условиями. Подставив краевые условия, найдём:
\begin{equation*}
\begin{cases}
2a(t)0 + b(t) = 0, \\
2a(t)l + b(t) = p
\end{cases}
\Rightarrow
\begin{cases}
a(t) = \frac{p}{2l}, \\
b(t) = 0.
\end{cases}
\end{equation*}
Получили, что
\begin{equation}
u = v + \frac{p}{2l}x^2.
\end{equation}
Подставив это представление в (8), получим задачу на $v$:
\begin{equation}
\begin{dcases}
v_t = a^2\left(v_{xx} + \frac{p}l\right) = a^2v_{xx} + \frac{pa^2}l, 0 < x < l, t > 0, \\
v_x(0, t) = v_x(l, t) = 0, t > 0, \\
v(x, 0) = -\frac{p}{2l}x^2, 0 \leq x \leq \pi.
\end{dcases}
\end{equation}
Собственные значения и собственные функции соответствующей ЗШЛ:
\begin{equation*}
\begin{cases}
\lambda_n = \left(\frac{\pi n}l\right)^2, \\
X_n = \cos\frac{\pi n}lx.
\end{cases}
\end{equation*}
Ищем решение в виде:
\begin{equation}
v(x, t) = \sum_{n = 0}^{\infty}v_n(t)\cos\frac{\pi n}lx.
\end{equation}
Найдём разложения $\frac{pa^2}l$ и $v(x, 0)$:
\begin{equation*}
\frac{pa^2}{l} = \sum_{n = 0}^{\infty}f_n\cos\frac{\pi n}l \Rightarrow f_0 = \frac{pa^2}l, f_n = 0, n \neq 0,
\end{equation*}
\begin{equation*}
v(x, 0) = \sum_{n = 0}^{\infty}v_n(0)\cos\frac{\pi n}lx = -\frac{p}{2l}x^2
\end{equation*}
\begin{equation*}
v_0(0) = \frac1l\int_0^l-\frac{p}{2l}x^2dx = -\frac{p}{2l^2}\frac{l^3}3 = -\frac{pl}6, \\
\end{equation*}
\begin{multline*}
v_n(0) = \frac2l\int_0^l-\frac{p}{2l}x^2\cos\frac{\pi n}lxdx =
-\frac{p}{l^2}\frac{l}{\pi n}\int_0^lx^2d\left(\sin\frac{\pi n}lx\right) = \\
= -\frac{p}{\pi nl}\left(x^2\sin\frac{\pi n}lx\bigg|_0^l - 2\int_0^lx\sin\frac{\pi n}lxdx\right) =
-\frac{2p}{\pi nl}\frac{l}{\pi n}\int_0^lxd\left(\cos\frac{\pi n}lx\right) = \\
= -\frac{2p}{\pi^2n^2}\left(x\cos\frac{\pi n}lx\bigg|_0^l - \int_0^l\cos\frac{\pi n}lxdx\right) =
\frac{2pl}{\pi^2n^2}(-1)^{n + 1} + \frac{2p}{\pi^2n^2}\frac{l}{\pi n}\sin\frac{\pi n}lx\bigg|_0^l =
\frac{2pl}{\pi^2n^2}(-1)^{n + 1}
\end{multline*}
Получаем систему задач Коши для $v_n(t)$:
\begin{equation}
\begin{cases}
v_n' = -\left(\frac{\pi na}l\right)^2v_n + f_n, \\
v_n(0) = v_n(0).
\end{cases}
\end{equation}
При $n = 0$:
\begin{equation*}
\begin{dcases}
v_0' = -\left(\frac{\pi na}l\right)^2v_n + \frac{pa^2}l, \\
v_0(0) = -\frac{pl}6.
\end{dcases}
\end{equation*}
Ищем решение в виде
\begin{equation*}
v(t) = C(t)\exp\left\{-\left(\frac{\pi na}l\right)^2t\right\}
\end{equation*}
Тогда
\begin{equation*}
C'(t)\exp\left\{-\left(\frac{\pi na}l\right)^2t\right\} = \frac{pa^2}l
\end{equation*}
\begin{equation*}
C'(t) = \frac{pa^2}l\exp\left\{\left(\frac{\pi na}l\right)^2t\right\},
\end{equation*}
\begin{equation*}
C(t) = \frac{pa^2}l\frac{l}{\pi na}\exp\left\{\left(\frac{\pi na}l\right)^2t\right\} + C_0 =
\frac{pa}{\pi n}\exp\left\{\left(\frac{\pi na}l\right)^2t\right\} + C_0.
\end{equation*}
Для $v_0(t)$ получаем:
\begin{equation*}
v_0(t) = \frac{pa}{\pi n} + C_0\exp\left\{-\left(\frac{\pi na}l\right)^2t\right\}
\end{equation*}
Подставляя начальные условия:
\begin{equation*}
v_0(0) = \frac{pa}{\pi n} + C_0 = -\frac{pl}6 \Rightarrow C_0 = -\frac{p(6a + \pi nl)}{6\pi n},
\end{equation*}
т. е.
\begin{equation}
v_0(t) = \frac{pa}{\pi n} - \frac{p(6a + \pi nl)}{6\pi n}\exp\left\{-\left(\frac{\pi na}l\right)^2t\right\}
\end{equation}
При $n \neq 0$:
\begin{equation}
\begin{dcases}
v'_n = -\left(\frac{\pi na}l\right)^2v_n, \\
v_n(0) = \frac{2pl}{\pi^2n^2}(-1)^{n + 1}.
\end{dcases}
\end{equation}
Решением этой задачи будет функция:
\begin{equation}
v_n(t) = \frac{2pl}{\pi^2n^2}(-1)^{n + 1}\exp\left\{-\left(\frac{\pi na}l\right)^2t\right\}.
\end{equation}
Подставляя найденные функции $v_n(t)$ в (11) и (9), находим окончательно выражение для $u(x, t)$:
\begin{equation}
u(x, t) = \frac{p}{2l}x^2 + \frac{pa}{\pi n} - \frac{p(6a + \pi nl)}{6\pi n}\exp\left\{-\left(\frac{\pi na}l\right)^2t\right\} +
\sum_{n = 1}^{\infty}\frac{2pl}{\pi^2n^2}(-1)^{n + 1}\exp\left\{-\left(\frac{\pi na}l\right)^2t\right\}\cos\frac{\pi n}lx
\end{equation}
\section{Задача 4.8}
\label{sec:org56f231a}
\begin{equation}
\begin{cases}
u_t = u_{xx} + \cos(t), 0 < x < \pi, t > 0, \\
u(0, t) = u(\pi, t) = 0, t > 0, \\
u(x, 0) = 0, 0 \leq x \leq \pi.
\end{cases}
\end{equation}
\subsection{Решение}
\label{sec:org169c7d6}
Собственные значения и собственные функции соответствующей ЗШЛ:
\begin{equation*}
\begin{cases}
\lambda_n = n^2, \\
X_n = \sin nx.
\end{cases}
\end{equation*}
Ищем решение в виде:
\begin{equation}
u(x, t) = \sum_{n = 0}^{\infty}u_n(t)\sin nx
\end{equation}
Найдём разложение неоднородности:
\begin{equation*}
\cos t = \sum_{n = 0}^{\infty}f_n(t)\sin nx, \text{ где}
\end{equation}
\begin{equation*}
f_n(t) = \int_0^{\pi}\cos t\sin nxdx = -\frac{\cos t}n\cos{nx}|_0^{\pi}
= \frac{\cos t}n(1 - (-1)^n) = \begin{cases}
\frac{2\cos t}n, n = 2k + 1, k \in \mathbb{N}_0, \\
0, k \in 2k, k \in \mathbb{N}_0.
\end{cases}
\end{equation*}
Получаем систему задач Коши для $u_n(t)$:
\begin{equation}
\begin{cases}
u_n' = -n^2u_n + f_n, \\
u_n(0) = 0.
\end{cases}
\end{equation}
Для чётных $n$ $u_n = 0$. Для нечётных $n$:
\begin{equation*}
\begin{dcases}
u_n' = -n^2u_n + \frac{2\cos t}n, \\
u_n(0) = 0.
\end{dcases}
\end{equation*}
Ищем решение в виде:
\begin{equation*}
u_n = C(t)e^{-n^2t}.
\end{equation*}
Тогда
\begin{equation}
C'(t)e^{-n^2t} = \frac{2\cos t}n \Rightarrow C'(t) = \frac{2e^{-n^2t}\cos t}n
\end{equation}
Пусть $I_a = \int_0^ae^{-n^2t}\cos tdt$. Тогда:
\begin{multline*}
I_a = \int_0^ae^{-n^2t}d(\sin t) = e^{-n^2t}\sin t\bigg|_0^a + n^2\int_0^ae^{-n^2t}\sin tdt = \\
= e^{-n^2a}\sin a - n^2\int_0^ae^{-n^2t}d(\cos t) = \\
= e^{-n^2a}\sin a - n^2\left(e^{-n^2t}\cos t\bigg|_0^a + n^2\int_0^ae^{-n^2t}\cos tdt\right) = \\
= e^{-n^2a}\sin a - n^2(e^{-n^2a}\cos a - 1) - n^4I_a
\end{multline*}
Отсюда получаем выражение для $I_a$:
\begin{equation*}
I_a = \frac{e^{-n^2a}\sin a - n^2(e^{-n^2a}\cos a - 1)}{n^4 + 1}
\end{equation*}
Тогда для $C(t)$:
\begin{equation}
C(t) = \frac{2e^{-n^2t}\sin t - 2n^2e^{-n^2t}\cos t + 2n^2}{n(n^4 + 1)} + C_0.
\end{equation}
Поскольку $u_n(0) = 0$, то $C(0) = 0$, поэтому $C_0 = 0$, что даёт для $u_n$:
\begin{equation}
u_n(t) = \frac{2e^{-2n^2t}\sin t - 2n^2e^{-2n^2t}\cos t + 2ne^{-n^2t}}{n^5 + n}.
\end{equation}
Подставляя в (18), получим для $u(x, t)$:
\begin{equation}
u(x, t) = \sum_{k = 0}^{\infty}\frac{2e^{-2(2k + 1)^2t}\sin t - 2(2k + 1)^2e^{-2(2k + 1)^2t}\cos t + 2(2k + 1)e^{-(2k + 1)^2t}}{(2k + 1)^5 + (2k + 1)}\sin(2k + 1)x
\end{equation}
\section{Задача 4.11}
\label{sec:orgd172c28}
\begin{equation}
\begin{cases}
u_t = u_{xx} + x, 0 < x < \pi, t > 0, \\
u(0, t) = u(\pi, t) = 0, t > 0, \\
u(x, 0) = \sin 2x, 0 \leq x \leq \pi.
\end{cases}
\end{equation}
\subsection{Решение}
\label{sec:org108bdc9}
Собственные значения и собственные функции соответствующей ЗШЛ:
\begin{equation}
\begin{cases}
\lambda_n = n^2, \\
X_n = \sin nx.
\end{cases}
\end{equation}
Ищем решение в виде:
\begin{equation}
u(x, t) = \sum_{n = 0}^{\infty}u_n(t)\sin nx.
\end{equation}
Найдём разложения $x$ и $u(x, 0)$:
\begin{equation*}
u(x, 0) = \sum_{n = 0}^{\infty}v_n\sin nx = \sin 2x \Rightarrow v_2 = 1, v_n = 0, n \neq 2.
\end{equation*}
\begin{equation*}
x = \sum_{n = 0}^{\infty}f_n\sin nx, \text{ где}
\end{equation*}
\begin{multline*}
f_n = \int_0^{\pi}x\sin nxdx = -\frac1n\int_0^{\pi}xd(\cos nx)dx = \\
= -\frac1n\left(x\cos nx\bigg|_0^{\pi} - \int_0^{\pi}\cos nxdx\right) =
\frac{\pi}n(-1)^{n + 1} + \frac1{n^2}\sin nx\bigg|_0^{\pi} = \frac{\pi}n(-1)^{n + 1}
\end{multline*}
Подставив найденные коэффициенты в задачу (24), получим систему задач Коши:
\begin{equation}
\begin{dcases}
u_n' = -n^2u_n + \frac{\pi}n(-1)^{n + 1}, \\
u_n(0) = v_n.
\end{dcases}
\end{equation}
При $n = 2$:
\begin{equation*}
\begin{cases}
u_n' = -n^2u_n + \frac{\pi}n(-1)^{n + 1}, \\
u_n(0) = 1.
\end{cases}
\end{equation*}
Ищем решение в виде
\begin{equation*}
u_n = C(t)e^{-n^2t}.
\end{equation*}
Тогда для $C(t)$:
\begin{equation*}
C'(t)e^{-n^2t} = \frac{\pi}n(-1)^{n + 1} \Rightarrow C'(t) = \frac{\pi}n(-1)^{n + 1}e^{n^2t}
\Rightarrow C(t) = \frac{\pi}{n^3}(-1)^{n + 1}e^{n^2t} + C_0.
\end{equation*}
Так как $u_n(0) = 1$, то $C(0) = 1 \Rightarrow C_0 = 1 + \frac{pi}8$, откуда получаем выражение
для $u_2(t)$:
\begin{equation}
u_2(t) = -\frac{\pi}8 + \left(1 + \frac{\pi}8\right)e^{-4t}.
\end{equation}
При $n \neq 2$:
\begin{equation}
\begin{cases}
u'_n = -n^2u_n + \frac{\pi}n(-1)^{n + 1}, \\
u_n(0) = 0.
\end{cases}
\end{equation}
Ищем решение в виде
\begin{equation*}
u_n = C(t)e^{-n^2t}.
\end{equation*}
Тогда для $C(t)$ получаем:
\begin{equation*}
\begin{cases}
C'(t)e^{-n^2t} = \frac{\pi}n(-1)^{n + 1}, \\
C(0) = 0.
\end{cases}
\Rightarrow
\begin{cases}
C'(t) = \frac{\pi}n(-1)^{n + 1}e^{n^2t}, \\
C(0) = 0.
\end{cases}
\Rightarrow
C = \frac{\pi}{n^3}(-1)^{n + 1}(e^{n^2t} - 1)
\end{equation*}
Откуда следует, что
\begin{equation}
u_n(t) = \frac{\pi}{n^3}(-1)^{n + 1}(1 - e^{-n^2t}).
\end{equation}
Окончательно для $u(x, t)$ имеем:
\begin{equation}
u(x, t) = \pi(1 - e^{-t}) + \left(-\frac{\pi}8 + \left(1 + \frac{\pi}8\right)e^{-4t}\right) +
\sum_{n = 3}^{\infty}\frac{\pi}n^3(-1)^{n + 1}(1 - e^{-n^2t})\sin nx
\end{equation}
\end{document}
