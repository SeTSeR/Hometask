% Created 2020-02-18 Tue 22:10
% Intended LaTeX compiler: pdflatex
\documentclass[11pt]{article}
\usepackage[utf8]{inputenc}
\usepackage[T1]{fontenc}
\usepackage{graphicx}
\usepackage{grffile}
\usepackage{longtable}
\usepackage{wrapfig}
\usepackage{rotating}
\usepackage[normalem]{ulem}
\usepackage{amsmath}
\usepackage{textcomp}
\usepackage{amssymb}
\usepackage{capt-of}
\usepackage{hyperref}
\usepackage{minted}
\usepackage{amsmath}
\usepackage{esint}
\usepackage[english, russian]{babel}
\usepackage{mathtools}
\usepackage{amsthm}
\usepackage[top=0.8in, bottom=0.75in, left=0.625in, right=0.625in]{geometry}
\def\zall{\setcounter{lem}{0}\setcounter{cnsqnc}{0}\setcounter{th}{0}\setcounter{Cmt}{0}\setcounter{equation}{0}}
\newcounter{lem}\setcounter{lem}{0}
\def\lm{\par\smallskip\refstepcounter{lem}\textbf{\arabic{lem}}}
\newtheorem*{Lemma}{Лемма \lm}
\newcounter{th}\setcounter{th}{0}
\def\th{\par\smallskip\refstepcounter{th}\textbf{\arabic{th}}}
\newtheorem*{Theorem}{Теорема \th}
\newcounter{cnsqnc}\setcounter{cnsqnc}{0}
\def\cnsqnc{\par\smallskip\refstepcounter{cnsqnc}\textbf{\arabic{cnsqnc}}}
\newtheorem*{Consequence}{Следствие \cnsqnc}
\newcounter{Cmt}\setcounter{Cmt}{0}
\def\cmt{\par\smallskip\refstepcounter{Cmt}\textbf{\arabic{Cmt}}}
\newtheorem*{Note}{Замечание \cmt}
\renewcommand{\div}{\operatorname{div}}
\newcommand{\rot}{\operatorname{rot}}
\newcommand{\grad}{\operatorname{grad}}
\author{Sergey Makarov}
\date{\today}
\title{}
\hypersetup{
 pdfauthor={Sergey Makarov},
 pdftitle={},
 pdfkeywords={},
 pdfsubject={},
 pdfcreator={Emacs 26.3 (Org mode 9.3)}, 
 pdflang={English}}
\begin{document}

\zall

ДЗ: 8.6, 8.9, 9.1-9.7(прочитать), 9.12, 9.20, 9.13, 9.10, 9.17
\section{Задача 8.6}
\label{sec:org005edef}
Найти функцию Грина и решить задачу Дирихле в области:
\begin{equation}
y > 0, z > 0
\end{equation}
\subsection{Решение}
\label{sec:org5834be6}
Рассмотрим точку $Q(x_0, y_0, z_0)$. Введём точки $Q_1(x_0, -y_0, z_0), Q_2(x_0, -y_0, -z_0), Q_3(x_0, y_0, -z_0)$.
Тогда, поскольку $G(P, Q)$ на границе равна нулю, она имеет вид:
\begin{multline}
G(P, Q) = \frac1{4\pi r_{PQ}} - \frac1{4\pi r_{PQ_1}} + \frac1{4\pi r_{PQ_2}} - \frac1{4\pi r_{PQ_3}} =
\frac1{4\sqrt{(x - x_0)^2 + (y - y_0)^2 + (z - z_0)^2}} - \\
- \frac1{4\sqrt{(x - x_0)^2 + (y + y_0)^2 + (z - z_0)^2}} +
\frac1{4\sqrt{(x - x_0)^2 + (y + y_0)^2 + (z + z_0)^2}} - \\
- \frac1{4\sqrt{(x - x_0)^2 + (y - y_0)^2 + (z + z_0)^2}}
\end{multline}
Далее решение можно найти из формулы Грина:
\begin{equation}
u(Q) = -\iint_Sf(P)\frac{\partial G}{\partial n_P}dS_P
\end{equation}
\section{Задача 8.9}
\label{sec:orga395a55}
Найти функцию Грина и решить задачу Дирихле в области:
\begin{equation}
x^2 + y^2 + z^2 < 1
\end{equation}
\subsection{Решение}
\label{sec:org06df165}
Рассмотрим точку $Q(x_0, y_0, z_0)$. Рассмотрим симметричную ей относительно окружности точку
$$Q'\left(\frac{x_0}{x_0^2 + y_0^2 + z_0^2}, \frac{y_0}{x_0^2 + y_0^2 + z_0^2}, \frac{z_0}{x_0^2 + y_0^2 + z_0^2}\right)$$
Тогда функция Грина будет иметь вид:
\begin{equation}
G(P, Q) = \frac1{4\pi r_{PQ}} - \frac1{\sqrt{x_0^2 + y_0^2 + z_0^2}}\frac1{4\pi r_{PQ'}}
\end{equation}
Решение можно найти из теоремы о среднем:
\begin{equation}
u(Q) = \frac1{4\pi a^2}\oiint_Sudl
\end{equation}
\end{document}
