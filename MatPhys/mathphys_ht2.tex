% Created 2019-09-26 Thu 16:26
% Intended LaTeX compiler: pdflatex
\documentclass[11pt]{article}
\usepackage[utf8]{inputenc}
\usepackage[T1]{fontenc}
\usepackage{graphicx}
\usepackage{grffile}
\usepackage{longtable}
\usepackage{wrapfig}
\usepackage{rotating}
\usepackage[normalem]{ulem}
\usepackage{amsmath}
\usepackage{textcomp}
\usepackage{amssymb}
\usepackage{capt-of}
\usepackage{hyperref}
\usepackage{amsmath}
\usepackage{esint}
\usepackage[english, russian]{babel}
\usepackage{mathtools}
\usepackage{amsthm}
\usepackage[top=0.8in, bottom=0.75in, left=0.625in, right=0.625in]{geometry}
\author{Sergey Makarov}
\date{\today}
\title{}
\hypersetup{
 pdfauthor={Sergey Makarov},
 pdftitle={},
 pdfkeywords={},
 pdfsubject={},
 pdfcreator={Emacs 26.3 (Org mode 9.1.9)}, 
 pdflang={English}}
\begin{document}


\section{Задача 2.5}
\label{sec:orgd769ad6}
\begin{equation}
u_t = u_{xx}
\end{equation}
$U(t, x)$ - решение (1). $U_5(t, x) = (1 + 4ct)^{-1/2}\exp\{-\frac{cx^2}{1 + 4ct}\}\cdot U\left(\frac{x}{1 + 4ct}, \frac{t}{1 + 4ct}\right)$ - решение (1)?

\begin{multline}
U_{5t} = -\frac124c(1 + 4ct)^{-3/2}\exp\left\{-\frac{cx^2}{1 + 4ct}\right\}U\left(\frac{x}{1 + 4ct}, \frac{t}{1 + 4ct}\right) + \\
+ (1 + 4ct)^{-1/2}\left(-\exp\left\{-\frac{cx^2}{1 + 4ct}\right\}\frac{x^2}4\left(-\frac1{\left(\frac1{4c} + t\right)^2}\right)\right)U\left(\frac{x}{1 + 4ct}, \frac{t}{1 + 4ct}\right) + \\
+ (1 + 4ct)^{-1/2}\exp\left\{-\frac{cx^2}{1 + 4ct}\right\}(U_x\left(\frac{x}{1 + 4ct}, \frac{t}{1 + 4ct}\right)\left(-\frac{4cx}{(1 + 4ct)^2}\right) + \\
+ U_t\left(-\frac{x}{1 + 4ct}, \frac{t}{1 + 4ct}\right)\frac1{16c^2t^2})
\end{multline}
\begin{equation*}
\left(\frac{x}{1 + 4ct}\right)_t = \frac{x}{4c}\left(-\frac1{\left(\frac1{4c} + t\right)^2}\right) =
-\frac{4cx}{1 + 8ct + 16ct^2}
\end{equation*}
\begin{equation*}
\left(\frac{t}{1 + 4ct}\right)_t = \frac1{4c}\left(\frac{4ct}{1 + 4ct}\right)_t = \frac1{4c}\left(1 - \frac1{4ct}\right)_t = \frac1{16c^2t^2}
\end{equation*}
\begin{multline*}
\left(\exp\left\{-\frac{cx^2}{1 + 4ct}\right\}\right)_{xx} = \left(-\frac{2cx}{1 + 4ct}\exp\left\{-\frac{cx^2}{1 + 4ct}\right\}\right)_x = \\
= -\left(\frac{2c}{1 + 4ct}\exp\left\{-\frac{cx^2}{1 + 4ct}\right\} - \frac{4c^2x^2}{(1 + 4ct)^2}\exp\left\{-\frac{cx^2}{1 + 4ct}\right\}\right) = \\
= \exp\left\{-\frac{cx^2}{1 + 4ct}\right\}\left(\left(\frac{2cx}{1 + 4ct}\right)^2 - \frac{2cx}{1 + 4ct}\right)
\end{multline*}
\begin{multline}
U_{5xx} = (1 + 4ct)^{-1/2}(\exp\left\{-\frac{cx^2}{1 + 4ct}\right\}\left(\left(\frac{2cx}{1 + 4ct}\right)^2 - \frac{2cx}{1 + 4ct}\right)
U\left(\frac{x}{1 + 4ct}, \frac{t}{1 + 4ct}\right) + \\
+ 2\left(-\frac{2cx}{1 + 4ct}\exp\left\{-\frac{cx^2}{1 + 4ct}\right\}\frac1{1 + 4ct}U_x\left(\frac{x}{1 + 4ct}, \frac{t}{1 + 4ct}\right)\right) + \\
+ \exp\left\{-\frac{cx^2}{1 + 4ct}\right\}U_{xx}\left(\frac{x}{1 + 4ct}\right)^2)
\end{multline}
Сопоставляя коэффициенты при $U$, $U_x$, $U_{xx}$ и учитывая, что $U_{xx} = U_t$ получаем, что $U_5$
является решением (1).
\section{Задача 2.13}
\label{sec:orgebf0dba}
Если источник подключен к левому концу:
\begin{equation}
\begin{cases}
u_x(0, t) = -\frac{Q}{kS}, \\
u_x(l, t) = 0.
\end{cases}
\end{equation}
К правому:
\begin{equation}
\begin{cases}
u_x(0, t) = 0, \\
u_x(l, t) = \frac{Q}{kS}.
\end{cases}
\end{equation}
\section{Задача 2.15}
\label{sec:org58ae717}
\begin{equation}
\begin{cases}
u_t = a^2u_{xx}, \\
u(0, t) = u(l, t) = 0, \\
u(x, 0) = 0.
\end{cases}
\end{equation}
\end{document}
