% Created 2019-09-30 Mon 18:50
% Intended LaTeX compiler: pdflatex
\documentclass[11pt]{article}
\usepackage[utf8]{inputenc}
\usepackage[T1]{fontenc}
\usepackage{graphicx}
\usepackage{grffile}
\usepackage{longtable}
\usepackage{wrapfig}
\usepackage{rotating}
\usepackage[normalem]{ulem}
\usepackage{amsmath}
\usepackage{textcomp}
\usepackage{amssymb}
\usepackage{capt-of}
\usepackage{hyperref}
\usepackage{amsmath}
\usepackage{esint}
\usepackage[english, russian]{babel}
\usepackage{mathtools}
\usepackage{amsthm}
\usepackage[top=0.8in, bottom=0.75in, left=0.625in, right=0.625in]{geometry}
\def\zall{\setcounter{lem}{0}\setcounter{cnsqnc}{0}\setcounter{th}{0}\setcounter{Cmt}{0}\setcounter{equation}{0}}
\newcounter{lem}\setcounter{lem}{0}
\def\lm{\par\smallskip\refstepcounter{lem}\textbf{\arabic{lem}}}
\newtheorem*{Lemma}{Лемма \lm}
\newcounter{th}\setcounter{th}{0}
\def\th{\par\smallskip\refstepcounter{th}\textbf{\arabic{th}}}
\newtheorem*{Theorem}{Теорема \th}
\newcounter{cnsqnc}\setcounter{cnsqnc}{0}
\def\cnsqnc{\par\smallskip\refstepcounter{cnsqnc}\textbf{\arabic{cnsqnc}}}
\newtheorem*{Consequence}{Следствие \cnsqnc}
\newcounter{Cmt}\setcounter{Cmt}{0}
\def\cmt{\par\smallskip\refstepcounter{Cmt}\textbf{\arabic{Cmt}}}
\newtheorem*{Note}{Замечание \cmt}
\author{Sergey Makarov}
\date{\today}
\title{}
\hypersetup{
 pdfauthor={Sergey Makarov},
 pdftitle={},
 pdfkeywords={},
 pdfsubject={},
 pdfcreator={Emacs 26.3 (Org mode 9.1.9)}, 
 pdflang={English}}
\begin{document}

Лектор Иновенков Игорь Николаевич. Рейтинговая система, экзамен.

Учебники:
\begin{enumerate}
\item Тихонов, Самарский. "уравнения математической физики"
\item Захаров, Дмитриев, ? "уравнения математической физики"
\end{enumerate}

\section{Классификация уравнений второго порядка с двумя независимыми переменными}
\label{sec:orgbb51327}
Это уравнения вида
\begin{equation}
F(x, y, u_x, u_y, u_{xx}, u_{xy}, u_{yy}) = 0
\end{equation}
Уравнение Монжа-Ампера:
\begin{equation*}
u_{xx}u_{yy} - u_{xy}^2 = \pm 1
\end{equation*}
Будем рассматривать более простой вариант уравнения второго порядка:
\begin{equation*}
a_{11}u_{xx} + 2a_{12}u_{xy} + a_{22}u_{yy} + F(x, y, u, u_x, u_y) = 0
\end{equation*}
Уравнение вида (1) называется \textbf{квазилинейным}, если \(a_{11}, a_{12}\) и \(a_{22}\) зависят от \(x, y, u, u_x, u_y\).

Уравнение вида
\begin{equation}
a_{11}u_{xx} + 2a_{12}u_{xy} + a_{22}u_{yy} + b_1 u_{x} + b_2 u_y + cu + f = 0,
\end{equation}
где \(a_{ij}, b_k, c, f\) - функции от \(x\) и \(y\).
Если \(f \equiv 0\), то уравнение (2) называется \textbf{однородным}.
Рассмотрим замену вида
\begin{equation*}
\begin{cases}
\xi = \varphi(x, y), \\
\eta = \psi(x, y).
\end{cases}
\end{equation*}
Тогда частные производные u запишутся в виде
\begin{equation*}
\begin{cases}
u_x = u_\xi\xi_x + u_\eta\eta_x, \\
u_y = u_\xi\xi_y + u_\eta\eta_y, \\
u_{xx} = u_{\xi\xi}\xi_x^2 + 2u_{\xi\eta}\xi_x\eta_x + u_{\eta\eta}\eta_x^2 + u_{\xi}\xi_{xx} + u_{\eta}\eta_{xx}, \\
u_{xy} = u_{\xi\xi}\xi_x\xi_y + u_{\xi\eta}\xi_x\eta_y + u_{\eta\xi}\eta_x\xi_y + u_{\eta}{\eta}\eta_x\eta_y + u_{\xi}\xi_{xy} + u_{\eta}\eta_{yx}, \\
u_{yy} = u_{\xi\xi}\xi_y^2 + 2u_{\xi\eta}\xi_y\eta_y + u_{\eta\eta}\eta_y^2 + u_\xi\xi_{yy} + u_\eta\eta_{yy}, \\
\end{cases}
\end{equation*}
Подставим эти формулы в (2):
\begin{multline*}
u_{\xi\xi}\left(a_{11}\xi_x^2 + 2a_{12}\xi_x\xi_y + a_{22}\xi_y^2\right) + 2u_{\xi\eta}\left(a_{11}\xi_x\eta_x + a_{12}(\xi_x\eta_y + \xi_y\eta_x) + a_{22}\xi_y\eta_y\right) + \\
+ u_{\eta\eta}\left(a_{11}\eta_x^2 + 2a_{12}\eta_x\eta_y + a_{22}\eta_y^2\right) + F = 0
\end{multline*}
Или:
\begin{equation}
\overline{a_{11}}u_{\xi\xi} + 2\overline{a_{12}}u_{\xi\eta} + \overline{a_{22}}u_{\eta\eta} + F = 0
\end{equation}
Потребуем, чтобы \(\overline{a_{11}} = 0\). Получим уравнение:
\begin{equation}
a_{11}\xi_x^2 + 2a_{12}\xi_x\xi_y + a_{22}\xi_y^2 = 0
\end{equation}
Если \(\varphi(x, y) = C\), то \(\varphi_xdx + \varphi_ydy = 0\), откуда, выразив, например,
\(\varphi_x\), получим
\begin{equation}
a_{11}dy^2 - 2a_{12}dxdy + a_{22}dx^2 = 0
\end{equation}
Уравнение (10) называется \textbf{характеристическим}.
\begin{equation*}
\overline{a_{12}} - \overline{a_{11}}\overline{a_{22}} = (a_{12}^2 - a_{11}a_{22})D^2,
D = \begin{vmatrix}
\xi_x   & \xi_y, \\
\eta_x  & \eta_y               
\end{vmatrix}
\end{equation*}
Расмотрим характеристическое уравнение:
\begin{equation*}
a_{11}\left(\frac{dy}{dx}\right)^2 - 2a_{12}\frac{dy}{dx} + a_{22} = 0, a_{11} \neq 0
\end{equation*}
Его решение имеет вид:
\begin{equation*}
\frac{dy}{dx} = \frac{a_{12} \pm \sqrt{a_{12}^2 - a_{11}a_{22}}}{a_{11}}
\end{equation*}
Если \(a_{12}^2 - a_{11}a_{22} > 0\), то уравнение называется \textbf{гиперболическим}.

Если \(a_{12}^2 - a_{11}a_{22} = 0\), то уравнение называется \textbf{параболическим}.

Остальные уравнения называются \textbf{эллиптическими}.

Пусть уравнение гиперболическое. Тогда \(\overline{a_{11}} = \overline{a_{22}} = 0\) и уравнение
записывается в \textbf{первой канонической форме}:
\begin{equation}
u_{\xi\eta} = \Phi, \Phi = -\frac{F}{2\overline{a_{12}}}
\end{equation}
Положим
\begin{equation}
\begin{cases}
\xi = \alpha + \beta, \\
\eta = \alpha - \beta,
\end{cases}
\end{equation}
Тогда уравнение записывается в виде:
\begin{equation}
u_{\alpha\alpha} - u_{\beta\beta} = \overline\Phi, \overline\Phi = 4\Phi,
\end{equation}
Называемом \textbf{второй канонической формой}.

Перейдём теперь к уравнениям параболического типа. Для них \(\xi\) - интеграл (7), \(\eta\) -
независимая от него функция, \(a_12 = \sqrt{a_{11}}\sqrt{a_{22}}\). Тогда
\begin{equation*}
a_{11}\xi_x^2 + 2\sqrt{a_{11}}\sqrt{a_{22}}\xi_x\xi_y + a_{22}\xi_y^2 =
(\sqrt{a_{11}}\xi_x + \sqrt{a_{22}}\xi_y)^2 = 0
\end{equation*}
\begin{equation*}
\overline{a_{12}} = a_{11}\xi_x\eta_y + \sqrt{a_{11}}\sqrt{a_{22}}(\xi_x\eta_y + \xi_y\eta_x)
+ a_{22}\xi_y\eta_y = (\sqrt{a_{11}}\xi_x + \sqrt{a_{22}}\xi_y)(\sqrt{a_{11}}\eta_x + \sqrt{a_{22}}\eta_y) = 0
\end{equation*}
Откуда
\begin{equation}
\overline{a_{11}} = \overline{a_{22}} = 0, u_{\eta\eta} = \tilde{\Phi}(\xi, \eta, u, u_\xi, u_\eta)
\end{equation}

Пусть теперь у нас уравнение эллиптического типа. Тогда \(\xi\) и \(\eta\) представимы в виде:
\begin{equation*}
\begin{cases}
\xi = \alpha + i\beta, \\
\eta = \alpha - i\beta
\end{cases}
\end{equation*}
\begin{equation*}
\xi_x = \alpha_x + i\beta_x,
\end{equation*}
\begin{multline*}
\overline{a_{11}} = a_{11}(\alpha_x + i\beta_x)^2 +
2a_{12}(\alpha_x + i\beta_x)(\alpha_y + i\beta_y) + a_{22}(\alpha_y + i\beta_y)^2 = \\
= a_{11} + \alpha_x^2 + 2a_{12}\alpha_x\alpha_y + a_{22}\alpha_y^2 + 2i(a_{11}\alpha_x\beta_x +
a_{12}(\alpha_x\beta_y + \alpha_y\beta_x) + a_{22}\alpha_y\beta_y) + (a_{11}\beta_x^2 +
2a_{12}\beta_x\beta_y + a_{22}\beta_y^2) = 0
\end{multline*}

Откуда
\begin{equation*}
\overline{a_{12}} = a_{11}\alpha_x\beta_x + a_{12}(\alpha_x\beta_y + \alpha_y\beta_x) + a_{22}\alpha_y\beta_y = 0
\end{equation*}

Рассмотрим случай постоянных коэффициентов:
\begin{equation}
\begin{cases}
u_{xx} - u_{yy} + b_1u_x + b_2u_y + cu + f = 0, \\
u_{xx} - u_{yy} + b_1u_x + cu + f = 0, \\
u_{xx} + u_{yy} + b_1u_x + b_2u_y + cu + f = 0.
\end{cases}
\end{equation}
В гиперболическом случае:

Сделаем замену \(u(x, y) = e^{\lambda x + \mu y}\nu(x, y)\). Получим:
\begin{multline*}
e^{\lambda x + \mu y}(\nu_{xx} + 2\lambda\nu_x + \lambda^2\nu) - e^{\lambda x + \mu y}(\nu_{yy} + 2\mu\nu_y + \mu^2\nu) + \\
+ e^{\lambda x + \mu y}(b_1\nu_x + b_2\mu\nu) + e^{\lambda x + \mu y}(b_2\nu_y + b_2\mu\nu) + ce^{\lambda x + \mu y}\nu + f = 0
\end{multline*}
Таким образом, уравнения с постоянными коэффициентами приводятся к виду:
\begin{equation}
\begin{cases}
u_{xx} - u_{yy} + cu + f = 0, \\
u_{xx} - u_y + cu + f = 0, \\
u_{xx} + u_{yy} + cu + f = 0.
\end{cases}
\end{equation}
Частные случаи, которые мы будем рассматривать:
\begin{equation}
\begin{cases}
u_t = u_{xx}, \\
u_{xx} + u_{yy} = 0 \text{ или } \Delta u = 0, \\
u_{tt} = u_{xx}.
\end{cases}
\end{equation}

\section{Уравнение непрерывности}
\label{sec:org6d15803}
Выведем \textbf{уравнение непрерывности}. Пусть P - плотность некоторой величины, \(\vec j\) - плотность
её потока, \(f\) - интенсивность её источника. Уравнение баланса для этой величины будет иметь вид:
\begin{equation}
\frac{\partial}{\partial t}\iiint_{\Delta V}Pd\tau = -\oiint\vec jd\vec j + \iiint_{\Delta V}fdt
\end{equation}
Применяя формулу Остроградского-Гаусса, получим:
\begin{equation}
\frac{\partial P}{\partial t} + \operatorname{div}\vec{j} = f
\end{equation}
Уравнение теплопроводности
\begin{equation}
C_{\rho}\frac{\partial u}{\partial t} = \operatorname{div}(k\operatorname{grad}u) + f(x, y, z, t)
\end{equation}
\section{Смешанная задача для уравнения в частных производных параболического типа}
\label{sec:orgd7cfbc1}
Уравнение + граничные условия + начальные условия:
\begin{equation}
\begin{cases}
u|_{\Sigma} = u(M, t), M \in \Sigma \text{ - граничные условия первого рода} \\
\vec{q}|_{\Sigma} = \vec{\nu}(M, t), q = -k\operatorname{grad}u \text{ - граничные условия второго рода}, \\
-k\frac{\partial u}{\partial n}|_{\Sigma} = \nu_n(M, t), \\
\frac{\partial u}{\partial n} + \alpha u_{\sigma} + \chi(M, t), M \in \Sigma \text{ - граничные условия третьего рода}.
\end{cases}
\end{equation}
Рассмотрим смешанную задачу для уравнения теплопроводности:
\begin{equation}
\begin{cases}
\frac{\partial u}{\partial t} = a^2\frac{\partial^2 u}{\partial x^2} + f(x, t), 0 < x < l, 0 < t < +\infty, \\
u(0, t) = \mu_1(t), \\
u(l, t) = \mu_2(t), \\
u(x, 0) = \varphi(x),
\end{cases}
\end{equation}
\begin{equation}
\begin{cases}
u_x(0, t) = \nu_1(t), \\
u_x(l, t) = \nu_2(t), \\
u_x - h_1u|_{x = 0} = \theta_1(t), \\
u_x + h_2u|_{x = l} = \theta_2(t).
\end{cases}
\end{equation}
Ищем решение типа $u(x, t) \in C^{2, 1}((0, l) \times [0, T]) \cap C([0, l] \times [0, T])$.
Тогда $f(x, t), \nu_1(t), \nu_2(t), \varphi(x)$ непрерывны.
Также $\nu_1(0) = \varphi(0), \nu_2(0) = \varphi(l)$.
\subsection{Метод разделения переменных}
\label{sec:orgb5277a3}
Рассмотрим простейший вариант смешанной задачи:
\begin{equation}
\begin{cases}
\frac{\partial u}{\partial t} = a\frac{\partial^2 u}{\partial x^2}, 0 < x < l, 0 < t < +\infty, \\
u(0, t) = 0, \\
u(l, t) = 0, \\
u(x, 0) = \varphi(x).
\end{cases}
\end{equation}
Ищем частное решение в виде $u(x, t) = X(x)T(t)$. Получим:
\begin{equation}
\begin{cases}
XT' = a^2X''T, \\
X(0)T(t) = 0, \\
X(l)T(t) = 0.
\end{cases}
\end{equation}
Откуда
\begin{equation}
\begin{cases}
\frac{T'}{a^2T} = \frac{X''}{X} = -\lambda, \\
X(0) = X(l) = 0.
\end{cases}
\end{equation}
Обозначим
\begin{equation}
Lv = \frac{d^2v}{dx^2}, \\
Lv = \mu v \Rightarrow (Lv, v) \leq 0
\end{equation}
Решая уравнения, получим:
\begin{equation}
\begin{cases}
X'' + \lambda X = 0, \\
X(0) = X(l) = 0.
\end{cases}
\Rightarrow
\begin{cases}
X(x) = C_1\cos\sqrt{\lambda}x + C_2\sin\sqrt{\lambda}x, \\
X(0) = C_1 = 0, X(l) = C_2\sin\sqrt{\lambda}l = 0.
\end{cases}
\Rightarrow
X_{n}(x) = \sin\frac{\pi nx}l.
\end{equation}
\begin{equation}
T' + \left(\frac{\pi n a}l\right)^2 = 0
\Rightarrow
T_n(t) = e^{-\frac{\pi n a}lt}a_n
\end{equation}
Получим систему частных решений $u_n(x, t) = a_ne^{-\left(\frac{\pi na}l\right)^2t}\sin\frac{\pi n x}l$.
Тогда общее решение будет иметь вид:
\begin{equation}
u(x, t) = \sum_{n = 1}^{\infty}\varphi_ne^{-\left(\frac{\pi na}l\right)^2t}\sin\frac{\pi nx}t,
\end{equation}
Где $\varphi_n$ - коэффициенты разложения $\varphi(x)$ в ряд Фурье.

Такая схема плохо работает при более общих начальных условиях.
\subsection{Общая схема метода разделения переменных}
\label{sec:orgf61d355}
Рассмотрим задачу
\begin{equation}
\begin{cases}
\frac{\partial u}{\partial t} = Lu + f(x, t), \\
l_1u|_{x = 0} = \mu_1(t), \\
l_2u|_{x = l} = \mu_2(t), \\
u|_{t = 0} = \varphi(x).
\end{cases}
\end{equation}
\begin{enumerate}
\item Проверяем самосопряжённость задачи, т. е. что \(L = L^*\). Если она не является самосопряжённой,
\end{enumerate}
пытаемся привести её к таковой.

a) \(L = \frac{d^2}{dx^2} + \frac{d}{dx}\). Привести можно с помощью замены \(u(x, t) = e^{\mu t}\nu(x, t)\).

b) \(L = \frac{d^2}{dx^2}\). Привести не получается.
Задача Самарского-Ионкина:
\begin{equation}
\begin{cases}
\frac{\partial u}{\partial t} = \frac{\partial^2u}{\partial x^2}, 0 < x < \pi, \\
u(0, t) = 0, \\
u_x(0, t) = u_x(\pi, t)
\end{cases}
\end{equation}
2. Приводим граничные условия к однородным.($f(kx, ky) = kf(x, y)$.)
a) \begin{equation}
\begin{cases}
u(0, t) = \mu_1(t), \\
u(l, t) = \mu_2(t).
\end{cases}
\end{equation}
Ищем решения в виде $u(x, t) = v(x, t) + w(x, t)$, где
\begin{equation}
\begin{cases}
w(0, t) = \mu_1, \\
w(l, t) = \mu_2.
\end{cases}
\end{equation}
Проще всего взять $w(x, t) = (\mu_2 - \mu_1)\frac{x}l + \mu_1$.
b) \begin{equation}
\begin{cases}
u(0, t) = \mu_1(t), \\
u_x(l, t) = \mu_2(t).
\end{cases}
\end{equation}
Возьмём $w(x, t) = \mu_2(t)x +\mu_1$.
c) \begin{equation}
\begin{cases}
u_x(0, t) = \mu_1, \\
u_x(l, t) = \mu_2.
\end{cases}
\end{equation}
Ищем $w(x)$ в виде $w(x, t) = ax^2 + bx$.
\begin{enumerate}
\item Получили \textbf{задачу Штурма-Лиувилля}:
\end{enumerate}
\begin{equation}
\begin{cases}
LX + \lambda X = 0, \\
l_1X|_{x = 0} = 0, \\
l_2X|_{x = l} = 0.
\end{cases}
\end{equation}
\begin{Theorem}
Собственные функции, отвечающие различным собственным значениям, ортогональны.
\end{Theorem}
\begin{enumerate}
\item Раскладываем правую часть, граничные условия и решение по собственным функциям:
\end{enumerate}
\begin{equation}
\begin{dcases}
v(x, t) = \sum_{n = 1}^{\infty}v_n(t)X_n(x), \\
\overline{f}(x, t) = \sum_{n = 1}^{\infty}f_n(t)X_n(x), f_n(t) = \frac1{||x_n||^2}\int_{o}^l\overline{f}(x, t)X_n(x)dx, \\
\overline{\varphi}(x) = \sum_{n = 1}^{\infty}\varphi_nX_n(x), \varphi_n = \frac1{||x_n||^2}\int_0^l\overline{\varphi}(x)X_n(x)dx.
\end{dcases}
\end{equation}
Подставляя в исходное уравнение, получаем:
\begin{equation}
\begin{dcases}
\sum_{n = 1}^{\infty}\left(\frac{dv_n}{dt} + \lambda_nv_n + f_n(t)\right)X_n(x) = 0, \\
\sum_{n = 1}^{\infty}(v_{n}(0) - \varphi_n)X_n(x) = 0.
\end{dcases}
\end{equation}
$$u(x, t) = \sum_{n = 1}^{\infty}v_{n}(t)X(x) + w(x, t)$$
\subsection{Теорема существования}
\label{sec:org3719d07}
Рассмотрим задачу
\begin{equation}
\begin{dcases}
\frac{\partial u}{\partial t} = a^2\frac{\partial^2 u}{\partial x^2}, 0 < x < l, 0 < t < +\infty, \\
u(0, t) = 0, \\
u(l, t) = 0, \\
u(x, 0) = \varphi(x).
\end{dcases}
\end{equation}
Методом, рассмотренным выше, получаем разложение:
\begin{equation}
u(x, t) = \sum_{n = 1}^{\infty}\varphi_ne^{-\left(\frac{\pi n a}l\right)^2t}\sin{\frac{\pi n x}l}
\end{equation}
$$u_t(x, t) \approx \sum_{n = 1}^{\infty}\varphi_n\left(-\left(\frac{\pi n a}l\right)^2\right)e^{-\left(\frac{\pi n a}l\right)^2t}\sin{\frac{\pi n x}l}$$
$$u_{xx}(x, t) \approx \sum_{n = 1}^{\infty}-\varphi_n\left(\frac{\pi n}l\right)^2e^{-\left(\frac{\pi n x}l\right)^2}\sin\frac{\pi n x}l$$
Оба ряда сходятся равномерно и являются непрерывными функциями $\Rightarrow$ решение является хорошим.

Рассмотрим теперь неоднородную задачу:
\begin{equation}
\begin{dcases}
\frac{dz_n}{dt} + \left(\frac{\pi na}lz_n\right)^2 = f_n(t), \\
z_n(0) = 0.
\end{dcases}
\end{equation}
Тогда $z_n(t) = \int_0^tU(t - \tau)f_n(\tau)d\tau$, где
\begin{equation}
\begin{dcases}
\frac{dU}{dt} + \left(\frac{\pi n a}l\right)^2U = 0, \\
U(0) = 1
\end{dcases}
\Rightarrow U(t) = e^{-\left(\frac{\pi n a}l\right)^2t}
\end{equation}

Рассмотрим задачу:
\begin{equation}
\begin{dcases}
\frac{\partial u}{\partial t} = \frac{\partial^2 u}{\partial x^2} + \cos x\cdot e^{-t}, 0 < x < \frac{\pi}2, \\
u_x(0, t) = 0, \\
u\left(\frac{\pi}2, t\right) = 0, \\
u(x, t) = 0.
\end{dcases}
\end{equation}

Рассмотрим задачу:
\begin{equation}
\begin{cases}
\frac{\partial u}{\partial t} = a^2\frac{\partial^2u}{\partial x^2} + f(x, t), 0 < x < l, 0 < t < +\infty, \\
u(0, t) = 0, u(l, t) = 0, \\
u(x, 0) = \varphi(x).
\end{cases}
\end{equation}
Ищем решение в виде $u(x, t) = u_1(x, t) + u_2(x, t)$, где $u_1$ и $u_2$ - решения систем
\begin{equation}
\begin{cases}
\frac{\partial u_1}{\partial t} = a^2\frac{\partial^2 u}{\partial x^2}, 0 < x < l, 0 < t < +\infty, \\
u_1(0, t) = u_1(l, t) = 0, \\
u_1(x, 0) = \varphi(x)
\end{cases}
\end{equation}
и
\begin{equation}
\begin{cases}
\frac{\partial u_2}{\partial t} = a^2\frac{\partial^2 u}{\partial x^2} + f(x, t), 0 < x < l, 0 < t < +\infty, \\
u_2(0, t) = u_2(l, t) = 0, \\
u_2(x, 0) = 0.
\end{cases}
\end{equation}
Решение задачи (41) записывается в виде:
\begin{equation}
u_1(x, t) = \int_0^lG(x, \xi, t)\varphi(\xi)d\xi,
\end{equation}
где
\begin{equation*}
G(x, \xi, t) = \frac2l\sum_{n = 1}^{\infty}e^{-\left(\frac{\pi n a}l\right)^2t}\sin\frac{\pi nx}l\sin\frac{\pi n\xi}l \text{ - функция-источник.}
\end{equation*}
Выясним физический смысл функции-источника:
\begin{equation}
u(x, t) = \int_0^lG(x, \xi, t)\varphi(\xi)d\xi = \int_{x^* - \varepsilon}^{x^* + \varepsilon}\varphi_{\varepsilon}(\xi)G(x, \xi, t)d\xi =
G(x, \tilde{\xi}, t)\int_{x^* - \varepsilon}^{x^* + \varepsilon}\varphi(\xi)d\xi = \frac{Q}{\varepsilon\rho}G(x, \xi, t).
\end{equation}
Таким образом, функция-источник описывает влияние точечного источника тепла.

Дельта-функция Дирака:
\begin{equation}
\delta(x) = \begin{cases}
0, x \neq 0, \\
+\infty, x = 0, \\
\end{cases}
\end{equation}
причём $\int_{-\infty}^{+\infty}\delta(x) = 1$.
$$\forall g(x) \in C(R) \int_{-\infty}^{+\infty}\delta(x)g(x)dx = g(0).$$
Введём функцию
\begin{equation}
H(x) =
\begin{cases}
1, x > 0, \\
0, x < 0.
\end{cases}
\end{equation}
$$\int_{-\infty}^{+\infty}H'(x) = H(x)$$
$$H'(x) = \delta(x)$$
Решение задачи (42) имеет вид:
\begin{equation}
u_2(x, t) = \int_0^t\int_0^lG(x, \xi, t - \tau)f(\xi, \tau)d\xi d\tau.
\end{equation}
Запись решений задач (41) и (42) в явном виде доказывает теорему существования.
\begin{Theorem}[Принцип максимума]
Если $u(x, t) \in C([0, l] \times [0, t])$ и $u_t = a^2u_{xx}, 0 < x < l, 0 < t \leq T$, то
$\max_{[0, l] \times [0, T]}u(x, t) = \max_{x = 0, x = l, t = 0}u(x, t)$.
\begin{proof}
Пусть $\max_{x = 0, x = l, t = 0}u(x, t) = M$ и $\exists (x_0, t_0) \in (0, l) \times (0, T)$, в
которой $u(x_0, t_0) = M + \varepsilon$. Тогда $u_x(x_0, t_0) = 0, u_{xx}(x_0, t_0) \leq 0, u_t(x_0, t_0) \geq 0$.
Введём функцию
\begin{equation}
v(x, t) = u(x, t) + k(t_0 - t).
\end{equation}
Для неё $v(x_0, t_0) = u(x_0, t_0) = M + \varepsilon$. Выберем k так, чтобы $k(t_0 - t) < kT < \frac{\varepsilon}2$.
При таком k
\begin{equation}
max_{x = 0, x = l, t = 0}v(x, t) \leq M + \frac{\varepsilon}2.
\end{equation}
Поскольку $v(x, t) \in C([0, l]\times[0, T])$, то $\exists (x_1, t_1)$, в которой $v(x_1, t_1) \geq M + \varepsilon$.
В этой точке
\begin{equation}
v(x_1, t_1) = \max_{[0, l]\times[0, T]}v(x, t), v_{xx} = u_{xx} \leq 0, v_t = u_t - k \geq 0.
\end{equation}
Противоречие.
\end{proof}
\end{Theorem}
\begin{Consequence}
Если
\begin{equation}
\begin{cases}
(u_1)_t = a^2(u_1)_{xx}, 0 < x < l, 0 < t, \\
u_1(0, t) = \nu_1(t), u_1(l, t) = nu_2(t), \\
u_1(x, 0) = \varphi_1(x)
\end{cases}
\text{ и }
\begin{cases}
(u_2)_t = a^2(u_2)_{xx}, \\
u_2(0, t) = \overline{\nu_1}(t), u_2(l, t) = \overline{\nu_2}(t), \\
u_2(x, 0) = \overline{\varphi_1}(x),
\end{cases}
\end{equation}
причём
\begin{equation}
\nu_1 \leq \overline{\nu_1}, \nu_2 \leq \overline{\nu_2}, \varphi \leq \overline{\varphi} \forall (x, t) \in [0, l] \times [0, T],
\end{equation}
то $u_1(x, t) \leq u_2(x, t)$.
\end{Consequence}
\begin{Consequence}
Если $|\nu_1(t)| \leq \varepsilon, |\nu_2(t)| \leq \varepsilon, |\varphi(x)| \leq \varepsilon$, то
$|u(x, t)| \leq \varepsilon$, т. е. задача (35) устойчива.
\end{Consequence}
\subsection{Теорема единственности}
\label{sec:orgbe76f3c}
\begin{Theorem}[Теорема единственности]
Задача (35) имеет одно решение.
\begin{proof}
Пусть есть два решения задачи (35): $u_1(x, t)$ и $u_2(x, t)$. Положим $v(x, t)$. Тогда $v(x, t)$ - решение задачи:
\begin{equation}
\begin{cases}
v_t = a^2v_{xx}, 0 < x < l, 0 < t < T, \\
v(0, t) = v(l, t) = 0, \\
v(x, 0) = 0.
\end{cases}
\end{equation}
Эта задача имеет только нулевое решение.
\end{proof}
\end{Theorem}
Рассмотрим задачу
\begin{equation}
\begin{cases}
-u_t = u_{xx}, 0 < x < \pi, 0 < t < +\infty, \\
u(0, t) = u(\pi, t) = 0, \\
u(x, 0) = \varepsilon\sin x.
\end{cases}
\end{equation}
Её решением будет функция $u(x, t) = \varepsilon\sin xe^t \Rightarrow$ задача поставлена некорректно.

Рассмотрим теперь задачу другого типа:
\begin{equation}
\begin{cases}
u_t = u_{xx}, 0 < x < l, 0 < t \leq T, \\
u_x(0, t) = u_x(l, t) = 0, \\
u(x, 0) = 0.
\end{cases}
\end{equation}
Рассмотрим функционал $E(t) = \frac12\int_0^lu^2dx$. Так как $uu_t = uu_{xx}$, то
\begin{equation}
\frac12\frac{d}{dt}\int_0^lu^2dx = \int_0^luu_{xx}dx = uu_x - \int_0^lu_x^2dx.
\end{equation}
Тогда $E(t)$ - решение системы:
\begin{equation}
\frac{dE}{dt} \leq 0, \\
E(0) = 0,
\end{equation}
откуда $E(t) = 0 \Rightarrow u(x, t) = 0$.
\zall
\subsection{Задача Коши для уравнения теплопроводности}
\label{sec:org9bba658}
Рассмотрим задачи
\begin{equation}
\begin{cases}
\frac{\partial u}{\partial t} = a^2\frac{\partial^2 u}{\partial t^2}, -\infty < x < +\infty, 0 < t \leq T, \\
u(x, 0) = \varphi(x)
\end{cases}
\end{equation}
и
\begin{equation}
\begin{cases}
u_t = a^2u_{xx} + f(x, t), -\infty < x < +\infty, 0 < t \leq T, \\
u(x, 0) = 0.
\end{cases}
\end{equation}
В такой постановке решение задачи не единственно, нужно добавить условие ограниченности(либо условие "достаточно медленного" роста).
\begin{Theorem}[Теорема единственности]
Задача
\begin{equation}
\begin{cases}
v_t = a^2v_{xx}, -\infty < x < +\infty, 0 < t, \\
v(x, 0) = 0, \\
|v(x, t)| < 2M
\end{cases}
\end{equation}
где $v(x, t) = u_1(x, t) - u_2(x ,t)$ имеет только нулевое решение.
\begin{proof}
Введём функцию $V(x, t) = \frac{4M}{L^2}\left(\frac{x^2}2 + a^2t\right)$.
Тогда $v|_{t = 0} = \frac{2M}{L^2}x^2 \geq 0, v|_{x = \pm L} = 2M + \frac{a^2t4M}{L^2} > 2M$.
Но $|v(x, t)| < V(x, t)$, а $\lim_{h \to +\infty}V(x, t) = 0 \Rightarrow v(x, t) \equiv 0$.
\end{proof}
\end{Theorem}
\begin{Theorem}[Теорема существования]
Существует решение задачи Коши:
\begin{equation}
\begin{cases}
u_t = a^2u_{xx}, -\infty < x < +\infty, 0 < t \leq T, \\
u(x, 0) = \varphi(x), \\
|u(x, t)| < M.
\end{cases}
\end{equation}
\begin{proof}
Ищем решение в виде $u(x, t) = X(x)T(t)$. Тогда
\begin{equation}
\frac{X''}X = \frac{T'}T = -\lambda^2.
\end{equation}
Тогда $X(x) = C_1e^{i\lambda x} + C_2e^{-i\lambda x}, T(t) = e^{-a^2\lambda^2t}$. Решение записывается
в виде $u_{\lambda}(x, t) = A(\lambda)e^{i\lambda x}e^{-a^2\lambda^2t}, -\infty < \lambda < +\infty$.
\begin{equation}
u(x, t) = \int_{-\infty}^{+\infty}A(\lambda)e^{i\lambda x}e^{-a^2\lambda^2t}d\lambda,
\end{equation}
где
\begin{equation}
A(\lambda) = \frac{1}{2\pi}\int_{-\infty}\varphi(\xi)e^{-i\lambda\xi}d\xi
\end{equation}
или
\begin{equation}
u(x, t) = \int_{-\infty}^{+\infty}\varphi(\xi)G(x, \xi, t)d\xi,
\end{equation}
где
\begin{equation}
G(x, \xi, t) = \frac1{2\pi}\int_{-\infty}^{+\infty}e^{-a^2\lambda^2t + i\lambda(x - \xi)}d\lambda.
\end{equation}
Найдём аналитическое выражение для $G(x, \xi, t)$:
\begin{multline}
G(x, \xi, t) = \frac1{2\pi}\int_{-\infty}^{+\infty}e^{-a^2\lambda^2t + i\lambda(x - \xi)}d\lambda =
\frac1{2\pi}\int_{-\infty}^{+\infty}e^{-a^2t\left(\lambda^2 - 2i\lambda\frac{x - \xi}{2a^2t} + \left(i\frac{x - \xi}{2a^2t}\right)^2 - \left(i\frac{x - \xi}{2a^2t}\right)^2\right)} = \\
= \frac1{2\pi}\int_{-\infty - i\sigma}^{+\infty - i\sigma}e^{-a^2tp^2}e^{-\frac{(x - \xi)^2}{4at}}dp
= \frac1{2\pi}\frac1{a\sqrt{t}}\int_{-\infty - i\sigma}^{+\infty - i\sigma}e^{-z^2}dze^{\frac{-(x - \xi)^2}{4a^2t}} = \frac1{2a\sqrt{\pi t}e^{\frac{(x - \xi)^2}{4a^2t}}}.
\end{multline}
\end{proof}
\end{Theorem}
\subsection{Метод подобия}
\label{sec:org04d8200}
\begin{equation*}
u_t = a^2u_{xx}
\end{equation*}
Сделаем замену $x' = kx, t' = k^2t$, тогда $u_{t'} = a^2u_{xx}$,
\begin{equation}
\frac{\partial u}{\partial t} = \frac{\partial u}{\partial t'}k^2,
\frac{\partial u}{\partial x} = \frac{\partial u}{\partial x'}k,
\frac{\partial^2 u}{\partial x^2} = \frac{\partial^2 u}{\partial x^2}k^2.
\end{equation}
Рассмотрим задачи
\begin{equation}
\begin{cases}
u_t = a^2u_{xx}, \\
u(x, 0) = \begin{cases}
u_0, x > 0, \\
0, x < 0.
\end{cases}
\end{cases}
\end{equation}
и
\begin{equation}
\begin{cases}
u_{t'} = a^2u_{xx}, \\
u(x', 0) = \begin{cases}
u_0, x > 0, \\
0, x < 0.
\end{cases}
\end{cases}
\end{equation}
$$u(x, t) = u(x', t') = u(kx, k^2t)$$
Положим $k = \frac1{2\sqrt{t}}$, тогда $u(x, t) = u\left(\frac{x}{2\sqrt{t}}4\right)$.

Ищем решение задачи (12) в виде $u(x, t) = u_0f\left(\frac{x}{2\sqrt{t}}\right)$.
\section{Уравнения эллиптического типа}
\label{sec:orga067f13}
\zall
Уравнение Лапласа имеет вид
\begin{equation}
\Delta u = u_{xx} + u_{yy} + u_{zz} = 0,
\end{equation}
Неоднородное уравнение Лапласа
\begin{equation}
\Delta u = -f(x, y, z)
\end{equation}
называется уравнением Пуассона.
\subsection{Стационарная теплопроводность}
\label{sec:orgfd53b58}
В нестационарном случае
\begin{equation}
c\rho\frac{\partial u}{\partial t} = \operatorname{div}(k\operatorname{grad} u) + f.
\end{equation}
Если $\frac{\partial u}{\partial t} = 0, k = k_0 = const$, получаем:
\begin{equation}
\operatorname{div}(\operatorname{grad} u) = -\frac{f}{k_0}
\end{equation}
В декартовых координатах:
\begin{equation}
\operatorname{div}(\operatorname{grad} u) = u_{xx} + u_{yy} + u_{zz},
\end{equation}
В цилиндрических:
\begin{equation}
\operatorname{div}(\operatorname{grad} u) = \frac1r\frac{\partial}{\partial u}(r\frac{\partial u}{\partial r}) +
\frac1{r^2}\frac{\partial^2 u}{\partial \varphi^2} + \frac{\partial^2u}{\partial z^2}
\end{equation}
В сферических:
\begin{equation}
\Delta u = \Delta_ru + \frac1{r^2}\Delta_{\theta, \varphi}u, { где}
\end{equation}
\begin{equation}
\Delta_ru = \frac1{r^2}\frac{d}{dr}\left(r^2\frac{du}{dr}\right) = \frac1r\frac{d^2}{dr^2}(ru) =
\frac{d^2u}{dr^2} + \frac2r\frac{du}{dr},
\end{equation}
\begin{equation}
\Delta_{\theta, \varphi}u = \frac1{\sin\theta}\frac{\partial}{\partial\theta}\left(\sin\theta\frac{\partial u}{\partial\theta}\right) +
\frac1{\sin^2\theta}\frac{\partial^2u}{\partial\varphi^2}
\end{equation}
Видно, что оператор Лапласа имеет вид суммы вторых производных \textbf{только в декартовых
координатах}.

Задача Дирихле:
\begin{equation}
\begin{cases}
\Delta u = -f, \\
u|_{Sigma} = g(p),
\end{cases}
\end{equation}
Задача Неймана:
\begin{equation}
\begin{cases}
\Delta u = -f, \\
\frac{\partial u}{\partial n}|_{\Sigma} = h(p),
\end{cases}
\end{equation}
Краевое условие третьего рода:
\begin{equation}
\Delta u = -f, \\
\frac{\partial u}{\partial n} + hu|_{\Sigma} = p(t).
\end{equation}
\subsection{Потенциальное течение жидкости}
\label{sec:org80aa412}
\begin{equation}
\begin{cases}
\operatorname{div}\vec v = 0, \\
\operatorname{rot}\vec v = 0, \text{ нет вихрей.}
\end{cases}
\end{equation}
Так как $\operatorname{rot}\vec v$, то $\vec v = \operatorname{grad}u$, тогда $\Delta u = 0$.
Получаем задачу
\begin{equation}
\begin{cases}
\Delta u = 0, \\
v_n|_{\Sigma} = 0, \frac{\partial u}{\partial n}|_{\Sigma} = 0, \\
v_x|+{\pm \infty} = v_0, \frac{\partial u}{\partial x}|_{x = \pm \infty} = v_0.
\end{cases}
\end{equation}
\subsection{Задачи электростатики}
\label{sec:orgaec48aa}
Уравнения Максвелла: \\
Гауссова система:
\begin{equation}
\begin{cases}
\operatorname{div}\vec E = 4\pi\rho, \\
rd\vec E = -\frac1c\frac{\partial\vec B}{\partial t}
\end{cases}
\end{equation}
Получаем уравнение
\begin{equation}
\Delta u = -4\pi\rho.
\end{equation}

Система СИ:
\begin{equation}
\begin{cases}
\operatorname{div}\vec D = \rho, \\
\operatorname{rot}\vec E = -\frac{\partial\vec B}{\partial t} = 0, \\
\vec D = \varepsilon\varepsilon_0\vec E.
\end{cases}
\end{equation}
\subsection{Фундаментальные решения уравнения Лапласа}
\label{sec:org9543c89}
В трёхмерном случае $u = u(r), r = \sqrt{x^2 + y^2 + z^2}$, тогда
\begin{equation}
\frac1{r^2}\frac{d}{dr}\left(r^2\frac{du}{dr}\right) = 0, r \neq 0.
\end{equation}
\begin{equation}
\frac1r\frac{d^2}{dr^2}(ru) = 0 \Rightarrow ru = C_2r + C_1 \Rightarrow u(r) = \frac1r, r \neq 0.
\end{equation}
В двумерном случае:
\begin{equation}
\frac1r\frac{d}{dr}\left(r\frac{du}{dr}\right) = 0,
\end{equation}
\begin{equation}
d\frac{du}{dr} = C_1,
\end{equation}
\begin{equation}
\frac{du}{dr} = \frac{C_1}r,
\end{equation}
\begin{equation}
u(r) = C_1\ln r + C_2, \\
u(r) = \ln\frac1r, r = \sqrt{x^2 + y^2}.
\end{equation}
Гармоническая функция, зависящая только от x:
\begin{equation}
\Delta u = 0, u = u(x), u_{xx} = 0 \Rightarrow u = C_1x + C_2.
\end{equation}
\subsection{Формулы Грина}
\label{sec:org614efc1}
\begin{equation}
\iiint_T\operatorname{div}\vec Ad\tau = \iint_{\Sigma}A_nd\sigma,
\vec A \in C^1(T) \cap C(\overline T), \overline T = T \cup \Sigma
\end{equation}
Пусть теперь $u, v \in C^2(T) \cap C^1(\overline T), \vec A = u\operatorname{grad}v, \div A = \operatorname{div}(u\operatorname{grad}v)
= \nabla(u\nabla v) = u\operatorname{div}\operatorname{grad}v + \operatorname{grad}u\operatorname{grad}v = u\Delta v + \operatorname{grad}u\operatorname{grad}v$. Тогда
\begin{equation}
\iiint_Tu\Delta vd\tau = \iint_{\Sigma}u\frac{\partial v}{\partial n}d\sigma -
\iiint_T(\operatorname{grad}u - \operatorname{grad}v)d\tau, \text{ - первая формула Грина.}
\end{equation}
Поменяем местами $u$ и $v$:
\begin{equation}
\iiint_T(v\Delta u)d\tau = \iint_{\Sigma}v\frac{\partial u}{\partial n}d\sigma -
\iiint_T(\operatorname{grad}u - \operatorname{grad}v)d\tau
\end{equation}
Вычтя из первого равенства второе, получим \textbf{вторую формулу Грина}:
\begin{equation}
\iiint_T(u\Delta v - v\Delta u) = \iint_{\Sigma}\left(u\frac{\partial v}{\partial n} -
v\frac{\partial u}{\partial n}\right)\delta\sigma
\end{equation}
\textbf{Интегральная формула Грина(третья формула Грина)}:
\begin{equation}
v(M) = v(x, y, z) = \frac1{R_{M_0M}}
\end{equation}
Применим к области T вторую формулу Грина, "выкинув" из неё сферу $K_{\varepsilon}$:
\begin{multline}
p\iiint_{K_{\varepsilon}(M_o)}\left(u\Delta\frac1{R_{M_0M}}-\frac1{R_{M_0M}}\Delta u\right)d\tau =
\iint_{\Sigma}\left(u\frac{\partial}{\partial n}\left(\frac1{R_{M_0M}}\right) - \frac1{R_{M_0M}}\frac{\partial u}{\partial n}\right)d\sigma + \\
+ \iint_{S_{\varepsilon}}\left(u\frac{\partial}{\partial u}\left(\frac1{R_{M_0M}}\right) - \frac1{R_{M_0M}}\frac{\partial u}{\partial n}\right)d\sigma
\end{multline}
Если $M \neq M_0$, то $\Delta\left(\frac1{R_{M_0M}}\right) = 0$. При $\varepsilon \to 0$ получаем:
\begin{equation}
-\iiint_T\frac{\Delta u}{R_{M_0M}}d\tau = \iint_{\Sigma}\left(u\frac{\partial}{\partial n}\left(\frac1{R_{M_0M}}\right) - \frac1R_{M_0M}\frac{\partial u}{\partial n}\right)d\sigma
+ 4\pi u(M_0)
\end{equation}
Или, если $M_0 \in T$, то:
\begin{equation}
u(M_0) = \frac1{4\pi}\iint_{\Sigma}\frac1{R_{M_0M}}\frac{\partial u}{\partial n}\left(\frac1{R_{M_0M}}\right)d\sigma -
\frac1{4\pi}\iiint_T\frac{\Delta u}{R_{M_0M}}d\tau
\end{equation}
Если дополнительно $\Delta u = 0$, то $u(M_0)$ представляется в виде:
\begin{equation}
u(M_0) = u_1(M_0) + u_2(M_0), \text{ где}
\end{equation}
\begin{equation}
u_1(M_0) = \iint_{\Sigma}\frac{\mu(p)}{R_{M_0M}}d\sigma,
\end{equation}
\begin{equation}
u_2(M_0) = \iint_{\Sigma}v(p)\frac{\partial}{\partial n}\left(\frac1{R_{M_0M}}\right)d\sigma.
\end{equation}
Поскольку $u_1(M_0)$ и $u_2(M_0) \in C^{\infty}$, то $u(M_0) \in C^{\infty}(R^3)$.

В двумерном случае $v(M) = \ln\frac1{R_{M_0M}}$:
\begin{equation}
u(M_0) = \frac1{2\pi}\int_C\ln\frac1{R_{M_0M}}\frac{\partial u}{\partial n}dl -
\frac1{2\pi}\int_Cu\frac{\partial}{\partial n}\left(\frac1{R_{M_0M}}\right)dl -
\frac1{2\pi}\iint_D\ln\frac1{R_{M_0M}}\Delta ud\sigma.
\end{equation}
Если $\Delta u = 0$:
\begin{equation}
u(M_0) = \frac1{2\pi}\int_C\ln\frac1{R_{M_0P}}\frac{\partial u}{\partial n}dl -
\frac1{2\pi}\int_Cu\frac{\partial}{\partial n}\ln\frac1{R_{M_0P}}dP.
\end{equation}
\subsection{Свойства гармонических функций}
\label{sec:org2307692}
\begin{Theorem}
Если $\Delta u = 0$ в $T$, $S \subset T$, $S$ - поверхность, то
\begin{equation}
\iint_S\frac{\partial u}{\partial n}d\sigma = 0.
\end{equation}
\begin{proof}
По второй формуле Грина:
\begin{equation}
\iiint_{K_S}u(\Delta 1)d\tau - \iiint_{K_S}1\Delta ud\tau =
\iint_S\left(u\frac{\partial}{\partial n}(1) - 1\frac{\partial u}{\partial n}\right)d\sigma
\Rightarrow \iint_S\frac{\partial u}{\partial n}d\sigma = 0.
\end{equation}
\end{proof}
\end{Theorem}
\begin{Consequence}
Если дана задача Неймана
\begin{equation}
\begin{cases}
\Delta u = 0 \text{ в T}, \\
\frac{\partial u}{\partial n}|_{\Sigma} = g(P)
\end{cases}
\end{equation}
и она разрешима, то
\begin{equation}
\iint_{\Sigma}g(P)d\sigma = 0
\end{equation}
\end{Consequence}
\begin{Theorem}
Пусть $\Delta u = 0$. Если $T = K_a(M_0)$, то
\begin{equation}
u(M_0) = \frac1{4\pi a^2}\iint_{S_a}ud\sigma.
\end{equation}
\begin{proof}
По третьей формуле Грина:
\begin{equation}
u(M_0) = -\frac1{4\pi}\iint_{S_a}\frac{\partial}{\partial n}\left(\frac1{R_{M_0P}}\right)d\sigma +
\frac1{4\pi}\iint_{S_a}\frac1{R_{M_0P}}\frac{\partial u}{\partial n}d\sigma =
\frac1{4\pi a^2}\iint_{S_a}ud\sigma + \frac1{4\pi a}\iint_{S_a}\frac{\partial u}{\partial n}d\sigma =
\frac1{4\pi a^2}\iint_{S_a}ud\sigma.
\end{equation}
\end{proof}
\end{Theorem}
\begin{Theorem}[Принцип максимума]
Если $u(M) \in C^2(T) \cap C(\overline T), \Delta u = 0$ в $T$. Тогда экстремумы $u(M)$
достигаются на границе.
\begin{proof}
Пусть $\max_{\overline T} u(M_0)$. Окружим эту точку сферой радиуса $\rho$. Получим, что
\begin{equation}
u(M_0) = \frac1{4\pi a^2}\iint_{S_{\rho}(M_0)}u(P)d\sigma
\end{equation}
Если $\exists P \in S_{\rho}(M_0), u(P) < u(M_0)$, тогда
$u(M_0) = \frac1{4\pi a^2}\iint_{S_{\rho}(M_0)}u(P)d\sigma < \frac1{4\pi a^2}u(M_0)\iint_{S_{\rho}}d\sigma = u(M_0)$.
Получили противоречие.
\end{proof}
\end{Theorem}
\begin{Consequence}
Если $\Delta u_1 = \Delta u_2 = 0$ в $T$, $u_1|_{\Sigma} \leq u_2|_{\Sigma}$, то $u_1(M) \leq u_2(M) \forall M \in \overline T$.
\begin{proof}
Нужно применить принцип максимума к функции $V = u_1 - u_2$.
\end{proof}
\end{Consequence}
\end{document}
