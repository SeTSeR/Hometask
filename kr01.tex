% Created 2020-04-23 Thu 12:35
% Intended LaTeX compiler: pdflatex
\documentclass[11pt]{article}
\usepackage[utf8]{inputenc}
\usepackage[T1]{fontenc}
\usepackage{graphicx}
\usepackage{grffile}
\usepackage{longtable}
\usepackage{wrapfig}
\usepackage{rotating}
\usepackage[normalem]{ulem}
\usepackage{amsmath}
\usepackage{textcomp}
\usepackage{amssymb}
\usepackage{capt-of}
\usepackage{hyperref}
\usepackage{minted}
\usepackage{amsmath}
\usepackage{esint}
\usepackage[english, russian]{babel}
\usepackage{mathtools}
\usepackage{amsthm}
\usepackage{listings}
\usepackage[top=0.8in, bottom=0.75in, left=0.625in, right=0.625in]{geometry}
\def\zall{\setcounter{lem}{0}\setcounter{cnsqnc}{0}\setcounter{th}{0}\setcounter{Cmt}{0}\setcounter{equation}{0}}
\newcounter{lem}\setcounter{lem}{0}
\def\lm{\par\smallskip\refstepcounter{lem}\textbf{\arabic{lem}}}
\newtheorem*{Lemma}{Лемма \lm}
\newcounter{th}\setcounter{th}{0}
\def\th{\par\smallskip\refstepcounter{th}\textbf{\arabic{th}}}
\newtheorem*{Theorem}{Теорема \th}
\newcounter{cnsqnc}\setcounter{cnsqnc}{0}
\def\cnsqnc{\par\smallskip\refstepcounter{cnsqnc}\textbf{\arabic{cnsqnc}}}
\newtheorem*{Consequence}{Следствие \cnsqnc}
\newcounter{Cmt}\setcounter{Cmt}{0}
\def\cmt{\par\smallskip\refstepcounter{Cmt}\textbf{\arabic{Cmt}}}
\newtheorem*{Note}{Замечание \cmt}
\author{Sergey Makarov}
\date{\today}
\title{}
\hypersetup{
 pdfauthor={Sergey Makarov},
 pdftitle={},
 pdfkeywords={},
 pdfsubject={},
 pdfcreator={Emacs 26.3 (Org mode 9.3)}, 
 pdflang={English}}
\begin{document}


\section{Задача 1}
\label{sec:orgdf0d08d}
Выписать 3 шага градиентного спуска с постоянным шагом для задачи поиска минимума функции
\begin{equation}
f(x_1, x_2) = x_1^2 + 2x_2^2
\end{equation}
при начальном приближении $(x_1, x_2) = (-1, 1)$ и шаге $a = 1$. При каком значении шага метод сойдётся?
\subsection{Решение}
\label{sec:orgeaa121a}
Аналитическая формула для градиента:
\begin{equation}
\operatorname{grad} f = \left\{\frac{\partial f}{\partial x_1}, \frac{\partial f}{\partial x_2}\right\} = \{2x_1, 4x_2\}
\end{equation}
\begin{equation}
M_0 = (-1, 1)
\end{equation}
\begin{equation}
M_1 = M_0 - \operatorname{grad}f(M_0) = (-1, 1) - (-2, 4) = (1, -3)
\end{equation}
\begin{equation}
M_2 = M_1 - \operatorname{grad}f(M_1) = (1, -3) - (2, -12) = (-1, 9)
\end{equation}
\begin{equation}
M_3 = M_2 - \operatorname{grad}f(M_2) = (-1, 9) - (-2, 36) = (1, -27)
\end{equation}

Чтобы найти значение \(a\), при котором метод сойдётся, рассмотрим шаг при произвольном \(M_n(x_0, x_1)\)
\begin{equation}
M_{n + 1} = M_n - a\operatorname{grad}f(M_n) \Rightarrow |M_{n + 1} - M_n| = a|\operatorname{grad}(M_n)| = a\sqrt{4x_0^2 + 16x_1^2}
\end{equation}
При \(a < 1\) метод сходится.

\begin{multline}
|M_{n + 1} - M_n| = a|\operatorname{grad}f(M_n)| = a|\operatorname{grad}f(M_n) - \operatorname{grad}f(M_0) + \operatorname{grad}f(M_0)| \leq \\
a(|\operatorname{grad}f(M_n) - \operatorname{grad}f(M_{n + 1})| + |\operatoname{grad}|f(M_{n + 1})) \leq \\
aL|M_n - M_{n + 1}| + a|\operatorname{grad}f(M_{n + 1})| \Rightarrow (1 - aL)|M_{n} - M_{n + 1}| \leq a|\operatorname{grad}f(M_{n + 1})|
\end{multline}
\begin{equation}
|M_{n} - M_{n + 1}| \leq \frac{a}{1 - aL}|\operatorname{grad}f(M_{n + 1})|
\end{equation}
\end{document}
