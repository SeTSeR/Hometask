% Created 2019-09-17 Tue 11:49
% Intended LaTeX compiler: pdflatex
\documentclass[11pt]{article}
\usepackage[utf8]{inputenc}
\usepackage[T1]{fontenc}
\usepackage{graphicx}
\usepackage{grffile}
\usepackage{longtable}
\usepackage{wrapfig}
\usepackage{rotating}
\usepackage[normalem]{ulem}
\usepackage{amsmath}
\usepackage{textcomp}
\usepackage{amssymb}
\usepackage{capt-of}
\usepackage{hyperref}
\usepackage{amsmath}
\usepackage{esint}
\usepackage[english, russian]{babel}
\usepackage{mathtools}
\usepackage{amsthm}
\usepackage[top=0.8in, bottom=0.75in, left=0.625in, right=0.625in]{geometry}
\def\zall{\setcounter{lem}{0}\setcounter{cnsqnc}{0}\setcounter{th}{0}\setcounter{Cmt}{0}\setcounter{equation}{0}}
\newcounter{lem}\setcounter{lem}{0}
\def\lm{\par\smallskip\refstepcounter{lem}\textbf{\arabic{lem}}}
\newtheorem*{Lemma}{Лемма \lm}
\newcounter{th}\setcounter{th}{0}
\def\th{\par\smallskip\refstepcounter{th}\textbf{\arabic{th}}}
\newtheorem*{Theorem}{Теорема \th}
\newcounter{cnsqnc}\setcounter{cnsqnc}{0}
\def\cnsqnc{\par\smallskip\refstepcounter{cnsqnc}\textbf{\arabic{cnsqnc}}}
\newtheorem*{Consequence}{Следствие \cnsqnc}
\newcounter{Cmt}\setcounter{Cmt}{0}
\def\cmt{\par\smallskip\refstepcounter{Cmt}\textbf{\arabic{Cmt}}}
\newtheorem*{Note}{Замечание \cmt}
\author{Sergey Makarov}
\date{\today}
\title{}
\hypersetup{
 pdfauthor={Sergey Makarov},
 pdftitle={},
 pdfkeywords={},
 pdfsubject={},
 pdfcreator={Emacs 26.3 (Org mode 9.1.9)}, 
 pdflang={English}}
\begin{document}

\zall

Андрей Николаевич Оленин, andrei\(_{\text{olenin}}\)@mail.ru, comp.ilc.edu.ru

Темы:
\begin{enumerate}
\item Светодиоды и инжекционные схемы
\item Фотодиоды и их быстродействие
\item ПЗС и КМОП матрицы
\item ЦАП и АЦП
\end{enumerate}

Уравнение Шрёдингера:
\begin{equation}
-i\hslash\frac{\partial\Psi}{\partial t} = \frac{\hslash^2}{2m}\Delta\Psi - U(x, y, z, t)\Psi
\end{equation}
где
$i^2 = -1$, $\hslash$ - постоянная Планка, $m$ - масса электрона, $\Psi(x, y, z, t)$ - волновая функция,
$U$ - потенциальная энергия электрического взаимодействия.
$|\Psi(x, y, z, t)|^2$ - вероятность обнаружить электрон в (x, y, z) в момент времени t.
\end{document}
