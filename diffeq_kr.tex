% Created 2019-05-21 Tue 21:01
% Intended LaTeX compiler: pdflatex
\documentclass[11pt]{article}
\usepackage[utf8]{inputenc}
\usepackage[T1]{fontenc}
\usepackage{graphicx}
\usepackage{grffile}
\usepackage{longtable}
\usepackage{wrapfig}
\usepackage{rotating}
\usepackage[normalem]{ulem}
\usepackage{amsmath}
\usepackage{textcomp}
\usepackage{amssymb}
\usepackage{capt-of}
\usepackage{hyperref}
\usepackage{amsmath}
\usepackage[english, russian]{babel}
\usepackage{mathtools}
\author{Sergey Makarov}
\date{\today}
\title{}
\hypersetup{
 pdfauthor={Sergey Makarov},
 pdftitle={},
 pdfkeywords={},
 pdfsubject={},
 pdfcreator={Emacs 26.2 (Org mode 9.1.9)}, 
 pdflang={English}}
\begin{document}

\section{Задача 1}
\label{sec:orgd868644}
Найти \(\frac{\partial x}{\partial\mu}\) при \(\mu = 0\) для решения задачи
$$x' = \frac{x}t + \mu te^{-x}, x(1) = 1$$

Продифференцируем задачу по \(\mu\), обозначив \(u = \frac{\partial x}{\partial\mu}\):
\begin{equation*}
\begin{dcases}
u' = \frac ut + t(e^{-x} - \mu e^{-x}u), \\
u(1) = 0
\end{dcases}
\end{equation*}
Подставляя \(\mu = 0\), найдём:
\begin{equation}
\begin{dcases}
u' = \frac{u}t + te^{-x}, \\
u(1) = 0
\end{dcases}
\end{equation}
Теперь подставим \(\mu = 0\) в начальное уравнение системы, чтобы найти \(x\):
\begin{equation*}
\begin{dcases}
x' = \frac{x}t, \\
x(1) = 1
\end{dcases}
\Rightarrow x = t
\end{equation*}
Подставим полученное выражение для \(x\) в (1):
\begin{equation*}
\begin{dcases}
u' = \frac{u}t + te^{-t}, \\
u(1) = 0
\end{dcases}
\Rightarrow
\begin{dcases}
u' - \frac{u}t = te^{-t}, \\
u(1) = 0
\end{dcases}
\end{equation*}
Это линейное ОДУ первого порядка, общее решение однородного уравнения имеет вид:
$$u = Ct, \forall C$$
Найдём решение неоднородного уравнения в виде \(u = C(t)t\):
\begin{equation*}
u' - \frac{u}t = C't + C - C = te^{-t} \Rightarrow C' = e^{-t} \Rightarrow C(t) = C - e^{-t}
\end{equation*}
Тогда \(u = Ct - te^{-t}, \forall C\).
Вспоминая, что \(u(1) = 0\), найдём окончательно, что \(u(t) = \frac{t}e - \frac{t}{e^t}\).
\section{Задача 2}
\label{sec:orgc81d4a8}
Найти решение уравнения в виде степенного ряда до 4 степени:
\begin{equation*}
\begin{dcases}
y' = x^2 + y^3, \\
y(1) = 1
\end{dcases}
\end{equation*}

Ищем решение в виде: \(y = a_0 + a_1(x - 1) + a_2(x - 1)^2 + a_3(x - 1)^3 + a_4(x - 1)^4 + \ldots\). Тогда
$$y' = a_1 + 2a_2(x - 1) + 3a_3(x - 1)^2 + 4a_4(x - 1)^3 + \ldots,$$
$$y^3 = a_0^3 + 3a_0^2a_1(x - 1) + (3a_0^2a_2 + 3a_0a_1^2)(x - 1)^2 + (a_3 + 6a_0a_1a_2 + a_1^3)(x - 1)^3 + \ldots$$
Подставив это в начальное уравнение, получим:
\begin{equation*}
\begin{cases}
a_1 + 2a_2(x - 1) + 3a_3(x - 1)^2 + 4a_4(x - 1)^3 + \ldots = a_0^3 + 3a_0^2a_1(x - 1) + (3a_0^2a_2 + 3a_0a_1^2 + 1)(x - 1)^2 + (a_3 + 6a_0a_1a_2 + a_1^3)(x - 1)^3 + \ldots, \\
y(1) = 1
\end{cases}
\end{equation*}
Или:
\begin{equation*}
\begin{cases}
a_1 = a_0^3, \\
2a_2 = 3a_0^2a_1, \\
3a_3 = 3a_0^2a_2 + 3a_0a_1^2 + 1, \\
4a_4 = a_3 + 6a_0a_1a_2 + a_1^3, \\
a_0 = 1
\end{cases}
\Rightarrow
\begin{dcases}
a_0 = 1, \\
a_1 = 1, \\
a_2 = \frac{3}2, \\
a_3 = \frac{17}6, \\
a_4 = \frac{77}{24}
\end{dcases}
\end{equation*}
Т. е. разложение имеет вид:
$$y = 1 + (x - 1) + \frac{3}2(x - 1)^2 + \frac{17}6(x - 1)^3 + \frac{77}{24}(x - 1)^4 + \ldots$$
0:58:06
\section{Задача 3}
\label{sec:org5dc04e3}
Исследовать на устойчивость нулевое решение:
\begin{equation*}
\begin{cases}
\dot x = \tg(z - y) - 2x, \\
\dot y = \sqrt{9 - 12x} - 3e^y, \\
\dot z = -3y
\end{cases}
\end{equation*}

Для начала линеаризуем систему:
$$\tg(z - y) - 2x = -2x - y + z + o(\rho)$$
$$\sqrt{9 - 12x} - 3e^y = 3\sqrt{1 - \frac{4}3x} - 3(1 + y + o(\rho)) = 3 - \frac{3}2\frac{4}3x - 3 - 3y + o(\rho) = -2x - 3y + o(\rho)$$
\begin{equation*}
\begin{cases}
\dot x = -2x - y + z + o(\rho), \\
\dot y = -2x - 3y + o(\rho), \\
\dot z = -3y
\end{cases}
\end{equation*}
Найдём собственные значения этой системы:
\begin{multline*}
|A - \lambda I| =
\begin{vmatrix}
-2 - \lambda & -1           & 1 \\
-2           & -3 - \lambda & 0 \\
0            & -3           & -\lambda
\end{vmatrix}
= \begin{vmatrix}
-2 & -3 - \lambda \\
0  & -3
\end{vmatrix} - \lambda
\begin{vmatrix}
-2 - \lambda & -1 \\
-2           & -3 - \lambda
\end{vmatrix}
= \\
= 6 + \lambda + 3 - \lambda(\lambda^2 + 5\lambda + 6 - 2) = -\lambda^3- 5\lambda^2
- 3\lambda + 9 = -\lambda^2(\lambda - 1) - 6\lambda(\lambda - 1) - 9(\lambda - 1) = \\
= (\lambda - 1)((\lambda - 3)^2 - 18) = (\lambda - 1)(\lambda - (3 - 3\sqrt 2))(\lambda - (3 + 3\sqrt 2))
\end{multline*}
В данном случае есть два положительных собственных значения \(\lambda = 1\) и \(\lambda = 3(1 + \sqrt 2)\),
поэтому система является неустойчивой.
1:25:30
\section{Задача 4}
\label{sec:org98bebd5}
Исследовать на устойчивость по определению:
\begin{equation*}
\begin{cases}
2ty' = y - y^3, \\
y(1) = 0
\end{cases}
\end{equation*}
Решим уравнение:
$$2ty' = y - y^3 \Rightarrow \frac{2dy}{y - y^3} = \frac{dt}t$$
$$\frac{2}{y - y^3} = \frac{A}{y} + \frac{B}{y - 1} + \frac{C}{y + 1} =
\frac{A(y^2 - 1) + B(y^2 + y) + C(y^2 - y)}{y + 1}$$
\begin{equation*}
\begin{cases}
A + B + C = 0, \\
B - C = 0, \\
A = -2
\end{cases}\
\Rightarrow
\begin{cases}
A = -2, \\
B = C = 1
\end{cases}
\end{equation*}
$$\int\frac{2dy}{y - y^3} = -2\int\frac{dy}y + \int\frac{d(y - 1)}{y - 1} +
\int\frac{d(y + 1)}{y + 1} = \ln\left(C\frac{y^2 - 1}{y^2}\right), C \neq 0$$
$$\int\frac{dt}t = \ln t$$
$$\frac{y^2 - 1}{y^2} = Ct \Rightarrow \frac{1}{y^2} = 1 - Ct, C \neq 0 \Rightarrow
y = \sqrt{\frac{1}{1 - Ct}}, C \neq 0$$
При делении на \(y - y^3\) мы не учли функции \(y = 0\) и \(y = \pm 1\). Все три функции также
являются решениями.
Граничному условию \(y(1) = 0\) удовлетворяет только нулевое решение. Найдём решение, отвечающее
граничному условию \(y(1) = \delta, 0 < \delta < 1\):
$$\sqrt{\frac{1}{1 - C}} = \delta \Rightarrow C = 1 - \frac{1}{\delta^2}$$
Этому значению \(C\) соответствует функция:
$$y = \sqrt{\frac{\delta^2}{\delta^2 - (\delta^2 - 1)t}}$$
Эта функция является убывающей на \(\matbb{R}_+\), поэтому \(y(t) < y(0) = \delta \forall t\).
Поэтому \(\forall \varepsilon > 0 \exists \delta = \varepsilon: |y_{\delta}(t) - 0| < \varepsilon\, \forall t\),
т. е. решение является устойчивым. Более того, оно асимптотически устойчиво, так как \(\lim_{t \to \infty}y(t) = 0\).
\end{document}
